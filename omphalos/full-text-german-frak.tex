\documentclass[a4paper, 11pt, oneside]{article}
\usepackage{gfsbaskerville}
% Load encoding definitions (after font package)
\usepackage[LGR,T1]{fontenc}
% Load encoding definitions (after font package)
\usepackage{textalpha}
\usepackage{yfonts}
\usepackage{wasysym}
\usepackage{cjhebrew}

% Babel package:
\usepackage[main=german,polutonikogreek]{babel}
\usepackage{listings}
\lstset{basicstyle=\ttfamily}

% With XeTeX$\$LuaTeX, load fontspec after babel to use Unicode
% fonts for Latin script and LGR for Greek:
\ifdefined\luatexversion \usepackage{fontspec}\fi
\ifdefined\XeTeXrevision \usepackage{fontspec}\fi

% "````Lipsiakos"' italic font `cbleipzig`:
\newcommand*{\lishape}{\fontencoding{LGR}\fontfamily{cmr}%
		       \fontshape{li}\selectfont}
\DeclareTextFontCommand{\textli}{\lishape}

\usepackage{booktabs}
\usepackage{graphicx}
\setlength{\emergencystretch}{15pt}
\graphicspath{ {./ } }
\usepackage[figurename=]{caption}
\usepackage{float}
\usepackage{fancyhdr}
\usepackage{microtype}
\usepackage{svg}
\usepackage{pdflscape}

%define custom symbols
\newcommand*\svgAAA{\includesvg[height=0.7em]{svgs/svg001.svg}}
\newcommand*\svgAAB{\includesvg[height=0.7em]{svgs/svg002.svg}}
\newcommand*\svgAAC{\includesvg[height=0.7em]{svgs/svg003.svg}}
\newcommand*\svgAAD{\includesvg[height=0.7em]{svgs/svg004-1.svg}}
\newcommand*\svgAAE{\includesvg[height=0.7em]{svgs/svg004-2.svg}}
\newcommand*\svgAAF{\includesvg[height=0.7em]{svgs/svg004-3.svg}}
\newcommand*\svgAAG{\includesvg[height=0.7em]{svgs/svg005.svg}}
\newcommand*\svgAAH{\includesvg[height=0.7em]{svgs/svg006.svg}}
\newcommand*\svgAAI{\includesvg[height=0.7em]{svgs/svg007.svg}}
\newcommand*\svgAAJ{\includesvg[height=0.7em]{svgs/svg008.svg}}
\newcommand*\svgAAK{\includesvg[height=0.7em]{svgs/svg009.svg}}
\newcommand*\svgAAL{\includesvg[height=0.7em]{svgs/svg010.svg}}
\newcommand*\svgAAM{\includesvg[height=0.7em]{svgs/svg011.svg}}
\newcommand*\svgAAN{\includesvg[height=0.7em]{svgs/svg012-1.svg}}
\newcommand*\svgAAO{\includesvg[height=0.7em]{svgs/svg012-2.svg}}
\newcommand*\svgAAP{\includesvg[height=0.7em]{svgs/svg012-3.svg}}
\newcommand*\svgAAQ{\includesvg[height=0.7em]{svgs/svg012-4.svg}}
\newcommand*\svgAAR{\includesvg[height=0.7em]{svgs/svg013.svg}}
\newcommand*\svgAAS{\includesvg[height=0.7em]{svgs/svg014.svg}}
\newcommand*\svgAAT{\includesvg[height=0.7em]{svgs/svg015.svg}}
\newcommand*\svgAAU{\includesvg[height=0.7em]{svgs/svg016.svg}}
\newcommand*\svgAAV{\includesvg[height=0.7em]{svgs/svg017-1.svg}}
\newcommand*\svgAAW{\includesvg[height=0.7em]{svgs/svg017-2.svg}}
\newcommand*\svgAAX{\includesvg[height=0.7em]{svgs/svg017-3.svg}}
\newcommand*\svgAAY{\includesvg[height=0.7em]{svgs/svg018.svg}}
\newcommand*\svgAAZ{\includesvg[height=0.7em]{svgs/svg019.svg}}
\newcommand*\svgABA{\includesvg[height=0.7em]{svgs/svg020.svg}}
\newcommand*\svgABB{\includesvg[height=0.7em]{svgs/svg021-1.svg}}
\newcommand*\svgABC{\includesvg[height=0.7em]{svgs/svg021-2.svg}}
\newcommand*\svgABD{\includesvg[height=0.7em]{svgs/svg022.svg}}
\newcommand*\svgABE{\includesvg[height=0.7em]{svgs/svg023.svg}}
\newcommand*\svgABF{\includesvg[height=0.7em]{svgs/svg024.svg}}
\newcommand*\svgABG{\includesvg[height=0.7em]{svgs/svg025.svg}}
\newcommand*\svgABH{\includesvg[height=0.7em]{svgs/svg026.svg}}
\newcommand*\svgABI{\includesvg[height=0.7em]{svgs/svg027.svg}}
\newcommand*\svgABJ{\includesvg[height=0.7em]{svgs/svg028.svg}}
\newcommand*\svgABK{\includesvg[height=0.7em]{svgs/svg029.svg}}
\newcommand*\svgABL{\includesvg[height=0.7em]{svgs/svg030.svg}}
\newcommand*\svgABM{\includesvg[height=0.7em]{svgs/svg031.svg}}
\newcommand*\svgABN{\includesvg[height=0.7em]{svgs/svg032-1.svg}}
\newcommand*\svgABO{\includesvg[height=0.7em]{svgs/svg032-2.svg}}
\newcommand*\svgABP{\includesvg[height=0.7em]{svgs/svg033-1.svg}}
\newcommand*\svgABQ{\includesvg[height=0.7em]{svgs/svg033-2.svg}}
\newcommand*\svgABR{\includesvg[height=0.7em]{svgs/svg034.svg}}
\newcommand*\svgABS{\includesvg[height=0.7em]{svgs/svg035.svg}}
\newcommand*\svgABT{\includesvg[height=0.7em]{svgs/svg036.svg}}
\newcommand*\svgABU{\includesvg[height=0.7em]{svgs/svg037.svg}}
\newcommand*\svgABV{\includesvg[height=0.7em]{svgs/svg038-1.svg}}
\newcommand*\svgABW{\includesvg[height=0.7em]{svgs/svg038-2.svg}}
\newcommand*\svgABX{\includesvg[height=0.7em]{svgs/svg038-3.svg}}
\newcommand*\svgABY{\includesvg[height=0.7em]{svgs/svg038-4.svg}}
\newcommand*\svgABZ{\includesvg[height=0.7em]{svgs/svg038-5.svg}}
\newcommand*\svgACA{\includesvg[height=0.7em]{svgs/svg039.svg}}

\usepackage[titles]{tocloft}
\usepackage{sectsty}

\allsectionsfont{\frakfamily}
\sectionfont{\frakfamily\Huge}
\subsectionfont{\frakfamily\LARGE}
\subsubsectionfont{\frakfamily\Large}
\paragraphfont{\frakfamily\large}

\begin{document}
\frakfamily
\renewcommand{\contentsname}{
\frakfamily{Inhaltsverzeichnis}
}

\renewcommand{\cftsecfont}{\frakfamily}
\renewcommand{\cftsubsecfont}{\frakfamily}
\renewcommand{\cftsubsubsecfont}{\frakfamily}

% fix toc page numbers
\let\origcftsecfont\cft
\let\origcftsecpagefont\cftsecpagefont
\let\origcftsecafterpnum\cftsecafterpnum
\renewcommand{\cftsecpagefont}{\frakfamily{\origcftsecpagefont}}
\renewcommand{\cftsecafterpnum}{\frakfamily{\origcftsecafterpnum}}
\let\origcftsubsecpagefont\cftsubsecpagefont
\let\origcftsubsecafterpnum\cftsubsecafterpnum
\renewcommand{\cftsubsecpagefont}{\frakfamily{\origcftsubsecpagefont}}
\renewcommand{\cftsubsecafterpnum}{\frakfamily{\origcftsubsecafterpnum}}
\let\origcftsubsubsecpagefont\cftsubsubsecpagefont
\let\origcftsubsubsecafterpnum\cftsubsubsecafterpnum
\renewcommand{\cftsubsubsecpagefont}{\frakfamily{\origcftsubsubsecpagefont}}
\renewcommand{\cftsubsubsecafterpnum}{\frakfamily{\origcftsubsubsecafterpnum}}

\renewcommand{\thefigure}{\frakfamily{\arabic{figure}}}
\renewcommand\thefootnote{\frakfamily{\arabic{footnote}}}
\let\oldfootnote\footnote
    \renewcommand{\footnote}[1]{\oldfootnote{\frakfamily\large#1}}
\begin{titlepage} % Suppresses headers and footers on the title page
	\centering % Centre everything on the title page
	%\scshape % Use small caps for all text on the title page

	%------------------------------------------------
	%	Title
	%------------------------------------------------
	
	\rule{\textwidth}{1.6pt}\vspace*{-\baselineskip}\vspace*{2pt} % Thick horizontal rule
	\rule{\textwidth}{0.4pt} % Thin horizontal rule
	
	\vspace{1\baselineskip} % Whitespace above the title
	
	{\scshape\Huge Omphalos}
	
	\vspace{1\baselineskip} % Whitespace above the title

	\rule{\textwidth}{0.4pt}\vspace*{-\baselineskip}\vspace{3.2pt} % Thin horizontal rule
	\rule{\textwidth}{1.6pt} % Thick horizontal rule
	
	\vspace{1\baselineskip} % Whitespace after the title block
	
	%------------------------------------------------
	%	Subtitle
	%------------------------------------------------
	
        {\scshape Eine Philologisch-Archäologisch-Volkskundliche Abhandlung \\über die Vorstellungen der griechen und anderer Völker \\vom `Nabel der Erde'} % Subtitle or further description
	
	\vspace*{1\baselineskip} % Whitespace under the subtitle

        {\scshape Von \Large Dr. Wilhelm Heinrich Roscher}

	\vspace*{1\baselineskip} % Whitespace under the subtitle

        {\scshape\small Mit 68 Figuren auf 9 Tafeln und 3 Bildern im Text}

	\vspace{1\baselineskip}


        %------------------------------------------------
	%	Editor(s)
	%------------------------------------------------
        \vspace*{\fill}

        \begin{figure}[H]
        \centering
        \includegraphics[width=0.55\textwidth,keepaspectratio]{figs/cover.jpg}
        \end{figure}

	\vspace{1\baselineskip}

        \vspace*{\fill}

	{\small\scshape Leipzig 1913}
	
	{\small\scshape{Bei B. G. Teubner}}
 
	\vspace{0.5\baselineskip} % Whitespace after the title block

        \scshape Internet Archive Online Edition  % Publication year
	
	{\scshape\small Namensnennung Nicht-kommerziell Weitergabe unter gleichen Bedingungen 4.0 International} % Publisher
\end{titlepage}
\setlength{\parskip}{1mm plus1mm minus1mm}
\clearpage
\pagestyle{fancy}
\fancyhf{}
\cfoot{\frakfamily{\thepage}}
\Large
\tableofcontents
\clearpage
\section*{Register --- Systematische Inhaltsübersicht.}
\subsection*{Vorwort.}
\paragraph{}
Über Veranlassung und Zweck der Untersuchung, sowie summarische Übersicht des wesentlichen Inhalts.
\subsection*{Kap. 1: Über die Etymologie von ὀμφαλός (= \emph{umbilicus} etc.) und die Bedeutung des `Nabels' bei den Griechen und andern Völkern.}
\paragraph{}
Ὀμφαλός, verwandt mit lat. \emph{umbilicus}, sanskr. \emph{nabhis} (Nabel, Nabe) \emph{nabhilam} (Nabelvertiefung, Schamgegend), althochd. \emph{nabulo} (Nabel), \emph{naba} (Nabe) usw., bedeutet, zunächst nicht bloß die rundliche Vertiefung in der Mittellinie des Leibes, sondern auch die Nabelschnur, die öfters mit der Wurzel (ῥίζα) einer Pflanze verglichen wird: S. 6-8. --- Bisweilen wird auch der die Mittellinie (Achse) der Frucht fortsetzende Stiel (Stengel) der Baumfrüchte, besonders der Feigen, sowie der sich aus der Mitte des Samenkernes entwickelnde Pflanzenkeim ὀ. (umbilicus) genannt: S. 8. --- Sehr oft dient ferner der Ausdruck ὀ. in übertragener Bedeutung zur Bezeichnung des Zentrums irgendeines Raumes, Gegenstandes oder auch einer Masse von Einzelindividuen: S. 9. --- Odyss. 1, 50 wird Ogygia, die Insel der Kalypso, ὀμφαλὸς θαλάσσης genannt, was wahrscheinlich auch einen Nabel der Erde voraussetzt; vgl. Epimenides fr. 6 Ki.: S. 9-10. --- Über ὀμφαλός (\emph{umbo}) in der Bedeutung `Schildbuckel,' `Jochknauf,' `Heereszentrum': S. 10-11. --- Ergebnis: S. 11-12. --- Über die Bedeutung des Nabels und der Nabelschnur als Amulett, in Geburtsriten usw. bei den Polynesien, Afrikanern, Indonesiern, Grönländern, ferner den Germanen, Alt- und Neu-Griechen: S. 12-19. Vgl. ob. S. 131 f. --- Ergebnis: S. 19-20.

\subsection*{Kap. 2: Der Gedanke eines Zentrums (Nabels) der Erdoberfläche bei verschiedenen Völkern.}
\paragraph{}
Der Begriff des Erdnabels beruht auf der Vorstellung von der Erde als einer horizontalen, meist kreisrunden Fläche (Scheibe), die als Solche einen Mittelpunkt haben muss: S. 20. --- 1. Die Chinesen: S. 20 f. --- 2. Die Japaner: S. 21 f. --- 3. Die Malayen: S. 22. --- 4. Die Inder: S. 22. --- 5. Die Babylonier: S. 23 f. --- 6. Die Israeliten: S. 24 ff. --- 7. Die Araber und Perser: S. 28 f. --- 8. Die Phönizier: S. 29 f. --- 9. Die Ägypter: S. 31 ff. --- 10. Die Griechen (die ὀμφαλοί von Branchidai, Delphi, Phlius, Athen, Antiochia): S. 32 ff. --- 11. Die Italiker (Enna, lacus Cutiliae, Rom): S. 34 f. --- 12. Die Magyaren: S. 35 (vgl. S. 132). --- 13. Die Peruaner: S. 35. --- Ergebnis: S. 35-36.

\subsection*{Kap. 3: Branchidai (Didyma) und sein Orakel als Nabel der Erde.}
\paragraph{}
Die Existenz eines Omphalos im Tempel zu Didyma ist zwar bisher durch die neueren Ausgrabungen nicht bestätigt worden, doch lässt sich die Geltung Branchidais als ὀμφαλὸς γῆς auf einem andern Wege sehr wahrscheinlich machen. Die dafürsprechenden Tatsachen sind: 1. Die Bezeichnung Ioniens als Zwerchfell (φρένες, praecordia) der Erde in der sehr alten Kosmologie der ps.-hippokrat. Schrift von der Siebenzahl, wodurch Ionien zugleich als Zentrum der Oikumene und als Sitz der höchsten Kultur und Intelligenz bezeichnet werden soll: S. 36-40. --- 2. Die Bezeichnung des Orakels von Didyma als ἄξων, d. h. als Erd- oder Weltachse, bei Iamblichus de myst. 3, 11 p. 127 P.: S. 40-43. --- 3. Die außerordentliche Bedeutung des Orakels von Didyma in der Zeit vor der Zerstörung Milets durch die Perser und seine Rivalität mit Delphi macht es höchst wahrscheinlich, dass die altmilesischen Geographen Didyma und nicht Delphi zum Zentrum ihrer Erdkarten gemacht haben: S. 43-45. --- Über die älteste Methode des Kartenentwerfens nach Plinius h. n. 18, 326 ff., sowie über die διαφράγματα der Erdkarte des Dikaiarchos, deren ὀμφαλός die Insel Rhodos war: S. 45-46. --- 4. Der ὀμφαλός von Branchidai ist wahrscheinlich dargestellt auf Münzen von Milet, auf dem Relief von Archelaos von Priene (`Apotheose Homers'), auf Münzen von Kyzikos und Sinope, den Pflanzstädten Milets: S. 47-54.

\subsection*{Kap. 4: Delphi und sein Orakel als Mittelpunkt (ὀμφαλός) der Welt und sein Nabelstein.}
\subsubsection*{Die literarischen Zeugnisse.}
\paragraph{}
Epimenides frgm. 6 Kinkel: S. 55 f. --- Pindars Zeugnisse: S. 56 ff. --- Bakchylides 4, 4: S. 58. --- Zeugnisse des Aischylos: S. 58 ff. --- Sophokles: S. 59 f. --- Euripides: S. 60 ff. --- Platon de rep. 4, 5 p. 427 C: S. 65. --- Delphische Bauinschrift des 4. Jahrh.: S. 65. --- Fragm. eines römischen Tragikers des 3. oder 2. Jahrh.: S. 65. --- Delphischer Hymnus des Aristonoos aus der 2. Hälfte des 3. Jahrh.: S. 65. --- Varro de l. l. 7, 17: S. 66 ff. --- Strab. 9 p. 419: S. 68 ff. --- Liv. 38, 48 u. 41, 23: S. 70. --- Ovid. Met. 10, 167 u. 15, 630: S. 70. --- Lucan. Phars. 5, 71 u. Schol.: S. 70 f. --- Stat. Theb. 1, 561 ff. u. 1, 118: S. 71. --- Pausan. 10, 16, 3: S. 72 f. --- Claudian. prol. hy. in consul. Fl. Malli Theodori (16, 11 ff.): S. 74. --- Nonnos Dion. passim: S. 74. --- Argivische Inschr. des 3. Jahrh. v. Chr.: S. 75 f. --- Ergebnisse: S. 76-80.

\subsubsection*{Die monumentalen Zeugnisse.}
\begin{enumerate}
    \item Die plastischen Nachbildungen des delphischen Omphalos.
    \item Die Omphalosdarstellungen in Wandgemälden usw.
    \item Der delphische Omphalos auf Münzen.
    \item Der delphische Omphalos in Vasengemälden.
\end{enumerate}
\subsection*{Kap. 5: Weitere, wahrscheinlich nicht von Delphi abhängige Kulte des Apollon, Asklepios usw., in denen Omphaloi vorkamen.}
\paragraph{}
Der Omphalos im Kult von Thymbra: S. 106-107. --- Der Omphalos des Apollotempels zu Patara: S. 107-109. --- Der Omphalos im Tempel des Apollon Ἐρεθίμιος zu Kamiros auf Rhodos: S. 109. --- Der O. im Tempel des Apollon zu Gryneion b. Myrina in Aiolis: S. 110. --- Der O. im Kult des römischen Aesculapius, des epidaurischen und des pergamenischen Asklepios und seine Bedeutung: S. 110-114. --- Der O. im Kult der Lares Compitales: S. 114. --- Der O. im Kult des Hermes (Mercurius): S. 115.

\subsection*{Kap. 6: Grabmonumente, Baitylien und Altäre in Omphalosform.}
\paragraph{}
Kritik der in neuester Zeit namentlich von Miss J. Harrison vertretenen Ansicht, dass der delphische Nabelstein eigentlich und ursprünglich ein den Bestattungsort des Pythondrachens deckender Grabstein gewesen sei. --- Miss H. zieht diesen Schluss a. aus den Zeugnissen Varros (de l. l. 7, 17) und des Hesychius (Τοξίου βουνός), b. aus der tatsächlich bestehenden Ähnlichkeit, ja Gleichheit gewisser Grabdenkmäler mit den oben behandelten Formen des Omphalos, deren Liste sich noch weiter vermehren lässt: S. 115-ı20. --- Dem gegenüber muss hervorgehoben werden: 1. dass die bloße äußere Ähnlichkeit oder Gleichheit noch lange nicht genügt, um wirkliche Identität nachzuweisen, was namentlich an den Formen gewisser Buchstaben des griechischen Alphabets schlagend nachgewiesen wird: S. 120. --- 2. dass die von Miss H. angeführten Zeugnisse des Varro und Hesychius viel zu später Zeit angehören, als dass sie gegenüber den zahlreichen, viel älteren des Pindar usw., sowie gegenüber der sehr alten gut bezeugten delphischen Lokaltradition irgend in Betracht kommen könnten: S. 121. --- 3. werden noch verschiedene Argumente archäologischer und philologischer Art gegen Miss H.s Auffassung geltend gemacht, namentlich dies, dass die ὅρος-Steine, die termini und metae, die unter dem Schutze des Ζεὺς ὅριος = Juppiter Terminus standen, vielfach deutliche Omphalosform zeigen, und dass die beiden goldenen Zeusadler, die den delphischen Nabelstein flankierten, nur bezweckten, ihn deutlich als einen dem höchsten Himmelsgott geweihten, die Mitte zwischen dem äußersten Osten und Westen anzeigenden Markstein und damit als Sinnbild des Mittelpunkts der Erdscheibe zu bezeichnen: S. 121-124. --- Zuletzt wird noch die äußere Ähnlichkeit des delphischen Omphalos mit gewissen Baitylia kurz besprochen: S. 124-125.

\subsection*{7. Nachträge.}
\begin{enumerate}
    \item Das neugefundene Omphalosrelief von Aigina: S. 126-127. ---
    \item Der plastische Omphalos von Thermos in Aitolien: S. 128. ---
    \item Der an die Stelen des Apollon Agyieus erinnernde schlangenumwundene Omphalos von Delos (?): S. 128-129. ---
    \item Weitere Omphaloi auf Reliefs: S. 129-130. ---
    \item Weiteres zum Omphalos Jerusalems: S. 130.
\end{enumerate}
\subsection*{8. Berichtigungen und Zusätze.}
\paragraph{}
Neugriechische, aus ältester Zeit stammende abergläubische Riten beim Abschneiden und Aufbewahren der Nabelschnur und Glückshaube usw.
\clearpage
\section*{Vorwort.}
\paragraph{}
Die nachfolgende, `Omphalos' betitelte, Untersuchung hängt --- so wunderlich dies auf den ersten Blick auch scheinen mag --- mit meinen kürzlich veröffentlichten Arbeiten über die hippokratische Schrift von der Siebenzahl\footnote{1. Roscher, Die Hebdomadenlehren der griech. Philosophen u. Ärzte. Leipz. 1906. --- 2. Derselbe, Über Alter, Ursprung u. Bedeutung der hippokrat. Schrift von d. Siebenzahl. Leipz. 1911. --- 3. Derselbe, Die neuentdeckte Schrift eines altmilesischen Naturphilosophen u. ihre Beurteilung durch H. Diels in d. D. Lit. Ztg. 1911 Nr. 30 (Sonderabdruck aus `Memnon' Bd. 5. 3/4). Berl. Stuttg. Leipz. 1912. --- 4. Derselbe, D. Alter d. Weltkarte in Hippokrates' περὶ ἑβδομάδων und die Reichskarte des Darius Hystaspis, Philologus 70 (1911) S. 529 ff. --- 5. Derselbe, D. hippokratische Schrift von d. Siebenzahl in ihrer vierfachen Überlieferung zum erstenmal herausgegeben von W. H. R. Paderborn 1913. --- 6. S. auch meine Erwiderung auf Lortzings Kritik in der Berl. Philol. Wochenschr. 1912 Sp. 1876 ff.} sehr eng zusammen und ist im Grunde nur durch diese veranlasst worden. Zur Begründung dessen bemerke ich, das in der höchst altertümlichen, nur vom Standpunkt Altmilets aus verständlichen, siebenteiligen Weltkarte, welche uns in der kosmologischen Einleitung der genannten Schrift glücklich erhalten geblieben ist, Ionien als der dem Zwerchfell der Mikrokosmen entsprechende Teil der bewohnten Erde und zugleich als Sitz der höchsten menschlichen Kultur und Intelligenz bezeichnet wird, woraus allein schon seine zentrale Lage und hervorragende Bedeutung innerhalb der altmilesischen Weltkarte erhellt. Auf diese Weise gewinnen wir eine gegen die frühere wesentlich veränderte Anschauung von der Anlage der ältesten milesischen Erdkarten, als deren Mittelpunkt (ὀμφαλὸς γῆς) noch eine Autorität von dem Range Hugo Bergers das für die altmilesischen Geographen höchst unbequem gelegene Delphi angenommen hatte, an dessen Stelle nunmehr Milet mit seinem hochberühmten Orakel von Branchidai (Didyma) tritt.\footnote{Dies ist namentlich auch die Ansicht eines so hervorragenden Kenners wie Sal. Reinach, der in der Revue Archéol. 1911 (2) p. 390 bemerkt: `Cela [que 1° ce traité était antérieur à Hippocr.; 2° qu'il doit avoir été écrit avant 494, à Milet ou dans une ville ionienne, puisque l'horizon de l'auteur est celui de l'Ionie au 6e siècle] est particulièrement sensible dans le passage relatif aux 7 parties du monde, passage d'ailleurs capitale pour l'histoire de la géographie antique.'} Hierzu kommt noch die von mir nachgewiesene, während des 7. und 6. vorchristlichen Jahrhunderts zwischen Branchidai und Delphi herrschende scharfe Rivalität,\footnote{S. meine dritte, oben Anm. 1 angeführte Schrift S. 26 f.} die es wenig glaublich erscheinen lässt, dass die patriotischen, nach Herodot auf ihre hohe Kultur und Eigenart stolzen Altmilesier\footnote{Herod. 1, 143: οἱ μέν νυν ἄλλοι Ἴωνες καὶ οἱ Ἀθηναῖοι ἔφυγον τὸ οὔνομα [τ. Ἰώνων], οὐ βουλόμενοι Ἴωνες κεκλῆσθαι, ἀλλὰ καὶ νῦν φαίνονταί μοι οἱ πολλοὶ αὐτῶν ἐπαισχύνεσθαι τῷ οὐνόματι· αἱ δὲ δυώδεκα πόλιες αὖται [an der Küste Asiens] τῷ τε οὐνόματι ἠγάλλοντο καὶ ἱρὸν ἱδρύσαντο ἐπὶ σφέων αὐτέων, τῷ οὔνομα ἔθεντο Πανιώνιον, ἐβουλεύσαντο δὲ αὐτοῦ μεταδοῦναι μηδαμοῖσι ἄλλοισι Ἰώνων. Mehr in meiner dritten u. fünften oben erwähnten Abhdlg.} gerade das von ihnen als Hauptkonkurrent ihres eigenen Nationalheiligtums betrachtete Delphi und nicht vielmehr Branchidai zum Zentralpunkt ihrer Erdkarte erwählt hätten. Durch solche Erwägungen bin ich ganz natürlicher Weise zu dem naheliegenden Entschluss gekommen, einmal die gesamten Anschauungen der Griechen vom Mittelpunkt der Erdscheibe möglichst eingehend zu behandeln, zumal da ich bei genauerer Betrachtung bald wahrnehmen musste, dass, trotz der anregenden und manches wertvolle Material bietenden Arbeiten von Miss Jane Harrison und G. Karo,\footnote{S. Journ. of Hellen. Stud. 19 (1899) S. 225 ff. Bull. de Corresp. Hellén. 25 (1900) S. 254 ff. G. Karo, Artikel `Omphalos' im Dict. des antiquités. Svoronos im Journ. Internat. d'archéol. numism. 13 (1911) S. 313 f.} das Problem des Erdnabels noch nicht in befriedigender Weise gelöst, ja nicht einmal das zugehörige Zeugnismaterial vollständig gesammelt war. Vor allem kam es mir darauf an, die sämtlichen literarischen und monumentalen Zeugnisse des griechisch-römischen Altertums nicht bloß zu sammeln, sondern auch kritisch und exegetisch zu behandeln, was bisher nur in ganz unzureichendem Maße geschehen ist (Kap. 3: Branchidai, Kap. 4: Delphi). Außerdem versuchte ich dem Gedanken des Erdnabels bei anderen Völkern, z. B. den Japanern, Chinesen, Malayen, Indern, Phöniziern, Israeliten, Arabern, nachzugehen, wobei mir verschiedene hervorragende Kenner der betreffenden Literaturen überaus wertvolle Hilfe geleistet haben (Kap. 2). Auch die Etymologie und Semasiologie der Ausdrücke für `Nabel' und `Nabelschnur' (ὀμφαλός, \emph{umbilicus} usw.), sowie gewisse uralte und weitverbreitete an das Abschneiden der Nabelschnur sich anknüpfende Geburtsriten erforderten eine eingehendere Untersuchung (Kap. 1). So ergab sich ganz von selbst das nicht unwichtige Resultat, dass es außer Delphi noch zahlreiche andere Orte, nicht bloß innerhalb des von den Hellenen besiedelten oder ihnen leicht zugänglichen Gebietes, sondern auch außerhalb desselben, gegeben hat, die sich rühmen durften, im Besitze des `Erdnabels' zu sein, und dass Delphi es nur seiner fast vollkommen gesicherten geographischen Lage während der Blütezeit Griechenlands zu verdanken hat, dass es Orte wie Branchidai, das ihm bis zu seiner Zerstörung durch die Perser (494) eine sehr ernstliche Konkurrenz gemacht hatte, in der Folgezeit überstrahlte. In Kap. 5 werden mehrere wahrscheinlich nicht von Delphi abhängige Kulte des Apollon, Asklepios usw. behandelt, in denen ebenfalls `Omphaloi' vorkamen. Kap. 6 endlich sucht die kürzlich von Miss Harrison auf Grund der äußeren Ähnlichkeit oder Gleichheit gewisser Grabdenkmäler (Grabsteine) mit dem Omphalos zu Delphi ausgesprochene Ansicht zu widerlegen, dass dieser eigentlich und ursprünglich nur der das Grab des Drachen Python bezeichnende Grabstein gewesen sei. Eigentlich war es auch meine Absicht, in einem letzten Kapitel durch eine eingehende Vergleichung der Riten und Mythen der kleinasiatischen Apollonorakel mit denen Delphis den Beweis zu führen, dass der delphische Apollonkult im Grunde nur ein Ableger der viel älteren kleinasiatischen Kulte ist, doch habe ich, um den Umfang dieser Untersuchung nicht allzu sehr anschwellen zu lassen, einstweilen von diesem Nachweise absehen müssen und hoffe ihn bei einer anderen Gelegenheit nachliefern zu können.

Ich kann dieses Vorwort nicht schließen, ohne allen, die mich bei der Sammlung des in dieser Untersuchung verarbeiteten sehr mannigfaltigen und z. T. schwer zugänglichen Materials bereitwilligst unterstützt haben, meinen herzlichsten Dank zu sagen. Diesen schulde ich vor allen den Herren A. Forke (Charlottenburg), I. Goldziher (Buda-Pest), Imhoof-Blumer (Winterthur), R. Lange (Berlin-Steglitz), N. G. Politis (Athen), H. Pomtow (Berlin), Th. Wiegand (Berlin), E. Windisch (Leipzig); alle übrigen Gelehrten, die mich zu Dank verpflichtet haben, werden `suo quisque loco' von mir genannt werden.
\clearpage
\section{Über die Etymologie von ὀμφαλός (= \emph{umbilicus} etc.) und die Bedeutung des `Nabels' bei den Griechen und anderen Völkern.}
\paragraph{}
Um zu einem möglichst gründlichen Verständnis der sämtlichen von den Griechen an den Begriff des `Omphalos' geknüpften Vorstellungen zu gelangen, müssen wir vor allem von der Etymologie und den aus dem Sprachgebrauch ersichtlichen verschiedenen Bedeutungen des Wortes ὀμφαλός ausgehen. Nach G. Curtius (Grundz. d. griech. Etym. 5 S. 294) und A. Fick (Vgl. Wörterb. d. indog. Spr. 2 3) ist \emph{ὀμφ-αλό-ς} = Nabel urverwandt mit lat. \emph{umbil-īcu-s}, mit sanskr. \emph{nābh-i-s} = Nabel, Nabe, Verwandtschaft und \emph{nābhī-la-m} = Nabelvertiefung, ferner mit althochd. \emph{nab-uló} = Nabel, \emph{nab-a} = Nabe, altpreuß. \emph{nabis} --- Nabel, Nabe usw. `Während wir also für das Griechische und Lateinische auf eine Wurzel \emph{ambh} geführt werden, gehen die entsprechenden Wörter der übrigen Sprachen auf die Wurzel \emph{nabh} zurück,' die sich von jener lediglich, wie es scheint, durch die im Griechischen und Lateinischen vollzogene Metathesis (\emph{nabh} --- \emph{anbh} --- \emph{ambh}) unterscheidet (vgl. Curtius a. a. O. S. 536). Ob das Urwort für Nabel, wie Curtius unter Verweisung auf sanskr. \emph{nabh} = bersten, reißen meint, ursprünglich `Riss, Bruch' bedeutet habe, welcher Sinn leicht auf das unmittelbar nach der Entbindung notwendige `Abreißen, Brechen, Trennen, Abschneiden' der Nabelschnur des Embryo bezogen werden könnte, lasse ich dahingestellt sein.

Was ferner den Sprachgebrauch von ὀμφαλός anlangt, so bezeichnet das Wort ebenso wie das verwandte lateinische \emph{umbilicus} zunächst nicht bloß `die rundliche Vertiefung in der Mittellinie des Leibes, welche die Stelle bezeichnet, wo am fötalen Körper die vordere offene Leibeshöhle sich geschlossen hat,'\footnote{Vgl. Plat. Symp. 190 E: ἕν στόμα ποιῶν [ὁ Ἀπόλλων] ἀπέδει [τὸ δέρμα τ γαστρός] κατὰ μέσην τὴν γαστέρα, ὃ δὴ τὸν ὀμφαλὸν καλοῦσι. --- Poll. on. 2, 169. τὸ δὲ κατὰ μέσην γαστέρα κοῖλον ὀμφαλὸς καὶ μεσομφάλιον. --- Schol. Nic. Al. 347. Brockhaus, Konversationslex. 14 unter Nabel. Aristot. de an. hist. 1, 13 (in einer Aufzählung der einzelnen Körperteile): μετὰ δὲ τὸν θώρακα ἐν τοῖς προσθίοις γαστὴρ καὶ ταύτης ῥίζα ὀμφαλός. Vgl. zum Verständnis des Ausdrucks ῥίζα Philolaos in Theol. Arithm. 4 p. 22: τέσσαρες ἀρχαὶ τοῦ ζῴου τοῦ λογικοῦ... ἐγκέφαλος, καρδία, ὀμφαλός, αἰδοῖον· "`Κεφαλὰ μὲν νόω, καρδία δὲ ψυχᾶς καὶ αἰσθάσιος, ὁμφαλὸς δὲ ῥιζώσιος καὶ ἀναφύσιος τῶ πρώτω... ἐγκέφαλος δὲ τὰν ἀνθρώπω ἀρχάν, καρδία δὲ τὰν ζῴω, ὀμφαλὸς δὲ τὰν φυτῶ... πάντα γὰρ θάλλοντι καὶ βλαστάνοντι."' Aristoteles scheint also den Nabel deshalb für die `Wurzel' des Bauches gehalten zu haben, weil der Fötus ursprünglich seine Nahrung durch die Nabelschnur wie die Pflanze durch die Wurzel erhält; vgl. Anaxag. b. Censor. d. n. 6, 3 u. Aristot. de an. gen. 2, 4 p. 140 ἡ μὲν οὖν αὔξησις τῷ κυήματι γίνεται διὰ τοῦ ὀμφαλοῦ τὸν αὐτὸν τρόπον ὅνπερ διὰ τῶν ῥιζῶν τοῖς φυτοῖς. S. auch außer der stoischen δόξα b. Plut. plac. phil. 5, 16 Simon Magus b. Hippol. refut. 6, 14 p. 244 D. et Schn.: Εἰ δὲ πλάσσει ὁ θεὸς ἐν μήτρᾳ μητρὸς τὸν ἄνθρωπον, τουτέστιν ἐν παραδείσῳ... ἔστω παράδεισος ἡ μήτρα,... ποταμὸς ἐκπορενόμενος ἐξ Ἐδὲμ ποτίζων τὸν παράδεισον ὁ ὀμφαλός. Οὗτος, φησίν, ἀφορίζεται ὁ ὀμφ. εἰς τέσσαρας ἀρχάς, ἑκατέρωθεν γὰρ τοῦ ὀμφαλοῦ δύο εἰσὶν ἀρτηρίαι παρατεταμέναι, ὀχετοὶ πνεύματος, καὶ δύο φλέβες, ὀχετοὶ αἵματος... Πλαττόμενον γὰρ τὸ βρέφος ἐν τῷ παραδείσῳ οὔτε τῷ στόματι τροφὴν λαμβάνει, οὔτε ταῖς ῥισὶν ἀναπνέει... τρέφεται δὲ δι᾽ ὀμφαλοῦ... Ich verdanke den Hinweis auf diese Stelle W. Schultz. --- S. auch unt. Anm. 20.} sondern ebenso auch die bei der Geburt von der Hebamme (μαῖα ὀμφαλ-ητόμος, -οτόμος), ursprünglich infolge eines uralten Aberglaubens, mit einem Nagel oder einem geschärften Schilfrohr oder einer scharfen Muschel oder Scherbe (ὄστρακον)\footnote{Die Maori Neuseelands legen den vom Kinde abgefallenen Nabelstrang in die Muschel, mit welcher man ihn durchschnitten hatte, und die man dann mit demselben auf das Wasser eines Stromes legt. Wenn die Muschel mit ihrem Inhalte schwimmt, so bedeutet das Glück für das Kind, sonst das Gegenteil (Ploss, D. Kind 2 1, 15).} oder auch einer scharfen und harten Brotrinde, später gewöhnlich (in einer Entfernung von 4 Fingerbreiten vom Bauche) mit einem eisernen Messer (ὀμφαλιστήρ, σμιλίον) durchschnittene Nabelschnur (funiculus umbilicalis), durch die der Fötus seine Nahrung erhält,\footnote{Soran. gynaec. p. 250 ed. Rose: δεῖ δὲ τέσσαρας δακτύλους διαστήσαντα ἀπὸ τῆς γαστρὸς ἀποκόπτειν τὴν ὀμφαλίδα διά τινος ἐπάκμου... πάσης δὲ ὕλης τμητικώτατός ἐστιν ὁ σίδηρος. αἱ πολλαὶ δὲ τῶν μαιουμένων ἥλῳ ἢ καλάμῳ ἢ ὀστράκῳ ἢ τῷ λεπίῳ τοῦ ἄρτου δοκιμάζουσι τὴν ἀποκοπὴν ἢ λίνῳ βιαίως ἀποσφίγξασαι, τῷ δυσώνιστον εἶναι τὴν ἐν τῷ πρώτῳ χρόνῳ σιδήρου τομήν, ὅπερ παντελῶς καταγέλαστόν ἐστιν... ἄμεινον ἀδεισιδαιμονέστερον σμιλίῳ μᾶλλον τὸν ὀμφαλὸν κόπτειν. --- Aristot. de an. hist. 7, 8, 1: Αὐξάνεται δὲ τὰ ζῷα πάντα, ὅσα ἔχει ὀμφαλόν, διὰ τοῦ ὀμφαλοῦ... 3: ὁ δ᾽ ὀμφαλός ἔστι κέλυφος περὶ φλέβας, ὧν ἡ ἀρχὴ ἐκ τῆς ὑστέρας ἐστί. --- Vgl. de an. gen. 2, 4, p. 740. --- Poll. on. 2, 169: καὶ ᾧ ἀποτέμνει [ἡ ὀμφαλητόμος, ἡ μαιεύτρια] τοὺς ὀμφαλοὺς τῶν βρεφῶν, ὀμφαλιστήρ. --- Vgl. über eine entsprechende, noch heute in Hellas (in Syme) bestehende Sitte N. G. Politis in Λαογραφία γ΄ (1912) p. 699: Ἐν Σύμῃ ὁ ὀμφαλὸς κόπτεται, τιθέμενος ἐπὶ τοῦ σκληροῦ τοῦ ἄρτου (Ζωγραφ. ἀγὼν σ. 211).} weshalb sie auch von den Alten öfters mit der Wurzel (ῥίζα) einer Pflanze verglichen wird (s. ob. Anm. 6 u. Suid. s. v. ὀμφαλός· οἷόν τις ῥίζα τοῦ βρέφους κ. τ. λ.).

Offenbar ganz ähnlich ist es zu erklären, wenn bisweilen der fast durchweg rundliche und in der Mittellinie (Achse) der Frucht sitzende Stiel oder Stengel der Baumfrüchte, insbesondere der Feigen, ὀμφαλός genannt wird,\footnote{Poll on. 2, 170: οἱ δὲ Ἀττικοὶ καὶ τὸν τῶν σύκων πυθμένα ὀμφαλὸν ὠνόμαζον. --- Geopon. 10, 56, 2: εἶτα ἀφαιρῶν τὰ σῦκα μικρὸν ὠμότερα μετὰ τῶν πεισμάτων, ἤτοι τῶν ὀμφαλῶν, τουτέστι μετὰ τοῦ μέρους ἀφ᾽ οὗ ἐπὶ τῷ δένδρῳ ἤρτηται, ἐντίθει ταῦτα εἰς τὴν χύτραν. --- Anders Schol z. Nik. Alex. 347: πόσιν ὀμφαλόεσσαν] τὴν ἐκ σύκων τῶν ὀμφαλοὺς ἐχόντων· τὰ γὰρ σῦκα κάτω τρύπας ἔχουσι δίκην ὀμφαλοῦ (s. oben Anm. 6). --- Vgl. Theophr. h. pl. 3, 7, 5. Ps.-Aristot. probl. 12, 8. --- Bei Pallad. 12, 7 p. 194 Gesn.: `bene servantur [poma], si umbilicum pomi gutta picis calentis oppleveris' soll dagegen umbilicus nicht den Stengel, sondern den `Krübs' der Frucht bedeuten. Vgl. Magerstedt, Bilder a. d. röm. Landwirtsch. 4, 228.} ohne Zweifel deshalb, weil der Stiel, an dem die Frucht hängt, diese ganz ähnlich mit dem Baume, wie die Nabelschnur den Fötus mit der Mutter verbindet und so durch Zuführung der nötigen Säfte dessen Wachsen und Gedeihen bedingt.\footnote{Nach einigen Philosophen sollte auch das Atmen des Embryo durch den Nabel erfolgen; vgl. Etym. M. 625, 39: Ὀμφαλός... παρὰ οὖν τὸ ἐμπνεῖν, ὅ ἐστιν ἀναπνεῖν, γίνεται ὀμπναλὸς καὶ... ὀμφαλός, δί οὗ τὸ ἔμβρυον ἀναπνεῖ... ἢ παρὰ τὸ φαλ<λ>ῷ ἐοικέναι· ἐκκρεμὴς γάρ ἐστιν ἐν ἀρχῇ πρὸ τῆς ἀποτομῆς.}

Nahe verwandt ist auch die Anschauung, welche zu dem Vergleiche des meist aus der Mitte (= dem `Nabel') des Samenkernes hervorsprießenden meist rundlichen Pflanzenkeimes mit einer Nabelschnur und zu seiner Bezeichnung als ὀμφαλός (umbilicus) geführt hat. Vgl. z. B. Plinius h. n. 18, 136 (von der Lupine): Condi in fumo maxime convenit, quoniam in humido vermiculi umbilicum [= oscillum b. Colum. 2, 10, 1] eius in sterilitatem castrant. --- Ders. 15, 89 (von den nuces) Umbilicus illis intus in ventre medio. --- Ders. 13, 32 (vom Samenkern der Dattelpalme): Est autem [semen] oblongum, ...praeterea caesum a dorso pulvinata fissura et in alvo media plerisque umbilicatum, unde primum spargitur radix [= ἔκφυσις b. Theophr. h. pl. 2, 6].\footnote{Vgl. auch zu den angeführten Belegen Lenz, Botanik d. alt. Griech. u. Römer S. 47. Magerstedt a. a. O. 5, 309 (für die Lupine); Lenz a. a. O. S. 396 Anm. 874 (von den Nüssen); Lenz a. a. O. 336 f. (von den Datteln).}

Während also in allen bisher von mir angeführten Fällen des Sprachgebrauchs die Bedeutung von ὀμφαλός als ein für die erste Entwickelung oder Entstehung der animalischen und pflanzlichen Individuen überaus wichtiger, rundlicher und in der Mitte des Leibes oder Samenkernes befindlicher Körperteil klar hervortritt, dient in zahlreichen anderen Fällen derselbe Ausdruck nur zur Bezeichnung der Mitte oder des Zentrums (Mittelpunktes) irgend eines Raumes,\footnote{Hierher gehört auch die Bedeutung des ὀμφαλός als des Mittelstücks eines Gewölbes (Schwibbogens, Kuppelbaues); vgl. Ps.-Aristot. de mu. 6 p. 400: Ἔοικε [ὁ Ἐμπεδοκλῆς] παραβάλλειν τὸν κόσμον τοῖς ὀμφαλοῖς λεγομένοις τοῖς ἐν ταῖς ψαλῖσι λίθοις, οἳ μέσοι κείμενοι κατὰ τὴν εἰς ἑκάτερον μέρος ἔνδεσιν ἐν ἁρμονίᾳ τηροῦσι καὶ ἐν τάξει τὸ πᾶν σχῆμα τῆς ψαλῖδος καὶ ἀκίνητον. Vgl. dazu Hippolyt. 5, 20 p. 208. --- Ebenso wird der circulus parvus in der Mitte der genau orientierten Windrose von Plinius (n. h. 18, 327; 331; 332) umbilicus genannt.} Gegenstandes oder auch einer versammelten Menge von Einzelindividuen.

Von besonderer Wichtigkeit für unsere Zwecke ist in dieser Beziehung Odyssee 1, 50, wo es von der in der Mitte des nordwestlichen Meeres, von jeder Festlandsküste gleich weit entfernt gedachten Insel der Kalypso, Ogygia, heißt:

νήσῳ ἐν ἀμφιρύτῃ ὃθι τ᾽ ὀμφαλός ἐστι θαλάσσης.\footnote{Dieselbe Vorstellung von einem `Nabel des Meeres' findet sich noch in neugriechischen Märchen. Vgl. Politis, Μελέται π. τοῦ βίον... τοῦ Ἑλληνικοῦ λαοῦ. Παραδόσεις α΄ p. 310 (Anfang eines Märchens aus Bizye). ᾽Σ τὸν ἀφαλὸ [= ὀμφαλὸν] τῆς θάλασσας, ἐκεῖ ποῦ τὸ νερὸ γυρίζει γύρω γύρω καὶ γίνεται μιὰ τρῦπα μέσ᾽ ᾿ς τὴ μέση εὑρίσκεται καὶ ἡ Φώκια, ἡ μάννα τ᾿ Ἀλεξάνδρου κ. τ. λ. Mehr bei Politis in der Λαογραφία γ΄ (1912) S. 700. Ferner mache ich schon hier darauf aufmerksam, dass sich der Gedanke eines `Nabels der Erde' auch noch bei vielen anderen Völkern findet, z. B. den Peruanern, den Bewohnern von Celebes, den Arabern, Persern, den Phöniziern auf Kypros usw. (s. unten Kap. 2). Ich verdanke die meisten Nachweisungen der Güte I. Goldzihers.}

Diese hier in einem altionischen Gedicht zum erstenmale deutlich ausgesprochene Idee eines `Nabels des Meeres' im kosmologischen Sinne\footnote{Ich glaube kaum zu irren, wenn ich annehme, dass die Auffassung der mythischen Insel Ogygia als ὀμφαλὸς θαλάσσης gewisse praktische Erfahrungen der ältesten ionischen Seefahrer voraussetzt. Ich denke dabei an die zentrale Lage gewisser Inseln wie z. B. Delos und Melite. Vgl. ἱστίη νήσων, von Delos gesagt, bei Kallim. hy. in Del. 325, wozu der Schol. bemerkt: ἔστι μὲν κυρίως ὁ βωμὸς ὁ ἐν μέσῳ τῷ δόμῳ ἑστώς. ἐπειδὴ οὖν ἡ Δῆλος ἐν μέσῳ τ. Κυκλάδων ἕστηκε, δοκεῖ ὥσπερ ἑστία τις καὶ βωμὸς εἶναι.} ist umso beachtenswerter, als sie mit einer gewissen Notwendigkeit auch den Begriff eines `Nabels der Erde' (ὀμφαλὸς γῆς) vorauszusetzen scheint, obwohl von einem solchen weder bei Homer noch bei Hesiod jemals die Rede ist, vor allem nicht an den zahlreichen Stellen, wo Pytho (= Delphi) als apollinische Kult- und Orakelstätte erwähnt wird.\footnote{Wir werden später zu zeigen versuchen, dass in der älteren Zeit namentlich Branchidai bei Milet (neben Pytho) als Nabel der Welt angesehen wurde.} Sonach bilden das älteste wirkliche Zeugnis für die Idee eines `Nabels der Erde und des Meeres' bis jetzt die später in weiterem Zusammenhang zu besprechenden Verse des Epimenides (b. Plut. Mor. p. 409 E = Epic. gr. fr. ed. Kinkel 1 p. 234, fr. 6):
\begin{quotation}
Οὔτε γὰρ ἦν γαίης μέσος ὀμφαλός, οὐδὲ θαλάσσης·

εἰ δέ τίς ἐστι, θεοῖς δῆλος, θνητοῖσι δ᾽ ἄφαντος.
\end{quotation}
\paragraph{}
An mehreren Stellen der Ilias wird ferner der runde erhabene Mittelteil des nach außen konvexen, nach innen konkav geformten Rundschildes (ἀσπίς), den die Römer mit dem etymologisch beinahe identischen Ausdruck \emph{umbo} bezeichnen, ὀμφαλός genannt (Il. 13, 192; vgl. 11, 35 f., wo sogar von mehreren ὀ-οί des Schildes die Rede ist). Von diesem Buckel erhält der Schild (ἀσπίς) bekanntlich das Epitheton ὀμφαλόεσσα (Od. 19, 32. Il. 4, 448 u. öft.).\footnote{Vgl. Daremberg-Saglio, Dictionn. d. ant. s. v. clipeus p. 1250b. Bisweilen war der Omphalos des Schildes so spitz und erhaben, daf er fast einer massiven Nabelschnur verglichen werden kann. Vgl. Daremberg-Saglio a. a. O. Fig. 1638. Rich, Illustr. Wörterb. d. röm. Alt. unter `Umbo' S. auch Hesych. s. v. ὀμφαλόεσσα· ἡ Ἄρκτος· διὰ τὸ μέσον τὸν βόρειον πόλον περιέχειν. τινὲς δὲ τὴν εὔτροφον χώραν. καὶ ἀσπίδες ὀμφαλόεσσαι. ἢ ὀμφαλοὺς ἔχουσαι, ἢ στρογγύλαι. Der erste Teil der Glosse bezieht sich auf Nikander, Alexiph. v. 7 (ἄρκτον ὑπ᾽ ὀμφαλόεσσαν ἐνάσσαο), wozu der Scholiast bemerkt: ἀρκτικωτέρα γὰρ ἡ Κύζικος τῆς Κολοφῶνος, ὀμφαλὸν δὲ καλεῖ τὸν βόρειον πόλον, ὡς μεσαίτατον. --- Aristides in der Rede auf Kyzikos (I, 1 p. 237 Jebb) nennt übrigens Kyzikos den Omphalos zwischen Gadeira und Phasis, was so ziemlich auf den Mittelpunkt der Oikumene hinausläuft. Mehr darüber unten!}

Unter dem ὀμφαλός des Jochs (ζυγόν) ist ein Knopf oder Knauf in der Mitte des ζυγόν zu verstehen, welcher dazu diente, den Jochriemen darum zu schlingen und so zu befestigen (vgl. Il. 24, 269 ff. und dazu die Erklärer).\footnote{Schol. zu Ω 269: ὀμφαλόεν: ὑπεροχὰς ἔχον ἐν μέσῳ τινάς, αἷς περιειλοῦνται οἱ ἱμάντες. οἱ δὲ, ἕνα μέσον ἔχοντες ὀμφαλόν, ᾧ προσδεῖται ὁ ῥυμός. --- Hesych. ὀμφαλὸς· ζυγοῦ τὸ μέσον. --- Anders Hesych. s. v. ὀμφαλόεντα· ὀμφαλοὺς ἔχοντα ζυγόν. Ὀμφαλοὺς δὲ λέγονσι τὰς ἐν τῷ ζυγῷ τρώγλας ἐφ᾽ [ἀφ᾽?] ὧν αἱ ἡνίαι δέδενται.}

Recht alt scheint auch der vom Zentrum des in Schlachtordnung aufgestellten Heeres gebrauchte Ausdruck ὀμφαλός zu sein (Pollux, on. 1, 126: τῶν μαχομένων τὸ μὲν ἔμπροσθεν μέτωπον καὶ ζυγὸν καὶ πρόσωπον... τὸ δὲ μέσον ὀμφαλός), obwohl er sonst nur bei Byzantinern nachweisbar ist. Hinsichtlich der an den hervorragenden Enden in der Mitte der Bücherrollen angebrachten Knöpfe (ὀμφαλοί) verweise ich auf Beckers Gallus 2 2 p. 320 u. Baumeister, Denkm. Fig. 331).

Suchen wir nunmehr aus den sämtlichen bisher angeführten Beispielen für den verschiedenartigen Gebrauch der Ausdrücke ὀμφαλός und \emph{umbilicus} das Fazit zu ziehen, so ergibt sich als durchgehendes Hauptmerkmal das der zentralen Lage (Stellung) sämtlicher durch ὀμφαλός und \emph{umbilicus} bezeichneten Begriffe. In dieser Hinsicht kommen auch folgende Darlegungen Vitruvs (de archit. 3, 3 --- p. 65 f. Rose u. Strüb.) in Betracht: item corporis centrum medium est umbilicus.\footnote{Dieselbe Anschauung haben auch noch die heutigen Griechen. Vgl. Politis Λαογραφία γʹ S. 700: "`Ἡ Γοργόνα ἀπὸ τὴ μέση καὶ κάτω ἤτανε ψάρι"' καὶ "`ἡ φώκια ἀπὸ τὸν ἀφαλὸ καὶ κάτω ἦταν ψάρι"' λέγονται πρὸς δήλωσιν τοῦ αὐτοῦ πράγματος (Πολίτου, Παραδόσεις σ. 307. 309. 310).} namque si homo conlocatus fuerit supinus manibus et pedibus pansis circinique conlocatum centrum in umbilico eius, circumagendo rotundationem utrarumque manuum et pedum digiti linea tangentur.\footnote{S. Middleton im Journ. of Hell. Stud. 9 [1888] S. 294 A. 1, der auf eine Zeichnung Lionardos d. V. in Mailand verweist. --- Vgl. jedoch auch Varro l. l. 7, 17: umbilicum [certum terrarum = Delphos] dictum aiunt ab umbilico nostro, quod is medius locus sit terrarum, ut umbilicus in nobis; quod utrumque est falsum: neque hic locus [= Delphi] est terrarum medius neque noster umbilicus est hominis medius. itaque pingitur quae vocatur <ἀντι>χθὼν Πυθαγόρα, ut media caeli ac terrae linea ducatur infra umbilicum per id quo discernitur homo mas an femina sit, ubi ortus humanus similis ut in mundo: ibi enim omnia nascuntur in medio, quod terra mundi media. pr<a>eterea si quod medium id est umbilicus + ut pila terra[e?], non Delphi medium; et terrae medium non hoc [sed], quod vocant Delphis in aede ad latus, est [quiddam ut] thesauri specie, quod Graeci vocant ὀμφαλόν, quem Pythonos aiunt esse tumul<um> [-os Hss.]; ab eo nostri interpretes ὀμφαλόν umbilicum dixerunt. --- Trotz des arg verderbten Wortlautes glaube ich folgenden Sinn der Sätze Varros zu erkennen. Varro vertritt einerseits den Satz, dass nicht der Nabel sondern das Genitale der Mittelpunkt des menschlichen Körpers sei, andererseits huldigt er der Lehre von der zentralen Lage der Erde als Kugel (pila), die deshalb auch auf ihrer Oberfläche kein Zentrum (Delphi) haben kann, und erklärt dementsprechend den ὀμφαλός von Delphi nach dem Vorgang gewisser Griechen nicht für das Zeichen des Mittelpunktes der Erde (Welt), sondern vielmehr für das Grab des Drachen Python (s. unten). Etwas anders Jebb bei Middleton im Journ. of Hell. Stud. 9 [1888] S. 294, 1.} Übrigens darf man aus dieser Lage des Nabels genau im Mittelpunkt des äußeren Körpers, sowie in unmittelbarer Nähe des als inneres Zentrum des Leibes und zugleich als Sitz der Seele (Lebenskraft) aufgefassten Zwerchfells wohl den Schluss ziehen, dass wie dieses so auch der Nabel in den engsten Beziehungen zur Seele stehend gedacht wurde.\footnote{Vgl. z. B. Ps.-Hippocr. π. ἑβδ. 52 = 8 p. 672 u. 9 p. 463 L: Ὅρος δὲ θανάτου ἐὰν τὸ τῆς ψυχῆς θερμὸν ἐπανέλθῃ ὑπὲρ τοῦ ὀμφαλοῦ εἰς τὸν ἄνω τῶν φρενῶν τόπον. --- Ebenda 48 = 9 p. 66 L: Definitio autem superiora partium et inferiora corporum umbilicus. Vgl. Roscher, Über Alter, Ursprung u. Bedeutung d. hippokr. Schr. v. d. Siebenzahl S. 16 A. 21 f. und Galen. 16 p. 284 K. Philolaos ob. Anm. 6.} Wir werden diesen zunächst nur als wahrscheinliche Vermutung sich ergebenden Schluss alsbald durch zahlreiche Tatsachen aus dem Folklore verschiedener Natur- und Kulturvölker bis zu einem gewissen Grade bestätigt finden.\footnote{Mir ist nur eine einzige Stelle bekannt geworden, wo ὀμφαλός nicht die genaue Mitte bezeichnet; ich meine die siebenfache Gliederung des νόμος κιθαρῳδικός des Terpander nach Pollux On. 4, 66: μέρη δὲ τοῦ κιθαρῳδικοῦ νόμον, Τερπάνδρου κατανείμαντος ἑπτά· ἀρχά, μεταρχά, κατατροπά, μετακατατροπά, ὀμφαλός, σφραγίς, ἐπίλογος. Hier steht ὀμφαλός in auffallender Weise an 5. statt an 4. Stelle.}

Als zweites beinahe ebenso wichtiges Merkmal sämtlicher durch ὀ. bezeichneten Begriffe kommt die rundliche Form in Betracht, die bald als eine Vertiefung (wie beim normalen menschlichen Nabel) bald als eine Nabelschnur, d. h. eine mehr oder weniger in die Länge gezogene Erhöhung (wie bei dem Keim der Pflanze oder dem Buckel des Schildes usw.) erscheint.

Weiteres hochwichtiges Material zur Erkenntnis der großen Bedeutung, welche in der ältesten Zeit dem Nabel und der Nabelschnur beigelegt wurde, liefert uns die Betrachtung gewisser bei fast allen Natur- und Kulturvölkern üblichen Geburtsriten, die wir ganz notwendig in diesem Zusammenhang zu besprechen haben. Wir erörtern hier zunächst diejenigen Gebräuche, welche deutliche Beziehungen des Nabels und der Nabelschnur zur Seele, d. h. zur Lebensfähigkeit, Fruchtbarkeit und Lebenskraft der Menschen, verraten.

Nach Foy (Arch. f. Rel. Wiss. 10 (1907) S. 557) wird beim Tuhoestamme (in Polynesien) Bäumen, die mit den Nabelschnüren bestimmter mythischer Vorfahren assoziiert und mit den Nabelschnüren aller Kinder behängt sind, die Macht zugeschrieben, Frauen fruchtbar zu machen. "`Anderwärts wird die Nabelschnur an geheiligten Orten vergraben und ein Bäumchen darüber gepflanzt, das als Lebensbaum fungiert. Die Nabelschnur eines Häuptlingssohnes wird dagegen oft unter einen Felsblock oder Baum an der Stammesgrenze gelegt, um als Grenzschutz zu dienen. Überall blickt die Anschauung durch, dass die Nabelschnur eine Repräsentation der Seele des betreffenden Individuums ist."' --- Von ganz besonderem Interesse ist das, was Frazer, The golden bough 2 1, 56 von der Nabelschnur des Königs von Uganda berichtet: "`The navel-string of the king of Uganda is preserved with the greatest care all through his life. It is wrapt in cloth and the wrappers increase in number as the king grows from infancy to manhood, until is assumes the appearence of a human figure swathed in cloth. The official who has charge of it is one of the highest ministers of state, and it is in his duty from time to time to present the precious bundle to the king."'\footnote{Ein ähnlicher Brauch herrscht bei den Wahuma im Reiche des Königs Kamrasi von Unyoro. Hier werden die Nabelstränge von der Geburt an aufbewahrt; beim Tode werden die von Männern innerhalb der Türschwelle, die von Weibern außerhalb begraben (Perty, Anthrop. 2 (1874), S. 272; Ploss, D. Kind 2, 1, 16).} Derselbe Frazer berichtet a. a. O. S. 262 von den Dyaks auf Borneo: "`Among the Dyaks of the Kajan and Lower Melawie districts you will often see, in houses where there are children, a basket of a peculiar shape with shells and dried fruits attached to it. These shells contain the remains of the children's navel-strings and the basket to which they are fastened is commonly hung beside the place where the children sleep. When a child is frightened, for example by being bathed or by the bursting of a thunderstorm, its soul flees from its body and nestles beside its old familiar friend the navel-string in the basket, from which the mother easily induces it to return by shaking the basket and pressing it to the childs body."' Von den Grönländern bezeugt H. Rink, Danish Greenland, London 1877 S. 205 (nach Ploss, D. Kind 2 1, 16): "`An extraordinary effective amulet for the purpose of restoring health to a child and conferring longevity on it is supplied by its navel-string, which for this reason, is sometimes carefully preserved."' Auch die Japaner heben die Nabelschnur des Kindes sorgfältig bis an das Ende seines Lebens auf und geben sie dem Toten mit in das Grab, gewiss nur deshalb, weil sie darin ebenfalls ein für das Heil und Leben der Kinder überaus wichtiges Amulett erblicken (Ploss a. a. O. 2, 199). Aus diesen und zahlreichen anderen von Ploss a. a. O. und ihm selbst gesammelten Beispielen schließt Frazer gewiss mit Recht: "`that the... navel-string remain through life, or at least for some considerable time, in sympathetic connection with the child, and that whatever is done to them produces a corresponding effect for good or ill on him or her. Thus the magic practised on them is sympathetic in the strict sense, for it rests on the principle that what is done to a thing affects simultaneously a person with whom the thing was formerly in contact."'

Aber diese außerordentliche Bedeutung der Nabelschnur reicht noch weit über den Rahmen der soeben angeführten sehr verschiedenartigen Naturvölker hinaus und lässt sich ganz ähnlich auch noch bei hochkultivierten, insbesondere bei indogermanischen Nationen nachweisen.

In Deutschland z. B. ist die Vorstellung, dass die sorgfältig aufgehobene Nabelschnur ein wirksames Amulett sei, von dem die geistige und körperliche Entwicklung bis zu einem gewissen Grade abhänge, stark verbreitet. So heißt es in der Schweiz: "`Bewahrt man einem Kinde die Nabelschnur bis in sein siebentes Jahr auf und gibt sie ihm dann zum Zerschneiden, so bekommt es eine große Fertigkeit in Handarbeiten und wird auch sonst geschickt. In der bayrischen Rheinpfalz wird der Nabelschnurrest in Leinwand eingewickelt und später, wenn das Kind ein Knabe war, verhakt, bei einem Mädchen verstochen, damit jener ein tüchtiger Geschäftsmann, dieses eine geschickte Näherin werde. Damit das Kind leicht lesen lerne, lässt man es in Oldenburg das A der Fibel durch das Loch der Nabelschnur anschauen. Damit es leicht gehe, legt man in Franken einen Hasenkopf mit recht starken Zähnen unter das Kopfkissen und dazu die getrocknete Nabelschnur.\footnote{In Franken wird die sorgfältig aufgehobene Nabelschnur dem Kinde nach zurückgelegtem sechsten Jahre in eine Eierspeise gehackt zu essen gegeben; so wird der Verstand geöffnet; in Hessen näht man sie dem Kinde in die Kleider, damit es nicht verloren geht, man steckt sie in Ostpreußen dem Kinde, wenn es zum ersten Mal in die Schule geht, in den Busen, dann lernt es gut usw. (Ploss, a. a. O. 2 1, 16).} Um sich hieb- und schussfest zu machen, näht man sich in Hessen ein Stückchen Nabelschnur in die Kleider; und schon Fischart sagt von den Soldaten, welche feldflüchtig ihr Leben zu retten suchten: "`Etliche zogen ihre Kinderpelglin herfür, meinten also dem Teufel zu entfliehen."' Die abgefallene Nabelschnur wird von der Mutter in einem Blechlöffel zu Pulver gebrannt, das man dem Kinde an drei aufeinanderfolgenden Freitagen der ersten 6 Wochen\footnote{Vgl. zum Verständnis dieser Frist Roscher, D. Tessarakontaden u. Tessarakontadenlehren d. Griechen u. and. Völker, Leipz. 1909, S. 28 A. 11. 30 A. 16. 41. 89 A. 112. 92. 99. 150. 162. 171. 200.} mit Wasser eingibt.\footnote{Vgl. dazu Politis in der Λαογραφία γ΄ (1912) p. 699: Ἐν Ζακύνθῳ τὸν ὀμφάλιον λῶρον, ἀλλὰ μόνον τῶν ἀρρένων "`φυλάττει ἡ μήτηρ, ζυμώνει δὲ μετ᾽ ἀλεύρου καὶ δίδει εἰς σκύλλον ἐν ἡμέρᾳ Σαββάτου διὰ νὰ κάνῃ ἀρσενικά"' (ΑΟΔΟ [Ἀπ᾽ ὅλα δἰ ὅλους] Ἀθ. 1904 ἀρ. 46 σ. 755).} Dann bekommt das Kind keine Krämpfe (in der Altmark). Zur Taufe wird dem Kinde Salz, Geld und die Nabelschnur mitgegeben (in Königsberg)"' (Ploss a. a. O. 1, S. 17 f., Wuttke, D. Volksaberglaube 2, § 579, S. 357. Frazer a. a. O., S. 54 f.). In Memel wird der getrocknete Nabelschnurrest aufbewahrt und dem Kinde, wenn es an Krämpfen leidet, eingegeben.\footnote{Auch die alten Peruaner im Inka-Reiche legten dem aufgehobenen Nabelschnurrest die Bedeutung eines Heilmittels bei und ließen das Kind daran saugen, wenn es etwa erkrankte (Ploss 2 1, 17). --- Bei den Mohammedanern zu Bagdad ist es unerlässlich, dass das Kind ein Stück von der Nabelschnur mindestens so lange behält, als es saugt (Ploss 2 1, 18, Globus 1868 (14) S. 52).} In Schwaben behauptet man: "`wenn man einen Kindsnabel in einen goldenen oder silbernen Fingerring fassen lässt und am linken Goldfinger trägt, so hilft er gegen das Grimmen; auch glaubt man dort, dass das Pulver von einem abgefallenen Kindsnabel eingegeben gegen `Kindsgichten' hilft. So heißt es auch in Bruggers altem Receptir-Handbuch: Heb' der Kinder Nabelgertlein wohl auf; bekommt es einmal Anmal oder Flecken; so leg' selbiges Näbeli in Feldwickenwasser und leg's täglich dreimal zum Trocknen auf's Anmal also lange, als es war, da das neugeborene Kind die Flecken empfangen hatte"' (Ploss a. a. O. 2 1, 17).

Übrigens gilt ziemlich dasselbe wie von der Nabelschnur auch von der sogenannten `Glückshaube,' d. h. von denjenigen Eihäuten, welche nicht, wie es gewöhnlich bei der Entbindung geschieht, vor Durchtritt des Kindeskopfs zerreißen, sondern vielmehr den Kopf des Neugeborenen wie eine Kappe überziehen, was fast allgemein für eine überaus glückliche Vorbedeutung aufgefasst wird (Ploss a. a. O. 2 1, 12 ff., Wuttke a. a. O. § 182 u. 579) "`Kinder, die mit einer solchen überdeckt zur Welt kommen, gelten in Deutschland für Glückskinder; diese Haut wird sorgfältig aufgehoben oder in die Kleider genäht, auch bei der Taufe mitgenommen, und bringt dem, der sie bei sich trägt, Glück im Handel, bei Prozessen\footnote{Dieselbe Bedeutung hat auch die aufgehobene Nabelschnur bei den Alfurus auf Celebes, bei den Soongaren, den Kalmücken, Mongolen und Chinesen (Ploss a. a. O. 16 f.).} und anderen Geschäften (Ostpreußen, Schlesien, Sachsen, Pfalz); die Hebammen entwenden sie gern, um sie ihren eigenen Kindern zu geben (Wuttke a. a. O.). --- In der Literatur der Griechen und Römer herrscht im allgemeinen tiefes Schweigen über diesen Aberglauben; dass er aber doch auch hier einst existiert hat, lehrt unwiderleglich folgende Stelle aus des Aelius Lampridius Vita des Antoninus Diadumenus (= Script. Hist. Aug. cap. 4 = 1, 197 ed. Peter): Nunc veniamus ad omina imperii, quae cum in aliis tum in hoc praecipue sunt stupenda... solent deinde pueri pilleo insigniri naturali [= Glückshaube], quod obstetrices rapiunt et advocatis credulis vendunt, si quidem causidici hoc iuvari dicuntur. at iste puer pilleum non habuit sed diadema tenue, sed ita forte ut rumpi non potuerit, fibris intercedentibus specie nervi sagittari. Ferunt denique Diadematum puerum appellatum, sed ubi adolevit, avi sui nomine materni Diadumenum vocatum, quamvis non multum abhorruerit ab illo signo Diademati nomen Diadumeni. --- Dies ist meines Wissens das einzige erhaltene Zeugnis des klassischen Altertums, welches die in Rede stehende Bedeutung der `Glückshaube' für die Griechen und Römer klar und deutlich ausspricht; es gilt nunmehr ähnliche Zeugnisse für die von uns auch für die Griechen vermutete Bedeutung der Nabelschnur ausfindig zu machen. Wenn ich nicht irre, so gibt es auch für diese wenigstens zwei Belege, auf die ich hier verweisen möchte. Der wichtigste findet sich bei Pollux (On. 2, 170) und lautet: τὸ δὲ περὶ τῷ ὀμφαλῷ δέρμα γραῖα ὀνομάζεται, ὅτι ῥυσούμενον γὴρως σύμβολον γίνεται. Dieser Satz kann sich m. E. kaum auf etwas anderes als auf die Haut der Nabelschnur beziehen, die γραῖα genannt wurde, weil sie, runzlig geworden, also getrocknet und wohl verwahrt, eine gute Vorbedeutung (σύμβολον) dafür war, dass das Kind das Greisenalter (γῆρας) erreichen würde. Diese Deutung lag umso näher, als ja γῆρας = Alter und alte, abgestreifte Schlangenhaut,\footnote{Bei dieser Gelegenheit sei daran erinnert, dass die Schlangenhaut von den antiken Ärzten als ein wirksames Arzeneimittel verwertet wurde; vgl. Plin. h. n. 28, 174; 30, 69. Marcell. de medicam. 9, 102. --- Dass auch die getrocknete und pulverisierte Nabelschnur ebenfalls für ein wirksames Heilmittel namentlich bei Kinderkrankheiten galt, haben wir oben gesehen. Es fehlt dafür nur noch ein bestimmtes Zeugnis aus der griechischen Literatur. Hinsichtlich der Bedeutung der Schlangen in der antiken Pharmakopöe s. meine Darlegungen in Fleckeisens Jahrb. f. kl. Philol. 1886, S. 243 f.} γραῦς = alte Frau und runzelige Haut über der (gekochten) Milch und γραῦς = altes Weib und getrocknete Nabelschnur etymologisch ganz nahe miteinander verwandt sind, wie Curtius (Grundz. d. gr. Etym. 5 176) nachgewiesen hat.

Ein zweites Zeugnis dafür, dass auch im ältesten Hellas die Nabelschnur in den Geburtsriten eine Rolle gespielt haben muss, entnehme ich dem jedenfalls sehr alten Mythos von der Geburt des Zeus, der ja, wie allgemein bekannt ist, vorzugsweise in Kreta lokalisiert war. Von dem so ziemlich im Mittelpunkte dieser Insel, nicht weit von Knossos,\footnote{Im Hinblick auf die Tatsache, dass die älteste Priesterschaft des delphischen Apollon nach dem homerischen Hymnus aus Kreta, und zwar gerade aus Knossos stammte, nimmt Gruppe (Gr. Mythol. 1, 103, 4) an, dass auch die Vorstellung vom ὀμφαλὸς γῆς aus Kreta nach Delphi verpflanzt sei.} gelegenen Orte Ὀμφάλιον\footnote{Ob der Ort Ὀμφάλιον, der nach Steph. Byz. s. v. in Thessalien, nach Ptolem. 3, 14, 7 in Epirus lag (so auch Rhian. b. Steph. Byz. s. v. Παραυαῖοι· ἔθνος Θεσπρωτικόν. Ῥιανὸς ἐν δ΄ Θεσσαλικῶν· σὺν δὲ Παραυαίοις καὶ ἀμύμονας Ὀμφαλιῆας), seinen Namen einer zentralen Lage oder einem andern Umstande verdankte, muss dahingestellt bleiben. Vgl. auch Tümpel im Lex. d. Mythol. 3 Sp. 870.} und der ihn umgebenden Ebene, dem Ὀμφάλιον πεδίον, singt Kallimachos (hy. in Jov. 43 ff.):
\begin{quotation}
Εὗτε Θενὰς ἀπέλειπεν ἀπὸ Κνωσοῖο φέρουσα,

Ζεῦ πάτερ, ἡ νύμφη σέ (Θεναὶ δ᾽ ἔσαν ἐγγύθι Κνωσοῦ),

τουτάκι τοι πέσε, δαῖμον, ἄπ᾽ ὀμφαλός. ἔνθεν ἐκεῖνο

Ὀμφάλιον μετέπειτα πέδον καλέουσι Κύδωνες.\footnote{Vgl. auch Schol. z. Nikand. Al. 7: [Ἄρκτον ὑπ᾽ ὀμφαλόεσσαν]... ὀμφαλόεσσαν εἴρηκε διὰ τὸ μέσον τοῦ βορείου κεῖσθαι [τ. ἄρκτον]. Τινὲς δ᾽ ἐπειδὴ δοκεῖ ὁ κατὰ τὰς ἄρκτους τόπος εὐβοτώτατος ὀμφαλόεσσαν εἰρῆσθαί φασι τὴν τροφώδη· ὀμφαλὸς γὰρ ἀπὸ τῆς ὄμπνης εἴρηται, ὅ ἐστι τροφή [!!], ἀφ᾽ οὗ καὶ ἡ Δημήτηρ ὀμπνία, ἄλλοι δὲ τὴν Κρητικήν· Ὀμφαλὸς γὰρ τόπος ἐν Κρήτῃ, ὡς καὶ Καλλίμαχος· πέσε, δαῖμον,... Κύδωνες. --- Steph. Byz. Ὀμφάλιον, τόπος Κρήτης πλησίον Θενῶν καὶ Κνωσσοῦ κ. τ. λ.}
\end{quotation}
\paragraph{}
Und Diodor (5, 70) erzählt: Σημεῖα δὲ πολλὰ μέχρι τοῦ νῦν διαμένειν τῆς γενέσεως καὶ διατροφῆς τοῦ θεοῦ τούτου [= Διός] κατὰ τὴν νῆσον. φερομένου μὲν γὰρ ὑπὸ τῶν Κουρήτων αὐτοῦ νηπίου φασὶν ἀποπεσεῖν τὸν ὀμφαλὸν περὶ τὸν ποταμὸν τὸν καλούμενον Τρίτωνα, καὶ τὸ χωρίον τὲ τοῦτο καθιερωθὲν ἀπὸ τοῦ συμβάντος Ὀμφαλὸν προσαγορευθῆναι, καὶ τὸ περικείμενον πεδίον ὁμοίως Ὀμφάλειον\footnote{Vgl. auch Bursian, Geogr. v. Gr. 2, 570 u. Anm. 2.} κ. τ. λ. Wer bedenkt, dass die Geburt des Zeus ebenso für das Prototyp aller menschlichen Geburten galt wie seine heilige Hochzeit (ἱερὸς γάμος) für das Ur- und Vorbild aller menschlichen Hochzeiten, der wird es doch wohl mit mir für recht wahrscheinlich erklären, dass der Mythos von der Nabelschnur des höchsten Gottes auch eine ähnliche Bedeutung und Behandlung dieses Organes bei den menschlichen Geburten voraussetzt. Ja, es scheint nicht unmöglich, dass man in uralter Zeit zu Omphalion die Nabelschnur des Zeus ebenso als kostbare Reliquie zeigte und verehrte, wie in Delphi den Stein, den Kronos nach der Geburt des Gottes ausgespien haben sollte (Paus. 10, 24, 6) oder zu Tegea die Locke der Gorgo Medusa (Paus. 8, 47, 5) usw.

Als weitere Zeugnisse für dieselbe Sache lassen sich endlich auch mehrere auf meine Bitte in höchst dankenswerter Weise von N. G. Politis, dem hervorragendsten Forscher auf dem Gebiete des griechischen Folklore, gesammelte Bräuche der heutigen Griechen anführen, die sich wohl nur als Überlebsel (survivals) aus dem klassischen Altertum erklären lassen. Politis berichtet in der Λαογραφία γʹ (1912) S. 898 ff. unter anderem Folgendes: Ἐν Λέσβῳ περιτυλίσσουν τὸν ἀποκοπέντα ὀμφάλιον λῶρον εἰς παννίον καὶ τὸν ῥίπτουν εἰς τὸ σχολεῖον ἢ τὴν ἐκκλησίαν ἢ τοὺς ἀγρούς, πιστεύοντες ὅτι τὸ παιδίον θὰ γίνῃ ἂν εἰς τὸ σχολεῖον ἔπεσεν ὁ λῶρος διδάσκαλος, ἂν εἰς τὴν ἐκκλησίαν παπᾶς, ἢν εἰς τοὺς ἀγροὺς γεωργός. Διὰ τοῦτο ὅταν παιδίον τι συχνάζῃ πολλάκις εἴς τινα τόπον ἡ μήτηρ του τῷ λέγει θυμωμένη· "`Τοὺν ἀφαλό σ᾽ ῥῆξαν αὐτοῦ;"' (Georgeakis et Pineau, Le Folk-lore de Lesbos S. 331-332). Mit Recht bemerkt Politis a. a. O. S. 699, 1 dazu: Δεικνύει δ᾽ ἡ φράσις αὕτη τὴν πίστιν περὶ τοῦ μυστηριώδους συνδέσμου τοῦ βίου τοῦ ἀνθρώπου πρὸς τὸ ἀποκοπὲν μέλος αὐτοῦ. Τὴν αὐτὴν ἔννοιαν φαίνεται ὅτι ἔχει καὶ ἡ παροιμιώθης Κεφαλληνιακὴ φράσις· "`Σοῦ φαίνεται καὶ κεῖ τοῦ κόψανα τἀφάλι."' (Πολίτου, Παροίμ. Α΄ 663 λ. ἀφαλὸς 2.) --- Ὁμοίως ἐν Βουρδουρίῳ τῆς Μικρᾶς Ἀσίας "`ὁ ὀμφαλὸς καταπίπτων ἔπρεπε νὰ ῥίπτηται εἰς τὴν ἐκκλησίαν, ὅπως τὸ παιδίον ἀγαπᾷ αὐτὴν καὶ γίνῃ εὐλαβής, ἣ ἐθάπτετο εἰς γωνίαν τινα τῆς οἰκίας διὰ νὰ μὴ ἔχῃ "`τὸ μάτι ἐξω,"' ἀλλὰ νὰ ἀγαπᾷ τὴν οἰκίαν καὶ τὴν ἐν αὐτῇ διαμονήν. (Ξενοφάνης, Ἀθ. 1906 τ. Γ΄ ς. 275.) --- Ἐν Σινασῷ δὲ τῆς Καππαδοκίας "`ἀφοῦ πέσῃ ὁ ὀμφάλιος λῶρος λαμβάνεται καὶ φέρεται ἢ ἐν τῇ ἐκκλησίᾳ ἢ ἐν τῇ σχολῇ καὶ κρύπτεται εἰς ὁπήν τινα, διότι πιστεύουσιν ὅτι διὰ τοῦ τρόπου τούτου ὁ παῖς καθίσταται εὐσεβὴς ἢ φιλομαθής"' (Ι. Σαραντίδου Ἀρχελάον, Ἡ Σίνατος. Ἀθ. 1899 S. 68).

Wir schließen nunmehr diesen Abschnitt ab mit einer summarischen Aufzählung der bis jetzt durch unsere Erörterungen erzielten Ergebnisse. Als solche glauben wir annehmen zu dürfen:
\begin{enumerate}
    \item die zentrale Lage oder Stellung aller mit ὀμφαλός (umbilicus) bezeichneten Begriffe innerhalb eines größeren Ganzen\footnote{Auch im Indischen bedeutet \emph{nābhi} (Nabel) in übertragenem Sinn nach dem Petersburger Wörterbuch "`Mittelpunkt,"' und zwar "`sowohl die räumliche Mitte als das die Teile Zusammenhaltende."' Im Rgveda wird Agni der Nabel der Erde genannt, \emph{nābhir Agniḥ pṛthivyāḥ} 1, 59, 2; das Opfer ist der Nabel der Welt: \emph{bhuvanasya nābhiḥ}: 1, 164, 34 (gütige Mitteilung E. Windischs).};

    \item der Ausdruck ὀμφαλός (umbilicus), vom menschlichen Körper gebraucht, bezeichnet entweder die rundliche Vertiefung in der Mittellinie des Körpers oder die rundliche (röhrenförmige), damit zusammenhängende Nabelschnur;

    \item Nabel und Nabelschnur scheinen auch bei den ältesten Griechen ebenso wie bei unzähligen anderen verwandten und nichtverwandten Völkern eine gewisse Heiligkeit und religiöse Bedeutung gehabt zu haben.\footnote{Auf eine ähnliche Bedeutung führt auch das, was Artemidor Oneirokr. 43 (= p. 41, 21 ff. H.) vom ὀμφαλός sagt: αὐτὸς δὲ ὁ ὀμφαλὸς ὄντων μὲν γονέων τοὺς γονεῖς, οὐκ ὄντων δὲ τὴν πατρίδα σημαίνει, ἧς ἐξέφυ τις καὶ ἐξεγένετο, ὥσπερ καὶ τοῦ ὀμφαλοῦ. ἐὰν οὖν τι περὶ τὸν ὀμφαλὸν δυσχερὲς γένηται, στερηθῆναι γονέων ἢ τῆς πατρίδος σημαίνει, καὶ τὸν ἐπὶ ξένης ὄντα οὐκ ἐπανάγει. Vgl. auch oben Philolaos in Anm. 6, der den ὀ. als ἀρχὰ ῥιζώσιος καὶ ἀναφύσιος τῶ πρώτω definiert, und sanskr. \emph{nābh-i-s} = Nabel, Nabe, Verwandtschaft.} Vielleicht gelingt es, dafür aus dem Bereiche von Alt- und Neuhellas noch weitere Belege zu sammeln.
\end{enumerate}
\clearpage
\section{Der Gedanke eines Zentrums (`Nabels') der Erdoberfläche bei verschiedenen Völkern.}
\paragraph{}
Die\footnote{Ich verdanke die im Folgenden dargelegten Anschauungen der Chinesen, Semiten, Perser usw. größtenteils den mir äußerst wertvollen brieflichen Mitteilungen von A. Forke in Berlin, I. Goldziher in Budapest, A. Jeremias in Leipzig und Rud. Lange in Berlin-Steglitz.} Vorstellung eines `Nabels' der Erdoberfläche hängt naturgemäß mit der uralten und überall verbreiteten Anschauung zusammen, dass die Erde eine kreisrunde (oder auch viereckige; s. unten!) Fläche sei, die als solche mit mathematischer Folgerichtigkeit einen Mittelpunkt oder `Nabel' haben müsse. Selbstverständlich verschwand die Idee eines Zentrums der Erdscheibe sofort oder ging in eine andere Vorstellung über, sobald man erkannte, dass die Erde keine runde, horizontale Ebene (Scheibe), sondern vielmehr eine Kugel sei. Wir werden später sehen, dass dieser Übergang bereits in vorchristlicher Zeit bei den Griechen stattgefunden hat, die demgemäß genötigt waren, ihre Anschauung vom Erdnabel (Delphi, Branchidai) in die von der Erd- und Weltachse (ἄξων, axis) zu verwandeln. Wir beginnen unsere Übersicht der einschlägigen Vorstellungen mit den Völkern des äußersten Ostens, um schließlich bei denen des äußersten Westens zu enden.

1. Das uralte Kulturvolk der Chinesen, die bekanntlich ihr Land Tschung Kwo, d. i. `Reich der Mitte' (nicht aber `Himmlisches Reich') nennen,\footnote{Meyers Konvers.-Lex. 6 4, 34 b.} `betrachtet,' wie mir A. Forke freundlichst mitteilt, `seit dem Ende des 12. Jahrhunderts vor Chr. Loyang, auch Loyi genannt, das heutige Honanfu in der Provinz Honan als den Mittelpunkt Chinas und der Welt.\footnote{Man beachte, dass auch hier der `Nabel' eines beschränkteren Gebietes mit dem der Erde zusammenfällt. Wir werden später das gleiche auch bei den Israeliten und Griechen beobachten. Wahrscheinlich ist in solchen Fällen der `Nabel' der beschränkteren Gebiete der ältere.} Loyi wurde um 1098 vor Chr. vom Herzog von Tschou gegründet und zur zweiten (östlichen) Hauptstadt der Tschou-Dynastie gemacht. Die Gründung wird erzählt im `Buch der Geschichte,' dem Schuking. Wie der Ort für die neue Hauptstadt gefunden wurde, er sieht man aus dem Tschouli, dass aus dem 11. Jahrhundert stammen soll und das Beamtenwesen der Tschou-Dynastie schildert; vgl. Le Tcheou Li ou Rites des Tcheou, trad. par Ep. Bror, Paris 1851 Ba. I 8. 200. Wie hier ausführlich geschildert wird, versuchte man den Mittelpunkt der Erde durch Messungen mit der Sonnenuhr zu bestimmen.\footnote{S. unten S. 28 Anm. 50.}

Von Loyang als dem Zentrum der Welt handelt auch Way Tschung, ein Schriftsteller des 1. Jahrhunderts nach Chr.; vgl. meine (A. Forkes) Übersetzung: Lun Hêng, Part 1, Philosophical Essays of Wang Chung S. 256. Von Loyang ausgehend versucht er die Erdfläche zu berechnen. Nach seiner Ansicht und der der meisten Chinesen ist die Erdoberfläche viereckig, das Himmelsgewölbe dagegen rund.

Als Mittelpunkt der Erde gilt den Chinesen zugleich der Sung-schan, der mittlere der fünf heiligen Berge, auch in der Präfektur Honanfu gelegen. Dies wird in einer Stelle des Po-hu-tung des Historikers Pan Ku aus dem 1. Jahrhundert nach Chr. ausgesprochen.

2. Die Japaner. Auch in Japan scheint die Vorstellung von einem Mittelpunkt (Nabel) der Erde sehr alt zu sein. Prof. Rup. Lance in Berlin-Steglitz, der ausgezeichnete Kenner der japanischen Sprache, teilt mir brieflich darüber Folgendes mit:

`Hinter einem der ältesten Tempel Japans, an der Ostküste der Hauptinsel (in der Provinz Hitachi, nahe der Stadt Chöshi), namens Kashimajinja, d. h. "`Shintotempel der Hirschinsel,"' befindet sich ein Stein, namens Kaname-ishi, d. h. Zapfenstein. Man glaubt allgemein, dass sein Ende im Zentrum der Erde liegt. [Diese Vorstellung setzt doch wohl die Verwandlung des Begriffes der Erdscheibe in den der Erdkugel voraus. Roscher.] Es wird von dem berühmten Fürsten von Mito, namens Mitsukuni, der dort in der Nähe residierte, erzählt, dass er viele Tage lang nach dem untersten Ende des Steins habe graben lassen, aber ohne Erfolg. Nach einem Bericht soll unter diesem Stein auch der riesige Wels (namazu) liegen, dessen Bewegungen nach dem Volksglauben die Erdbeben verursachen.

Das Wort kaname bedeutet der Zapfen, der Stift (z. B. beim Fächer, der denselben zusammenhält), übertragen dann auch der wesentliche Punkt von etwas. ishi bedeutet Stein.'

3. Die Malayen. `In der Regentschaft von Kadjang auf Celebes nennt die Bevölkerung den heiligen Ort Papurianga, welcher sich auf einem Hügel befindet, den Bambus umringt. Il y a là une fosse de grandes dimensions... on l'appelle \emph{potgi-tanah} (le nombril de la terre), nom qui est rarement prononcé par respect et par crainte de quelque malheur. C'est pour cette raison qu'on a donné à cet endroit le nom de \emph{Papurianga}, ce qui veut dire possesseur de tout, et tout puissant (L'Anthropologie, Paris 1893 (4) 621.' Gütige Mitteilung Goldzihers.

4. Die Inder. Wie mir E. Windisch schreibt, `liegt in dem späteren mythischen Weltbild der Inder der mythische aus Gold bestehende Berg Meru im Nabel des innersten Weltteils, \emph{nābhyām} (Lokativ von \emph{nābhi} = Nabel): Bhägavata Purāna 5 16, 7.\footnote{Vgl. auch über die Mitte der Erde (Meru mit Jambu dwipa) nach indischer Anschauung Baldvin, Prehistoric Nations, New York 1859 p. 65. Über indische und chinesische Vorstellungen von der Mitte der Erde s. auch Reinaud, Introduction zur Géographie d'Aboulféda. T. 1 (Paris 1848) p. 215 (Mitteilungen I. Goldzihers).} Das Wort \emph{nābhi} bedeutet im übertragenen Sinne nach dem Petersburger Wörterbuch `Mittelpunkt,' und zwar "`sowohl die räumliche Stätte als das die Teile Zusammenhaltende."' Merkwürdig ist eine Stelle im Pāraskara Gṛbyasūtra 3, 4, 4, beim Hausbau: "`Die Eckbalken soll man aufrichten mit einem Spruche, der mit den Worten: «Hier richte ich auf den Nabel der Welt (bhuvanasya nābhim)» beginnt. Im \d{R}gveda wird Agni der Nabel der Erde genannt, nābhir Agniḥ pṛthivyāḥ 1, 59, 2; das Opfer ist der Nabel der Welt (bhuvanasya nābhiḥ 1, 164, 34).' Wie mir aus der letzteren Anschauung sowie aus dem Ritus beim Hausbau hervorzugehen scheint, nahm der religiös begeisterte Inder bei der feierlichen Zeremonie des Opfers und der Aufrichtung der Eckbalken an, dass er in diesem Augenblicke im Zentrum der Erde sich befinde, ein Glaube, der natürlich dazu betrug, die feierliche Stimmung des Moments zu erhöhen. Dieser Glaube rechtfertigt sich ganz einfach durch die Tatsache, dass allgemein unter Zenith oder Scheitelpunkt derjenige Punkt am Himmel betrachtet wird, der gerade über dem Haupte oder dem Scheitel des jeweiligen Beobachters steht und zugleich als der mittelste und höchste Punkt des Himmels angesehen wird.

5. Die Babylonier. Aus der, wie es scheint, von Jensen (Kosmol. d. Bab. 161 f.) festgestellten Tatsache, dass die Babylonier (ebenso wie die älteren Griechen) die Erde und den Himmel, nicht nur als `optisch, sondern als der Wirklichkeit entsprechend,' für kreisrund ansahen, folgt eigentlich unmittelbar, dass sie auch einen Mittelpunkt (`Nabel') dieser Erdscheibe angenommen haben müssen. Um Näheres über diese Vorstellung zu hören, habe ich mich an A. Jeremias gewandt und von diesem erfahren, dass sumerisch DUR. AN. KI oder babylonisch markas šamê u irṣitim `Band des Himmels und der Erde' bedeutet, und dass darunter wahrscheinlich die Nabelschnur, `das Mutterband' zu verstehen sei.\footnote{Vgl. weiter unten die ähnliche Anschauung gewisser Gnostiker.} Zugleich bedeutet aber der Ausdruck nach J. `den Höhepunkt des irdischen Alls, das in dem dreigeteilten himmlischen All hängt' (s. D. Alte Test. u. d. alt. Or. 2 S. 8). Von Ninib, der Manifestation der Gottheit im Höhepunkt des Kosmos, heißt es: mukil markas šamê u irṣitim = `der hält das Band Himmels und der Erde.' Einer der kosmischen Stufentürme in Nippur heißt DUR. AN. KI. = m. š. u i. So vermutet denn J. auch mit einer gewissen Wahrscheinlichkeit, dass Babylon \emph{bâb-ili} = `Tor des Himmels' heiße, weil es den Mittelpunkt darstelle (A. T. A. Or. 2 375). Wenn auch Jeremias' anregende Darlegungen noch nicht vermocht haben, Babylon unzweifelhaft als Nabel der Erde nach altbabylonischer Anschauung zu erweisen, so ist es mir doch schon deshalb gar nicht unwahrscheinlich, dass auch bei den Babyloniern der Begriff des ὀμφαλὸς γῆς eine Rolle gespielt hat, weil dieser (wie wir gleich sehen werden) auch bei den so nahe verwandten Juden, Phöniziern, Arabern, sowie bei den benachbarten Persern und Indern unzweifelhaft vorkommt. Eine weitere Stütze erhält diese Vermutung durch eine wahrscheinlich auf altorientalischen (persisch-babylonischen?) Vorstellungen beruhende Lehre der Sethianer bei Hippolytos ref. 5, 19 p. 202, 11, wo es heißt: γέγονεν οὖν ἐκ πρώτης τῶν τριῶν ἀρχῶν συνδρομῆς μεγάλη τις ἰδέα σφραγῖδος, οὐρανὸς καὶ γῆ. σχῆμα δὲ ἔχουσιν ὁ οὐρανὸς καὶ ἡ γῆ μήτρᾳ παραπλήσιον τὸν ὀμφαλὸν ἐχούσῃ μέσον, καὶ εἴ φησιν, ὑπὸ ὄψιν ἀγαγεῖν θέλει τις τὸ σχῆμα τοῦτο, ἔγκυον μήτραν ὁποίου βούλεται ζῴου τεχνικῶς ἐρευνησάτω, καὶ εὑρήσει τὸ ἐκτύπωμα τοῦ οὐρανοῦ καὶ τῆς γῆς καὶ τῶν ἐν μέσῳ πάντων ἀπαραλλάκτως ὑποκειμένων· γέγονε δὲ οὐρανοῦ καὶ γῆς τὸ σχῆμα τοιοῦτον οἱονεὶ μήτρᾳ παραπλήσιον κατὰ τὴν πρώτην συνδρομήν. In der Tat verbindet die Nabelschnur wie ein Band den Mutterkuchen (= γῆ) mit dem (darüberliegenden) Fötus (= οὐρανός), sobald man sich das Ganze in der Gestalt eines liegenden Θ (= $\odot$) vorstellt.\footnote{Vgl. Wolfg. Schultz, Dokumente der Gnosis S. 109 f. u. Anm. 1. Eine ähnliche Vorstellung von der Lage des Himmels und der Erde hatten die Ägypter. Vgl. die in meinen beiden Abhandlungen `Üb. Alter, Ursprung u. Bedeutg. d. hippokrat. Schrift von D. Siebenzahl' S. 12 und `Die neuentdeckte Schr. e. altmiles. Naturphilosophen usw.' Bildertaf. Fig. 1 u. 2 mitgeteilten Figuren. Hier liegt der Nabel der Himmelsgöttinnen wie der Erdgötter, über dem jene gelagert sind, genau in der Mitte der Darstellung des Kosmos (s. unt. S. 32 ob.). Ich verdanke obigen Hinweis auf Hippol. einem Briefe von W. Schultz.}

6. Die Israeliten. Eine berühmte Stelle des Propheten Ezechiel (um 595 v. Chr) 5, 5 lautet nach der Übersetzung von Kautzsch: `So spricht der Herr Jahwe: Dies ist Jerusalem, die ich mitten unter die Volker gestellt habe, und rings um sie her Länder.' Wie mir Prof. Winter in Dresden, der ausgezeichnete Talmudist, mitteilte, ist es nicht ganz sicher, ob hier wirklich eine zentrale Lage Jerusalems in der Mitte der Erdscheibe gemeint ist,\footnote{Ebenso urteilt Ginzberg, Die Haggada bei den Kirchenvätern und in der apokryphen Literatur = Monatsschrift f. Gesch. u. Wissensch. d. Judenthums 43 (1899) S. 68 f. Ich verdanke den Hinweis darauf I. Goldziher. S. auch Grünbaum in der Ztschr. d. D. Morgenl. Ges. 31 (1877) S. 199 u. Gruppe, Gr. Mythol. u. Rel.-Gesch. S. 725, 4.} aber schon die altjüdischen Erklärer haben unzweifelhaft die Sache so aufgefasst. So heißt es z. B. im Midrasch Tanchuma z. Abschn. Kedoschim gegen Ende (vgl. Monumenta Jadaica p. 237 nr. 792): `So wie der Nabel in der Mitte des Menschen ist, so ist das Land Israel in der Mitte der Welt,' wie es heißt: "`Die wohnen in der Höhe (im Nabel) der Erde"' (vgl. Ezech. 38, 12: `Leute, die auf dem Nabel der Erde wohnen')... Das Land Israel liegt in der Mitte der Welt und Jerusalem in der Mitte des Landes Israel\footnote{Vgl. Joseph. bell. jud. 3, 3, 5: μεσαιτάτη δὲ αὐτῆς [τ. Ἰουδαίας] πόλις τὰ Ἱεροσόλυμα κεῖται, παρ᾽ ὃ καί τινες οὐκ ἀσκόπως ὀμφαλὸν τὸ ἄστυ τῆς χώρας ἐκάλεσαν. --- Gervas. v. Tilb. ed. Liebrecht S. 1.} und das Heiligtum in der Mitte Jerusalems und die Tempelhalle in der Mitte des Heiligtums und die Bundeslade in der Mitte der Halle und der Grundstein der Welt vor der Lade, denn es heißt, dass von ihm aus die Welt gegründet wurde.'\footnote{`Im Madchal des `Abdarī (ed. Alexandrien 1293 H.) 3 266, 4 wird davor gewarnt, dass Jerusalempilger bei dem Ort, den man "`den Nabel der Welt nennt,"' gewisse verwerfliche Riten üben (sie entblößen ihren eigenen Nabel und berühren damit den Stein).' [Mitteilung I. Goldzihers.]}

Vgl. auch Monumenta Judaica C Weltbild p. 237 nr. 791 = Joma 54 b: `Und Grundstein wurde er genannt. Es wird gelehrt: Auf seinem Ausmaß wurde die Welt gegründet. Also lernten wir in der Mischna, dass es so ist, wie derjenige behauptet, welcher sagt, die Welt sei von Zijon aus erschaffen worden; denn es wird gelehrt: R. Eliezer sagte: Die Welt ist von ihrer Mitte aus erschaffen worden; denn es heißt: "`Wenn der Staub zum Gusswerk zusammenfließt und die Schollen aneinander kleben"' (Hiob 38, 38)... R. Jizchak, der Schmied, sagte: Der Heilige, gebenedeit sei er, warf einen Stein ins Meer, und von diesem aus wurde die Welt gegründet; denn es heißt: "`Worauf sind ihre Pfeiler eingesenkt, oder wer hat ihren Eckstein hingeworfen?"' (ebenda 38, 6).'\footnote{Prof. Winter schreibt mir am 5. Februar 1913: `Derselbe Salomo Jizchaki, der zu Ez. 5, 5 bemerkt: "`In die Mitte der Welt,"' erklärt 38, 12: "`Sitzende auf dem Nabel des Landes"' (nicht der Erde) mit den Worten: "`Auf der Höhe und Feste des Landes, wie dieser Nabel, der die Mitte des Menschen ist und nach allen seinen Seiten schräg abfällt."' Ähnlich zu Richter 9, 37 ("`Siehe Volk herabsteigend vom Nabel des Landes"'): "`Von der Feste des Landes, von dem Berge, der der höchste von allen ist"' So und ähnlich erklären Ez. 38, 12 und Richter 9, 37 alle mir zugänglichen hebräischen Kommentatoren, wohl gestützt auf das Targum, den aramäischen Paraphrasten.'}

Auch Hieronymos und Theodoretos in ihren Kommentaren zu Ezech. 5, 5 fassen die Worte des Propheten in dem Sinne auf, dass damit Jerusalem als Zentrum der Erde bezeichnet werde. Ebenso wird im Buche Henoch (ed. Dillmann, cap. 26 Anfang) eine Beschreibung von der Mitte der Erde, wo sich ein heiliger Berg befindet, gegeben, die keinen Zweifel darüber zulässt, dass der Verfasser Jerusalem, oder noch genauer den Tempelberg, für die Mitte der Erde hält. Ungefähr derselben Zeit entstammt das Buch der Jubiläen, wo 8 p. 251 der Berg Zion als Mittelpunkt der Erde und der Berg Sinai als Mittelpunkt des Nabels der Wüste bezeichnet wird.\footnote{Ginzberg a. a. O.}

Ginzberg (a. a. O.) fährt fort: `Steht demnach fest, dass Jerusalem schon in alter Zeit als Mittelpunkt der Erde galt, so lag es für die Adamlegende sehr nahe, Adam, den Mittelpunkt der Schöpfung, in den Mittelpunkt der Erde zu versetzen, und wir haben auch mehr als einen Beleg, dass dies wirklich geschah. Vgl. Pirke des R. Elieser cap. 11 (ed. Amsterd. p. 11 unt.): "`Und er knetete die Erde zum Körper Adams, an einem reinen --- heiligen --- Orte geschah dies, in der Mitte der Erde nämlich..."' Dies wird weiter unten dahin erklärt, dass diese Stätte das Heiligtum zu Jerusalem ist.'\footnote{Die alte Apokalypse Mosis (Apoc. apocr. 21 ed. Tischendorf) identifiziert das Zentrum der Erde, wo Adam erschaffen ward, ganz folgerichtig mit dem Paradiese. --- Ein arabischer Nachklang dieser Anschauung ist es wohl, wenn Al-Hākīm al-Termidī sagt: `Der Firdaus (= παράδεισος) ist der Nabel der ǵanna (surrat al-ǵanna) und ihre Mitte.' Murtaḍā, Itḥāf al Sāda 10 524, 8 v. u. [Gütige Mitteilung I. Goldzihers].} Jeder Kenner der griechischen Mythologie wird wohl in diesem Falle an die merkwürdige Parallele der am Nabel der Erde zu Delphi lokalisierten Deukalionsage denken.

Damit kombiniert Ginzberg a. a. O. S. 69 ff. weiter die alte christliche Sage von Golgatha, der Grabstätte Adams und dem Kreuzigungsort Jesu (Fabricius, Cod. Pseudep. 5 T. 1 p. 75 und 2 p. 37; Dillmann, Das christl. Adambuch p. 142), die eine Umarbeitung der obigen Adamlegende darstellt. Vgl. Fragen und Antworten der drei Heiligen in Denkschr. d. k. Akad. d. Wiss. zu Wien, phil-hist. Kl. Bd. 22 p. 63: `Et corpus Adae angeli susceperunt et portantes sepelierunt in medio loco terrae, in Jerusalem, eo loco ubi deum crucifixerunt.' Offenbar in Anlehnung an solche Gedanken `entstand [wie Guthe in seinem trefflichen Artikel Palästina in Herzogs Realenc. 3 14 S. 561 f. treffend bemerkt] in der alten christlichen Kirche eine Legende, die noch heute in der Grabeskirche Jerusalems ihr Denkmal hat. In dem Hauptschiffe sieht man dort auf einem etwa zwei Fuß hohen Ständer aus Marmor eine Halbkugel, die als der Mittelpunkt oder Nabel (ὀμφαλός) der Erde gilt. Die griechische Übersetzung von Ps. 74 (73) 12 [ὁ θεὸς] εἰργάσατο σωτηρίαν ἐν μέσῳ τῆς γῆς gilt als die Stütze dieser Annahme; die σωτηρία ist die Erlösung durch Christum.'\footnote{Mittelalterliche Karten zeigen in der Tat Jerusalem in der Mitte des runden `Orbis terrarum': s. A. Jeremias, D. A. Test. im Licht. d. alt. Or. 2 S. 584. --- S. auch Clem. Alex. Str. 5, 6, 33 S. 665, wo der Räucheraltar im Inneren des Tempels zu Jerusalem als σύμβολον τῆς ἐν μέσῳ τῷ κόσμῳ τῷδε κειμένης γῆς bezeichnet wird, was Gruppe a. a. O. 725, 4 mit Recht nur dann erklärlich findet, wenn er wirklich als Weltmittelpunkt galt. Gruppe verweist noch auf Liebrecht zu Gerv. v. Tilbury Ot. imper. S. 54. S. uns. Taf. 9, 4.} S. uns. Taf. 9, 3.

Ein hervorragender Talmudist in Berlin-Friedenau Dr. M. I. Berdyczewski, den ich um briefliche Auskunft über die Vorstellungen der späteren Israeliten vom Erdnabel ersucht hatte, schrieb mir am 3./7 1912 unter anderem: `Das ganze rabbinische und mystische Schrifttum ist von diesem Gedanken geradezu durchtränkt. Nicht nur ist Jerusalem, bezugsweise der Ort, wo der Tempel stand, der Mittelpunkt oder der Nabel der Welt, sondern von dieser Stelle ist die ganze Weltschöpfung ausgegangen. Nun aber sind diese Schriften durchaus synoptisch, sie haben von jeder Stelle unzählige Parallelen und Varianten. Dazu kommen noch die Nebenströmungen: die israelitische Tendenz, Beth-El\footnote{Vgl. dazu auch A. Jeremias, D. Alte Test. im Lichte d. alt. Or. 2 S. 190 u. S. 374 f. Anm. 4. --- Guthe a. a. O. S. 561, 45 ff. weist darauf hin, dass Palästina [u. Jerusalem] tatsächlich zwischen Babylonien und Ägypten, also den wichtigsten Kulturländern der ältesten Zeit, genau in der Mitte lag. Er hält es sogar für möglich, dass diese Vorstellung schon bei den Kanaanitern vorhanden war, aber in Israel erhöhte Bedeutung damit gewann, dass sich das Volk wegen seiner höheren Gottesverehrung zum Lehrer aller andern Völker berufen fühlte (Jes. 45, 14, 21 ff. 51, 4 f. 2, 1-4).} in den Mittelpunkt der Welt zu stellen, wie wiederum eine andere, den Berg Moria, auf dem Isaak geopfert werden sollte, als solchen anzusehen; daneben die Bestrebungen der Samaritaner, den Berg Garizim an Stelle des Zionsberges zu setzen. Kurz es ist ein förmliches Ringen zwischen diesen Meinungen durch alle Zeiten.' Man erkennt aus diesen Äußerungen und Darlegungen ganz deutlich, wie dankenswert eine erschöpfende historisch-kritische Darstellung aller einschlägigen Vorstellungen der Israeliten sein würde. Hoffentlich trägt diese meine Arbeit dazu bei, die baldige Erfüllung des hiermit ausgesprochenen Wunsches anzuregen. Zum Schluss mache ich noch auf folgende interessante Stelle bei Gervasius v. Tilbury (Otia imper. ed. Liebrecht S. 1; vgl. S. 54) aufmerksam: `Majores nostri civitatem sanctam Jerusalem in medio nostrae habitabilis [οἰκουμένης] sitam scripserunt secundum illud: "`Operatus est salutem in medio terrae..."' Hoc autem circumferentiae centrum arbitrantur quidam in illo loco esse, ubi Dominus locutus est ad Samaritanam ad puteum [also beim Berge Garizim: s. ob.]; illic enim in solstitio aestivo meridiana hora sol recto tramite descendit in aquam putei, umbram nullam aliqua parte monstrans, quod apud Syenen fieri tradunt philosophi.'\footnote{Vielleicht dient diese Notiz zum Verständnis der oben mitgeteilten Angabe, dass die Chinesen die Mitte der Erde durch die Sonnenuhr festgestellt hätten.} Hierzu bemerkt Liebrecht (S. 54): `Über den Glauben, dass Jerusalem in der Mitte der Erde liege, s. Cellarius, Not. Orb. Ant. l. 1 c. 4; cf. Felix Faber 1, 306 ff. (nach welchen der eigentliche Mittelpunkt ein Stein in der Kirche auf Golgatha sein soll [s. ob. S. 26]. Auch die Mohamedaner hegen jenen Glauben; s. d'Herbelot s. v. Scheith. Gleiche Vorstellungen in bezug auf ihre heilig gehaltenen Orte finden sich noch bei anderen Völkern: so in betreff Asgards bei den alten Skandinaviern,\footnote{Vgl. E. H. Meyer, Germ. Mythologie § 250 S. 187 f.: `Ásgađr liegt nach E. 2, 258 U \emph{i mi{\thorn}jum heimi} d. h. mitten in der Welt oder nach einem Zusatz anderer Hss. [E. 1, 54] mitten auf der Erde.' S. auch Grimm, Deutsche Mythol. 3 S. 778.} des Berges Meru (Himalaya) bei den Hindus [s. ob. S. 22], des Berges Bordsch (Kaukasus) bei den Parsen (s. Lex. Myth. p. 285. 313), des Berges Righiel bei den Tibetanern, Delphis (daher ὀμφαλός genannt, s. Kap. 4) bei den Griechen usw. S. auch Temme, Volkss. d. Altmark S. 33. Kuhn und Schwartz No. 244.' Ich muss es andern überlassen, den von Liebrecht hier gegebenen Anregungen in bezug auf die Anschauungen der Parsen und Tibetaner weiter nachzugehen. Ich selbst bin leider nicht dazu imstande.

7. Die Araber und Perser kennen ebenfalls die Vorstellungen vom Nabel als dem Symbol der Mitte und vom Nabel der Welt. So nennt \d{T}abarī, Annales (ed. Leiden) 1 p. 1068 Mittelsyrien und Palästina surrat al-Scham, d. i. den Nabel Syrien.\footnote{`Osmanische Schriftsteller nennen den dritten der Hügel der Hauptstadt Konstantinopel den Nabel der Stadt; Hammer, Gesch. d. osman. Reiches 2 370. 628. Fundgruben d. Orients 2 47 ff. Journal of Royal Asiat. Society 1830 2 57. Erdmann, Schöne vom Schlosse 11 Anm. 3.' [Goldziher].} --- Ferner sagt Ja'\d{k}ūbī, Kitāb al-boldan (Bibl. Geogr. Arab. Bd. 7 p. 233, 19 ff.), dass seine Beschreibung "`deswegen mit dem `Irā\d{k} beginne, weil es die Mitte der Welt und der Nabel (surra) der Erde ist; Baghdad wieder ist die Mitte vom `Irā\d{k}."' --- Und nach Mu\d{k}addasī, Descriptio imperii moslemici (Bibl. Geogr. Arab. Bd. 3 p. 445, 13 "`befindet sich eine halbe Parasange von Kāzerūn [in Schiras] eine Kuppel, von der sie sagen, dass dort die Mitte der Welt sei."'\footnote{Mitteilungen I. Goldzihers.}

Meine Frage, ob nicht die Ka'ba in Mekka den Mohamedanern als Nabel der Erde gelte, hat Goldziher mit einem entschiedenen `Nein' beantwortet, weil er ein entsprechendes Zeugnis nicht gefunden habe. Wäre es der Fall gewesen, meint G. gewiss mit Recht, so würden islamische Schriftsteller auch die anderen von ihnen besprochenen Traditionen nicht ohne entsprechende Bemerkung gelassen haben.\footnote{Vielmehr sah Mohammed ursprünglich Jerusalem als Zentrum der Erde an. Seine Luftreise dorthin soll so viel als Himmelfahrt, Besuch des obersten Himmels, bedeuten: v. Landau, Mitteil. d. Vorderasiat. Gesellsch. 1904 S. 57. A. Jeremias a. a. O. S. 584.}

8. Die Phönizier. Auch bei den Phöniziern muss der Gedanke eines in ihrem Gebiete befindlichen Nabels der Erde verbreitet gewesen sein. Wir schließen das mit vollster Sicherheit aus der Glosse Hesychs γῆς ὀμφαλός· ἡ Πάφος καὶ Δελφοί, aus der hervorgeht, dass auch Paphos auf der von Phöniziern besiedelten Insel Kypros, bekanntlich neben Amathus der Hauptkultort der phönizischen Aphrodite (= Astarte, die seit ältester Zeit von ihrer Heimat Παφία und Κύπρις, Κυπρογενής, Κυπρογένεια\footnote{Vgl. die Nachweise bei Preller-Robert, Griech. Mythol. 4 1 346, 1 u. 2.} hieß), als Nabel der Erde angesehen wurde. Von der paphischen Venus berichtet aber Tacitus Hist. 2, 2: "`Illum [Titum] cupido incessit adeundi visendique templum Paphiae Veneris, inclytum per indigenas advenasque... Conditorem templi regem Aërian vetus memoria, quidam ipsius Deae nomen id perhibent. Fama recentior tradit a Cinyra sacratum templum, Deamque ipsam, conceptam mari huc adpulsam. ...Simulacrum Deae non effigie humana: continuus orbis latiore initio tenuem in ambitum, metae modo exsurgens."' Ähnlich schildert Serv. z. Verg. Aen. 1, 720 das Bild der Göttin, indem er sagt: Apud Cyprios Venus in modum umbilici [= ὀμφαλοῦ], vel, ut quidam volunt, metae colitur.\footnote{Vgl. auch Maxim. Tyr. diss. 8, 8: τὸ ἄγαλμα οὐκ ἂν εἰκάσαις ἄλλῳ τῳ ἢ πυραμίδι λευκῇ, ἡ δὲ ὕλη ἀγνοεῖται. Wenn der Stein wirklich weiße Farbe hatte, so kann er kaum ein Aerolith gewesen sein, der immer schwärzlich aussieht, oder man müsste annehmen, dass der schwärzliche Stein mit weißem Stuck oder weißer Farbe überzogen war.} Dieses Idol erscheint oft auf cyprischen Kaisermünzen; abgebildet im Catal. of the greek coins in the Brit. Mus. pl. 16 2 ff. 26, 3. 6 ff. p. 80 ff. pl. 17 4 ff.\footnote{S. auch Head, Hist. num. 2 S. 741.} G. Franc. Hill, der Herausgeber dieser Abteilung der griechischen Münzen im Brit. Museum (p. 80 Anm. 4), nimmt im Hinblick auf die Glosse des Hesychius mit einer gewissen Wahrscheinlichkeit an, dass der Omphalos zu Paphos mit dem Steinkegel der dort verehrten Aphrodite identisch sei. Sehr merkwürdig ist, dass in Paphos mit dem Kult der Aphrodite nicht bloß ein berühmtes Orakel verbunden war (Tac. a. a. O.), sondern auch ein bedeutender Apollotempel bestand mit einer Statue, die diesen Gott auf einem bienenkorbförmigen, basislosen, mit Netzwerk versehenen Omphalos darstellte (vgl. a. a. O. p. 79 f. nebst pl. 22 10 u. 11). Die Beschreibung lautet: "`On the obverse is a head of Aphrodite, wearing a tall stephanos... Behind the head are the letters $\svgAAA$. On the reverse is Apollo seated on the omphalos, holding arrow and bow; in field l., in two straight lines: $\svgAAB$."'\footnote{Babelon, Rois de Syrie p. 47 vermutet wohl richtig, dass dieser Apoll auf dem Omphalos eine Statue voraussetzt, welche mit andern Bildwerken unter den cyprischen Kunstwerken sich befand, von denen Libanios berichtet, sie seien `by trickery' nach Antiochien übergeführt worden. In der Tat erscheint ziemlich derselbe Apollontypus auf den Münzen der Seleuciden (Catal. Brit. Mus. Seleucid kings Pl. 3, 3. 4. 5. 6. 7 u. pag. 19). S. Taf. 1, 13. Der Typus der Seleucidenmünzen ging dann über auf die Münzen der Parthischen Könige: Head, Hist. num. 2 S. 818.} Dass der Omphalos, auf dem der paphische Apollon sitzt, mit der `meta' oder dem `umbilicus' der dortigen Venus identisch sein könne, verbietet freilich schon die sehr verschiedene Form der beiden Kultobjekte; dagegen ist es allerdings nicht nur möglich, sondern sogar wahrscheinlich, dass der Omphalos des paphischen Apollon nicht der delphische, sondern der paphische sein soll, da ja, wie Hesychius bezeugt, Paphos ebenso wie Delphi sich rühmte, der Nabel der Erde und zugleich im Besitze eines bedeutenden Orakels zu sein.\footnote{Möglicherweise bezieht sich auf ganz ähnliche Vorstellungen das phönikisch-karthagische Relief, abgebildet in Memnon 3 Taf. 3 Fig. 30. Vgl. unten Kap. 6.}

\begin{figure}[H]
\centering
\includegraphics[width=0.55\textwidth,keepaspectratio]{figs/fig01.jpg}
\caption*{\frakfamily Münze von Kypros nach Roschers Mythol. Lex. 1 Sp. 747).}
\end{figure}
\paragraph{}
9. Die Ägypter. Um zu erfahren, ob auch dieses uralte Kulturvolk in älterer Zeit die Vorstellung von einem Nabel des Landes oder der Erde gekannt hat, habe ich mich an Dr. Günther Roeder in Breslau, einen ganz hervorragenden jüngeren Ägyptologen, mit verschiedenen Fragen gewandt, aber leider von diesem die Antwort erhalten, dass ihm nichts über die in meinen Fragen berührten Punkte bekannt sei. Vielleicht gewähren künftige Funde und Ausgrabungen im Nillande die Möglichkeit, dass für Ägypten, die Heimat der ältesten Geometer, nicht unwichtige Problem durch den Nachweis ausdrücklicher Zeugnisse aus älterer Zeit zu lösen. Dass die hiermit ausgesprochene Hoffnung nicht unbegründet ist, scheint eine interessante Stelle des aus dem späteren Ägypten stammenden hermetischen Traktats Κόρη κόσμου (= Stob. Ecl. 1 p. 302 ed. Meineke) zu bestätigen. Hier antwortet Isis auf die Frage ihres Sohnes Horos auf dessen Frage: Διὰ τίνα αἰτίαν οἱ ἔξω τῆς ἱερωτάτης ἡμῶν χώρας ἄνθρωποι ταῖς διανοίαις οὐκ ὄντως εἰσὶ συνετοὶ ὡς οἱ ἡμέτεροι; --- Ἡ γῆ μέσον τοῦ παντὸς ὑπτία κεῖται, καὶ κεῖται ὥσπερ ἄνθρωπος οὐρανὸν βλέπουσα, μεμερισμένη δὲ καὶ ὅσα μέλη ὁ ἄνθρωπος μελίζεται· ἐμβλέπει δ᾽ οὐρανῷ καθάπερ πατρὶ ἰδίῳ, ὅπως ταῖς ἐκείνου μεταβολαῖς καὶ αὐτὴ τὰ ἴδια συμμεταβάλλῃ. καὶ πρὸς μὲν τῷ νότῳ τοῦ παντὸς κειμένην ἔχει τὴν κεφαλήν [1], πρὸς δὲ τῷ ἀπηλιώτῃ τὸν δεξιὸν ὦμον [2], <πρὸς δὲ τῷ λιβὶ τὸν εὐώνυμον> [3], ὑπὸ τῆς ἄρκτου τοὺς πόδας, <τὸν μὲν δεξιὸν ὑπὸ τὴν οὐράν> [4], τὸν δὲ εὐώνυμον ὑπὸ τὴν κεφαλὴν τῆς ἄρκτου [5], τοὺς δὲ μηροὺς ἐν τοῖς μετὰ τὸν ἄρκτον [6], τὰ δὲ μέσα ἐν τοῖς μέσοις [7],... ἐπειδὴ δὲ ἐν τῷ μέσῳ τῆς γῆς κεῖται ἡ τῶν προγόνων ἡμῖν ἱερωτάτη χώρα [= Αἴγυπτος], τὸ δὲ μέσον τοῦ ἀνθρωπίνου σώματος μόνης τῆς καρδίας ἐστὶ σηκός, τῆς δὲ φυχῆς δρμητήριόν ἐστι καρδία, παρὰ ταύτην τὴν αἰτίαν, ὦ τέκνον, οἱ ἐνταῦθα ἄνθρωποι τὰ μὲν ἄλλα ἔχουσιν οὐχ ἧττον ὅσα καὶ πάντες, ἐξαίρετον δὲ τῶν πάντων νοερώτεροί εἰσι καὶ σώφρονες, ὡς ἂν ἐπὶ καρδίας γεννώμενοι καὶ τραφέντες. Illustriert werden diese höchst wahrscheinlich auf einer altägyptischen Vorstellung beruhenden Sätze, wie Fr. Boll (Die Lebensalter S. 50 f.) erkannt hat, einerseits durch das beistehend abgebildete altägyptische (bereits von mir zum Verständnis der in Kap. 11 der hippokratischen Schrift von der Siebenzahl erhaltenen siebenteiligen Weltkarte herangezogene) Gemälde, anderseits durch diese Weltkarte selbst, die so nahe verwandte, ja fast identische Anschauungen enthält, dass auch Boll a. a. O. in verstärktem Maße dazu neigt, mit mir an Anregung des Hippokrateers durch eine ägyptische Vorstellung zu glauben.\footnote{Man beachte vor allem, dass nicht bloß die Siebenteilung der Erde (des Makrokosmos) und des menschlichen Körpers (des Mikrokosmos) in π. ἑβδομ. dieselbe ist wie in der Κόρη κόσμου, sondern dass auch die einzelnen Körperteile hier wie dort so ziemlich identisch sind. Nur ist zu beachten, dass statt der χεῖρες in π. ἑβδ. hier die ὦμοι, statt der φρένες (praecordia) dort hier die καρδία als der mittelste Teil und zugleich als Sitz der Seele und Denkkraft genannt werden, und dass natürlich bei dem Ägypter dessen Vaterland genau dieselbe Rolle spielen musste, die für den Ionier (Milesier) Ionien (als Sitz höchster Kultur!) spielt. Vgl. Roscher, Über Alter, Ursprung u. Bedeutg. d. hippokrat. Schrift von d. Siebenzahl S. 12 Anm. 14. Derselbe, Die neuentdeckte Schrift e. altmiles. Naturphilosophen usw. S. 6 nebst Taf. 2 Fig. 1 u. 2. Derselbe, D. hippokrat. Schr. von d. Siebenzahl in ihrer vierfachen Überlieferung z. ersten Mal herausgeg. von W. H. R. Paderborn 1913 S. 107 ff., wo nicht weniger als 8 hebdomadische Listen der wesentlichsten Körperteile mitgeteilt sind, die mit der Liste in der Schrift Κόρη κόσμου nahe Verwandtschaft verraten.}

\begin{figure}[H]
\centering
\includegraphics[width=0.55\textwidth,keepaspectratio]{figs/fig02.jpg}
\caption*{\frakfamily Zwei Himmelsgöttinnen und Erdgott der Ägypter. Nach Brugsch, Religion u. Mythologie der alten Ägypter S. 211. Man beachte, dass hier der Nabel aller drei Figuren genau in der Mitte liegt.}
\end{figure}
\paragraph{}
10. Die Griechen. Da über die beiden Hauptorte, die sich im Bereiche Griechenlands rühmten, im Besitze des ὀμφαλὸς γῆς zu sein, nämlich Branchidai und Delphi, in zwei größeren Sonderabschnitten (Kap. 3 u. 4) gehandelt werden wird, so wollen wir hier nur ganz kurz diejenigen Städte aufzählen, die beanspruchten das Zentrum (ὀμφαλός) eines mehr oder weniger umfassenden Teiles von Hellas zu sein.

Hier ist vor allen zu nennen die Stadt Phlius, von der Pausanias (2, 13, 7) berichtet: οὐ πόρρω δέ ἐστιν [nämlich unweit des οἶκος μαντικός des Amphiaraos] ὁ καλούμενος Ὀμφαλός, Πελοποννήσου δὲ πάσης μέσον, εἰ δὴ τὰ ὄντα εἰρήκασιν. Bursian (Geogr. v. Griechenland 2, S. 34) behauptet von diesem Nabelstein, den `die Fremdenführer von Phlius mit naiver Unverschämtheit als den Mittelpunkt der ganzen Halbinsel zeigten, er sei ursprünglich jedenfalls das Symbol irgendeiner nicht anthropomorphisch dargestellten Gottheit gewesen.' Unmöglich ist diese Annahme B.s an sich natürlich nicht, doch ist daneben, namentlich wenn man dabei die gleich zu behandelnde Analogie der Omphaloi von Athen und Antiochia und des umbilicus urbis Romae in Betracht zieht, auch die Vermutung nicht von der Hand zu weisen, dass der phliasische Nabelstein ursprünglich zwar nicht den Mittelpunkt der Peloponnes, wohl aber der Stadt Phlius und ihres Gebietes und Wegenetzes bezeichnen sollte, wofür auch der Umstand sprechen dürfte, dass er nach Pausanias ganz in der Nahe des ältesten und angesehensten Tempels in der Unterstadt und der Agora stand.\footnote{Auf den Omphalos von Phlius hat man auch das Bild älterer phliasischer Münzen beziehen wollen, das ein vierspeichiges Rad mit einem Punkt in der Mitte darstellt. Man vergleicht dieses Rad mit einem gewissen Typus älterer Münzen von Delphi, der jüngst als φιάλη, früher als `orbis terrarum' mit dem Omphalos in der Mitte gedeutet wurde. Vgl. Catal. of gr. coins in the Brit. Mus. Peloponnesus S. 33 und unten unter Delphi (Kap. 4).}

Dass auch Athen einen ὀμφαλὸς ἄστεος besaß, erfahren wir aus dem schönen Fragmente eines von Pindar für die Athener gedichteten Dithyrambos (= fr. 45 Boeckh = 53 Bergk):
\begin{quotation}
Ἴδετ᾽ ἐν χορόν, Ὀλύμπιοι,

ἔπι τε κλυτὰν πέμπετε χάριν, θεοί,

πολύβατον οἵτ᾽ ἄστεος ὀμφαλὸν θυόεντα

ἐν ταῖς ἰεραῖς Ἀθάναις,

οἰχνεῖτε πανδαίδαλόν τ᾽ εὐκλέ᾽ ἀγοράν κ. τ. λ.
\end{quotation}
\paragraph{}
Boeckh (z. d. St.: 2 p. 576) dachte an die Tholos als Mittelpunkt der Stadt, während Dissen in seiner Ausgabe Pindars (2, p. 617) vielmehr für den Zwölfgötteraltar eintrat. Jetzt ist wohl allgemein die Erklärung Dissens durchgedrungen; denn der von den Peisistratiden gegen Ende des 6. Jahrhunderts gestiftete Altar der Zwölfgötter lag nicht bloß so ziemlich in der Mitte der Stadt und konnte also als deren `Nabel' bezeichnet werden, sondern diente auch als religiöser Mittelpunkt und Asyl, vor allem aber als Zentralmeilenstein für das gesamte Straßennetz Athens und Attikas.\footnote{Herod. 2, 7: ἔστι δὲ ὁδὸς ἐς Ἡλίου πόλιν ἀπὸ θαλάσσης ἄνω ἰόντι παραπλησίη τὸ μῆκος τῇ ἐξ Ἀθηνέων ὁδῷ τῇ ἀπὸ τῶν δυώδεκα θεῶν τοῦ βωμοῦ φερούσῃ ἔς τε Πῖσαν καὶ ἐπὶ τὸν νηὸν τοῦ Διὸς τοῦ Ὀλυμπίου. Vgl. dazu auch d. Inschr. C. I. Att. 1 522. Plat. Hipparch. 228 D. ff. Mehr bei Milchhöfer b. Baumeister, Denkm. 1, p. 165 a. Iudeich, Topogr. v. Athen, S. 60 u. 312 Anm. 25. Wachsmuth, Athen 1, S. 525. Curtius, Gesch. d. Wegebaues, S. 39. Es fragt sich übrigens, ob dieser Altar rund oder viereckig war. Ich möchte eher runde als viereckige Form vermuten. --- Nach Boeckh a. a. O. und im Prooem. catal. lect. Univ. Berol. aest. a. 1818 hatte auch Simonides von einem ὀμφαλὸς πόληος gesprochen, welche Bezeichnung Boeckh auf das athenische Buleuterion beziehen möchte. Es ist mir aber bis jetzt nicht gelungen, das betreffende Bruchstück aufzufinden.}

Wahrscheinlich nach dem Vorbild Athens erhielt die gewaltige in der Diadochenzeit gegründete Hauptstadt Syriens, nämlich Antiochia ad Orontem, einen das Zentrum der Stadt bezeichnenden Omphalos, nach dem auch der ganze ihn umgebende Platz so genannt wurde. Vgl. Malalas Chronogr. 10, p. 233 Dind. ὅστις τόπος κέκληται ὁ ὀμφαλὸς τῆς πόλεως, ἔχων καὶ τύπον ἐγγεγλυμμένον ἐν λίθῳ ὀμφαλοῦ.\footnote{Hier bieten zwar die Hss. ὀφθαλμοῦ, aber es scheint mir ganz evident, dass hier, wie auch sonst öfters, ὀμφαλός und ὀφθαλμός verwechselt sind (s. Geopon. 2, 39, 8 u. 17, 2, 1 u. Niclas z. d. St.; Pierson ad Moer. p. 178; unt. S. 59). Bötticher freilich, D. Omphalos zu Delphi, 19. Winckelmannsprogr. Berlin 1859 S. 6, will ὀφθαλμοῦ zu halten suchen, indem er annimmt, dass der Stein in der Mitte der Stadt `das allsehende Auge der göttlichen Providenz' [!] bedeuten solle.} O. Müller u. Wieseler (Denkm. a. Kunst 1 Taf. 49 Nr. 220 i) vermuten, dass der dem Apollon als Sitz dienende Omphalos auf Silbermünzen des Königs Antiochos 2 mit dem `die Mitte Antiochiens bezeichnenden Omphalosstein' identisch sei.\footnote{Vgl. Otfr. Müller, Antiquit. Antiochen. Baumeister, Denkm. S. 953, Fig. 1108 ff. Overbeck, Kunstmythol. Apollon, Münztaf. 3 19. 41. 42. S. 300 f. Catal. of gr. coins in the Brit. Mus. Seleucid kings S. 114. Cyprus Introd. p. 80.}

11. Die Italiker. In Sizilien galt Enna (Henna), jetzt Castrogiovanni, die tatsächlich im Zentrum der Insel auf einer beträchtlichen Anhöhe gelegene uralte sikulische Stadt, als ὀμφαλός oder `umbilicus' der ganzen Insel. Vgl. Kallim. hymn. in Cer. 15: τρὶς δ᾽ ἐπὶ καλλίστης νήσου δράμες ὀμφαλὸν Ἔνναν. Diod. 5, 2: δοκεῖ δ᾽ ἐν μέσῳ κεῖσθαι τῆς ὅλης νήσου, διὸ καὶ Σικελίας ὀμφαλός ὑπό τινων προσαγορεύεται. Cicero in Verr. 3, 192: Henna mediterranea est maxime. 4, 106: qui locus, quod in media est insula situs, umbilicus Siciliae nominatur.\footnote{S. auch Rossbach, Castrogiovanni. Leipz. 1912, S. 4 f. u. S. 14 f.}

Dieselbe Rolle spielte im eigentlichen Italien der See von Cutilia im Sabinischen Gebiete nach Varro bei Plinius n. h. 3, 109 (In agro Reatino Cutiliae lacum, in quo fluctuetur insula, Italiae umbilicum esse M. Varro tradit).

In Rom gab es bekanntlich zwei inmitten der Stadt auf dem Forum Romanum ganz in der Nähe der Rednerbühne und des Concordiatempels errichtete Monumente, deren Bedeutung einerseits an den Nabelstein von Delphi, anderseits an den Zwölfgötteraltar von Athen erinnerte: den von Constantin erbauten Umbilicus urbis Romae, und das Milliarium aureum,\footnote{Suet. Oth. 6. Tac. h. 1, 27. Plin. h. n. 3, 66.} dass Augustus im Jahre 28 v. Chr. errichtete. Ersterer galt als ideeller Mittelpunkt der Stadt und des Reiches, letzteres als Generalmeilenzeiger oder als Zentrum des gesamten römischen Straßennetzes (vgl. O. Richter b. Baumeister Denkm. S. 1464 u. Dens., Rekonstruktion u. Gesch. d. röm. Rednerbühne).

12. Die Magyaren. Der Freundlichkeit Goldzihers in Budapest verdanke ich auch die Notiz, dass nicht weit von Tata (Totis) im Komitate Komárom in Ungarn ein kleiner Ort namens Naszály existiert. Die Bewohner dieser Ortschaft haben die Überlieferung, dass dort der Mittelpunkt der Erde (Scherz?) sei. Dies hat G. von einer in Tata heimischen Verwandten gehört.

13. Die Peruaner. An letzter Stelle gedenke ich noch der von I. G. Müller, Gesch. der amerikan. Urreligionen S. 304 berichteten Tatsache, dass die Peruaner ihre Hauptstadt, die nach ihrem Kulturmythus ihre erste feste Ansiedlung war, Cuzco, d. i. `Nabel,' nannten (Mitteilung Goldzihers).

Versuchen wir nunmehr das Fazit aus den vorstehenden zwar etwas langweiligen und monotonen, aber doch für unsern Zweck ganz notwendigen Darlegungen zu ziehen, so ist folgendes zu sagen.

a. Wir finden die Idee eines Mittelpunkts (Nabels) der Erde bei den verschiedensten Völkern, und zwar meist seit uralter Zeit.

b. Diese Idee ist immer mit der Vorstellung der Erde als einer horizontalen Fläche verbunden, die fast überall (mit Ausnahme Chinas) als kreisrund (`orbis terrarum') aufgefasst wird.

c. Sehr oft wird ein Berg (Hügel) als `Nabel' der Erde betrachtet.

d. Nicht selten gilt der Erdnabel auch zugleich als Zentrum oder Nabel eines bestimmten Landes (z. B. Chinas, Palästinas, Iraks usw.). Es ist zu vermuten, dass die letztere Bedeutung in der Regel primären, die andere sekundären Charakter hat.

e. Höchst wahrscheinlich hat es außer den bisher bekannten, von uns aufgezählten Nabeln der Erde noch zahlreiche andere gegeben, von denen entweder gar nichts,\footnote{So ist es mir nicht unwahrscheinlich, dass es, wie in Athen, Rom, Antiochia, auch in anderen `Weltstädten,' wie in Konstantinopel und Alexandria, ὀμφαλοί gegeben hat, die zugleich als Mittelpunkte der Städte und Generalmeilenzeiger galten und nahe daran waren, als ὀμφαλοὶ γῆς angesehen zu werden.} oder nur so Weniges und Unbestimmtes überliefert ist, dass wir über Vermutungen nicht hinausgelangen können. Wir werden später sehen, dass namentlich die Griechen außer Delphi und Branchidai vielleicht noch mehrere andere apollinische Orakel mit dem Erdnabel identifiziert haben, z. B. Delos,\footnote{S. auch ob. Anm. 14 u. unt. Anm. 74.} Gryneion, Klaros und Patara.

f. Die Vorstellung des Erdmittelpunkts scheint nicht, wie die `Panbabylonisten' annehmen dürften, in Babylonien entstanden zu sein und sich von dort aus über die ganze Erde (bis nach Celebes und Peru) verbreitet zu haben, vielmehr hat wohl jeder Leser der vorstehenden Darlegungen den bestimmten Eindruck gewonnen, dass es sich auch hier um ganz spontane, an verschiedenen Orten, vielfach unabhängig voneinander, entstandene Anschauungen handelt. Doch will ich natürlich damit nicht in Abrede stellen, dass einzelne Orte, wie z. B. Branchidai, Delphi, Athen, andere, z. B. Rom, in dieser Hinsicht beeinflusst haben können.
\clearpage
\section{Branchidai (Didyma) und sein Orakel als Nabel der Erde.}
\paragraph{}
Bereits in der Monographie `Die neuentdeckte Schrift eines altmilesischen Naturphilosophen und ihre Beurteilung durch H. Diels in der D. Lit. Ztg. 1911 Nr. 30' [Stuttg. 1912] S. 28 f. habe ich dem berühmten Forscher gegenüber nachzuweisen versucht, dass im 6. und 7. vorchristlichen Jahrhundert, d. h. vor der Zerstörung Milets und seines Didymaions durch die Perser im J. 494, das uralte Apollonorakel zu Branchidai, bis zu dem angegebenen Termin unstreitig der erfolgreichste Konkurrent des delphischen, ebenso wie dieses den Anspruch erhoben habe, der Omphalos, d. h. das Zentrum der damals den Griechen bekannten Erde, zu sein. Die a. a. O. von mir angeführten Beweismomente bedürfen jetzt infolge erweiterter Studien einerseits der Korrektur andererseits der Ergünzung.\footnote{Eine solche Ergänzung liefert auch mein im Philologus 70 (N. F. 24 1911) S. 529 ff. erschienener Aufsatz über "`Das Alter der Weltkarte in `Hippokrates' περὶ ἕβδομάδων und die Reichskarte des Darius Hystaspis."' Ich habe darin vor allem bewiesen, dass der am Grabe des Darius 1 bildlich und inschriftlich angebrachten Länder- und Völkerliste des Perserreichs ein wesentlich jüngeres Weltbild zugrunde liegt als der heptadischen Weltkarte des Kosmologen in `Hipp.' Schrift von der Siebenzahl. Weitere Ergänzungen s. in meiner soeben erscheinenden Ausgabe der hippokrat. Schr. von d. Siebenzahl. Paderborn 1913 (passim!).} Für unrichtig muss ich es jetzt vor allem erklären, wenn ich S. 28 auf Grund des trefflichen Artikels `Didyma' von Bürchner bei Pauly-Wissowa die durch neuere Ausgrabungen bestätigte einstige Existenz eines `Omphalos' im Didymaion behauptet habe. Bürchner, den ich brieflich um genauere Auskunft hierüber gebeten habe, hatte die Freundlichkeit, mir Folgendes mitzuteilen:

`Auf der die Rekonstruktion des Apollontempels zu Didyma darstellenden Tafel 37 des Werkes von Rayet et Thomas, Milet et le Golfe Latmique, sowie auf Taf. 35 und 38 ist deutlich der Omphalos an einem Erdspalt, umgeben von Gebüsch, dargestellt, und im Text p. 64 heißt es ausdrücklich: "`Force est donc de voir l'adyton dans le naos et de placer au milieu le χάσμα, l'omphalos et l'arbre sacré."' Aber ein paar Seiten vorher (p. 62) wird zugegeben: "`Nous nous expliquons d'une manière très naturelle et très claire pourquoi le sol du naos est si bas à Didymes, et pourquoi nous n'avons trouvé dans les fouilles, au milieu du naos, aucune trace de dallage, quoique nous soyons descendu plusieurs pieds au dessous de bas du soubassement. C'est que le milieu du naos était un trou et qu'autour de ce trou, le sol naturel était à nu. On voyait là dans un éspace restreint, la fissure où était la source sainte, l'omphalos, les lauriers sacrés, par derrière la statue colossale du Dieu, dominant toute la scène. C'est cet aspect qui lui a été donné dans la restauration, bien entendu d'ailleurs que la forme du χάσμα et la position exacte de l'omphalos et de l'arbre sacré par rapport au χάσμα sont conjecturales."' Auch eine weitere briefliche Anfrage bei Th. Wiegand, dem hochverdienten Leiter der neuesten (deutschen) Ausgrabungen in und bei Milet, ergab leider nur ein negatives Resultat. Wiegand hatte die Güte, mir am 29/5 1912 zu schreiben:

"`Spuren eines `Omphalos' haben sich bei den Französischen Grabungen im Innern des Didymaions, wo man die Orakelquelle vermuten darf, nicht gefunden. Über die Einrichtung der Quelle haben die neueren Grabungen der Berliner Museen so viel ergeben:

a. Die Quelle befand sich im hinteren, westlichen Drittel des Adyton. Unter Adyton verstehe ich den großen unbedeckten westlichen Hauptraum des Tempels.\footnote{Vgl. einstweilen den Plan des Didymaions (nach den französ. Ausgrabungen!) im Dictionn. d. antiquités 4, 1 p. 217 [Roscher].}

b. Die Stelle, wo die Quelle floss, war mit einer besonderen, sehr feinen Gebäudearchitektur überbaut. Von dieser haben sich eine größere Anzahl Marmorfragmente bisher gefunden. Die Stelle der Quelle selbst ist von den Grabungen noch nicht erreicht. Das wird aber in diesem Herbst [1912] der Fall sein.

c. Dies Quellgebäude wurde bei Einführung des Christentums abgerissen, seine Bruchstücke wurden in einer Kirche verbaut, die man im Adyton errichtete. Sollte sich im Herbst etwas über einen Omphalos an der Quelle ergeben, so werde ich mir erlauben, Sie sofort zu benachrichtigen. --- `Omphaloi' sind in Milet bisher nur in der Nekropole zu Tage gekommen. Einer davon ist mit einer großen Schlange umwunden."'\footnote{S. unten Taf. 6 Fig. 5. Ich verdanke die betreffende Photographie und 2 kleine erläuternde Skizzen der Güte Wiegands und Dr. Bruno Schröders.} ---

Trotz dieser, wie ich jetzt rückhaltlos zugebe, bisher durchaus negativen Ergebnisse der Ausgrabungen im Didymaion hinsichtlich des dort gesuchten Omphalos muss ich aber doch meine Behauptung, dass Milet und dessen Didymaion genau ebenso wie Delphi und dessen Haupttempel beansprucht habe, der Omphalos, d. h. das Zentrum der bewohnten Erde, ja des Weltalls zu sein, ganz bestimmt aufrechterhalten. Die für diese meine Annahme sprechenden Gründe sind folgende.

1. Die Bezeichnung Ioniens als `Zwerchfell' (φρένες = praecordia) der Erde, die sich in der von mir dem 6. Jahrhundert zugewiesenen Kosmologie des hippokratischen Buches von der Siebenzahl findet, beweist m. E. unwiderleglich, dass jener ionisch schreibende, nur hocharchaische Anschauungen verratende und die ganze ihm bekannte Welt lediglich vom Standpunkte eines Altmilesiers des 6. und 7. Jahrhunderts aus betrachtende Verfasser\footnote{Bereits haben mir zahlreiche hervorragende Forscher öffentlich oder privatim in der Hauptsache zugestimmt. Vgl. z. B. E. Drerup im Lit. Zentralbl. 1911 Sp. 1310 ff. 1912 Sp. 231. R. Fritzsche in Vierteljahrsschr. f. Philos. u. Soziologie 1912 S. 119 ff. S. R[einach] in Rev. archéol. 1911 2 p. 390. N. G. Politis, Λαογραφία 3 (1911) S. 336 ff. Pagel in Wochenschr. f. klass. Philol. 1911 nr. 42 Sp. 1137 ff. W. Nestle, Wochenschr f. kl. Philol. 1912 Sp. 901 ff.} Ionien zunächst als `Zwerchfell,' d. h. als Zentrum, der Oikumene betrachtet hat.\footnote{So versteht den Ausdruck `Zwerchfell' auch Diels a. a. O. Sp. 1863.} Aber der Ausdruck φρένες = praecordia bedeutet in diesem Falle noch mehr. Der Kosmologe des genannten Buches vertritt durchweg den Standpunkt, dass das Weltall als lebendiger Makrokosmos durchaus in derselben Weise wie der menschliche Körper als Mikrokosmos organisiert sei, und verlegt deshalb die Vernunft und Seele des Kosmos in die mittelste der sieben Sphären, d. h. die des Mondes, weil er den Sitz der Seele und Vernunft des Menschen ebenso wie Homer im Zwerchfell, d. h. in der Mitte des menschlichen Leibes, annimmt.\footnote{Wie der Nabel (ὀμφαλός) die äußere Mitte des menschlichen Leibes oder die Grenze zwischen oben und unten, so bezeichnet das Zwerchfell die innere. Vgl. `Hipp.' π. ἑβδ. 48 der lat. Übers.: Definitio (= ὅρος) ...superiorum partium et inferiorum corporis umbilicus. --- Ebenso bezeichnet nach Kap. 11 das Zwerchfell (= Ionien) die mittelste Zone der Erde. --- Galen 16 284 K: ὅρον... τίθησι [ὁ Ἱπποκρ. ἐν τῷ ἕκτῳ τ. ἐπιδημ.] τὸ ἧπαρ τῶν ἄνω τε καὶ κάτω τοῦ σώματος ἁπάντων μορίων· τὰ γὰρ κυρτὰ αὐτοῦ τῷ διαφράγματί ἐστι συνεζευγμένα. ὅσα οὖν ἐστιν ἄνω τοῦ ἥπατος ταῦτα καὶ ἄνω τοῦ διαφράγματός ἐστιν, ὡς τὰ κατὰ καρδίαν καὶ πνεύμονα κ. τ. λ. --- Daher werden in π. ἑβδ. Kap. 52 ὀμφαλός und φρένες beinahe wie Synonyma gebraucht: Ὅρος δὲ θανάτου, ἐὰν τὸ τῆς ψυχῆς θερμὸν ἐπανέλθῃ ὑπὲρ τοῦ ὀμφαλοῦ εἰς τὸν ἄνω τὸν φρενῶν τόπον καὶ συγκαυθῇ τὸ ὑγρὸν ἅπαν...} So besagt die Bezeichnung Ioniens als `Zwerchfell' (φρένες) der Erde nicht bloß, dass Ionien das genau in der Mitte zwischen dem kalten Norden (Skythien) und dem heißen Süden (Ägypten, Libyen) gelegene, durch das gemäßigtste Klima ausgezeichnete Land sei,\footnote{Herod. 1, 142: οἱ δὲ Ἴωνες οὗτοι... τοῦ μὲν οὐρανοῦ καὶ τῶν ὡρέων ἐν τῷ καλλίστῳ ἐτύγχανον ἱδρυσάμενοι πόλιας πάντων ἀνθρώπων τῶν ἡμεῖς ἴδμεν. `Hippocr.' π. ἀέρ. 1 p. 548 Kühn. --- π. νούσ. 4, 34 = V 546 L.: ἡ... Ἰωνίη χώρη... τοῦ ἡλίου καὶ τῶν ὡρέων οὐ κάκιστα κέεται. --- `Hippocr.' προγν. a. E. = I 119 K. betrachtet die ionische Insel Delos [= ὀμφαλὸς θαλάσσης? = ἱστίη νήσων, εὐέστιος Call. in Del. 325], die ungefähr auf demselben Breitengrade liegt wie Milet, als Mitte zwischen dem kalten Skythien und dem heißen Libyen. Vgl. Galen. 18, 2 p. 314 K. z. d. St: Λιβύης μὲν γὰρ ὡς θερμῆς, Σκυθίας δὲ ὡς ψυχρᾶς, Δήλου δ᾽ ὡς εὐκράτου καὶ μέσης ἀμφοῖν ἕνεκα παραδείγματος ἐμνημόνευσεν. --- Auf dieser mittleren Lage Ioniens beruht natürlich auch die die ganze milesische Naturphilosophie beherrschende Lehre vom θερμόν und ψυχρόν, deren entgegengesetzte Wirkung die Milesier in ihren nördlichen Kolonien ebenso wie in Naukratis kennen zu lernen so reichliche Gelegenheit hatten.} sondern auch, dass es als Mittelpunkt aller Kultur und Intelligenz angesehen werden müsse, was ja tatsächlich vom 8. bis zum Anfang des 5. Jahrhunderts, d. h. bis zur Eroberung Ioniens durch die Perser (vgl. Herod. 1, 170: Ἰωνίη διεφθάρη), der Fall war. Und als ein schlagender Beweis für die Richtigkeit dieser Annahme kann jetzt das Zeugnis Ps.-Galens angesehen werden, der (nach seinem jetzt von Bergsträßer aus dem Arabischen übersetzten Kommentar zu der Stelle) in seinem Exemplar von π. ἑβδ. hier einen Zusatz gelesen hat, der besagte, dass `die Bewohner Ioniens im höchsten Grade verständig, einsichtig und weise seien.' Wenn also unser Verfasser Ionien nicht als `Nabel' (= Omphalos) sondern als `Zwerchfell' der Welt bezeichnet hat, so wird ihn dazu sicherlich die Erwägung veranlasst haben, dass dem Ausdruck für Zwerchfell (φρένες) eine viel prägnantere Bedeutung (Zentrum und Seele = Vernunft) eignete als dem Worte ὀμφαλός, dass in diesem Falle nur die zentrale Lage Ioniens nicht aber zugleich dessen einzigartige kulturelle Bedeutung bezeichnet hätte.\footnote{Außerdem kommt hier noch der Umstand in Betracht, dass der altionische Verfasser der Hebdomadenlehre in `Hippocr.' π. ἑβδ. sich die Erde nicht mehr als flache Rundscheibe sondern bereits als Kugel (wie Pythagoras) vorstellt, weshalb bei ihm genau genommen kaum noch von einem `Nabel' sondern nur von einer `Achse' (ἄξων; s. unt. S. 41 ff.) der Erdkugel die Rede sein kann.}

2. Ein zweites kaum minder bedeutsames Zeugnis für dieselbe Sache lässt sich aus dem, was Jamblichos de myster. 3, 11 p. 127 Parthey über das Orakel von Branchidai bemerkt, gewinnen. Es heißt dort (im unmittelbaren Anschluss an die Schilderung der vielfach übereinstimmenden Orakelsitte zu Kolophon und Delphi)\footnote{S. einstweilen oben S. 5.}:

Καὶ μὴν ἥ γε ἐν Βραγχίδαις γυνὴ χρησμῳδός, εἴτε ῥάβδον ἔχουσα τὴν πρώτως ὑπὸ θεοῦ τινὸς παραδοθεῖσαν\footnote{Natürlich ist hier ein Lorberstab gemeint, wie ihn Apollon selbst und die in seinem Dienst stehenden Propheten, Dichter und Rhapsoden trugen. Vgl. Boetticher, D. Baumkultus d. Hellenen S. 350 ff.; Boeckh, Explan. Pind. Isthm. 3, 56 p. 506 und zahlreiche Bildwerke.} πληροῦται τῆς θείας αὐγῆς, εἴτε ἐπὶ ἄξονος καθημένη προλέγει τὸ μέλλον, εἴτε τοὺς πόδας ἢ κράσπεδόν τι τέγγουσα τῷ ὕδατι ἢ ἐκ τοῦ ὕδατος ἁτμιζομένη\footnote{Wir werden später zeigen, dass die rechts vom Omphalos in einer Grotte (Adyton) dem Apollon Kitharodos gegenüberstehende, eine Trinkschale zum Munde oder zur Nase führende Priesterin der lauter ionische (nicht delphische!) Vorstellungen enthaltenden `Apotheose Homers' des Archelaos von Priene nicht die Pythia sein kann, sondern wahrscheinlich die Priesterin von Branchidai oder einer anderen kleinasiatischen Kultstätte Apollons darstellt.} δέχεται τὸν θεόν, ἐξ ἁπάντων τούτων ἐπιτηδεία παρασκευαζομένη πρὸς τὴν ὑποδοχὴν ἔξωθεν αὐτοῦ μεταλαμβάνει. Δηλοῖ δὲ καὶ τὸ τῶν θυσιῶν πλῆθος καὶ ὁ θεσμὸς τῆς ὅλης ἁγιστείας καὶ ὅσα ἄλλα δρᾶται πρὸ τῆς χρησμῳδίας θεοπρεπῶς, τά τε λουτρὰ τῆς προφήτιδος καὶ ἡ τριῶν ὅλων ἡμερῶν ἀσιτία καὶ ἡ ἐν ἀδύτοις αὐτῆς διατριβὴ... --- καὶ γὰρ αὐτὰ πάντα παράκλησιν τοῦ θεοῦ ὥστε παραγενέσθαι καὶ παρουσίαν ἔξωθεν ἐπιδείκνυσιν ἐπίπνοιάν τε θαυμασίαν... καὶ ἐν αὐτῷ τῷ πνεύματι τῷ ἀπὸ τῆς πηγῆς ἀναφερομένῳ, ἕτερόν τινα πρεσβύτερον χωριστὸν ἀπὸ τοῦ τόπον θεὸν ἀποφαίνει τὸν αἴτιον καὶ τοῦ τόπου καὶ τῆς πηγῆς αὐτῆς καὶ τῆς μαντικῆς ὅλης.

Was bedeutet nun in dieser hochinteressanten Schilderung der Orakelsitte zu Branchidai der merkwürdige Ausdruck ἐπὶ ἄξονος καθημένη προλέγει τὸ μέλλον? Ich habe unter Berücksichtigung der oben unter 1. angeführten Gründe und der Tatsache, dass ἄξων (= πόλος, axis) bei Aristoteles und Plutarch die Erd- und Himmelsachse bezeichnet,\footnote{Ps.-Aristot. de mu. 2: Ὁ μὲν οὖν κόσμος ἐν κύκλῳ περιστρέφεται, καλοῦνται δ᾽ οὗτοι πόλοι· δι᾽ ὧν εἰ νοήσαιμεν ἐπεζευγμένην εὐθεῖαν, ἥν τινες ἄξονα καλοῦσι, διάμετρος ἔσται τοῦ κόσμου, μέσην μὲν ἔχουσα τὴν γῆν, τοὺς δὲ δύο πόλους πέρατα. --- Aristarch. Sam. b. Plut. de fac. in o. lun. 6: μένειν τὸν οὐρανὸν ὑποτιθέμενος, ἐξελίττεσθαι δὲ κατὰ λοξοῦ κύκλου τὴν γῆν, ἅμα καὶ περὶ τὸν αὑτῆς ἄξονα δινουμένην. --- Plut. Q. conv. 9, 14, 6, 6: Πλάτων ὡς ἀτράκτους καὶ ἠλακάτας, τοὺς ἄξονας... ὀνομάζει. --- Plat. Tim. 40 B: γῆν δὲ τροφὸν μὲν ἡμετέραν εἰλλομένην δὲ περὶ τὸν διὰ παντὸς πόλον τεταμένον, φύλακα καὶ δημιουργὸν νυκτός τε καὶ ἡμέρας ἐμηχανήσατο = Cic. Tim. 10, 32: Iam vero terram, altricem nostram, quae traiecto axe continetur, diei noctisque effectricem. --- Cic. Ac. pr. 2, 39 123: Hicetas Syracusius, ut ait Theophrastus, caelum, solem, lunam, stellas, supera denique omnia stare censet, neque praeter terram rem ullam in mundo moveri, quae eum circum axem se summa celeritate convertat et torqueat, eadem effici omnia, quasi stante terra caelum moveretur. --- De nat. deor. 1, 20, 52: nullo puncto temporis iutermisso versari circum axem caeli admirabili celeritate.} schon in meiner Abhandlung über `Die neuentdeckte Schrift eines altmilesischen Naturphilosophen' etc. S. 28 Anm. 4 die Ansicht ausgesprochen, dass hier unter ἄξων die Erdachse, deren Endpunkt gewissermaßen der ὀμφαλὸς γῆς war, zu verstehen sei.\footnote{Vgl. auch (oben Anm. 19) bei Varro de l. l. 7, 17 die Worte: itaque pingitur quae vocatur <ἀντί>χθων Πυθαγόρα, ut media caeli ac terrae linea ducatur...} Dabei verwies ich vor allem auf Nikanders Alexiph. 7, wo das im Mittelpunkt des Himmels (Zenith) befindliche Bärengestirn ἄρκτος ὀμφαλόεσσα genannt wird. Vgl. dazu auch Hesych. s. v. ὀμφαλόεσσα· ἡ Ἄρκτος, διὰ τὸ μέσον τὸν βόρειον πόλον περιέχειν und den Schol. a. a. O. ὀμφαλὸν δὲ καλεῖ τὸν βόρειον πόλον ὡς μεσαίτατον ἢ αὐτὴν τὴν ἄρκτον... ὁμφαλόεσσαν εἴρηκε διὰ τὸ μέσον τοῦ βορείον [πόλου] κεῖσθαι.

Diese meine Deutung steht nun freilich in ziemlich schroffem Gegensatz zu der bisherigen von anderen gegebenen. So sagt Bötticher, Der Baumkultus d. Hellenen S. 350: "`Bei den Branchiden war ein Weib die leitende Wahrsagerin, welche mit göttlichem Licht erfüllt ward, indem sie einen Stab hielt, den ihr ein Gott übergeben hatte, oder auf einem Axon (\textbf{Dreifuß}) sitzend die Zukunft offenbarte."' Und auf einen ganz ähnlichen Sinn scheint eine in dem griechisch-deutschen Wörterbuch von Jacobitz u. Seiler unter ἄξων gegebene Erklärung hinauszulaufen, wo unter Verweisung auf `Nonnos' behauptet wird, unter ἄξων "`sei das drehbare Schallgefäß am Dreifuß"' zu verstehen.\footnote{Wie ich später gefunden habe, stammt diese falsche Deutung von keinem Geringeren als Otfr. Müller in seiner Abhandlg. `Über die Tripoden' in Amalthea, herausg. v. Böttiger 1 (1820) S. 121: `Dazu kam ein Schallgefäß, von Nonnos ἄξων, von den Römern cortina genannt, von dünnem Erzblech gebildet und schön verziert, wie wir es auf dem Dresdner Kandelaberfuße zwischen dem Räuber Herakles und... Apollon liegen sehen.' --- Jetzt wird bekanntlich der Gegenstand allgemein als Omphalos gedeutet. Übrigens bemerkt schon Böttiger selbst a. a. O. S. 29 Anm., dass `ἄξων nie der eigentliche Name der cortina gewesen sein könne. Diesen könne sie nur als eine Art von Drehmaschine zuweilen gehabt haben.' Aus der Amalthea ist dann der O. Müllersche Irrtum zunächst in das Wörterbuch von Passow und daraus wieder in das von Jacobitz-Seiler übergegangen. Vgl. auch Bouché-Leclercq, Hist. de la divination dans l'ant. 3 p. 90 f. Anm., der ebenfalls ἄξων und cortina identifiziert.}

Um nun zu erfahren, was es mit der Berufung auf Nonnos für eine Bewandtnis hat, habe ich mich unter Darlegung meiner eigenen Interpretation der betreffenden Stelle des Iamblichos an den besten lebenden Kenner des Nonnos, A. Ludwich in Königsberg, gewendet und von diesem mit freundlichster Bereitwilligkeit folgende für mich hocherfreuliche Antwort erhalten: "`Mir ist es nicht zweifelhaft, dass Ihre Erklärung das Richtige trifft. Nonnos sah die Weltachse (ἄξων) als das Ganze und Delphi als den Mittelpunkt (ὀμφαλός) zwischen den beiden Polen an (vgl. 32, 205 von der Hüfte des menschlichen Leibes ἄξονα μηροῦ): daher die Verbindung μεσόμφαλος ἄξων Dion. 2, 697. 4, 290 (das für Delphi geprägte Adjektivum kommt bekanntlich schon bei Aeschylus und andern vor).\footnote{Nonn. Dion. 2, 697: σὺ δὲ, Κάδμε, μεσόμφαλον ἄξονα βαίνων || Δελφίδος αὐδήεντα μετέρχεο τέμπεα Πυθοῦς. --- ib. 4, 289 ff.: ἔνθα κιχήσας || Δελφὸν ἀσιγήτοιο μεσόμφαλον ἄξονα Πυθοῦς || μαντοσύνην ἐρέεινε, καὶ ἔμφρονα Πύθιος ἄξων κυκλόθεν αὐτοβόητος ἐθέσπισε κοιλάδι φωνῇ... Vgl. dazu Bouché-Leclercq, Hist. de la divin. 3 p. 90 Anm. u. p. 92 Anm. 3.} Diesen ἄξων nannte er ὀμφήεις (7, 72) oder ὀμφαῖος (27, 252),\footnote{ib. 7, 71 f.: καὶ Κρονίδης Αἰῶνι θεηγόρον ἴαχε φωνὴν || ἄξονος ὀμφήεντος ὑπέρτερα θέσφατα φαίνων. --- 27, 252: ἄξονος ὀμφαίοιο θεηγόρε κοίρανε Πυθοῦς [= Ἄπολλον]. Ebenso nennt Claudianus 16, 16 den delphischen `Nabel' nicht umbilicus sondern axis.} indem er annahm, dass die göttliche Orakelstimme aus dem Innern der Erde (4, 350)\footnote{4, 349 f.: ἀλλ᾽ ὅτε Κάδμῳ || Πύθιον οὐδαίης ἐτελείετο θέσφατον ἠχοῦς... ---} durch eine kreisrunde Felsöffnung kam, über welcher der Dreifuß (13, 132), der Sitz der Pythia, stand.\footnote{13, 131 f.: ἀγειρομένοισι δὲ λαοῖς || Πυθιὰς ὀμφήεσσα θεηγόρος ἔκλαγε πέτρη || καὶ τρίπος αὐτοβόητος...} Dass Nonnos irgendwo mit ἄξων `das drehbare Schallgefäß am Dreifuß' bezeichnet habe, halte ich für mehr als unwahrscheinlich; ich wenigstens kenne keinen Beleg dafür, auch kein Lexikon, das dafür eintritt. Vermutlich beruht jene irrige Annahme allein auf 4, 290 ff., einer Stelle, die allerdings unverständlich überliefert, aber von Lehrs, Qu. ep. 282 sicher gebessert ist."' Wir ersehen also aus diesen dankenswerten Darlegungen A. Ludwichs, dass der ἄξων von Branchidai genau dieselbe Bedeutung hat wie der mit dem ὀμφαλὸς γῆς identische von Delphi, dass folglich amblichos eine Überlieferung kannte, nach der Branchidai sich ebenso wie Delphi und viele andere Orte rühmte, der Mittelpunkt der Erde und des Weltalls zu sein.

3. Weiter erwäge man Folgendes. Herodot (1, 157) berichtet: οἱ δὲ Κυμαῖοι ἔγνωσαν συμβουλῆς πέρι ἐς θεὸν ἀνοῖσαι τὸν ἐν Βραγχίδῃσι. ἦν γὰρ αὐτόθι μαντήιον ἐκ παλαιοῦ ἰδρυμένον, τῷ Ἴωνές τε πάντες καὶ Αἰολέες ἐώθεσαν χρᾶσθαι κ. τ. λ. So fragen die Kymaier in Aiolis wiederholt zur Zeit des Kyros und Kroisos in Branchidai an, was mit dem Lyder Paktyes geschehen solle (Her. a. a. O), ebenso die Karer zur Zeit des Darius, ob sie die Milesier zu Bundesgenossen wühlen sollten (Zenob. 5, 80); so sendet Kroisos, um die Glaubwürdigkeit verschiedener Orakel zu prüfen, seine Boten nicht bloß nach Delphi, Abai, Dodona, sowie zu den Orakeln des Amphiaraos und Trophonios, sondern auch nach Branchidai (Her. 1, 46), dessen Tempel er, wie Her. 1, 92 hervorhebt, genau ebenso reich mit Weihgeschenken bedenkt wie den delphischen. Überhaupt muss das Orakel zu Branchidai im 7. und 6. Jahrhundert in Osthellas, Kleinasien und Ägypten ebenso angesehen gewesen sein wie das delphische: das beweist vor allem die kostbare Kriegsrüstung, welche um 600 der ägyptische König Necho 2. nach seinem Siege über die Syrer ins Didymaion weihte (Her. 2, 159), höchst wahrscheinlich um seiner Dankbarkeit für eine günstige Antwort Ausdruck zu geben. Darum erscheint es auch nicht wunderbar, wenn Herodot (5, 36) berichtet, unmittelbar vor dem Aufstand der Ionier seien die im Didymaion aufgehäuften Schätze so groß gewesen, dass Hekataios seinen Landsleuten raten konnte, sie zum Bau einer gewaltigen Kriegsflotte zu verwenden. Aus anderen zum Teil sehr alten Zeugnissen, die ich in meiner Monographie `Die neuentdeckte Schrift e. altmiles. Naturphilosophen etc.' S. 26 f. gesammelt und besprochen habe, erhellt deutlich, dass im 6., 7. und 8. Jahrhundert eine starke Rivalität und Eifersucht zwischen Delphi und Branchidai geherrscht hat, die erst mit der Zerstörung und Plünderung von Milet und Didyma im Jahr 494 aufhörte, seit welchem Zeitpunkt natürlich Delphis Autorität dauernd die von Branchidai überwog.\footnote{S. die Belege a. a. O. S. 26 f., wo noch nachzutragen ist das für die Eifersucht Delphis gegenüber Milet und Branchidai höchst bezeichnende Orakel, das die Pythia gegen Ende des 6. Jahrh. den Argivern erteilte. Vgl. Herod. 6, 19:\\\hspace*{10mm}Τὰ δὲ τοῖσι Μιλησίοισι οὐ παρεοῦσι ἔχρησε ἔχει ὧδε·\\\hspace*{10mm}Καὶ τότε δή, Μίλητε, κακῶν ἐπιμήχανε ἔργων,\\\hspace*{10mm}Πολλοῖσιν δεῖπνόν τε καὶ ἀγλαὰ δῶρα γενήσῃ,\\\hspace*{10mm}Σαὶ δ᾽ ἄλοχοι πολλοῖσι πόδας νίψουσι κομήταις,\\\hspace*{10mm}Νηοῦ δ᾽ ἡμετέρου Διδύμοις ἄλλοισι μελήσει.} Aus allen diesen Zeugnissen aber habe ich schon a. a. O. den naheliegenden Schluss gezogen, dass die ältesten in Milet heimischen Geographen, insbesondere Anaximandros und Hekataios, die bei den Entwürfen ihrer Erdkarten notwendigerweise von einem festen und möglichst allgemein anerkannten Mittelpunkte (ὀμφαλός) ausgehen mussten (s. H. Berger, Gesch. d. wiss. Erdkunde d. Griechen 2 S. 110 f.), als solchen schwerlich den gefährlichsten und erfolgreichsten Konkurrenten oder Rivalen ihres eigenen ebenfalls hochberühmten Orakels, sondern vielmehr das ihnen zu solchem Zwecke sehr viel bequemer gelegene und nach Herodot sonst regelmäßig von ihren Landsleuten und Mitbürgern als Orakel benutzte Didyma (d. h. Milet) erwählt haben werden. Den allerdeutlichsten Beweis für die Richtigkeit aller dieser Annahmen liefert uns aber wiederum die Weltkarte des Kosmologen in `Hippocr.' π. ἑβδ., die als `Zwerchfell,' d. h. als Zentrum der Oikumene, eben nicht Phokis und Delphi sondern vielmehr Ionien betrachtet wissen will.\footnote{In dieser Hinsicht stimmt mir kein Geringerer bei als Sal. Reinach in seiner Anzeige meiner Abhandlung in der Revue Archéol. 1911 2 p. 390.}

Wie man übrigens in jener alten Zeit bei der Bestimmung des Mittelpunktes einer Land- und Erdkarte zu Werke ging, können wir mit Leichtigkeit der Schilderung des beim Entwerfen von Windrosen (orbes ventorum: Varro r. r. 3, 5 a. E.) üblichen einfachen Verfahrens, das uns Plinius (18, 326 ff.) aufbewahrt hat, entnehmen. Es heißt dort nach Mayhoff:

"`Ventorum [ratio] paulo scrupulosior. Observato solis ortu, quocunque die libeat, stantibus hora diei sexta [= meridie] sic, ut ortum eum a sinistro humero habeant, contra mediam faciem meridies et a vertice septentrio erit. Qui ita limes per agrum curret, cardo [= ἄξων, axis, πόλος = διάφραγμα b. Dikaiarch\footnote{Vgl. Roscher, Die hippokrat. Schrift von d. Siebenzahl, in ihrer 4fachen Überlieferg. zum ersten Mal herausg. v. W. H. R. S. 118 A. 173.}] appellabitur. Circumagi deinde melius est, ut umbram suam quisque cernat; alioquin post hominem erit. Ergo permutatis lateribus, ut ortus illius diei a dextro humero fiat, occasus a sinistro, tunc erit hora sexta, cum minima umbra contra medium fiet hominem. Per huius mediam longitudinem duci sarculo sulcum vel <vom?>ere lineam, verbi gratia, pedum viginti conveniat, mediamque mensuram, hoc est in decimo pede circumscribi circulo parvo, qui vocetur umbilicus. Quae pars fuerit a vertice umbrae, haec erit ventus septentrionis... 331: Diximus ut in media linea designaretur umbilicus. Per hunc medium transversa currat alia. Haec erit ab exortu aequinoctiali ad occasum aequinoctialem, et limes, qui ita secabit agrum, decumanus vocabitur. Ducantur deinde aliae duae lineae in decussis obliquae, ita ut a septentrionis dextra laevaque ad Austri dextram laevamque descendant. Omnes per eundem currant umbilicum, omnes inter se pares sint, omnium intervalla paria. Quae ratio semel in quoque agro ineunda erit, vel si saepius libeat uti, e ligno facienda, regulis paribus in tympanum exiguum sed circinatum adactis."'

Nehmen wir also an, dass nach dieser überaus einfachen und sicher hochaltertümlichen Methode die ältesten Kartographen der Milesier verfuhren, so erkennt man sofort, wie infolge der geographischen Lage Milets, das ganz genau in der Mitte zwischen den äußersten Nord- und Südpunkten (Maiotis --- Naukratis) der den Ioniern bekannten Welt gelegen war, dieses zum `Omphalos' oder `umbilicus' der ältesten griechischen Erdkarte werden musste. Nun nennt aber Dikaiarchos, der Schüler des Aristoteles, welcher um 320 eine Erdkarte entwarf, deren `umbilicus' Rhodos war, die beiden Linien, die sich im umbilicus Rhodos schnitten, διαφράγματα; διάφραγμα aber bedeutet bei den Ärzten ganz gewöhnlich `Zwerchfell' (= φρένες), insofern es die innere Scheidewand oder Grenze des menschlichen Körpers zwischen den oberen und unteren Eingeweiden bezeichnet.\footnote{S. Kiepert, Lehrb. d. alt. Geogr. S. 5 A. 2 u. oben Anm. 87 b.} Es liegt demnach die Vermutung nahe, dass Dikaiarch jenen Ausdruck direkt oder indirekt älteren Vorgängern, also den älteren ionischen Geo- und Kartographen, entlehnt hat.\footnote{Anders Berger, Gesch. d. wiss. Erdkunde 2 S. 418, der glaubt, `dass der Name δ. (Scheidewand) zuerst vom Taurus, diesem großen Scheidegebirge, auf die ganze natürliche Grenze, dann auf die mit derselben in engster Verbindung bleibende Längenlinie und endlich auf deren Teile übertragen worden sei.' Gegen diese Erklärung spricht vor allem die Tatsache, dass der Taurus nur ein verschwindend kleiner Teil der von den Säulen des Herkules durch Sardinien, Sizilien, die Peloponnes, Karien, Taurus und Imaus bis zum östlichen Ozean gezogenen Linie darstellt. --- Vgl. auch meine ob. Anm. 87b angeführte Schrift S. 118 Anm. 173.} Auf diese Weise gelangen wir wiederum auf einem neuen Wege zu dem Verständnis jenes bedeutungsvollen Satzes der alten in der hippokratischen Schrift von der Siebenzahl glücklich erhaltenen Kosmologie, dass Ionien das `Zwerchfell' der als lebendiger Makrokosmos gedachten Welt darstelle.\footnote{Ein weiteres sehr schätzbares Argument für meine These, dass auch Didyma sich rühmte, das Zentrum der Erdscheibe zu sein, würde sich aus der Tatsache ergeben, dass an einem milesischen Apollonfeste von einer Sängergilde vor den Türen des Didymaions ein mit Binden geschmückter, γυλλός genannter Stein gesetzt und mit ungemischtem Wein begossen wurde (Nilsson, Gr. Feste S. 168), falls sich erweisen ließe, dass diesem Steine die Bedeutung eines Omphalos zukamen. Wir werden später sehen, dass auch vor dem delphischen Tempel ein steinerner Omphalos stand und dass außer diesem noch weitere kleinere Omphaloi in Delphi ausgegraben worden sind, ebenso wie in Milet. Inbetreff der Bedeutung von γυλλός vgl. Hesych. s. v. γύλιος... καὶ ἀγγεῖον ὁδοιπορικὸν εἰς ἀπόθεσιν τῶν ἀναγκαίων ᾧ ἐχρῶντο οἱ στρατιῶται. --- γύλλιον· ἀγγεῖον πλεκτόν. --- γυλλάς· εἶδος ποτηρίου, παρὰ Μακεδόσιν. --- γυλλός· κύβος, ἢ τετράγωνος λίθος. --- Die letzte Glosse deutet auf viereckige Form des Steines, die Gleichsetzung mit einem ἀγγεῖον oder ποτήριον auf rundliche Gestalt. Vgl. dazu γαυλός rundes Gefäß, Kübel, γαῦλος rundes Kauffahrzeug und Curtius, Grundz. d. gr. Etym. 5 174 f.}

4. Eine sehr willkommene Bestätigung aber für unsere Annahme, dass das Apollonorakel von Didyma, ebenso wie das von Delphi, für den Mittelpunkt der bewohnten Erde galt und deshalb auch einen "`Omphalos"' enthielt, scheint eine schöne Bronzemünze von Milet aus der Zeit des Commodus zu liefern, die Overbeck in seiner Kunstmythologie des Apollon auf Münztafel 4 unter nr. 47 abgebildet und S. 304 u. 308 besprochen hat (uns. Taf. 1, 2). Sie gehörte einst der Sammlung Imhoof-Blumer an und ist Overbeck von diesem trefflichen Forscher und Sammler in einem ausgezeichneten Abdruck zur Verfügung gestellt worden. Der Obvers zeigt den jugendlichen Kopf des Commodus mit dem Paludamentum rechtshin und trägt die Inschrift $\svgAAC$, der Revers stellt den Apoll von Didyma attributlos und nackt in bequemer Lage links hin auf einem Felsen sitzend und rechtshin (in die Ferne?) blickend dar; den rechten Arm legt er anmutig auf seinen Kopf, mit dem linken stützt er sich auf einen unmittelbar neben seinem Felsensitze stehenden bienenkorbartigen,\footnote{Vgl. den antiken Bienenkorb bei Baumeister, Denkm. 1, S. 317, Fig. 333.} auf niedriger Basis stehenden, ganz oben sich plötzlich stark verjüngenden, schlangenumwundenen, ziemlich hohen Omphalos, der den Felsensitz noch etwas überragt. Dabei die Inschrift: $\svgAAD[\svgAAE]\:\svgAAF$. Overbeck bemerkt dazu (S. 304), jedenfalls auf Grund von Mitteilungen Imhoofs: "`Falsch beschrieben bei Mionnet Suppl. 6 277. 1273 (les légendes retouchées). Ähnlich Rev. Numism. 1884, p. 18. 12."' Nach Overbeck (a. a. O. S. 299) gehört dieser Apollontypus zu denjenigen Münzbildern, `in denen entweder sicher oder doch wahrscheinlich Darstellungen anderer Kunstgattungen, namentlich statuarische Kompositionen auf die Münzen übertragen worden sind.' Es scheint also in Milet ein schönes statuarisches Bildwerk oder Relief aus bester Zeit gegeben zu haben, welches der betreffende Münzstempelschneider bei seiner Darstellung des Gottes vor Augen gehabt hat. Vielleicht kann dieser unzweifelhaft apollinische schlangenumwundene Omphalos auch zum Verständnis des unten anzuführenden großen Schlangenomphalos dienen, den die neuesten deutschen Ausgrabungen in der Nekropole von Milet zutage gefördert haben, doch ist freilich, wie wir später sehen werden, in diesem Falle auch noch eine andere Erklärung möglich. S. Taf. 6 Fig. 5.

5. Hierher gehört wahrscheinlich auch der bienenkorbförmige, ziemlich niedrige, basislose Omphalos des bekannten, die `Apotheose' Homers darstellenden Reliefs von der Hand des ionischen Künstlers Archelaos von Priene, dem Milet benachbarten Städtchen.\footnote{Ich verdanke den Hinweis auf dieses für meine Zwecke wichtige Monument der Güte G. Treus.} Dass es sich in diesem sicher in der Zeit zwischen Alexander d. Gr. und Tiberius entstandenen Bildwerk (Brunn, Künstlergesch. 1, 590) tatsächlich um ionische, nicht um delphische Vorstellungen und Riten handelt, scheint mir hervorzugehen nicht bloß aus der ionischen Herkunft des Archelaos, sondern namentlich auch aus dem Inhalt der betreffenden Darstellungen, die sich offenbar auf einen ionischen Kult Homers als des größten ionischen Dichters beziehen.\footnote{Solche Kulte Homers bestanden bekanntlich in Ios (Monat Ὁμηρεών), Smyrna (Strab. 646), Alexandria (Ael. v. h. 13, 22), Argos (Cert. Hes. et. Hom. 325 Göttl.; Ael. v. h. 9, 15). Vgl. Anth. 12, 168. Anth. Plan. 301 (θεός), Catal. gr. coins Brit. Mus. Ionia S. 41 ff. (Kolophon), 238 ff. (Smyrna), 346 (Chios), Head, Hist. nu. 2 486 (Ios), 554 (Kyme).} Hierzu kommt noch, dass der dem Dichterheros geopferte Stier deutlich als Buckelochse (Zebu) gebildet ist, eine Rinderspezies, die sich nicht im eigentlichen Hellas, wohl aber in Kleinasien (Ionien, Karien, Phrygien, Syrien) nachweisen lässt (vgl. Aristot. h. an. 8, 28, 3; Plin. h. n. 8, 179; Opp. Cyn. 2, 91 f.\footnote{Vgl. auch Lenz, Zool. d. Griech. u. Römer 245; Magerstedt, Bilder aus d. römischen Landwirtschaft 2 S. 15 und als Gegensatz dazu das apollinische (pythische) Stieropfer zu Delphi im Monat Bukatios, dargestellt auf dem schönen Wandgemälde aus dem Hause der Vettier zu Pompeji.} Demnach muss sich die im untersten Streifen des Reliefs dargestellte Apotheose Homers auf einen ionischen Heroenkult des großen Ioniers und ebenso die in dem folgenden Streifen abgebildete Grotte (Adyton) mit dem (untersatzlosen Omphalos in der Mitte, dem Apollon Kitharodos zur Linken und der eine Trinkschale voll heiligen Quellwassers zum Munde führenden kleiner gebildeten Prophetin (s. ob. S. 41)\footnote{Dass die Prophetin die Schale voll heiligen Wassers selbst trinken oder ihren Duft einatmen will (vgl. die εὐωδία der Adytonquelle b. Plut. def. or. 50 und das εὐῶδες ἀμβροσίων ἐκ μυχῶν ἐραννὸν ὕδωρ des Simonides, sowie die ἀναπνοὴ τοῦ νάματος b. Plut. Pyth. or. 17), geht einfach aus der Haltung des Apollon hervor, der mit der L. die große Kithara, mit der gesenkten R. das Plektron hält und zugleich den Kopf von der Priesterin abwendet (vgl. Overbeck, Apollon 270).} zur Rechten auf einen ionischen Apollokult, wahrscheinlich auf den des unweit von Priene gelegenen Branchidai, beziehen. Die früheren Erklärungen, welche die Grotte als die korykische Höhle,\footnote{Gegen diese spricht schon der Umstand, dass sie, soviel wir wissen, nicht dem Apollon, sondern dem Pan und den Nymphen geheiligt war: Preller-Robert, Gr. Myth. 1 4, 722, 8. Bursian, Geogr. v. Gr. 1, 179, 3. Baedeker, Griechenl. 4 155.} den Apollon als den pythischen Gott und die Priesterin als Pythia auffassen (Friederichs, Berl. ant. Bildw. 1, S. 449), Julius in Baumeisters Denkmälern 1, 111; Overbeck, Apollo, S. 270 nr. 25), müssen also als irrig bezeichnet werden. Auch fehlt ja der vom pythischen Gotte fast untrennbare Dreifuß auf dem Relief des Prieners. Der dargestellte Berg aber, auf dessen Spitze Zeus (Ὀλύμπιος?) tront, kann kaum ein anderer sein, als entweder der durch Zeus- Musen- und Apollonkult ausgezeichnete thessalische Olymp oder dessen kleinasiatischer Namensvetter in Mysien, der ebenso durch Zeuskult hervorragt (Head, Hist. nu. 2 517 unten) S. Taf. 7, 3.

Haben wir sonach mit allen Mitteln philologischer und archäologischer Kritik die einstige Existenz eines "`Omphalos"' im Tempel von Didyma und dessen Bedeutung als Symbol der zentralen Lage von Milet auf der ältesten Weltkarte der Griechen festgestellt, so müssen wir nunmehr weiterzusehen, welches Licht von dieser Erkenntnis aus auf die von Didyma und Milet ausgegangenen Apollonkulte in den z. T. sehr alten Kolonialstädten der Milesier fällt. In dieser Beziehung kommt für uns vor allem der bedeutende Apollonkult von Kyzikos in Betracht. Bereits Marquardt in seiner immer noch lesenswerten Monographie über `Cyzicus u. sein Gebiet' (S. 129) hat erkannt, dass der Apoll von Kyzikos, der daselbst als ἀρχηγέτης verehrt wurde (Aristid. or. 1 p. 383 D.),\footnote{Vgl. auch Apollon Rh. 1, 958: ἀτὰρ κεῖνόν [λίθον εὐν.] γε θεοπροπίαις Ἑκάτοιο || Νηλεῖδαι μετόπισθεν Ἰάονες [= Μιλήσιοι] ἱδρύσαντο || ἱερόν.} kein anderer als der didymäische Gott von Milet ist, dessen Orakel bei allen Kolonialgründungen der Milesier (vgl. Herod. 1, 157) ebenso befragt wurde, wie es ja nach der Gründungssage der Mutterstadt schon bei der Gründung dieser selbst der Fall gewesen war (Tz. z. Lyk. 1385). Ein wie starkes Bewusstsein zu jeder Zeit, auch noch später, von der milesischen Herkunft seines Apollonkultes in Kyzikos vorhanden war, beweisen vor allem die von Marquardt a. a. O. S. 129 angeführten milesischen Inschriften (C. I. Gr. 2, 2855 l. 21 u. 2858), in denen verschiedene von den Kyzikenern ins Didymaion von Milet gestiftete Weihgeschenke erwähnt werden. Nun gibt es aber eine dem 5. Jahrhundert angehörende, also ziemlich alte Elektronmünze von Kyzikos, die im Catalogue of the Greek coins of Mysia (in the Brit. Mus.) S. 32 also beschrieben wird (uns. Taf. 1, 1):

Obv.: Netted Omphalos of Delphi [??], on which two eagles with closed wings facing one another: beneath, tunny r. --- Rev.: Incuse square of mill-sail pattern. [Pl. 8, 7; vergrößert auch abgebildet bei Studniczka im Hermes 37 (1902) S. 266]. Wenn hier unbedenklich der Omphalos als der von Delphi bezeichnet wird, so hege ich in dieser Beziehung doch sehr starke Zweifel. Diese gründen sich erstens auf die unleugbare Tatsache, dass die Adler des Omphalos von Kyzikos wesentlich anders aufgefasst und dargestellt sind als die des delphischen Nabelsteins, wie er auf dem interessanten von Wolters (Mitteil. d. d. arch. Inst. Athen 1887 Tf. 12, S. 378 ff., vgl. auch Studniczka a. a. O. S. 267) herausgegebenen und erläuterten Marmorrelief zu Sparta sowie auf einigen athenischen, auf den Kult des attischen Apollon Pythios bezüglichen Bildwerken der zweiten Hälfte des 5. Jahrhunderts (s. unt. Kap. 4 B) erscheint. Ob auf dem Kyzikenischen Stäter "`die Vögel um den Preis des Verschwindens ihrer Füße hinauf- und herangerückt sind,"' weil dies, wie Studniczka a. a. O. meint, "`der enge Rahmen"' erfordere, ist mir deshalb einigermaßen zweifelhaft, weil ja der Münzstempelschneider, um den für die genauere Wiedergabe des Motivs nötigen Raum zu gewinnen, nur nötig hatte, den Omphalos und die Adler in etwas kleinerem Maßstab darzustellen. Noch viel bedeutsamer aber erscheint mir der weitere Unterschied, dass auf der Münze von Kyzikos die beiden Vögel ihre Köpfe und Schnäbel einander zukehren, während sie auf dem spartanischen Relief diese voneinander abwenden, sowie dass der Omphalos von Delphi eine deutliche, etwas breitere Basis (Stufe) hat, auf deren Enden die Adler sitzen.\footnote{Auf den weiteren Unterschied, dass der Omphalos von Kyzikos mit einem Bindennetze bedeckt, dagegen der delphische des spartanischen Reliefs kahl ist, lege ich kein Gewicht, zumal da ja das Bindennetz des Reliefs ursprünglich aufgemalt sein konnte.} Diese Unterschiede zwischen beiden Darstellungen scheinen mir in der Tat groß genug, um die Annahme zu rechtfertigen, dass der auf dem Kyzikener Stäter dargestellte Omphalos (basislos!) nicht der delphische sondern vielmehr ein anderer, und zwar wahrscheinlich der alte milesische im Jahre 494 bei der Zerstörung und Plünderung Milets durch die Perser zugrunde gegangene ist oder wenigstens sein kann.

Aus der Zeit zwischen 330 und 280 vor Chr. stammt ferner die kyzikenische Silbermünze, welche im Catalogue des Brit. Mus. Mysia Taf. 9 13 abgebildet und daselbst S. 36 so beschrieben wird:

Obv.: $\svgAAG$ Head of Kore Soteira l.; wearing earring, necklace, stephane, corn-wreath, and veil wound round head; beneath tunny l. Rev: $\svgAAH$ Apollo wearing himation over lower limbs, seated l. on omphalos, in extended r., patera; l. elbow is supported by lyre. --- in front, cock l.; behind $\svgAAI$. --- Ganz ähnlich die Abbildung bei Overbeck a. a. O. Münztafel 3 nr. 21; vgl. S. 300 nr. 8 u. S. 307. Vgl. auch unsere Taf. 1 nr. 12.

Eine Bronzemünze von Kyzikos aus der Zeit des Commodus zeigt nach dem Catalogue von Mysia S. 51 auf ihrem Revers: $\svgAAJ$ Apollo, naked, standing r.; l. foot rests on omphalos; in r. hand, branch; l. hand rests on knee; before him (raven?). Dazu die Note: `A similar specimen is published by Overbeck, Griech. Kunstmythologie, Münztaf. 5, 9: p. 304, no. 101; p. 314, who describes the Apollo as holding the gorgoneion as well as the branch, but the supposed gorgoneion is probably only an abrasion of the surface. With the attitude of the figure, cp. no. 24, p. 12, supra (Apollonia ad Rhyndacum), and note, ib. --- Auch dieser O. ist basislos!

Mehrfach erscheint auch auf dem Revers der Kaisermünzen von Kyzikos ein 8-säuliger Tempel (octastyle temple), den ich für das Heiligtum des Apollon ἀρχηγέτης halte, weil in dessen Giebelfeld ein Omphalos von genau derselben Form [$\odot$] angebracht ist, wie auf gewissen Münzen von Delphi.\footnote{Andere, z. B. Svoronos und Head (hist. nu. 2), freilich halten jetzt das Zeichen $\odot$ nicht für den Omphalos inmitten des `orbis' terrarum, sondern für eine φιάλη ὀμφαλωτή, die aber auch recht wohl als Symbol der runden Erdscheibe mit dem Nabel in der Mitte gelten kann (s. unt. S. 75 Anm. 136).} Man vergleiche vor allem a. a. O. Taf. 12, 14 (Münze des Antoninus Pius) und Taf. 13, 10 (Commodus). Auf letzterer Münze ist freilich der Punkt in dem umgebenden Kreise undeutlich geworden. --- Hier an den delphischen Tempel zu denken, dürfte namentlich auch deshalb unstatthaft sein, weil dieser auf delphischen Münzen stets 6-säulig (Hexastylos) dargestellt ist. Vgl. Catal of the gr. c. in the Brit. Mus. Central Greece pl. 4, 22, p. 29 f. Imhoof-Blumer u. Percy Gardner, Num. Comment. on Pausanias Taf. 10, nr. 24 u. 25 (hier statt $\odot$ nur •!).

Infolge der eigentümlichen Lage der Stadt genau in der Mitte zwischen dem äußersten Osten und Westen des hellenischen Kolonialgebietes waren in Kyzikos alle diejenigen Sagen lokalisiert, in welchen Züge nach Osten vorkamen (Argonauten- und Orestessage). Daher wusste die Legende nicht bloß von den Argonauten in Kyzikos zu berichten, sondern ebenso auch von Orestes, der auf der Rückkehr von Taurike hier am Omphalos des Apollontempels gereinigt und gesühnt sein sollte. Zwar schweigt darüber die literarische Überlieferung, aber eine schon längst auf Orestes bezogene ältere Elektronmünze des 5. Jahrhunderts scheint deutlich die Lokalisierung jener Sage in Kyzikos zu bezeugen.\footnote{Die Reinigung und Entsühnung des Orestes war bekanntlich keineswegs auf Delphi beschränkt, sondern sollte noch an zahlreichen anderen Orten vollzogen worden sein, z. B. zu Gythion, wo man ebenfalls einen Stein (Omphalos?) zeigte, auf dem sitzend Orestes vom Wahnsinn geheilt sein sollte (Paus. 3, 22, 1); zu Ἄκη in Arkadien (Paus. 8, 34, 2); zu Troizen, wo ebenfalls ein `Orestesstein' gezeigt wurde (Paus. 2, 31, 4. 8); zu Keryneia in Achaja (Paus. 7, 25, 7. Schol. Soph. Oed. Col. 42). Mehr bei Höfer im Artikel Orestes des Mythol. Lexikons 3, Sp. 985 ff. und Sp. 998 ff., wo auch verschiedene Orte Kleinasiens, Kappadokiens usw., ja Siziliens und Italiens angegeben sind, an denen die Sage von der Entsühnung des Orestes und der Weihe des taurischen Artemisbildes lokalisiert war.} Die betreffende Beschreibung im Catalogue des Brit. Mus. Mysia, S. 28 (vgl. Taf. 7 1 und 2) lautet:

Obv: Bearded male figure (Orestes), wearing chlamys, kneeling l. beside Delphic [?] omphalos, on which his l. hand rests; in r., drawn sword: beneath, tunny l. --- Rev: Incuse square of millsail pattern. Vgl. auch Num. Chron. 1889, p. 257, no. 24, pl. 12, 11. Der O. ist auch hier wieder basislos! S. Taf. 1, 3.

So halte ich es denn im Hinblick auf alle für die einstige Existenz eines Omphalos im Apollotempel von Kyzikos angeführten Tatsachen nicht für zu kühn, auch folgende Worte des Aristides in seiner Lobrede auf Kyzikos auf ihn zu beziehen: τῆς γὰρ θαλάττης ἐν μέσῳ κειμένη συνάγει πάντας ἀνθρώπους εἰς ταὐτόν, τούς τε ἀπὸ τῆς εἴσω πρὸς τὴν ἔξω παραπέμπουσα καὶ τοὺς ἔξωθεν πρὸς τὰ εἴσω, ὥσπερ τις ὀμφαλὸς τοῦ μεταξὺ τόπου Γαδείρων καὶ Φάσιδος. Aristides hätte aber mit demselben Rechte auch behaupten können, dass Kyzikos ungefähr in der Mitte zwischen Libyen und Taurike oder Tanais gelegen sei. Es braucht kaum hervorgehoben zu werden, dass auf dieser wahrhaft zentralen Lage der Stadt ihre außerordentliche Blüte, namentlich in kommerzieller Beziehung, während des 5., 4. und 3. Jahrhunderts beruht.

So bleibt denn, um die Betrachtung der unzweifelhaft von Didyma abhängigen Apollokulte abzuschließen, nur noch die Untersuchung der `Omphalosmünzen' von Sinope zu erwähnen übrig. Vgl. Head hist. num. 1 S. 435 (s. uns. Taf. 1, 11):

`Obv: Head of Sinope, r. turreted (Num. Chron. 1885, Pl. 2 18). --- Rev. $\svgAAK$ Apollo naked, seated on omphalos [ohne Basis], holding lyre and plectrum. AR. Spread tetradrachm. The type of this tetradrachm is copied, with some modifications, from the tetradrachms of Antiochus 3. cf. Syria. Circ. B. C. 189-183.' Ob Head mit seiner Behauptung, diese Münze sei die Kopie einer solchen des Antiochos 3, Recht hat, ist mir doch sehr zweifelhaft geworden, als ich die von Overbeck a. a. 0. auf Münztafel 3 nebeneinander abgebildeten Münzen (mit dem Bilde des auf dem Omphalos sitzenden Apollon) von Chersonasos auf Kreta (nr. 36), von Sinope (nr. 37) und von Antiochos (nr. 41 u. 42)\footnote{Vgl. auch die treffliche Vergrößerung der Antiochosmünze b. Studniczka im Hermes 1902 (37), S. 258.} miteinander verglich. Dieser Vergleich zeigt sofort, dass die Münze von Sinope sehr viel mehr Ähnlichkeit mit der von Chersonasos als mit denen des Antiochos besitzt, welche letzteren wiederum mit dem Typus der Münze des Nikokles von Paphos (Overbeck a. a. O. nr. 41; vgl. Catal. of gr. coins in the Brit. Mus. Cyprus p. 80 u. pl. 22 nr. 10 u. 11) völlig übereinstimmen, wie auch meiner Erinnerung nach schon andere bemerkt haben.

Ob freilich allein aus diesem Münzbild von Sinope mit voller Sicherheit auf einen Omphalos im dortigen gewiss ursprünglich vom Didymaion stark beeinflussten Apollokult zu schließen ist, bezweifle ich. Vielleicht geben weitere Funde noch einmal sicherere Entscheidung.\footnote{Ein charakteristisches Beispiel dafür, wie treu die Sinopenser an ihren milesischen Traditionen auch noch in späterer Zeit festhielten, führt Thraemer (Pergamos S. 101, A 1) an. Th. weist nach, dass die Sinopenser den Ostwind immer noch Βερεκυντίας nannten, obwohl sie weit östlich von Βερεκύντιον ὄρος wohnten. `Der Name hat nur Sinn, wo Phrygien östlich liegt, ist also aus der Heimat Milet nach Sinope mitgewandert' (vgl. Grotefend b. Pauly s. v. Berecyntes). Dieses treue Festhalten an alten Überlieferungen zeigt sich aber auf keinem Gebiete deutlicher als auf dem der Religion und des Kultus.}

Am Schlusse dieser Untersuchung möchte ich auch für Didyma betonen, dass für den dortigen `Omphalos' höchstwahrscheinlich dasselbe gilt, was wir für Delphi annehmen zu dürfen glauben, dass nämlich auch zu Milet als der eigentliche Nabel nicht der sogenannte Omphalosstein, sondern vielmehr ursprünglich das χάσμα γῆς, aus der das von der Prophetin getrunkene, mit prophetischer Kraft begabte Quellwasser strömte, anzusehen ist. Der daneben angebrachte konische Stein hatte nur den Zweck, allen Tempelbesuchern von weitem sichtbar den heiligen Punkt zu bezeichnen, der als Zentrum der Erde angesehen wurde.
\clearpage
\section{Delphi und sein Orakel als Mittelpunkt (ὀμφαλός) der Welt und sein Nabelstein.}
\subsection{Die literarischen Zeugnisse.}
\paragraph{}
Um möglichst gründlich und übersichtlich zu verfahren, legen wir zunächst die sämtlichen literarischen Zeugnisse in historischer Folge vor und begleiten die einzelnen, soweit es nötig scheint, mit erklärenden und kritischen Bemerkungen.

1. Epimenides. Das älteste Zeugnis überliefert uns Plutarch, der delphische ἱερεὺς διὰ βίου, am Anfang der Schrift De defectu orac. 1: Ἀετούς τινας ἢ κύκνους,\footnote{Anders Strabo 9, p. 419: Τῆς Ἑλλάδος [ὁ τόπος τ. Δελφῶν] ἐν μέσῳ πώς ἐστι τῆς συμπάσης... ἐνομίσθη δὲ καὶ τῆς οἰκουμένης, καὶ ἐκάλεσαν τῆς γῆς ὀμφαλὸν, προσπλάσαντες καὶ μῦθον ὅν φησι Πίνδαρος [vgl. Pindari frgm. 27 Boeckh], ὅτι συμπέσοιεν ἐνταῦθα οἱ ἀετοὶ οἱ ἀφεθέντες ὑπὸ τοῦ Διός, ὁ μὲν ἀπὸ δύσεως ὁ δ᾽ ἀπὸ τῆς ἀνατολῆς· οἱ δὲ κόρακάς φασι. δείκνυται δὲ καὶ ὀμφαλός τις ἐν τῷ ναῷ τεταινιωμένος καὶ ἐπ᾽ αὐτῷ αἱ δύο εἰκόνες τοῦ μύθου. Da die erhaltenen Monumente (s. unten) unzweifelhaft 2 Adler, nicht aber Schwäne oder Raben darstellen, und, wie wir sehen werden, die (goldenen) Adler im heil. Kriege (354 v. Chr.) von Onomarchos und Philomelos geraubt waren, so folgt aus Strabon und Plutarch a. a. O., dass man später, als die Adler verschwunden waren, mehrfach an diesen Anstoß nahm, weil sie Vögel des Zeus, nicht des Apollon sind, und an ihrer Stelle vielmehr apollinische Vögel (Raben oder Schwäne) am Omphalos angenommen hat. Diese Annahme lag umso näher, wenn in anderen Apollotempeln die daselbst befindlichen Nabelsteine mit Raben oder Schwänen geschmückt waren, wie wir aus gewissen Münzen (s. unten) schließen dürfen.} ὦ Τερέντιε Πρίσκε, μυθολογοῦσιν ἀπὸ τῶν ἄκρων τῆς γῆς ἐπὶ τὸ μέσον φερομένους εἰς ταὐτὸ συμπεσεῖν Πυθοῖ, περὶ τὸν καλούμενον ὀμφαλόν· ὕστερον δὲ χρόνῳ τὸν Φαίστιον Ἐπιμενίδην ἐλέγχοντα τὸν μῦθον ἐπὶ τοῦ θεοῦ, καὶ λαβόντα χρησμὸν ἀσαφῆ καὶ ἀμφίβολον, εἰπεῖν·
\begin{quotation}
Οὔτε γὰρ ἦν γαίης μέσος ὀμφαλὸς οὔτε θαλάσσης·

εἰ δέ τις ἐστί, θεοῖς δῆλος, θνητοῖσι δ᾽ ἄφαντος.\footnote{Zum Verständnis des ἦν und ἄφαντος verweise ich auf Aristot. Rhet. 3 17 p. 1418 a 21: Ἐπιμενίδης ὁ Κρὴς... περὶ τῶν ἐσομένων οὐκ ἐμαντεύετο, ἀλλὰ περὶ τῶν γεγονότων μὲν ἀδήλων δέ.}
\end{quotation}
\paragraph{}
Ἐκεῖνον μὲν οὖν εἰκότως ὁ θεὸς ἠμύνατο, μύθου παλαιοῦ καθάπερ ζωγραφήματος ἁφῇ ἀποπειρώμενον (vgl. Kinkel, Epicor. gr. frgm. 1 p. 234, 6).\footnote{Worin die vom Gotte an Epimenides vollzogene Strafe für seine Ungläubigkeit bestand, ist m. W. bis jetzt unbekannt. War es vielleicht der langdauernde todesschlafähnliche Zustand, in den er verfallen sein sollte?}

Wir ersehen daraus, dass die Sage vom delphischen Omphalos und seiner Entstehung durch die von den entgegengesetzten Enden der Erde ausgesandten Adler des Zeus nach der dem Plutarch vorliegenden Tradition bereits dem alten, spätestens um 500 v. Chr.\footnote{Vgl. Plat. leg. 642 D, wo aber vielleicht statt ιʹ [= δέκα] ριʹ [= ἑκατὸν δέκα] zu lesen ist. Diese Lesung würde den Widerspruch mit den übrigen Zeugnissen für die Lebenszeit des Epimenides beseitigen.} lebenden Sühnpriester und Mantis Epimenides bekannt, aber von diesem in Zweifel gezogen worden war, weil er auf eine von ihm der Pythia vorgelegte darauf bezügliche Frage eine unklare Antwort erhalten hatte. Ob dies freilich der einzige Grund seines Zweifels war, darf billig bezweifelt werden. Wahrscheinlich war für Epimenides mindestens ebenso maßgebend der Umstand, dass außer Delphi noch mehrere andere ihm wohlbekannte Orte Anspruch darauf erhoben, Mittelpunkte (ὀμφαλοί) der Erde zu sein, insbesondere Branchidai bei Milet (s. das vorige Kapitel) und Paphos (s. ob. S. 29 f.).

Unklar bleibt auch bis auf weiteres, wie sich Epimenides oder die von ihm getadelten Vertreter der Anschauung vom ὀμφαλὸς γῆς und θαλάσσης das Verhältnis dieser beiden ὀμφαλοί zueinander vorgestellt haben, und welche Insel ursprünglich unter dem ὀμφαλὸς θαλάσσης zu verstehen ist (vgl. oben Anm. 14 u. 74).

2. Pindar.

a. Pyth. 6, 1 ff. Boeckh:
\begin{quotation}
Ἀκούσατ᾽· ἦ γὰρ ἑλικώπιδος Ἀφροδίτας

ἄρουραν ἢ Χαρίτων

ἀναπολίζομεν, ὀμφαλὸν ἐριβρόμου

χθονὸς ἀένναον προσοιχόμενοι...
\end{quotation}
\paragraph{}
Schol. ἀκούσατε... εἰς τὴν Πυθὼ παριόντες, ἥ ἐστιν ὀμφαλὸς τῆς γῆς. Gedichtet ist die Ode nach Boeckh Ol. 71, 3 = 494 v. Chr., als Pindar 28 Jahre alt war.

b. Pyth. 4, 6 ff. B.:
\begin{quotation}
ἔνθα ποτὲ χρυσέων Διὸς αἰητῶν πάρεδρος

οὐκ ἀποδάμου Ἀπόλλωνος τυχόντος ἰρέα

χρῆσεν...
\end{quotation}
\paragraph{}
Schol. Λόγος τις τοιοῦτος περιηχεῖ, ὅτι ὁ Ζεὺς τὸ μεσαίτατον τῆς οἰκουμένης καταμετρήσασθαι βουληθεὶς ἴσους κατὰ τὸ τάχος ἀετοὺς ἐκ δύσεως καὶ ἀνατολῆς ἀφῆκεν· οἱ δὲ διῒπτάμενοι συνέπεσον ἀλλήλοις κατὰ τὴν Πυθῶνα, ὥστε τὴν σύμπτωσιν ὁρίζειν αὐτόθι τῆς ὅλης οἰκουμένης τὸ μεσαίτατον. ὕστερον δὲ σημεῖον τοῦ γεγονότος καὶ χρυσοῦς ἀετοὺς κατασκευάσας ἀνέθηκε τῷ τοῦ θεοῦ τεμένει. --- Ἄλλως ὅτι ὑπὸ Διὸς ἀφεθέντες ἐκ τῶν περάτων τῆς γῆς συνέπεσον ἐνταῦθα, καὶ οὕτως ἐγνώσθη τὸ μέσον τῆς γῆς. ὧν εἰκόνες οἱ χρυσοὶ ἀνέκειντο παρὰ τὸν ὀμφαλὸν ἀετοί, ἤρθησαν δὲ ἐν τῷ Φωκικῷ πολέμῳ, ὃν Φιλόμηλος συνεστήσατο. Der in dieser Ode gefeierte Sieg des Arkesilaos vor Kyrene fällt in das Jahr Ol. 78, 3 = 466 v. Chr.

c. ebenda 4, 73 ff. (130 ff.):
\begin{quotation}
ἦλθε δέ οἱ [τῷ Πελίᾳ] κρυόεν πυκινῷ μάντευμα θυμῷ,

πὰρ μέσον ὀμφαλὸν εὐδένδροιο ῥηθὲν ματέρος·

τὸν μονοκρήπιδα πάντως ἐν φυλακᾷ σχεθέμεν μεγάλᾳ...
\end{quotation}
\paragraph{}
Schol. ἦλθε δὲ... τὸ φρικτὸν τοῦτο μάντευμα τῷ συνετῷ αὐτοῦ θυμῷ, τὸ κατὰ τὸν μέσον ὀμφαλὸν ῥηθέν, τουτέστι κατὰ τὸ Δελφικὸν χρηστήριον τὸ ἐν μέσῳ τῆς οἰκουμένης τῆς εὐδένδρου γῆς· λέγει δὲ τῆς Πυθῶνος.

d. Nem. 7, 33 (49) ff.:
\begin{quotation}
...τοὶ παρὰ\footnote{Es fragt sich, ob nicht hier unter dem μέγας ὀ. χθονός eigentlich der Parnass zu verstehen ist, der, wie wir später sehen werden, bisweilen als der gewaltige, weithin sichtbare Mittelpunkt Griechenlands und der Erde aufgefasst wird.} μέγαν ὀμφαλὸν εὐρυκόλπου

μόλον χθονός (d. i. Neoptolemos und seine Gefährten)...
\end{quotation}
\paragraph{}
Schol. ἔμολε παρὰ μέγαν ὀμφαλὸν εὐρυκόλπου χθονὸς Νεοπτόλεμος, Πριάμου πόλιν ἐπεὶ πράθε· τεθνηκότων δὴ τῶν βοηθῶν ἐν Πνθίοις δαπέδοις κεῖται... Das Gedicht ist entstanden um Ol. 79, 4 = 461 v. Chr.

e. Pyth. 8, 62 (85):
\begin{quotation}
ὑπάντασέ [ὁ Ἀλκμάων] τ᾿ ἰόντι γᾶς ὀμφαλὸν παρ᾽ ἀοίδιμον,

μαντευμάτων τ᾽ ἐφάψατο συγγόνοισι τέχναις.
\end{quotation}
\paragraph{}
Schol. Ταῦτα δὲ εἴρηκεν ὡς ὑπάρχοντος ἥρωος καὶ γειενιῶντος τῇ τοῦ νικηφόρου οἰκίᾳ, προσυποτίθεται δὲ ὅτι καὶ ὑπήντησε πορευομένῳ εἰς τὸν ἀγῶνα καὶ τῆς μαντείας ἐφήψατο καὶ αὐτὸς ὢν μάντις... Ἄλλως· ἐφηδόμενος δὲ καὶ αὐτὸς τὸν Ἀλκμάονα στεφανῷ τῇ ὠδῇ, ὅτι δή μοι ὁ τούτου πατὴρ Ἀμφιάραος γείτων ἐστὶ καὶ φύλαξ τῶν ἐμῶν κτημάτων, καὶ ὅτι ἀπιόντι εἰς τὸν τῆς γῆς ὀμφαλόν, τουτέστιν εἰς τὴν Πυθῶνα, ἀκήντησε καὶ τῶν μαντευμάτων ἐφήψατο τοῖς συγγεννηθεῖσιν αὐτῷ. --- Gedichtet Ol. 80, 3 = 458 v. Chr.

f. Pyth. 11, 9 (15) f.:
\begin{quotation}
ὄφρα Θέμιν ἱερὰν Πυθῶνά τε καὶ ὀρθοδίκαν

γᾶς ὀμφαλὸν κελαδήσετ᾽ ἄκρᾳ σὺν ἑσπέρᾳ...
\end{quotation}
\paragraph{}
Schol. ὅπως ἂν τὴν ἱερὰν Θέμιν καὶ τὴν Πυθῶνα καὶ τὸν ὄντα ὀμφαλὸν τῆς γῆς ὑμνήσητε σὺν παννυχίσι, χάριν κατατιθέμεναι ταῖς ἑπταπύλοις Θήβαις καὶ τῆς Κίρρας ἀγῶνι. --- Gedichtet Ol. 80, 3 = 458 v. Chr.

g. Pind. fr. 27 Boeckh (aus dem Paian auf den Delphischen Apollon, auf den Boeckh wohl mit Recht die Worte des Strabon [s. ob. Anm. 103] bezieht): Ἐκάλεσαν τῆς γῆς ὀμφαλὸν προσπλάσαντες καὶ μῦθον, ὅν φησι Πίνδαρος, ὅτι συμπέσοιεν ἐνταῦθα οἱ ἀετοὶ οἱ ἀφεθέντες ὑπὸ τοῦ Διός, ὁ μὲν ἀπὸ τῆς δύσεως, ὁ δ᾽ ἀπὸ τῆς ἀνατολῆς. Vgl. auch Pausanias 10, 16, 3: Τὸν δὲ ὑπὸ Δελφῶν καλούμενον ὀμφαλὸν λίθου πεποιημένου λευκοῦ, τοῦτο εἶναι τὸ ἐν μέσῳ γῆς πάσης αὐτοί τε λέγουσιν οἱ Δελφοί, καὶ ἐν ᾠδῇ τινι Πίνδαρος ὁμολογοῦντά σφισιν ἐποίησεν. Vgl. Boeckh 2 p. 570: `Igitur haec praeter Epinicia in alio carmine prodita uberius arbitror, in quo me illud confirmat, quod Pausanias non universe ad Pindarum passim haec tangentem, sed ad certam quandam odam provocat, in qua Pindarus consentientia Delphis pronunciaverit: illi vero Paeani, in quo de Delphicis templis poeta dixerat' illa optime vindicari quis negaverit?'

3. Auch der Zeitgenosse und schwächere Rivale Pindars, Bakchylides, gedenkt des Erdnabels Delphi in der vierten, zu Ehren eines pythischen Wagensiegs des Hieron von Syrakus gedichteten Ode, Vers 4:
\begin{quotation}
Έτι Συρακοσίαν φιλεῖ

πόλιν ὁ χρυσοκόμας Ἀπόλλων,

ἀστύθεμίν θ᾽ Ἱέρωνα γεραίρει·

τρίτον γὰρ π[αρ᾽ ὀμφα]λὸν ὑψιδείρου χθονὸς\footnote{Hier ist unter ὀ. ὑψιδ. χθονός entweder der Parnass oder der delphische Apollontempel mit dem Nabelstein zu verstehen.}

Πυθιόνικ[ος ἀείδε]ται...

ὠκυπόδ[ων ἀρετᾷ] σὺν ἵππων.
\end{quotation}
\paragraph{}
Der Wagensieg Hierons fällt in das Jahr Ol. 77, 3 = 470 v. Chr. (vgl. Blass, Praef. p. 50).

4. Aischylos.

a. In den Septem ante Thebas v. 745 ff. (aufgeführt im J. 467 v. Chr.) sagt der Chor:
\begin{quotation}
Ἀπόλλωνος εὖτε Λάϊος

βίᾳ, τρὶς εἰπόντος ἐν

μεσομφάλοις Πυθικοῖς

χρηστηρίοις θνάσκοντα γέννας ἄτερ σώζειν πόλιν...
\end{quotation}
\paragraph{}
b. Choeph. 1034 ff. sagt Orestes zum Chor:
\begin{quotation}
καὶ νῦν ὁρᾶτέ μ᾽, ὡς παρεσκευασμένος

ξὺν τῷδε θαλλῷ καὶ στέφει προσίξομαι

μεσόμφαλόν θ᾽ ἴδρυμα, Λοξίου πέδον,

πυρός τε φέγγος ἄφθιτον κεκλημένον...
\end{quotation}
\paragraph{}
Mit πυρὸς φέγγος ἄφθιτον scheint die Hestia des Tempels gemeint zu sein, auf der auch nach den Darstellungen mehrerer Vasenbilder Orestes sitzt.

c. Eumenid. 39 ff. sagt die durch die Eumeniden erschreckte Pythias:
\begin{quotation}
ἐγὼ μὲν ἕρπω πρὸς πολυστεφῇ μυχόν·\footnote{Unter dem πολυστεφὴς μυχός kann kaum etwas anderes zu verstehen sein als das Adyton des Tempels.}

ὁρῶ δ᾽ ἐπ᾽ ὀμφαλῷ μὲν ἄνδρα θεομυσῆ

ἕδραν ἔχοντα προστρόκαιον, αἵματι

στάζοντα χεῖρας καὶ νεοσπαδὲς ξίφος...
\end{quotation}
\paragraph{} 
Da Vasenbilder den Orestes öfters am Omphalos sitzend darstellen, so ist hier ὀ. wohl am besten als Nabelstein zu fassen. Die Pythia muss also, um zum Dreifuß im Adyton zu gelangen, am Nabelstein vorübergehen, was für die Beurteilung von dessen Lage im Tempel nicht unwichtig ist.

d. ebenda v. 166 Kirchh. singt der Eumenidenchor:
\begin{quotation}
πάρεστι γᾶς ὀμφαλόν

προσδρακεῖν αἰμάτων

βλοσυρὸν ἀρόμενον ἄγος ἔχειν.
\end{quotation}
\paragraph{}
Schol. τὸν θρόνον, οὗ Ὀρέστης καθῆστο, πάρεστιν ὁρᾶν φονολιβῆ ἀπὸ ποδῶν ἕως κεφαλῆς ὅλον. τὸ δὲ γᾶς ὀμφαλὸν ἀντὶ τοῦ τὸν ἐν Πυθοῖ... ὥστε τὸν ὀμφαλὸν (so Weil für das ὀφθαλμόν der Hss.; vgl. ob. Anm. 62) ἔχειν αἱμάτων ἄγος ἐπαίροντα.

Aufgeführt sind die beiden zu der gleichen Trilogie gehörigen Stücke bekanntlich 458 v. Chr.

5. Sophokles.

a. Oed. Rex 476 ff. (Chorgesang):
\begin{quotation}
φοιτᾷ γὰρ [der flüchtige Mörder] ὑπ᾽ ἀγρίαν

ὕλαν ἀνά τ᾽ ἄντρα καὶ

πέτρας ἅτε ταῦρος,

μέλεος μελέῳ ποδὶ χηρεύων,

τὰ μεσόμφαλα γᾶς ἀπονοσφίζων

μαντεῖα...
\end{quotation}
\paragraph{}
Schol. ὅτε δὲ ἡ Πυθὼ μεσόμφαλος, δηλοῖ καὶ ἡ περὶ τοὺς ἀετοὺς ἱστορία καὶ ὅτι χρύσεοι ἀετοὶ διὰ τοῦτο ἀνάκεινται, καὶ ὅτι τούτου χάριν ὁ Ζεὺς ἐκεῖσε τὸ μαντεῖον ἱδρύσατο.

b. Oedip. Rex 897 ff. (Chorgesang):
\begin{quotation}
οὐκ ἔτι τὸν ἄθικτον εἶμι γᾶς ἐπ᾿ ὀμφαλὸν σέβων

οὐδ᾽ ἐς τὸν Ἀβαῖσι ναὸν

οὐδὲ τὰν Ὀλυμπίαν,

εἰ μὴ τάδε χειρόδεικτα

πᾶσιν ἁρμόσει βροτοῖς.
\end{quotation}
\paragraph{}
Schol. Οὐκέτι ἄπειμι πρὸς τὸν Ἀπόλλωνα, οὐδὲ πρὸς τὸν ἄχραντον καὶ ἀπροσπέλαστον αὐτοῦ νεών.\footnote{Die Beziehung von ὀμφαλός auf den Tempel zu Delphi ist auch sonst die gewöhnliche (vgl. z. B. Suid. s. v. Γῆς ὀμφ.), jedoch fragt es sich sehr, ob dieser ἄθικτος genannt werden kann, da er doch allgemein zugänglich war. Ich glaube daher, dass Bötticher in seinem Berliner Winckelmannsprogramm über den Omphalos (1859) S. 9 Recht hat, wenn er hier unter ἄθικτος ὀ. den heiligen Nabelstein im Tempel zu Delphi versteht, der mit Binden verdeckt und dadurch vor allen profanen Berührungen geschützt war.}

6. Euripides.

a. Ion 5:
\begin{quotation}
ΕΡΜ:

ἥκω δὲ Δελφῶν τήνδε γῆν, ἵν᾽ ὀμφαλὸν

μέσον καθίζων Φοῖβος ὑμνῳδεῖ βροτοῖς

τά τ᾽ ὄντα καὶ μέλλοντα θεσπίζων ἀεί.
\end{quotation}
\paragraph{}
Man kann zweifelhaft sein, ob hier unter ὀμφ. μέσος, wie Wieseler (Gött. Gel. Anz. 1860 S. 167) meint, der delphische Tempel oder der eigentliche Nabelstein daselbst, auf dem der Gott oft sitzend dargestellt wird (s. unten) zu verstehen ist. Mir scheint die letztere Bedeutung annehmbarer.

b. Ion 222:
\begin{quotation}
ΧΟΡΟΣ:

ἆρ᾽ ὄντως μέσον ὀμφαλὸν

γᾶς Φοίβου κατέχει δόμος;

ΙΩΝ:

στέμμασί γ᾽ ἐνδυτόν, ἀμφὶ δὲ Γοργόνες.

ΧΟΡΟΣ:

οὕτω καὶ φάτις αὐδᾷ.
\end{quotation}
\paragraph{}
Dieser schwierigen Stelle hat neuerdings Studniczka eine lehrreiche Studie gewidmet (Hermes 37 (1902) S. 258 ff.), dem ich hinsichtlich der Beanstandung des verderbten Ausdrucks Γοργόνες vollkommen beistimme. Mit Recht macht St. (S. 262) vor allem gegen die Anbringung von Gorgonenbildern am Omphalos den Umstand geltend, dass sonst `von solch äußerst bedeutsamer Umgebung des Erdnabels kein Sterbenswörtchen verlaute'\footnote{An kleine am ἀγρηνόν wie an der damit identischen [?] αἰγίς angebrachte Gorgonenmasken denken Miss J. Harrison (Bull. de Corr. Hell. 24 [1900] p. 261 f.) und G. Karo in seinem Artikel Omphalos im Dict. des ant. 4 1 p. 199 b.} und dass von Pindar (s. ob) und anderen Zeugen vielmehr zwei goldene Adler zur Rechten und Linken des Nabelsteins bezeugt würden. Auch die Annahme G. Hermanns und Verrals, dass die `Gorgonen' des Euripides mit den χρύσεαι Κηληδόνες Pindars (fr. 30 Bergk 2) oder gar mit den beiden Moirenstatuen am Poseidonaltar des Tempels identisch seien, weist Studniczka mit überzeugenden Gründen zurück. Wenn er aber S. 269 zusammen mit C. Robert v. 224 zu verbessern sucht:
\begin{quotation}
στέμμασί γ᾽ ἐνδυτόν, ἀμφὶ δὲ γοργ<ὼ>

<χρυσοφαέννω Διὸς οἰωνώ>,
\end{quotation}
\paragraph{}
so gestehe ich offen, dass mir diese Emendation schon wegen ihrer zu großen Kühnheit nicht recht einleuchtet. Zwar nehme auch ich an, dass noch zur Zeit des Euripides die beiden goldenen Zeusadler am Omphalos zu sehen waren (s. oben S. 56 ff.), ob sie aber mit Notwendigkeit hier erwähnt gedacht werden müssen, bezweifle ich im Hinblick auf die meisten bis jetzt angeführten Zeugnisse durchaus. Ich nehme vielmehr eine viel leichtere Verderbnis an, indem ich statt ἀμφὶ δὲ Γοργόνες lese:
\begin{quotation}
ἀμφὶ δὲ γνώμονες
\end{quotation}
\paragraph{}
und diesen leichtverständlichen Ausdruck\footnote{Zum Verständnis von γνώμονες vgl. auch folgende Stellen: Aesch. Agam. 1130 ΧΟ. Οὐ κομπάσαιμ᾽ ἂν θεσφάτων γνώμων ἄκρος || εἶναι, κακῷ δὲ τῷ προσεικάζω τάδε (dies bezieht sich auf die unmittelbar vorausgehenden Prophezeiungen der Kassandra). --- Eurip. Phoen. 1757 f. ὦ πάτρας κλεινῆς πολῖται, λεύσσετ᾽, Οἰδίπους ὅδε || ὃς τὰ κλείν᾽ αἰνίγματ᾽ ἔγνω καὶ μέγιστος ἦν ἀνὴρ. --- Hippol. 346: οὐ μάντις εἰμὶ τἀφανῆ γνῶναι σαφῶς. --- Hesych. γνώμων· συνετὸς Σοφοκλῆς (= fr. 931 N. 1). --- S. auch Thukyd. 1, 138, 3. Lys. π. σηκοῦ 25. Xen. Mem. 1, 4, 5. Solon fr. 16 B. 2 (γνωμοσύνη). Et. M. s. v. --- Übrigens nehmen auch Karo (b. Daremberg-Saglio, Dict. d. ant. s. v. Omphalos p. 199 Anm. 11) und Svoronos, Journ. Internat. d. archéol. numism. 1911 [13] p. 313 Anstoß an der überkühnen Emendation Roberts. Svoronos will lesen γοργῶπες, was die Adler bezeichnen soll. Ich vermisse für diese Möglichkeit die nötigen Belege.} auf den Priester (προφήτης) und die fünf delphischen ὄσιοι beziehe, die regelmäßig der auf dem Dreifuß sitzenden Pythia assistierten, um deren oft undeutliche und unverständliche Worte zu interpretieren, d. h. verständlich zu machen. Zur weiteren Begründung dieser meiner Emendation berufe ich mich auf folgende Zeugnisse:

Ion 414 ff:
\begin{quotation}
ΙΩΝ:

ἡμεῖς τά γ᾽ ἔξω, τῶν ἔσω δ᾽ ἄλλοις μέλει,

οἵ πλησίον θάσσουσι τρίποδος, ὦ ξένε,

Δελφῶν ἀριστῆς, οὃς ἐκλήρωσεν πάλος.
\end{quotation}
\paragraph{}
Plut. Q. Gr. 9: Πέντε δέ εἰσιν Ὅσιοι διὰ βίου, καὶ τὰ πολλὰ μετὰ τῶν προφητῶν δρῶσιν οὗτοι, καὶ συνιερουργοῦσιν, ἅτε γεγονέναι δοκοῦντες ἀπὸ Δευκαλίωνος. --- Plut. de def. or. 51: Κατέβη μὲν [ἡ Πυθιὰς] εἰς τὸ μαντεῖον, ὥς φασιν, ἄκουσα καὶ ἀπρόθυμος, εὐθὺς δὲ περὶ τὰς πρώτας ἀποκρίσεις ἦν καταφανὴς τῇ τραχύτητι τῆς φωνῆς οὐκ ἀναφέρουσα, δίκην νεὼς ἐπειγομένης, ἀλάλου καὶ κακοῦ πνεύματος οὖσα πλήρης· τέλος δὲ παντάπασιν ἐκταραχθεῖσα καὶ μετὰ κραυγῆς φοβερᾶς φερομένη πρὸς τὴν ἔξοδον ἔρριψεν ἑαυτήν· ὥστε φυγεῖν μὴ μόνον τοὺς θεοπρόπους, ἀλλὰ καὶ τὸν προφήτην Νίκανδρον καὶ τοὺς παρόντας τῶν ὁσίων.

Habe ich also mit meiner Vermutung Recht, so würde sich Euripides hier den ursprünglichen Omphalos nicht in der von Pomtow angenommenen `aedicula des Omphalos' unweit der Hestia in der Cella des späteren Tempels (vgl. Pomtow im Philologus 1912 S. 59 ff.), sondern vielmehr im Adyton unmittelbar am χάσμα γῆς und in der Nähe des Dreifußes denken, eine Annahme, die im Hinblick darauf, dass es sich hier nicht um die historischen Verhältnisse des 4. Jahrhunderts, sondern um die der älteren Zeit handelt, wohl gerechtfertigt erscheinen dürfte.\footnote{Dass Euripides von der Einrichtung des delphischen Tempels keine ganz klare Vorstellung hatte, geht auch aus der folgenden Stelle (Ion 461 f.) hervor, wonach die delphische ἑστία neben (παρά) dem Orakeldreifuß gestanden und Orakel gespendet haben soll. Vgl. auch die Unklarheit des Pausanias hinsichtlich des eigentlichen Omphalos (10, 16, 3) und darüber Pomtow a. a. O. S. 59 f.}

c. Ion 461 f.
\begin{quotation}
ΧΟ:

Φοιβήιος ἔνθα γᾶς

μεσσόμφαλος ἑστία

παρὰ χορευομένῳ\footnote{Wieseler (Gött. Gel. Anz. 1860 S. 167) will hier περιχορευομένῳ τρίποδι lesen und den Ausdruck ἑστία von dem ganzen Tempel verstehen, nicht von dem Herde im Adyton, `auf welchem sich der Omphalos befand' [??].} τρίποδι

μαντεύματα κραίνει.
\end{quotation}
\paragraph{}
Diese Verse sind deshalb von ziemlicher Wichtigkeit für die Omphalosfrage, weil hauptsächlich aus ihnen sowie aus einem hocharchaischen den Kampf um Troilos darstellenden Vasengemälde in München (O. Jahn nr. 124, wo ein apollinischer Altar von `Omphalos'-form ausdrücklich als $\svgAAL$ bezeichnet ist; s. unt. Kap. 4 B), endlich aus dem Umstande, dass der Omphalos von Delphi öfters als auf einer Basis stehend dargestellt wird, Wieseler den Schluss gezogen hat, dass in Delphi Omphalos und Hestia identisch seien, also der Omphalos als ein Symbol der Hestia betrachtet werden müsse.\footnote{Vgl. Wieseler, Annali d. I. 29 (1857) p. 160 ff. Derselbe in Jahns Jahrbb. 75 (1857) S. 678 ff. und in den Gött. Gel. Anz. 1860 S. 161 ff. --- An Wieseler hat sich trotz Preuners Darlegungen neuerdings angeschlossen Baumeister, Denkmäler 1, 601.} Gewiss mit Recht hat dagegen Preuner in seinem Buche Hestia-Vesta (vgl. auch dessen Artikel `Hestia' im Lexikon d. Mythol. 1 Sp. 2640) außer anderen Gründen vor allem die absolute Unmöglichkeit geltend gemacht, einen mit (brennbaren) Wollenbinden bekleideten Omphalos mit dem auf der delphischen Hestia brennenden ewigen Feuer in unmittelbaren Zusammenhang zu bringen. Folglich ist nach Preuner (im Lexikon d. Mythol. a. a. O.) unter der Φοιβήιος γᾶς μεσσόμφαλος ἑστία auf keinen Fall der Nabelstein, sondern vielmehr die im Tempel von Delphi stehende κοινὴ ἑστία τῆς Ἑλλάδος zu verstehen, `deren ewiges Feuer zugleich zu Ehren Apollons brannte' (Preuner a. a. O. Sp. 2639, 39 ff.; vgl. auch Pomtow a. a. O. S. 58 f. u. Anm. 19, sowie den homer. Hymn. auf Hestia 1 ff.). Μεσσόμφαλος wird diese ἑστία natürlich nur deshalb genannt, weil sie in einem geheiligten Raume stand, der nach delphischer Lehre den Mittelpunkt der bewohnten Erde bildete.\footnote{Daneben kann freilich auch ἑστία in übertragener Bedeutung von δόμος (νεώς, ἄδυτον) in Betracht kommen. Vgl. unten Orest. 327 ff. μεσόμφαλοι μυχοί. Iph. T. 1251 ff. μέσον γᾶς μέλαθρον etc.}

Die Aufführung des Ion fällt wahrscheinlich in die Zeit zwischen 421 und 413 (A. Dieterich).

d. Medea 667:
\begin{quotation}
ΜΗΔ. Τί δ᾽ ὀμφαλὸν γῆς θεσπιῳφδὸν ἐστάλης;

ΑΙΓ. παίδων ἐρευνῶν σπέρμ᾽ ὅπως γένοιτό μοι.
\end{quotation}
\paragraph{}
Aufgeführt wurde die Medea Ol. 87, 1 = 431 v. Chr.

e. Orestes 327 ff.:
\begin{quotation}
ΧΟ...

φεῦ μόχθων,

οἵων, ὦ τάλας, ὀρεχθεὶς ἔρρεις,

τρίποδος ἄπο φάτιν, ἃν ὁ Φοῖβος

ἔλακεν ἔλακε, δεξάμενος ἀνὰ δάπεδον,

ἵνα μεσόμφαλοι λέγονται μυχοί.
\end{quotation}
\paragraph{}
Schol. ὀμφαλὸς λέγεται ἡ Πυθὼ παρὰ τὸ τὰς ὀμφὰς τὰς ὑπὸ θεοῦ χρηστηριαζομένας λέγειν ἢ παρὰ τὸ εἶναι ἐν μέσῳ τῆς οἰκουμένης, λέγεται γὰρ τὸν Δία μαθεῖν βουλόμενον τὸ μέσον τῆς γῆς, δύο ἀετοὺς ἰσοταχεῖς ἀφεῖναι, τὸν μὲν ἀπὸ δύσεως, τὸν δὲ ἀπὸ ἀνατολῆς, καὶ ἐκεῖ αὐτοὺς συναντῆσαι, ὅθεν ὀμφαλὸς ἐκλήθη. ἀνακεῖσθαί τε χρυσοῦς ἀετούς φασι τῶν μυθευομένων ἀετῶν ὑπομνήματα.\footnote{Vgl. oben S. 55 ff. Anm. 103 und die in vieler Hinsicht ganz ähnliche Sage von den βωμοὶ Φιλαίνων b. Sall. Iug. 79. Val. Max. 5, 6, Ext. 4. Strab. 3, 171.} Vgl. zu der Ableitung von ὀμφή unten (Anm. 126) Cornutus de nat. deor. 196 Os.

f. Orestes 591:
\begin{quotation}
ΟΡΕΣΤ.

ὁρᾷς, Ἀπόλλων ὃς μεσομφάλους ἕδρας

ναίων βροτοῖσι στόμα νέμει σαφέστατον.
\end{quotation}
\paragraph{}
Die Aufführung des Orestes fällt ins Jahr 408.

g. Iphig. Taur. 1251 ff.:
\begin{quotation}
ΧΟ.

...ὦ Φοῖβε, μαν-

τείων δ᾽ ἐπέβας ζαθέων,

τρίποδί τ᾽ ἐν χρυσέῳ

θάσσεις, ἐν ἀψευδεῖ θρόνῳ.

μαντείας βροτοῖς

θεσφάτων νέμων

ἀδύτων ὕπο, Κασταλίας ῥεέθρων

γείτων, μέσον γᾶς ἔχων μέλαθρον.
\end{quotation}
\paragraph{}
h. Phoeniss. 232 ff.:
\begin{quotation}
ΧΟ.

ζάθεά τ᾽ ἄντρα δράκοντος οὔ-

ρειαί τε σκοπιαὶ θεῶν

νιφόβολόν τ᾽ ὄρος ἱερόν, εἱ-

λίσσων ἀθανάτας θεοῦ

χορὸς γενοίμαν ἄφοβος

παρὰ μεσόμφαλα γύαλα Φοί-

βου Δίρκαν προλιποῦσα.
\end{quotation}
\paragraph{}
Aufgeführt zwischen 411 und 408 (Schol. Arist. ran. 53).

7. Platon.

De republ. 4, 5 p. 427 e: οὗτος γὰρ δήπον ὁ θεὸς [Ἀπόλλων] περὶ τὰ τοιαῦτα πᾶσιν ἀνθρώποις πάτριος ἐξηγητὴς ἐν μέσῳ τῆς γῆς ἐπὶ ὀμφαλοῦ καθήμενος ἐξηγεῖται. Vgl. zum Verständnis dieser Worte die unten angeführten Bilder und Münzen.

8. "`In einer Bauinschrift des 4. Jahrhunderts ist von einer πρόστασις ἡ πρὸ τοῦ ὀμφαλοῦ und von ἔργον τὸ περὶ τὸν ὀμφαλόν die Rede, beides sicher im Tempel, aber nicht genau lokalisiert: Acad. d. inscr. Comptes rendus 23 (1895) S. 335 mit Note"' (Studniczka im Hermes 37 (1902) S. 263 Anm. 5). S. jetzt darüber Pomtow a. a. O. S. 61, der annimmt, dass in dem Tempel des 4. Jahrh. der Omphalos in einer aedicula auf der einen Seite der Cella (s. S. 69), gegenüber der ἑστία, gestanden habe (vgl. Varro l. l. 7, 17 in aede ad latus).

Ob Pomtow a. a. O. S. 61 auch die leider etwas zusammenhangslosen Worte aus dem Paian des Philodamos (4. Jahrh.) ναὸν... νεοχρύσεον χρυσέοις τύποις... ἀργαίνοντε αὐτόχθονι κόσμῳ mit Recht auf den Omphalos und seine beiden goldenen Adler (die von den Phokiern eingeschmolzen waren, jetzt aber erneuert worden sein sollen) bezieht, ist mir schon wegen des ἀργαίνοντε = λευκαίνοντε (was von weißen, nicht von goldenen Gegenständen gilt) etwas zweifelhaft.

9. Römischer Tragiker des 3. oder 2. vorchr. Jahrhunderts (Frgm. tr. inc. inc. 19 f. R.).

Cicero (de divin. 2, 56, 115) und Varro de l. l. 7, 17 (p. 97 Götz-Schöll) zitieren folgende Verse:
\begin{quotation}
O sancte Apollo, qui umbilicum certum terrarum obtines,

Unde superstitiose primum saeva evasit vox fera.
\end{quotation}
\paragraph{}
Das Beiwort certus bezieht man wohl am besten auf die Tatsache, dass auch andere Orte (z. B. Branchidai, Paphos usw.) die Ehre beanspruchten, Mittelpunkt der Erde zu sein, denen gegenüber Delphi hier als der in Wahrheit einzig berechtigte Erdnabel hingestellt werden soll.

10. Der zweiten Hälfte des dritten vorchristlichen Jahrhunderts gehört der kürzlich in Delphi aufgefundene Hestiahymnus des Aristonoos an, dessen Anfang in der von Pomtow (Delphica 3 S. 248 ff.) mitgeteilten Fassung so lautet:
\begin{quotation}
[Ἱ]ερὰν ἱερῶν ἄνασσαν Ἑστίαν [ὑ-]

μνήσομεν, ἃ καὶ Ὀλύμπου καὶ υ[...]\footnote{Hier ergänzt ein Ungenannter: εὐ[ρέας], Br. Keil [τ]ύ[λον]. Mir scheint ein Epitheton zu γαίας zu fehlen. μεσόμφαλον fasse ich als Substantiv (= Mittelpunkt, Zentrum); vgl. Batrach. 129; μεσομφάλιον Poll. 2, 169; μεσόλοφον, μεσόριον, μεσομήνιον etc. --- Zum μεσόμφαλον Ὀλύμπου verweist P. Maas auf Philolaos b. Diels, Vorsokrat. 32 A 16 S. 237: Φιλόλαος πῦρ ἐν μέσῳ περὶ τὸ κέντρον ὅπερ ἑστίαν τοῦ παντὸς καλεῖ.}

γαίας μεσόμφαλον ἀεὶ Πυθίαν πα[ρὰ]

[δ]άφναν κατέχουσα ναὸν ἀν ἱ[ε-]

[ρ]ὸν Φοίβου χορεύεις τερπομένα τ[ε]

χολῶν θεσπίσμασιν κ. τ. λ.
\end{quotation}
\paragraph{}
11. Varro l. l. 7, 17 (p. 97 ed. Götz-Schöll):

`O sancte Apollo, qui umbilicum certum terrarum obtines.' Umbilicum dictum aiunt ab umbilico nostro, quod is medius locus sit terrarum, ut umbilicus in nobis; quod utrumque est falsum: neque hic locus est terrarum medius neque noster umbilicus est hominis medius. Itaque pingitur quae vocatur <ἀντί>χθων Πυθαγόρα [vgl. Philolaus 247, 17 D.; Cic. Tusc. 1, 68 f.] ut media caeli ac terrae linea\footnote{= ἄξων (πόλος): s. ob. S. 41 f. Anm. 79 ff.} ducatur infra umbilicum per id quo discernitur, homo mas an femina sit, ubi ortus humanus, similis ut in mundo: ibi enim omnia nascuntur in medio, quod terra mundi media. Praeterea si quod medium id est umbilicus, * ut pila terrae, non Delphi medium. Et terrae medium * non hoc sed quod vocant Delphis, in aede ad latus\footnote{Lobeck, der die schwierige Stelle Aglaoph. p. 1002 ff. behandelt und darauf hinweist, dass statt `\emph{ad latus}' auch gelesen wird `[\emph{foramen}] \emph{adlatum, allatum}, (\emph{ablatum}' [= \emph{lanatum}?], vermutet a. a. O. p. 1004 \emph{arquatum} (od. \emph{arcuatum}) \emph{quiddam}' d. h. ein Ding von gewölbter Gestalt, was allerdings die Form, die der Nabelstein, nach den maßgebendsten Monumenten hatte, außerordentlich treffend bezeichnen würde.} est quiddam ut thesauri\footnote{Gewöhnlich fasst man hier `thesaurus' im Sinne von `Kuppelbau' oder `Grabgewölbe,' doch macht G. Karo im Artikel Omphalos des Dict. des antiq. 4, 1 p. 198a sehr wahrscheinlich, dass Varro darunter die "`tirelires ovoïdes d'argile"' = `Sparbüchsen') verstanden hat, die in Italien einst (ebenso wie noch heute) gebräuchlich waren (vgl. Graeven, Arch. Jahrb. 1901 S. 160).} specie, quod Graeci vocant ὀμφαλόν, quem Pythonos [-is?]\footnote{Pomtow, Philolog. 1912 S. 60 A. 21.} aiunt esse tumulum [-os Hss.]; ab eo nostri interpretes ὀμφαλόν umbilicum dixerunt.

Der Sinn der leider stark verderbten Sätze scheint folgender zu sein. Vor allem bekämpft der skeptische Varro (wohl nach dem Vorgang griechischer Philosophen; vgl. oben das über Epimenides' Zweifel an der μεσομφαλία Delphis Gesagte!) die delphische Lehre vom ὀμφαλός im Apollontempel als dem anerkannten (certus) Mittelpunkt der Erde, wie sie namentlich auch der von ihm (und Cicero) zitierte römische Tragiker vertreten hatte. Zur Begründung seiner Ansicht nun scheint sich Varro auf Dreierlei zu berufen: erstens auf die vor allen von Pythagoras und später von den Stoikern (Posidonius) ausgesprochene Lehre von der Kugelgestalt der Erde (pila terrae), die deshalb wohl eine Achse, nicht aber auf ihrer Oberfläche einen Mittelpunkt haben könne; zweitens auf die auch sonst in der späteren Zeit mehrfach verbreitete Theorie, dass der sogenannte delphische, wie ein `thesaurus' (d. h. entweder Grabgewölbe, Grabkuppel oder Sparbüchse: s. ob. Anm. 121) gestaltete Omphalos nicht den Mittelpunkt der Erde, sondern vielmehr das Grab des Python darstelle.\footnote{Vgl. Hesych. s. v. Τοξίου βουνός· τοῦ Ἀπόλλωνος τοῦ ἐν Σικυῶνι. βέλτιον δὲ ἀκούειν τὴν ἐν Δελφοῖς Νάπην λεγομένην· ἐκεῖ γὰρ καὶ ὁ δράκων κατετοξεύθη. καὶ ὁ ὀμφαλὸς τῆς γῆς τάφος ἐστι τοῦ Πύθωνος. Vgl. dazu Meineke, Callim. hymn. et epigr. p. 154 f. --- Tatian c. Gr. 8, 251 identifiziert den ὀ. sogar mit dem nach Philochoros (fr. 22 f.) unmittelbar neben der goldenen Apollonstatue des Tempels befindlichen Grabe des Dionysos, indem er behauptet: ὁ δὲ ὀμφ. τάφος ἐστὶ Διονύσου. Vgl. Preller-Robert, Gr. Myth. 1, 686 f. Lobeck, Agl. 572 ff. etc. Rohde, Psyche 2 1, 132, 2.} Drittens endlich auf die Tatsache, dass der Nabel nicht genau die Mitte des menschlichen Leibes bezeichnet.

Wenn neuerdings ein so ausgezeichneter Forscher wie E. Rohde in seiner Psyche 2 1, 132 f. (ebenso Stengel, Gr. Kultusaltertümer 2 S. 65) die Behauptung ausgesprochen hat, dass die Ansicht, der delphische Nabelstein von der Form eines `tumulus' oder `thesaurus' habe in der Tat ursprünglich das Grab des Python dargestellt, und `zum "`Nabel"' d. h. Mittelpunkt der Erde hätten ihn erst Missverständnis und daraus hervorgesponnene Fabeln gemacht,' das Richtige treffe, so kann ich schon im Hinblick auf die angeführten Zeugnisse des Epimenides, Pindar usw. diese Deutung nur als höchst unwahrscheinlich bezeichnen.\footnote{Hierzu kommt jetzt noch der von Pomtow (Philologus 1912 S. 60 A. 21) gelieferte Nachweis, dass in römischer Zeit die Pythonsage willkürlich ausgeschmückt wurde. --- Hat vielleicht zu der Deutung des O. als `Pythonis tumulus' der Umstand beigetragen, dass er mehrfach von einer Schlange (= Python?) umringelt und hohl dargestellt wurde (s. unt.)?} Damit soll freilich nicht geleugnet werden, dass vielleicht in allerältester Zeit der möglicherweise als Baityl aufgefasste delphische Omphalos\footnote{S. jedoch unten Paus. 10, 16, 3.} eine andere Bedeutung hatte, als ihm von Pindar usw. zugeschrieben wird; welches aber diese seine ursprünglichste Bedeutung gewesen ist, kann mit unsern jetzigen Mitteln absolut nicht ausgemacht werden; das was allein sich feststellen lässt, ist die Tatsache, dass spätestens von Beginn des 5. vorchristlichen Jahrhunderts an Delphi von den Westgriechen für den Mittelpunkt der (scheibenförmigen) Erde und der dortige Nabelstein als äußeres Wahrzeichen dieser Anschauung erklärt wurde. Die Anschauung vom Omphalos als dem Grabe des Python ist wahrscheinlich erst dann entstanden, als man sich allgemein die Erde nicht mehr als runde Scheibe sondern mit Parmenides, Pythagoras und dem von mir neuentdeckten altionischen Kosmologen bei Ps.-Hippokr. π. ἑβδ. bereits als Kugel vorstellte, die als solche wohl eine Achse, aber auf ihrer Oberfläche keinen Mittelpunkt oder `Nabel' haben konnte.\footnote{Ähnlich wie Varro bestreitet auch der Stoiker Cornutus, De nat. deor. p. 196 Os. die zentrale Lage Delphis mit den Worten: ἐλέχθη δὲ καὶ ὁ τόπος [= Δελφοί] ὀμφαλὸς τῆς γῆς, οὐχ ὡς μεσαίτατος ὢν αὐτῆς, ἀλλ᾽ ἀπὸ τῆς ἀναδιδομένης ἐν αὐτῷ ὀμφῆς, ἥτις ἐστὶ θεία φωνή. Ja er versteigt sich dabei sogar zu einer neuen völlig unhaltbaren Deutung und Etymologie des Wortes ὀμφαλός. S. auch den Schol. zu Euripid. Or. 327 ff. (ob. S. 64). Diese unhaltbare Etymologie ist neuerdings wieder angenommen worden von der gelehrten und scharfsinnigen Miss J. Harrison (Bull. de Corr. Hell. 24 (1900) S. 258 f.: ὀ. = `le pierre qui parle'), vgl. G. Karo a. a. O.}

12. Strabo 9, p. 419: Ἡ μὲν οὖν ἐπὶ τὸ πλεῖον τιμὴ τῷ ἱερῷ τούτῳ διὰ τὸ χρηστήριον συνέβη δόξαντι ἀψευδεστάτῳ τῶν πάντων ὑπάρξαι, προσέλαβε δέ τι καὶ ἡ θέσις τοῦ τόπου. τῆς γὰρ Ἑλλάδος ἐν μέσῳ πώς ἐστι τῆς συμπάσης,\footnote{Vgl. Liv. 35, 18: Aetolos, qui umbilicum Graeciae incolerent (bezieht sich wohl auch auf Delphi, das die Ätoler im 3. Jahrh. eine Zeit lang okkupiert hatten); s. auch ib. 41, 23 (unten S. 70). Mit Strabo stimmt völlig überein Aristeides in seinem Panathenaikos b. Phot. bibl. p. 404, 2 Bekk.: τοσοῦτον παρελήλυθε τὸν ὀμφαλὸν τῆς γῆς καὶ τῆς Ἑλλάδος τοὺς Δελφούς.} τῆς τε ἐντὸς Ἰσθμοῦ καὶ τῆς ἐκτός, ἐνομίσθη δὲ καὶ τῆς οἰκουμένης, καὶ ἐκάλεσαν τῆς γῆς ὀμφαλόν, προσπλάσαντες καὶ μῦθον, ὃν φησι Πίνδαρος (s. ob. S. 56 u. 58), ὅτι συμπέσοιεν ἐνταῦθα οἱ ἀετοὶ οἱ ἀφεθέντες ὑπὸ τοῦ Διός, ὁ μὲν ἀπὸ τῆς δύσεως, ὁ δ᾽ ἀπὸ τῆς ἀνατολῆς· οἱ δὲ κόρακάς φασι. δείκνυται δὲ καὶ ὀμφαλός τις ἐν τῷ ναῷ τεταινιωμένος καὶ ἐπ᾽ αὐτῷ\footnote{Svoronos in seinem interessanten Aufsätze im Journ. Internat. d'archéol. numism. 13 (1911) S. 310 u. 316 will dieses ἐπ᾽ αὐτῷ nicht im Sinne von παρ᾽ αὐτῷ sondern vielmehr von ἐπ᾽ αὐτοῦ fassen, indem er annimmt, dass die beiden Adler ursprünglich nicht (wie auf dem unten zu besprechenden spartanischen und attischen Marmorrelief des 5. Jahrh.) neben, sondern wie auf dem Kyzikener Stäter auf dem Omphalos gesessen hätten und so auch wieder in der Zeit kurz vor Strabon ergänzt worden seien. Ich kann dieser Auffassung nicht beistimmen ausfolgenden Gründen: 1. Für Delphi sind nicht die Auffassungen der von Branchidai abhängigen Kyzikener, sondern die der Spartaner und Athener maßgebend; 2. das gleich anzuführende Zeugnis des Scholiasten zu Lucian beweist, dass die massiv-goldenen Adler später nicht durch gleichartige Bildwerke sondern durch Mosaiken ersetzt waren.} αἱ δύο εἰκόνες τοῦ μύθου.

Wir lernen aus dieser Notiz hauptsächlich zweierlei: erstens dass der Vorstellung vom delphischen Erdnabel eine ältere voranging, nach der Delphi --- was ja auch ganz richtig ist --- wegen seiner geographischen Lage zunächst nur als Mittelpunkt von Hellas aufgefasst wurde, und zweitens, dass zu Strabons Zeit die alten im phokischen Krieg (s. ob. S. 56) geraubten goldenen Adler durch ein neues Bildwerk ersetzt waren. Welcher Art dieses gewesen ist, lässt sich mit großer Wahrscheinlichkeit aus den Scholien zu Lukian de salt. 38 (= 4 p. 144 ed. Jacobitz) schließen,\footnote{Luc. de saltat. 38: καὶ τὰ διὰ μέσου μάλιστα ἴστω· Οὐρανοῦ τομὴν, Ἀφροδίτης γονάς, Τιτάνων μάχην... Δήλου πλάνην καὶ Λητοῦς ὠδῖνας καὶ Πύθωνος ἀναίρεσιν καὶ Τιτυοῦ ἐπιβουλὴν καὶ τὸ μέσον τῆς γῆς εὑρισκόμενον πτήσει τῶν ἀετῶν. Δευκαλίωνα ἐπὶ τούτους κ. τ. λ.} wo es heißt: λέγουσιν ἐν Δελφοῖς ὀμφαλὸν εἶναι ἐπὶ τοῦ ἐδάφους τοῦ νεὼ καὶ περὶ αὐτὸν αἰετώ (Hss. -όν) γεγράφθαι ἀπὸ συνθέσεως λίθων,\footnote{Damit ist also ein Mosaikbild gemeint, das an die Stelle der archaischen im phokischen Krieg (4. Jahrh.) geraubten goldenen Zeusadler getreten war. Vielleicht bezieht sich auf dieses Mosaik die delphische Bauinschrift des 4. Jahrhunderts, in der von einer πρόστασις ἡ πρὸ τοῦ ὀμφαλοῦ und von ἔργον τὸ περὶ τὸν ὀμφαλόν (im Tempel!) die Rede ist. Vgl. oben nr. 8 S. 65. Eine andere Möglichkeit der Erklärung ergibt sich, wenn man an die von Pomtow a. a. O. S. 59 ff. nachgewiesene (spätere) `aedicula des Omphalos' denkt.} καὶ τοῦτο ἔφασκον τὸ μέσον ἁπάσης τῆς γῆς. Die alten goldenen Adler waren demnach später durch entsprechende Mosaikbilder im Fußboden (ἔδαφος) ersetzt worden (vgl. Studniczka im Hermes 37 (1902) S. 264 A. 5). Übrigens ist Strabo außer Euripides (s. ob. S. 60 f.) der einzige antike Schriftsteller, welcher die Bekleidung des Omphalos mit Binden (ταινίαι) ausdrücklich bezeugt.

13. Livius 38, 48 lässt den Cn. Manlius Vulso, den Besieger der Galater, in seiner Rechtfertigungsrede vor dem Senate im J. 187 unter anderem sagen: Delphos quondam commune humani generis oraculum, umbilicum orbis terrarum, Galli spoliaverunt. --- Vgl. 41, 23 (Rede des Callicrates gegen Perseus): Inde transgressus Oetam, ut repente in medio umbilico Graeciae conspiceretur, Delphos escendit.

14. Auch Ovid gedenkt an zwei Stellen der Metamorphosen Delphis als des Mittelpunkts des Orbis terrarum. Vgl. 10, 167:
\begin{quotation}
Te (Hyacinthum) mens (Orphei) aut alios genitor dilexit, et orbe

In medio positi caruerunt praeside Delphi,

Dum Deus Eurotan immunitamque frequentat

Sparten...
\end{quotation}
\paragraph{}
15, 630:
\begin{quotation}
Auxilium caeleste petunt (Romani) mediamque tenentes

Orbis humum Delphos adeunt, oracula Phoebi.
\end{quotation}
\paragraph{}
15. Lucan. Phars. 5, 71 ff.
\begin{quotation}
Hesperio tantum quantum submotus Eoo

Cardine Parnasus gemino petit aethera colle

Mons Phoebo Bromioque sacer...

\bigskip

Hoc solum fluctu terras mergente cacumen

Emicuit pontoque fuit discrimen et astris.

Tu quoque vix summam seductus ab aequore rupem

Extuleras unaque iugo, Parnase, latebas.
\end{quotation}
\paragraph{}
Schol. Bernens. ed. Usener p. 156: Iovis scire volens quae pars terrarum media esset, alteram ab oriente aquilam, alteram ab occidente misit, ut pari volatu adversum tendentes iter ibi consisterent, ubi obviae sibi factae essent. hoc in loco occurrerunt ubi Delphicum est oraculum, eoque umbilicus terrae dictus est. In diluvio propter illud divinum specus hoc cacumen solum eminuit... Parnasus autem mundi dictus umbilicus... --- ib. p. 157, 21 ff.: Forsan scilicet illud, inquit, antrum terrae umbilicus est, quod inde aër exiens caelo conexus terras suspendit [wie eine Nabelschnur?]. huic loco congruit quod Virgilius [Aen. 6, 726 u. Georg. 4, 221 ff.] dixit: `spiritus intus alit' et `deum namque ire per omnes terrasque tractusque maris.'

Wir ersehen daraus, dass neben der Anschauung, dass Delphi oder genauer gesagt dessen Apollontempel und der darin befindliche Nabelstein der Mittelpunkt von Hellas und der Erde sei, auch noch die andere bestand, wonach dem allerdings von den meisten hochgelegenen Punkten Griechenlands aus sichtbaren Parnass diese Bedeutung zukomme. Wie es scheint, bezog man auf diese Bedeutung des Berges auch die bekannte Sage von der Deukalionischen Flut, wie aus den angeführten Versen Lucans hervorgehen dürfte. Bei dieser Gelegenheit möchte ich nicht unterlassen auf eine eigentümliche, nicht allzu junge, von Kallimachos und Statius überlieferte echt griechische Vorstellung hinzuweisen, wonach die Pythonschlange, die auf Bildwerken sonst den Nabelstein umringelnd dargestellt wird (s. unten!), bisweilen auch den Parnass mit 9 oder 7 Windungen umschlingend gedacht wurde. Vgl. Kallim. hy. in Del. 90 ff.
\begin{quotation}
Οὔπω μοι Πυθῶνι μέλει τριποδήιος ἕδρη,

Οὐδέ τί πω τέθνηκεν ὄφις μέγας, ἀλλ᾽ ἔτι κεῖνο

Θηρίον αἰνογένειον ἀπὸ Πλειστοῖο παρέρπον

Παρνησὸν νιφόεντα περιστέφει ἐννέα κύκλοις.
\end{quotation}
\paragraph{}
Statius Theb. 1, 561 ff. setzt an die Stelle des Parnass die Stadt Delphi:
\begin{quotation}
Postquam caerulei sinuosa volumina monstri,

Terrigenam Pythona, deus septem\footnote{Über die Vertauschung der beiden Zahlen 7 und 9 s. meine Abhandlungen `D. Sieben- u. Neunzahl im Kultus u. Mythus d. Griechen' S. 15 A. 39; S. 56 A. 131 u. `Enneadische Studien' S. 9 f.} orbibus atris

Amplexam Delphos squamisque annosa terentem

Robora, Castaliis dum fontibus ore trisulco

Fusus hiat nigro sitiens alimenta veneno,

Perculit, absumptis numerosa in vulnera telis,

Cirrhaeique dedit centum per iugera campi

Vix tandem explicitum, nova deinde piacula caedi

Perquirens nostri tecta haut opulenta Crotopi

Attigit...
\end{quotation}
\paragraph{}
Es scheint demnach die Frage gerechtfertigt, ob nicht mehrfach unter dem von einer Schlange umringelten apollinischen Omphalos eigentlich der als weithin sichtbarer Mittelpunkt Griechenlands und der Erde aufgefasste Parnass mit der an seinem Abhang gelegenen Stadt Delphi verstanden werden könne. Die Frage lässt sich gewiss nicht leicht beantworten, aber aufgeworfen werden muss sie in diesem Zusammenhang, und zwar hauptsächlich im Interesse der später zu betrachtenden einschlägigen Bildwerke, insbesondere gewisser Münzen.\footnote{Auch die charakteristische Rolle, welche der Parnass und Delphi in der Sage von der Deukalionischen Flut und der Neugeburt der Erde und ihrer Bewohner spielen --- man denke an die delphischen Hosier, die Nachkommen Deukalions --- lässt sich mit der Auffassung des Parnass als des erhabenen Mittelpunkts (Omphalos) der Erde leicht in Zusammenhang bringen. Unter diesem Gesichtspunkt könnte der Parnass die abgerissene Nabelschnur der Erde darstellen, und als sein verkleinertes (oft vom Pythondrachen mehrfach umschlungenes) Abbild wäre dann der delphische Nabelstein aufzufassen.}

16. Wenn Statius Theb. 1, 118 von einem `medius caeli Parnasus' redet, so soll damit offenbar dieser Berg als der die Mitte des unbeweglichen Fixsternhimmels wie der Erde bezeichnende Punkt oder als Teil der Weltachse (s. ob.) charakterisiert werden.

17. Von besonderer Wichtigkeit als eines Augenzeugen ist das Zeugnis des Pausanias (10, 16, 3) in seiner Beschreibung Delphis: Τὸν δὲ ὑπὸ Δελφῶν καλούμενον ὀμφαλόν, λίθου πεποιημένον λευκοῦ, τοῦτο εἶναι τὸ ἐν μέσῳ γῆς πάσης αὐτοί τε λέγουσιν οἱ Δελφοί καὶ ἐν ὠδῇ τινὶ Πίνδαρος ὁμολογοῦντά σφισιν ἐποίησε (s. ob. S. 58). Wir ersehen daraus nicht bloß, dass Pindar in einer leider verloren gegangenen Ode den Mythus von den zur Bestimmung des Erdnabels ausgesandten Adlern verherrlicht, sondern auch --- was besonders beachtenswert erscheint --- sich darin an eine alte von den Delphiern selbst erfundene oder geglaubte Überlieferung angeschlossen hatte, der gegenüber die viel spätere von Varro (s. ob. S. 66 nr. 11) und Hesychius vertretene Deutung des Nabelsteins als Pythongrabes kaum in Betracht kommen kann. Doch enthält diese Beschreibung eine sehr große Schwierigkeit, insofern nach P. der Nabelstein mitten unter den außerhalb des Tempels stehenden Thesauren und Weihgeschenken sich befand, während er doch --- wenigstens in der Zeit vor Pausanias --- unzweifelhaft nach allen sonstigen Zeugnissen zu dem Inventar der Tempelcella oder des Adytons gehörte. Um diese Schwierigkeit zu beseitigen, gibt es, soviel ich sehe, drei Wege: erstens die Annahme, dass die den Omphalos betreffende Notiz des Pausanias ursprünglich eine nachträglich gemachte Randbemerkung war, die infolge eines Versehens später an die unrichtige Stelle geriet. Oder aber, es wäre an eine kurz vor der Zeit des Pausanias beschlossene und ausgeführte Verlegung des Nabelsteins aus dem Innern des Tempels in dessen äußere Umgebung zu denken.\footnote{Dies nehmen tatsächlich Dießen, O. Müller und Frazer Paus. 5 S. 317 an; vgl. Studniczka a. a. O. S. 263 A. 1.} Oder man könnte auch, mit Studniczka a. a. O. S. 263, annehmen, `dass der Marmoromphalos draußen vor dem Tempel, welchen Pausanias für den echten hielt, nur eine von den Nachbildungen gewesen sei, deren sich auch in Delphi gefunden haben' (Bull. Corr. Hell. 18 1894 S. 180. Frazer Pausan. 5 S. 318).\footnote{Wir werden später sehen, dass solche `Omphaloi' auch anderwärts, z. B. in der Nekropole von Milet, also bei Branchidai (das ebenso wie Delphi sich rühmte der Nabel der Erde zu sein) vorkommen.} Mir persönlich will die zweite Alternative im Hinblick auf den sonst in der griechischen Religion überall wahrnehmbaren Konservativismus wenig glaublich erscheinen, so dass ich mich bis auf weiteres für Studniczka oder für die Annahme eines Versehens des Pausanias oder eines seiner Abschreiber entscheide.\footnote{Gelöst ist nunmehr die Streitfrage, und zwar endgültig, von Pomtow a. a. O. S. 59 f., der nachweist, dass Pausanias den eigentlichen O. des späteren Tempels ignoriert, dafür aber einen neuerdings vor dem Tempel in situ gefundenen großen Marmoromphalos (s. unten S. 81 u. Taf. 6 Fig. 1) nennt.} Im übrigen mache ich darauf aufmerksam, dass, wenn der von Pausanias gesehene Nabelstein wirklich von weißer Farbe, also doch wohl von Marmor war, die Meinung, er sei ursprünglich ein Baityl gewesen, hinfällig werden dürfte, weil solche Meteorsteine, soviel ich weiß, niemals eine weiße (helle), sondern wegen des in ihnen befindlichen Eisens stets eine dunkle Färbung aufweisen.

18. Den ersten Jahrhunderten nach Chr. gehört der mehrfach aus älteren Quellen schöpfende Geograph Agathemeros an, der in seinem ersten Kapitel (περὶ τῆς τῶν παλαιῶν γεωγραφίας = 1, 2) bemerkt: Οἱ μὲν οὖν παλαιοὶ τὴν οἰκουμένην ἔγραφον στρογγύλην, μέσην δὲ κεῖσθαι τὴν Ἑλλάδα, καὶ ταύτης Δελφούς, τὸν ὀμφαλὸν γὰρ ἔχειν τῆς γῆς.

Dass hier unter den παλαιοί nicht die von Delphi völlig unabhängigen, ja sogar dazu im Gegensatze stehenden altionischen Geographen (Anaximandros und Hekataios, die vielmehr Milet und Branchidai als Erdnabel betrachtet haben), sondern die nach der Zerstörung von Milet und Branchidai durch die Perser maßgebenden älteren Geographen der Griechen zu verstehen sind, ist in dem vorigen Abschnitt gezeigt worden.

19. Nicht unwichtig ist auch das nunmehr hier anzuführende Zeugnis des Claudianus im Prolog des Hymnus auf das Konsulat des Fl. Mallius Theodorus (16, 11 ff.):
\begin{quotation}
Jupiter, ut perhibent, spatium quum discere vellet

\hspace*{5mm}Naturae, regni nescius ipse sui,

Armigeros utrimque duos aequalibus alis

\hspace*{5mm}Misit ab eois occiduisque plagis.

Parnassus geminos fertur iunxisse volatus;

\hspace*{5mm}Contulit alternas Pythius axis aves.
\end{quotation}
\paragraph{}
Also auch Claudian kannte Quellen, nach denen der Parnass der eigentliche Nabel der Erde war. Der Ausdruck axis (= ἄξων, πόλος) erklärt sich dagegen aus der späteren Vorstellung, dass die Erde keine Scheibe, sondern eine Kugel sei, so dass das ursprüngliche Kreiszentrum sich mit Notwendigkeit in eine Achse verwandeln musste (s. oben S. 41 Anm. 79 ff. und die vom delphischen ἄξων zeugenden Stellen des Nonnos; siehe auch den nun folgenden Abschnitt Nr. 18).

20. An mehreren Stellen in Nonnos' Dionysiaka wird das delphische Orakel μεσόμφαλος ἄξων, Πύθιος ἄξων, ἄξων ὀμφήεις, ἄ. ὀμφαῖος genannt. Da ich über diese Ausdrücke auf Grund einer sehr dankenswerten Zuschrift A. Ludwichs in Königsberg im vorigen Kapitel ausführlicher gehandelt habe, so sei hier nur dies bemerkt, dass jenen Bezeichnungen einfach die aus der Vorstellung der Erde als Kugel notwendig abgeleitete Idee einer Erd- und Himmelsachse (in die sich der alte Omphalosbegriff verwandelt hat) zugrunde liegt. Vgl. das oben über Varro l. l. 7, 17 und Claudian 16, 11 Gesagte. Wir haben im vorigen Kapitel gezeigt, dass sich genau dieselbe Ideenentwicklung auch für das Orakel in Branchidai bei Milet nachweisen lässt, das ebenfalls in älterer Zeit Anspruch darauf erhob, der Mittelpunkt (ὀμφαλός) der scheibenförmigen Erde zu sein, später aber als der Punkt angesehen wurde, durch den die Erdachse hindurchging.

21. An letzter Stelle ist hier unter den literarischen Zeugnissen, wenn auch mit gebührender Reserve, eine hochinteressante, vor wenigen Jahren in Argos ausgegrabene Inschrift des 3. Jahrh. zu erwähnen, die W. Vollgraff im Bulletin de Corresp. Hellén. 1904 S. 270 ff. veröffentlicht und besprochen hat. Sie lautet:

Θεός. Προμάντιες ἀνέθεν || Ἀπόλλωνι Ἀρισ[τ]εὺς Σφυρή||δας, Φιλοκράτης Νατελιά||δας, προφῆται <Αἰ>σχύλος Ἀραχνά||δας, Τρυγῆς Αἰθωνίδας καὶ κα||τεσκεύασσαν καὶ ἕσσαντο <τὸν> || ἐκ μαντήας Γᾶς ὀμφαλὸν\footnote{Auf dem Obvers der Münzen von Argos aus dem 4. Jahrh. erscheint öfters neben dem Wappen der Stadt (Vorderteil eines Wolfs) ein Kreis mit einem Punkt in der Mitte $\odot$. Darin ist entweder der Buchstabe θ oder das auch auf delphischen Münzen erscheinende Zeichen des orbis terrarum mit dem Omphalos in der Mitte zu erblicken (s. ob. S. 51 f. Anm. 99 und vgl. den Catal. of gr. coins in the Brit. Mus. Peloponnesus p. 141 nr. 61; p. 142 nr. 79; p. 143 nr. 90 u. 95).} καὶ τ<ὰ>||ν περίστασιν καὶ τὸ φάργμα καὶ τὸν βωμὸν πρόσ<β>ορον ποτ᾽ ἀ<ϝ>ὼ καὶ πέτ||ρινον ῥόον καὶ τὰν ἀρχιθύραν || ὑπὲρ αὐτοῦ καὶ θἡ αυρὸν ἐν τῷ μαν||τήω κατεσκεύασσαν τοῖς πελά||νοις κλᾳκτὸν καὶ τὰν ὁδὸν ἐργάσ||σαντο κ. τ. λ.

Die Inschrift bezieht sich offenbar auf den Tempel des Apollon Deiradiotes am Abhange der Akropolis (Larisa) von Argos, von dem Pausanias 2, 24, 1 berichtet: Ἀνιόντων δὲ ἐς τὴν ἀκρόπολιν ἔστι μὲν τῆς Ἀκραίαρ Ἥρας τὸ ἱερόν, ἔστι δὲ [καὶ] ναὸς Ἀπόλλωνος, ὃν Πυθαεὺς\footnote{Nach Paus. 2, 35, 2 hieß der Ἀπ. Δειραδιώτης auch Πυθαεύς. Von ihm wird berichtet: τὸ μὲν δὴ [Ἀπ.] τοῦ Πυθαέως ὄνομα μεμαθήκασι [οἱ᾽ Ἀσιναῖοι] παρὰ Ἀργείων. τούτοις γὰρ Ἑλλήνων πρώτοις ἀφικέσθαι Τελέσιλλά φησι τὸν Πυθαέα ἐς τὴν χώραν Ἀπόλλωνος παῖδα ὄντα (vgl. Höfer im Lex. d. Mythol. unter Pythaeus; Preller-Robert 1 267, 2).} πρῶτος παραγενόμενος ἐκ Δελφῶν λέγεται ποιῆσαι. τὸ δὲ ἄγαλμα τὸ νῦν χαλκοῦν ἐστὶν ὀρθόν, Δειραδιώτης Ἀπόλλων καλούμενος, ὅτι καὶ ὁ τόπος οὗτος καλεῖται Δειράς. ἡ δὲ μαντική, μαντεύεται γὰρ ἔτι καὶ ἐς ἡμᾶς, καθέστηκε τρόπον τοῦτον. γυνὴ μὲν προφητεύουσά ἐστιν, ἀνδρὸς εὐνῆς εἰργομένη. θυομένης δὲ ἐν νυκτὶ ἀρνὸς κατὰ μῆνα ἕκαστον (wohl am 7. Tage), γευσαμένη δὴ τοῦ αἵματος ἡ γυνὴ κάτοχος ἐκ τοῦ θεοῦ γίνεται. ---

Leider ist nicht ohne weiteres klar, ob man diesen Kult als eine Filiale des delphischen ansehen darf oder nicht. Für starke Beeinflussung von Delphi spricht erstens die Legende vom Stifter Pythaeus, zweitens die mit der Pythia vergleichbare γυνὴ προφητεύουσα zur Zeit des Pausanias, endlich auch der in der Inschrift erwähnte ὀμφαλὸς γῆς, während anderseits das vom delphischen Ritus durchaus abweichende Trinken frischen Lammesblutes, die Beinamen Δειραδιώτης und Πυθαεύς statt Πύθιος, sowie die Tatsache, dass in der Inschrift nur männliche Priester (προμάντιες und προφῆται), keine Priesterin (Pythia) erwähnt werden, auf Unabhängigkeit von Delphi hinzuweisen scheinen. Wie dem aber auch sein möge: in jedem Falle bildet der ἐκ μαντῄας errichtete γᾶς ὀμφαλός mit der zu ihm gehörigen περίστασις eine höchst beachtenswerte Parallele zu der in der oben erwähnten Bauurkunde von Delphi erwähnten πρόστασις ἡ πρὸ τοῦ ὀμφαλοῦ, die umso schlagender wirkt, wenn man das μαντῇον in dem Satze ἐν τῷ μαντήῳ κατασκεύασσαν als das eigentliche χρηστήριον oder das Adyton des Tempels deuten darf, weil daraus hervorgehen würde, dass auch zu Delphi der alte Nabelstein im eigentlichen Orakelraume (χρηστήριον) oder Adyton gestanden haben muß.\footnote{Ähnlich Karo im Dictionn. d. antiq. s. v. `Omphalos' (4 1 Sp. 198 b), der auch darauf aufmerksam macht, dass die oben angeführte delphische Bauurkunde (πρόστασις ἁ πρὸ τοῦ ὀμφαλοῦ etc.) sich auf Arbeiten des Unternehmers Sion bezieht, dessen Name auf mehreren `blocs de l'adyton' wiederkehrt. Derselbe Gelehrte verweist für die Identität der Ausdrücke μαντῇον, χρηστήριον und ἄδυτον auf Plutarchs (des delphischen ἱερεύς) Schrift de Is. et Os. 35.} Dann würde unter περίστασις oder πρόστασις keine aedicula, wie Pomtow will, sondern nur eine um oder vor dem Nabelstein angebrachte niedrige Schranke zu verstehen sein. Nimmt man freilich in bezug auf den argivischen ὀμφαλὸς γᾶς Unabhängigkeit von Delphi an, so würde daraus der Schluss zu ziehen sein, dass auch das schon in ältester Zeit zu höchster Blüte und Macht gelangte Argos sich rühmte, der Mittelpunkt der Erde zu sein.

Suchen wir nunmehr möglichst kurz und bündig die wesentlichen Ergebnisse unserer kritischen Betrachtung sämtlicher Zeugnisse darzustellen, so ist Folgendes zu sagen.

Die aufgezählten Zeugnisse umfassen einen Zeitraum von ungefähr 900 Jahren und beweisen deutlich die gewaltige Rolle, welche das delphische Orakel innerhalb dieser langen Periode gespielt hat. Allerdings ist diese Rolle keine ganz gleichmäßige und uneingeschränkte gewesen, denn bereits vor 500 v. Chr. bezweifelt Epimenides, unser ältester Zeuge --- wahrscheinlich im Hinblick auf Orte wie Branchidai und Paphos, die ebenfalls beanspruchten, ὀμφαλοὶ γῆς zu sein ---, die Berechtigung Delphis, sich den Erdnabel κατ᾽ ἐξοχήν zu nennen, und 500 Jahre später stellen es Männer wie Varro und Cornutus, die wahrscheinlich aus etwas älteren Philosophen und Mathematikern schöpften, mit großer Entschiedenheit in Abrede, dass Delphi sich als Nabel der Erde betrachten dürfe, weil ja die Erde nicht eine Scheibe, sondern eine Kugel sei, die auf ihrer Oberfläche kein Zentrum haben könne. Varro will deshalb ebenso wie die Quelle des Hesychius (s. v. Τοξίου βουνός) den Nabelstein nicht als Wahrzeichen der Erdmitte, sondern als Grabmal des Python aufgefasst wissen, und Cornutus leitet sogar das Wort ὀμφαλός von ὀμφή (göttliche Orakelstimme) ab, um mit Hilfe dieser gewagten Etymologie die Deutung `Nabelstein' ablehnen zu können.

Ferner ist festzustellen, dass unter ὀμφαλὸς χθονός oder γῆς\footnote{Ich rechne hierher auch Ausdrücke wie μεσόμφαλον ἵδρυμα, Λοξίου πέδον (Aesch. b) und μεσόμφαλα Πυθικὰ χρηστήρια (Aesch. a), μεσόμφαλα γᾶς μαντεῖα (Sophokl. a), μες. ἕδραι (Eurip. f).} bald Delphi, d. h. der dortige Apollotempel samt dem Orakel,\footnote{Vgl. oben Pindar (a, c, d, e, f), Aeschylus (a, b), Bakchylides, Euripides (d, e), Strabo usw. Hie und da könnte allerdings auch vielleicht der Nabelstein oder das χάσμα γῆς gemeint sein.} bald der Nabelstein,\footnote{Pindar (b, g), Aeschylus (c, d), Sophokles (b?), Euripides (a? b), Platon, delph. Inschriften, Varro, Strabo, Pausanias.} bisweilen auch der Parnass\footnote{Pindar (d?), Lucan, Statius, Claudian. --- Auch in dem späteren mythischen Weltbild der Inder liegt der mythische, aus Gold bestehende Berg Meru im Nabel des innersten Weltteils, nābhyam (Locat. v. nābhi): Bhāgavata Purāṇa 5, 16, 7 (Mitteilung E. Windischs). Mehr oben S. 21 ff.} verstanden wird, auf dessen Abhang ja auch Delphi liegt. Von besonderer Wichtigkeit für die Bedeutung des Nabelsteins ist der zuerst von Pindar und später auch von Strabo, Plutarch und Lukian bezeugte Mythus von den beiden Zeusadlern, die dort zusammengetroffen sein sollten,\footnote{Schon hier sei hingewiesen auf das merkwürdige phönikisch-karthagische Relief (Memnon 3 Taf. 3 Fig. 30 und W. Schultz im `Weltall' Heft 26 Taf. 3 Fig. 53). Es zeigt in seinem Giebel eine göttliche rechte Hand, die nach außen geöffnet ist (vgl. dazu Weinreich Θεοῦ χείρ Antike Heilungswunder S. 12, 18, 42), unterhalb folgt die geflügelte karthagische Himmelskönigin, über der sich der Himmel wölbt; in den Händen hält sie eine Mondsichel nebst Kugel; ganz unten ruht auf einem omphalosähnlichen Kegel ein von einem Quadrat eingeschlossenes Rund mit einem Punkte in der Mitte (Becken, Nabel?), welchem von entgegengesetzten Seiten her Vögel (Tauben? Adler?) zufliegen.} wodurch, wie schon Bötticher in seiner Schrift über den delphischen Omphalos erkannt hat, der Nabelstein recht eigentlich für ein Heiligtum des Zeus erklärt wird, das anderen, ebenfalls im delphischen Apollotempel aufgestellten, dem Poseidon (Altar) und der Hestia (ἑστία) geweihten, nicht apollinischen Monumenten\footnote{Vgl. darüber namentlich A. Mommsen, Delphika S. 1 ff. und 11.} zur Seite tritt.

Sodann erfahren wir aus Euripides (6b) und Strabon (12), dass der Nabelstein ständig mit στέμματα oder ταινίαι bekleidet war, während sonst über seine Form nichts mitgeteilt wird.

Wenn Pausanias (17) uns recht berichtet, was allerdings nach Pomtow (Philol. 1912 S. 59 f.) sehr zweifelhaft ist, so kann der Nabelstein kaum ein Meteorstein (Baityl) gewesen sein,\footnote{Dies nimmt u. a. Ulrichs, Reisen 1 78 an. Vgl. auch A. Mommsen, Delphika S. 11, der mit Ulrichs den Nabelstein für ein Baityl, d. h. Behausung der Gottheit (hier der Gaia), hält und annimmt, er habe wohl anfänglich zugleich als Altar gedient, um der Gaia Opfergaben darzubringen; aber schon frühzeitig müsse das Bedürfnis neben dem Nabelstein eine eigentliche Opferstätte geschaffen haben: eben jenen ganz dicht bei dem ὀ. γῆς hergerichteten pythischen Herd (dagegen sprechen schon die Wollenbinden des Nabelsteines und seine Gestalt).} weil er selbst oder seine vor dem Tempel befindliche Kopie nach dem genannten Berichterstatter von weißer Farbe, also von Marmor war.

Hinsichtlich des Ortes seiner Aufstellung stimmen alle darüber Andeutungen enthaltende Zeugnisse mit einziger Ausnahme des Pausanias (17) dahin überein, dass er in der Tempelcella, und zwar in der unmittelbaren Nahe des Orakel spendenden στόμα oder χάσμα γῆς oder des pythischen Dreifußes, also im Adyton selbst oder dicht vor ihm, gestanden hat.\footnote{Das ist auch die Ansicht Studniczkas a. a. O. S. 263, der sich namentlich auf den Ausdruck μυχός in den Eumeniden des Aeschylos (39) sowie auf die Bezeichnung der auf dem Dreifuß im Adyton sitzenden Pythia als χρυσέων Διὸς αἰητῶν πάρεδρος b. Pindar (b, g) beruft. Vgl. auch Varro l. l. 7, 17 in aede ad latus. Eurip. Ion 222 μέσον ὀμφαλὸν γῆς Φοίβου κατέχει δόμος. --- Orestes: 327 ff.: μεσόμφαλοι μυχοί. Bauinschriften von Delphi oben nr. 8. Strab. a. a. O. ὀμφαλός τις ἐν τῷ ναῷ τεταινιωμένος. --- Wie Studniczka a. a. O. bemerkt, sind entscheidende Funde bei den Ausgrabungen leider ausgeblieben. `Aber die einzige, sehr ungewisse Vermutung, die sich Homolle aus einer Pflasterplatte zu ergeben scheint, würde die unmittelbare Nachbarschaft des Erdnabels mit dem Orakeldreifuß [und dem χάσμα γῆς] bestätigen.' --- Boetticher dagegen (a. a. O. S. 13) denkt sich den Omphalos zwischen der Hestiatholus und dem Adyton mitten in der Cella, also mitten unter dem Opaion des Daches und der Decke, d. h. sub divo, wogegen schon das varronische von B. nicht beachtete: "`in aede ad latus"' spricht (s. oben nr. 11). Auch Ulrichs (Reisen 1, 78) und Bursian (Geogr. v. Gr. 1, 176) verlegen den O. in die Cella, Fr. Wieseler dagegen (der ihn fälschlich mit der ἑστία identifiziert: s. oben) wohl richtiger ins Adyton (Jahns Jahrb. 75, 10, S. 678). Ebenso Karo, im Artikel `Omphalos' des Dictionn. des antiquités, p. 198 b f. Middleton im Journ. of Hell. Stud. 1888 (9) S. 294 ff. --- Von großer Bedeutung für die unmittelbare Nachbarschaft des über dem Erdspalt stehenden Dreifußes und des Omphalos ist das kürzlich in der Nähe von Phaleron ausgegrabene schöne Relief, das Staës in der Ἐφημ. Ἀρχαιολ. 3 (1909) Taf. 8 veröffentlicht und S. 239 ff. besprochen hat (s. unten!). Hier sitzt. Apollon Pythios auf dem Dreifuß, und seine Füße ruhen auf dem bienenkorbartig gebildeten Omphalos, neben dem ein Adler erscheint.} Mehr darüber im folgenden Abschnitt B.

In der späteren Zeit, als man die Vorstellung der Erde als einer kreisrunden Scheibe aufgegeben hatte und an deren Stelle die einer Kugel getreten war, auf deren Oberfläche kein Mittelpunkt mehr gefunden werden kann, verwandelte sich naturgemäß der `Erdnabel' in den Punkt auf der Kugeloberfläche, durch welchen die Erd- und Himmelsachse (ἄξων, πόλος, axis) hindurchgeht.\footnote{Varro a. a. O., Claudian (axis), Nonnos (ἄξων).} Wir haben im vorigen Kapitel gesehen, dass sich genau dieselbe Verwandlung des ὀμφαλός in einen ἄξων auch bei dem wahrscheinlich noch älteren Apollonorakel von Branchidai (Didyma) nachweisen lässt, das ebenfalls die Ehre beanspruchte, der `Mittelpunkt der Erde' zu sein.

Im übrigen dürfte wohl jeder gewissenhafte Betrachter sämtlicher Zeugnisse ebenso wie ich den Eindruck gewonnen haben, dass eigentlich und ursprünglich unter dem ὀμφαλὸς γῆς in Delphi wohl nicht der (emporragende) Nabelstein, sondern vielmehr das orakelspendende χάσμα γῆς, über dem der Dreifuß der Pythia stand, also eine Vertiefung, zu verstehen ist.\footnote{Vgl. Schol. in Lucan. ed. Usener p. 157, 21 ff.: illud antrum terrae umbilicus est, quod inde aër exiens caelo conexus terras suspendit.} Denn einerseits passt nur auf eine Vertiefung, nicht aber auf eine kegelförmige Erhöhung der Ausdruck `Nabel' (ὀμφαλός; s. oben S. 6 f.), anderseits hat offenbar der Nabelstein mit den beiden Zeusadlern ursprünglich gegenüber dem χάσμα γῆς nur eine sekundäre und symbolische Bedeutung: er sollte nur das den wenigen unmittelbar Herantretenden sichtbare orakelspendende χάσμα (στόμα) γῆς allen Tempelbesuchern als ein weithin sichtbares Zeichen (Symbol) der außerordentlichen Ehre und Heiligkeit kenntlich machen, deren sich Delphi und sein Tempel als Mittelpunkt der Oikumene rühmen durfte. Als leicht und allgemein verständliches Symbol von solcher Bedeutung genügte ein von zwei Zeusadlern flankierter, oben abgerundeter Kegel, d. h. eine niedrige στήλη, wie man sie seit uralter Zeit auch sonst an bemerkenswerten Punkten der Erde oder der Länder aufzustellen pflegte. Vgl. z. B. Strabo 3 p. 171 ἔθος γὰρ παλαιὸν ὑπῆρχε τὸ τίθεσθαι τοιούτους ὅρους, καθάπερ οἱ Ῥηγῖνοι τὴν στυλίδα ἔθεσαν τὴν ἐπὶ τῷ πορθμῷ κειμένην, πυργίον τι, καὶ ὁ τοῦ Πελώρου λεγόμενος πύργος ἀντίκειται ταύτῃ τῇ στυλίδι. καὶ οἱ Φιλαίνων λεγόμενοι βωμοὶ κατὰ μέσην που τὴν ματαξὺ τῶν σύρτεων γῆν. καὶ ἐπὶ τῷ ἰσθμῷ τῷ Κορινθιακῷ μνημονεύεται στήλη τις ἱδρυμένη κ. τ. λ.

\subsection{Die monumentalen Zeugnisse.}
\paragraph{}
Eine überaus wertvolle und hochwillkommene Ergänzung zu den soeben aus der antiken Literatur gewonnenen Ergebnissen bilden diejenigen Monumente, welche uns über die Form und Ausschmückung des delphischen Nabelsteins genauer unterrichten. Sie zerfallen je nach ihrer Zuverlässigkeit und Bedeutung in vier verschiedene Gruppen. Diese sind:
\begin{enumerate}
    \item die in Delphi selbst sowie in Sparta, Athen und anderwärts ausgegrabenen plastischen Nachbildungen und Darstellungen des delphischen Omphalos; sie gehören größtenteils der zweiten Hälfte des 5. Jahrhunderts an;
    
    \item die Omphalosbilder auf Wandgemälden, Spiegeln und Cisten;

    \item die Darstellungen auf Münzen, insbesondere den delphischen;

    \item die Vasenbilder.
\end{enumerate}
\subsubsection{Die plastischen Nachbildungen des delphischen Omphalos.}
\paragraph{}
Den eigentlichen und ursprünglichen Omphalos, der sich, wie wir sahen, nach den literarischen Zeugnissen entweder im Adyton selbst oder in dem diesem unmittelbar benachbarten Raum der Tempelcella befunden haben muss, haben die neueren Ausgrabungen leider ebenso wenig wie sichere Spuren seiner einstigen Situation zu Tage gefördert, doch verdienen in letzterer Hinsicht die Darlegungen Pomtows (Philologus 1912 S. 59 ff.) alle Beachtung. Pomtow sucht nämlich a. a. O. wahrscheinlich zu machen, dass der alte echte Nabelstein ursprünglich in der Tempelcella unweit der Hestia sich befand.\footnote{Vgl. a. a. O. S. 61 und Frickenhaus in den Athen. Mitteil. 1910 S. 271, 1.} Ob er freilich, wie P. annimmt, in einem besonderen Kapellchen (`aedicula') gestanden hat,\footnote{Diese `aedicula' (= ναΐσκος) mit dem Omphalos denkt sich Pomtow, wie aus seinem Plane S. 69 hervorgeht, im rechten Seitenschiffe der Cella zwischen der 2. und 3. Innensäule angebracht; ich möchte den O. lieber im Hinblick auf die literarischen Zeugnisse (s. ob S. 78 f.) entweder ins Adyton selbst oder doch in dessen unmittelbare Nähe versetzen, wo wohl auch die ἑστία stand.} ist deshalb unsicher, weil diese Annahme bisher durch kein einziges erhaltenes Bildwerk bestätigt wird; auch findet sich in den erhaltenen literarischen Zeugnissen keine sichere Andeutung eines derartigen Miniaturgebäudes im Tempel.

1. Gehen wir nunmehr auf eine genauere Betrachtung der erhaltenen Abbildungen des Nabelsteins ein, so ist unter den in Delphi selbst gefundenen plastischen Darstellungen an erster Stelle zu nennen ein neuerdings auf dem Vorplatz des Tempels unweit des großen Altars ausgegrabener, besonders prächtiger Omphalos von weißem Marmor (Taf. 6 Fig. 1).

Es ist dies, wie Pomtow erkannt hat, derselbe Stein, von dem Pausanias (10, 16, 3) berichtet: Τὸν δὲ ὑπὸ Δελφῶν καλούμενον ὀμφαλόν, λίθου πεποιημένον λευκοῦ, τοῦτο εἶναι τὸ ἐν μέσῳ γῆς πάσης αὐτοί τε λέγουσιν οἱ Δελφοί, καὶ ἐν ᾠδῇ τινὶ Πίνδαρος ὁμολογοῦντά σφισιν ἐποίησε (s. ob. S. 58). Die Tatsache, dass Pausanias sonach offenbar den eigentlichen O. im Tempel mit seiner prächtigen Nachbildung vor dem Tempel verwechselt hat, erklärt Pomtow (S. 59 f.) scharfsinnig so, indem er sagt: `Es ist bekannt, dass der Perieget Doppelerwähnungen ängstlich vermeidet, also hat er später das uralte, äußerlich unscheinbare Original übergangen, weil er vorher --- gelegentlich der prächtigen Nachbildung --- die Omphalossage und sein Pindarzitat angebracht hatte (10, 16, 3), aber auch hier drückte er sich so gewunden aus, dass man erst jetzt nach Auffindung dieses großen Marmorkegels merkt, dass er die Omphalosgeschichte angesichts jener Kopie antizipiert (vgl. Ulrichs, Reisen 1 92, 58), und dass er absichtlich nicht sagt, dass der Erdnabel selbst schon hier --- auf dem Vorplatz --- läge.' Dass diese Erklärung Pomtows das Richtige trifft, und dieser große und prächtige Omphalos des Tempelvorplatzes nicht mit dem eigentlichen Erdnabel (im Tempel!) identisch sein kann, sondern nur eine so zu sagen für die weiteste Öffentlichkeit (d. i. die zahlreiche vor dem Tempel harrende Menge) bestimmte Kopie des weniger leicht zuganglichen Originals im Tempel\footnote{Vgl. Paus. 10, 24, 5: Ἐς δὲ τοῦ ναοῦ τὸ ἐσωτάτω παρίασί τε ἐς αὐτὸ ὀλίγοι, καὶ χρυσοῦν Ἀπόλλωνος ἕτερον ἄγαλμα ἀνάκειται. Vgl. dazu Karo im Dict. d. antiq. s. v. Omphalos Sp. 199 b.} war, welche den ganzen Ort (d. h. Delphi und seinen Haupttempel) als Mittelpunkt der Erde bezeichnen sollte, geht aus verschiedenen Momenten deutlich hervor. Erstens aus dem Umstande, dass das Netzwerk (ἀγρηνόν) das bei dem Original im Tempel unzweifelhaft aus Wolle bestand, bei der Kopie von Marmor gebildet und aus demselben Steinblock herausgearbeitet ist, wie das Kernstück; zweitens aus der Fundstelle vor dem Tempel (s. Paus. a. a. O.); drittens aus der an der Spitze des Steines deutlich bemerkbaren etwa 40 cm betragenden Abplattung, während nach fast allen anderen Darstellungen des echten Omphalos (mit Ausnahme des im Dionysostheater von Athen gefundenen: s. unt) die Spitze des alten Nabelsteins nicht abgeplattet, sondern vielmehr gewölbt war. Welches der Grund jener Abplattung war, ist nicht klar: entweder kann man annehmen, dass sie, wie bei dem athenischen O., dazu diente ein Bildwerk (Apollon?) zu tragen, oder es war, da die Größe des Marmorblocks nicht ausreichte, um die natürliche Spitze darzustellen, wie so oft, auch hier ein besonderes, später verloren gegangenes Ergänzungsstück angesetzt gewesen (s. u.). Ob dieser Omphalos ebenso wie zwei andere ebenfalls in Delphi gefundene, die gleich zu erwähnen sind, hohl war oder nicht, lässt sich, wie mir Pomtow gütigst mitteilt, infolge der Vergipsung der unteren Partie leider zur Zeit nicht feststellen, ebenso wenig, ob auf der einst, wie es scheint, ursprünglich vorhandenen Basis, in die, nach dem Aussehen der untersten Partie zu urteilen, dieser O. eingelassen gewesen sein muss (Karo a. a. O. Sp. 199 b), zwei Adler angebracht waren oder nicht.\footnote{Wahrscheinlich ist, wie wir später sehen werden, dieser vor dem Tempel stehende O. auf gewissen Vasenbildern dargestellt, die den O. fast immer (mit Ausnahme der Vase C. R. de St. Pétersb. 1863 pl. 6 = Reinach, Répert. 1, 19, 5) in ziemlicher Größe vor dem Tempel stehend ohne die Adler zu beiden Seiten zeigen. Sehr merkwürdig ist übrigens die Tatsache, dass bei dem Zuge der milesischen Sängergilde (im Monat Taureon) nach dem Tempel in Didyma vor dessen Türen ein γυλλός genannter Stein gesetzt wurde, den man mit Binden behing und mit ungemischtem Weine begoss, also ähnlich wie den delphischen Omphalos oder die Steinsäulen des Apollon Agyieus behandelte: Nilsson, Gr. Feste S. 168 f. Nach Hesych. s. v. soll freilich der γυλλός nicht konisch sondern viereckig gewesen sein (s. ob. S. 46 f. Anm. 90).}

2. Pomtow, wohl der beste Kenner Delphis und seiner Überreste, hatte die Freundlichkeit mir am 6/9 12 brieflich Folgendes mitzuteilen: "`Außer dem Marmoromphalos vor dem Tempel sind im Temenos noch 2 bis 3 andere schmucklose Omphaloi aus Kalkstein, glatt, zuckerhutförmig, innen hohl gefunden. Einer liegt beim Thesauros der Athener, ein anderer auf der Agora. Von beiden lege ich Ihnen Photographien bei (s. Taf. 6 Fig. 2). Diese zwei Omphaloi sind spitzer als der aus Marmor, lassen aber vielleicht vermuten, dass auf letzterem einst eben solche Spitze auflag, als Extrastück gearbeitet. Warum sie hohl sind, vermag ich nicht zu sagen --- ist etwa auch der marmorne ausgehöhlt? Das kann wegen des leidigen Vergipsens heut Niemand mehr feststellen."' Vielleicht erklärt sich die Aushöhlung dieser Omphaloi aus dem Bestreben, den nach der Abschnürung und Durchschneidung zurückgebliebenen Rest der Nabelschnur, die ja auch ὀμφαλός hieß (s. ob.), recht naturalistisch darzustellen.\footnote{Näcke, der ausgezeichnete Arzt und Psychiater, schreibt im Archiv für Kriminal-Anthropologie u. Kriminalistik 1912 S. 350 darüber: "`Der Nabel [den der ὀμφαλός von Delphi darstellen sollte] war realistisch nachgebildet. Bei Kindern ist der Nabel nach Abfallen des Nabelstrangs einige Zeit noch leicht konisch gestaltet und zieht sich allmählich ein. Bei schlechter Pflege tritt leicht Entzündung ein und Nabelbruch ein, wie wir dies öfters bei Naturvölkern, besonders Negern, auch beim Erwachsenen sehen, als eine vorspringende Bandung"' usw.} Jedenfalls ist es von hohem Interesse zu sehen, dass es außer dem eigentlichen O. im Tempel und dessen Kopie vor dem Tempel noch mehrere andere Nabelsteine im Temenos gab, die wohl als Weihgeschenke zu gelten haben. Der heilige Omphalos scheint also in Delphi ebenso vervielfältigt worden zu sein wie der heilige Dreifuß, von dem es ebenfalls zahlreiche mehr oder minder kostbare Kopien in Delphi gegeben hat. Wir haben oben gezeigt, dass auch zu Branchidai bei Milet, dessen Orakel und Apollokult wahrscheinlich älter als Delphi und für dieses prototypisch waren, außer dem heiligen Omphalos im Tempel noch mehrere geweihte ὀμφαλοί existierten (darunter ein von einer Schlange umwundener), die sich bei den neueren Ausgrabungen in der Nekropole gefunden haben.

3. Von großem Interesse für die Frage des delphischen Omphalos ist ein im Jahre 1885 in dem zu Delphi bekanntlich von jeher in engsten Beziehungen stehenden Sparta in der Nähe des dortigen Museums gefundenes sehr schönes Votivrelief, das Wolters im 12. Bande der Athenischen Mitteilungen (1887) Taf. 12 abgebildet und S. 378 ff. besprochen hat (s. Taf. 7 Fig. 4). Es stellt den Kitharöden Apollon dar, dem Artemis (nach Middleton a. a. O. S. 295 soll es Nike sein!) einen Trunk kredenzt, ein Motiv, das häufige Darstellung gefunden hat.\footnote{Vgl. außer Wolters a. a. O. S. 378 A. 2 jetzt namentlich auch Overbeck, Kunstmythol. Apollon S. 259 ff. u. 263 ff. sowie Studniczka im Hermes 1902 [37] S. 267 Fig. 6, der auch darauf hinweist, dass das Netzwerk (ἀγρηνόν) nur scheinbar fehlt, weil es ursprünglich wohl in jetzt verschwundenen Farben dargestellt war.} Wolters bemerkt darüber: "`Eine Einzelheit verlangt unsere Aufmerksamkeit: der Omphalos, welcher sich zu den Füßen der göttlichen Geschwister\footnote{Nach Pomtows schöner Vermutung a. a. O. S. 48 befand sich im delphischen Tempel ganz in der Nähe der Statue Apollons auch eine Seitenkapelle mit einer Statue der Artemis und eine andere mit einer solchen der Athene, der λευκαὶ κόραι, deren Hilfe bei der Belagerung Delphis durch die Gallier der Sieg verdankt wurde. Ich brauche kaum zu bemerken, dass die Gruppierung von Apollon und Artemis zu beiden Seiten des Omphalos sehr für Pomtows Annahme spricht.} befindet. Er steht auf einer niedrigen Stufe, die, etwas breiter als er, noch Raum für die beiden Adler bietet, die rechts und links von ihm sitzen. Die ganz symmetrische und etwas leblose Haltung der Vögel zeigt deutlich genug, dass es nicht lebendige Wesen sind, die wir hier bei dem heiligen Steine von Delphi sehen, sondern Kunstwerke. Die Sage, welche ein Adlerpaar mit dem Omphalos in Beziehung setzt, ist bekannt genug."' Wolters setzt die Entstehung des Reliefs aus stilistischen Gründen in dieselbe Zeit wie die der Reliefs der Nikebalustrade in Athen, d. h. um 430-407 vor Chr. und schreibt es derselben Schule zu.

4. Sehr nahe Verwandtschaft mit diesem schönen Relief aus Sparta verrät ein zweites ganz ähnliches, das im Jahre 1898 παρὰ τὴν Πύλην τῆς Ἀγορᾶς in Athen gefunden, nebst der dazu gehörigen Inschrift (aus voreuklideischer Zeit) von Svoronos im Journal Internat. d'archéol. numismat. 13 (1911) S. 302 abgebildet und daselbst unter der Überschrift Ψήφισμα Ἀττικὸν ἀνέκδοτον καὶ οἱ ὀμφαλοὶ τῶν Πυθίων besprochen worden ist (Taf. 9, 5). Leider ist das Relief oben, unten und auf seiner linken Seite stark beschädigt, doch kann über den wesentlichen Inhalt seiner Darstellung kein Zweifel sein, zumal da die dazu gehörige Inschrift glücklicherweise in der Hauptsache ziemlich intakt geblieben ist. Es handelt sich in diesem Falle um den Beschluss, einen Athener, dessen Name fehlt, der aber als ἐχσεγετὲς γενόμενος Ἀθηναί[οις] bezeichnet wird, durch einen Ehrensitz im Prytaneion sowie im Theater neben dem Dionysospriester auszuzeichnen. Wie Svoronos a. a. O. S. 304 nachweist, ist hier unter dem ἐχσεγετές einer der bekannten ἐξηγηταὶ πυθόχρηστοι zu verstehen, deren Aufgabe nach Suidas s. v. ἐξηγηταί darin bestand, καθαίρειν τοὺς ἄγει τινὶ ἐνισχηθέντας.\footnote{Vgl. außer Toepffer, Att. Genealogie 69, 1 und Stengel, D. griech. Kultusalt. 2 S. 67 A. 7 f. (der Plat. Euthyphr. 4 c; Leg. 6, 759 c; R. Schoell, Herm. 6 36; [Demosth.] 43 66 f. anführt) noch folgende Zitate b. Svoronos: Isaios π. Κίρ. κλήρου 39. Tim. lex. Plat. s. v. ἐξηγ. Harpokr. s. v. Inscr. Gr. 3, 241; 267; 684; Ἐφημ. Ἀρχ. 1883 S. 144.} Den hohen Rang dieses Priestertumes beweist die Stellung seines θρόνος im Dionysostheater rechts vom Sitze des in der Mitte des Zuschauerraumes thronenden Priesters des Dionysos Eleuthereus auf das deutlichste. Schon Wilhelm (Oesterr. Jahreshefte 1 [1898] Beiblatt p. 43 = Anz. d. K. K. Akad. d. Wiss. in Wien 1899 S. 3) hat die Übereinstimmung dieses Reliefs mit dem von Wolters besprochenen in Sparta erkannt, die so groß ist, dass man beide für Kopien desselben Originals halten muss, nur mit dem Unterschied, dass auf dem athenischen Bildwerk vielleicht links noch eine dritte Gottheit vorhanden war, ὡς ἐνδεικνύει τοῦτο τὸ πλάτος τῆς συμμετρικῶς τῷ πίνακι ὑποκειμένης ἐπιγραφῆς συμπληρουμένης οὕτω τῆς Δελφικῆς τριάδος Ἀπόλλωνος Πυθίου, Λητοῦς καὶ Ἀρτέμιδος, ἣν ἔχομεν παρὰ τὸν αὐτὸν μετὰ πανομοίων χρυσῶν ἀετῶν ὀμφαλὸν ἐπὶ τρίτου, τῶν αὐτῶν ἀκριβῶς χρόνων, ἀττικοῦ ἀναγλύφου, ἤτοι ἐπὶ τοῦ ἐν τῷ Νυμφαίῳ τοῦ Φαλήρου ἐσχάτως ἀνακαλυφθέντος καλλίστου ἀληθῶς ἀναθήματος τῆς Ξενοκρατείας (Svoronos a. a. O. S. 308, der dazu auf Ἀρχ. Ἐφημ. 1909 Πιν. γʹ und auf Τὸ ἐθν. Μους. σελ. 492 ff. Πίν. 181 verweist; s. unten nr. 5 S. 86 f.). Ich stimme Svoronos völlig bei, wenn er vermutet, dass dieses offenbar mit einem der ἐξηγηταὶ Πυθόχρηστοι zusammenhängende Relief sich zugleich auf den Kult des athenischen Pythions\footnote{Über die Lage des Pythions in der Nähe des Olympieions auf dem Wege zum Ilissos s. Judeich, Topographie v. Athen S. 344 und Milchhöfer bei Baumeister, Denkmäler S. 179.} beziehe, kann ihm aber nicht beipflichten, wenn er weiter behauptet, dass der hier dargestellte Omphalos mit den beiden auf einer viereckigen Basis sitzenden Adlern nur eine ungenaue (freie), im Pythion zu Athen befindliche Kopie des ursprünglichen Omphalos im Adyton zu Delphi sei (a. a. O. S. 309 u. 312). Merkwürdiger Weise hält Svoronos die Darstellung des von zwei Adlern flankierten Omphalos auf einem kyzikenischen Elektronstater, der in wichtigen Einzelheiten von der Darstellung der hier besprochenen Reliefs von Sparta und Athen vollkommen abweicht (s. ob. S. 50), für die bei der Rekonstruktion des ältesten delphischen Nabelsteins einzig und allein maßgebende, eine Annahme, die, wie ich glaube, schon durch meine oben (S. 50) geltend gemachten Erwägungen hinreichend widerlegt sein dürfte.\footnote{Ein Hauptgrund für Svoronos' Annahme ist der Umstand, dass es bei Strabon 420 vom delphischen Nabelstein heißt: δείκνυται δὲ καὶ ὀμφαλός τις ἐν τῷ ναῷ τεταινιωμένος καὶ ἐπ᾿ αὐτῷ αἱ δύο εἰκόνες τοῦ μύθου (s. ob. Anm. 128). Da die Reliefs von Sparta und Athen das (nach Studniczka a. a. O. S. 267 ursprünglich nur in Farben dargestellte, aus Tänien bestehende) Netzwerk jetzt vermissen lassen, während es auf der Münze von Kyzikos ganz deutlich sichtbar ist, so will Sv. der Rekonstruktion des delphischen Originals diese letztere, nicht aber die an sich viel bedeutungsvolleren Votivreliefs zu Grunde legen und behauptet sogar, dass bei Strabon a. a. O. nicht ἐπ᾽ αὐτῷ, sondern vielmehr ἐπ᾽ αὐτοῦ nach Maßgabe des Kyzikeners zu lesen sei.} Ich füge jetzt noch hinzu, dass eine Münze des weit mehr von Milet und Branchidai als von Delphi abhängigen, weil von Milet aus gegründeten, Kyzikos, zumal im Hinblick auf die bekannte Freiheit und Ungenauigkeit der meisten Münzstempelschneider bei der Wiedergabe berühmter Originale, weit weniger für die Beurteilung delphischer Verhältnisse in Betracht kommen kann als hochkünstlerische Votivreliefs der besten Zeit aus dem Delphi so viel näher liegenden und mit ihm durch ihre eigenen Apollokulte so eng verbundenen Sparta und Athen.

5. Ein ganz wundervolles, figurenreiches Votivrelief (nebst Inschrift), welches den Omphalos genau ebenso darstellt wie die beiden soeben besprochenen Bildwerke, ist kürzlich in Phaleron, und zwar innerhalb der alten langen Mauern, im alten Demos der Echeliden, unweit des Kephissos, in einem den Nymphen und dem Kephissos geweihten Temenos ausgegraben, von Staës in der Ἐφημ. Ἀρχαιολογική Jahrg. 1909 Taf. 8 abgebildet und daselbst S. 239 ff. ausführlich besprochen worden.\footnote{Vgl. auch Svoronos, Ἀρχ. Ἐφ. 1912 p. 256 u. in Τὸ ἐν Ἀθήναις Ἐθν. Μουσ. Heft 19-20 p. 492 ff. Taf. 181 u. f. (der auch von diesem Relief a. a. O. S. 311 behauptet, es stelle nicht den in Delphi sondern den in Athen als πατρῷος θεός verehrten Apollon Pythios dar) und G. Karo im Archiv f. Relig.-Wiss. 16 (1913) S. 271 f.} Die dazu gehörige Inschrift lautet: Ξενοκράτεια Κηφισοῖ <ἰ>δερ || ὸν ἰδρύσατο καὶ ἀνέθηκεν || ξυνβώμοις τε θεοῖς διδασκαλ || ίας τόδε δῶρον Ξενιάδου θυγάτ || ηρ καὶ μήτηρ ἐκ Χολειδῶν || θύε<ι>ν τῷ βουλομένῳ. ἐπὶ || Τελεστῶν Ἀγάθωνος... Das Relief enthält nicht weniger als 13 Figuren (s. Taf. 8, 3). Am äußersten Ende links gewahren wir eine Gruppe von drei Gottheiten: 1. Apollon Pythios auf dem Dreifuß sitzend, der mit zwei ineinander geringelten Schlangen geschmückt ist, und seine Füße auf den bienenkorbförmig gebildeten Omphalos setzend, neben dem ein vollständiger Adler und von einem zweiten der Kopf sichtbar ist. Neben Apollon stehen 2. Leto und 3. Artemis, Leto in der Haltung einer διαδουμένη (ihr Diadem war gemalt), Artemis hielt in der linken erhobenen Hand eine in Farben ausgeführte, jetzt nicht mehr sichtbare Fackel. Vor Artemis steht eine 4. etwas kleiner gebildete männliche Gestalt (Xuthos?), mit der sich eine ebenfalls kleiner gebildete Frau (5 = Pythia?) angelegentlich unterredet.\footnote{Nach Svoronos a. a. O. ist unter der von Staës für Xuthos erklärten Gestalt vielmehr Kephisos zu verstehen, die vermeintliche Pythia hält er für Xenokrateia und den kleinen Knaben für deren Sohn Xeniades.} Zwischen den beiden letztgenannten Sterblichen steht ein kleines Knäblein (6 = Ion?), das zu der sterblichen Frau emporblickt und seinen rechten Arm emporstreckt. Nach der Deutung von Staës spielt die durch die ebengenannten 6 Figuren angedeutete Szene in Delphi, die nun folgende, die andere Hälfte des Reliefs einnehmende dagegen in Athen. Staës will nämlich in den weiteren 7 Figuren Hermes 7. und 4 Nymphen (8-11), endlich eine auf einem Bathron stehende als Kultbild dargestellte Göttin (12) mit Polos (Eileithyia? Artemis?) und in dem am äußersten rechten Ende (13) erscheinenden Stier mit Menschenkopf Acheloos (oder Kephissos?) erkennen, während Svoronos a. a. O. die bei Acheloos stehende Göttin mit Polos für Kallirrhoë, die beiden nach rechts gewandten Göttinnen für Ileithyia und Rhapso, die beiden nach links gewandten für die Geraistischen Nymphen der in der Nähe gefundenen Inschrift erklärt.\footnote{Die Inschrift lautet: Ἑστίᾳ, Κηφισῷ, Ἀπόλλωνι Πυθίῳ, Λητοῖ, Ἀρτέμιδι Λοχίᾳ, Ἰλειθυίᾳ, Ἀχελῴῳ, Καλλιρρόῃ, Γεραισταῖς νύμφαις γενεθλίαις, Ῥαψοῖ (Svoronos S. 495; vgl. G. Karo im Arch. f. Rel.-Wiss. 16 (1913) S. 271.)}

6. Aus der Zeit nach dem Phokischen Kriege, der den delphischen Omphalos der beiden goldenen Adler beraubte (s. ob. S. 56), stammt ein in Delphi gefundenes Relief, das den Nabelstein genau ebenso (fast halbkugelförmig!) wie die bisher erwähnten plastischen Bildwerke, aber ohne die Adler darstellt. Es ist veröffentlicht von Svoronos im Journ. Internat. d'archéol. numism. (13) 1911 S. 315 (Fig. 11). Die zu diesem Relief gehörige Inschrift ist publiziert und besprochen von M. Colin im Bull. de Corr. Hellén. 20 (1896) p. 675 ff. Sie lautet:
\begin{quotation}
[Ὁ δ]ῆμος ὁ Ἀθηναίων τῶι [Ἀπ]όλλωνι ἀν[έθηκεν]

[Ἱ]εροποιοὶ οἱ τὴν πυθιάδα\footnote{Über diese Pythiaden oder Pythaïden s. K. Fr. Hermann, Gottesd. Alt. 2 62, 4 u. Colin a. a. O. S. 639 ff.} ἀγαγόντες·

Φανόδημος Διύλλου.

Βόηθος Ναυσινίκου.

Λυκοῦργος Λυκόφρονος.

Δημάδης Δημέου etc. (folgen noch 6 weitere Namen).
\end{quotation}
\paragraph{}
Wie aus der Erwähnung der beiden bekannten attischen Redner Lykurgos und Demades hervorgeht (s. Colin a. a. O. p. 677), müssen Inschrift und Relief aus der Zeit zwischen 331 und 324 stammen. Ich vermute, dass das Relief nicht in Athen, sondern in Delphi angefertigt ist, weil sonst doch wohl die bei den athenischen Bildwerken traditionellen Adler kaum fehlen dürften, dass also das nach dem Phokischen Kriege in Delphi entstandene Relief tatsächlich den damaligen Zustand des Omphalos wiedergibt, den jeder Delphier kennen musste.

7. Aus Athen, und zwar aus dem Dionysischen Theater, stammt ein großer, marmorner, mit Netzwerk en relief versehener, Omphalos, der wie die obere abgeplattete Fläche mit den darauf noch deutlich erkennbaren Fußspuren zeigt, einst als Basis eines stehend dargestellten Apollon gedient hat.\footnote{Vgl. Overbeck, Gr. Kunstmythol. Apollon S. 164 f. Studniczka a. a. O. S. 261, Fig. 3. Middleton im Journ. of Hell. Stud. 1888 (9) S. 298, Fig. 7. Waldstein ebenda 1, S. 180. v. Sybel, Katalog d. Skulpturen zu Athen S. 53 nr. 291 (mit weiteren Literaturangaben). Vgl. auch ebenda S. 201 nr. 2791: "`Auf oblonger Plinthe l. Spielfluss einer Statuette des Apollon, z. L. Omphalos innerhalb des Dreifußes, davon die drei Füße (Löwentatzen) erhalten und ganz r. vorn Schwanz der Schlange (umringelt den Dreifuß?)."' Man vergleiche auch die Münzen von Kreta und von Tarsos (bei Overbeck, Apollon, Münztaf. 1, 27 u. 30; s. Text S. 25), die Apollon ebenfalls auf dem Omphalos stehend darstellen. S. Taf. 1, 4 u. 5.} Vgl. Taf. 6 Fig. 4.

Hinsichtlich seiner Form und Größe (auch hinsichtlich der Abplattung an der Spitze) steht er dem oben unter 1 besprochenen, in Delphi vor dem Tempel aufgestellten Nabelstein nahe.

An diese sieben plastischen, teils in Delphi selbst teils in Sparta und Athen ausgegrabenen Omphaloi schließen sich noch einige andere außerhalb der genannten Orte aufgefundene an, von denen mit ziemlicher Sicherheit angenommen werden darf, dass sie ebenfalls den Nabelstein des delphischen Adytons darstellen sollen. Hierher gehört vor allen:

8. ein großer Marmoromphalos, fast halbkugelförmig (0,45 hoch, 0,60 breit, sitzend auf einer damit unmittelbar zusammenhängenden, vierseitigen Basis, 0,90 breit und jetzt noch c. 0,75 hoch), gefunden in Vathia (Βάθεια), d. h. Eretria, auf Euboia, nicht weit von dem Heiligtum der Artemis Amarysia. Dieser Nabelstein ist versehen mit einem aus dem Marmor herausgearbeiteten Wollnetz, das nach unten in dreieckige Franzen ausläuft, und gehörte einst nach der Vermutung des Herausgebers (Kuruniotis in der Ἐφημ. Ἀρχαιολογική 1900, S. 19 f., wo er auch abgebildet ist) zu dem Heiligtum der Artemis Amarysia,\footnote{Vgl. über die Bedeutung dieses Kultes Preller-Robert, Griech. Mythol. 1, 310, 4. Catal. of greek coins Brit. Mus. Central Greece S. 123 ff. = Taf. 23 9 ff.} dem auch ein a. a. O. auf Taf. 2 (Fig. 1) wiedergegebenes Relief mit der Darstellung der beiden Letoiden und ihrer Mutter entstammt, woraus zu schließen ist, dass dort neben seiner Schwester auch Apollon verehrt wurde.\footnote{Ebenso auch in Cumae, der Kolonie von Eretria: Boll im Arch. f. Relig.-Wiss. 13 (1910) S. 572 und Roscher im Philologus 1911 S. 307 f.} S. Taf. 6 Fig. 6.

9. Zu Ikaria unweit von Marathon haben die Ausgrabungen der Amerikaner im Jahre 1887 die Reste eines Pythions zu Tage gefördert. Dazu gehörte ein noch vorhandener Schwellenstein mit der bemerkenswerten Inschrift $\svgAAM$ (vgl. Americ. Journ. of Archaeol. 5 (1889) S. 175). Ebenda wurde ausgegraben ein schönes wohlerhaltenes Votivrelief (besprochen ebendort S. 471 und abgebildet a. a. O. auf Taf. 11 unter nr. 3), das in der Mitte Apollon auf einem ziemlich hohen, basislosen bienenkorbförmigen Omphalos sitzend und in der Linken einen Lorberzweig, in der Rechten eine Phiale haltend darstellt. Vor Apollon steht ein Altar, vor diesem wieder ein Adorierender. Hinter Apollon steht Artemis. Auf dem oberen und unteren Rande liest man die Inschrift: $\svgAAN\enspace\svgAAO\enspace\svgAAP\enspace\svgAAQ$.\footnote{Vgl. dazu Toepffer im Hermes 1888 S. 321 ff.} --- Ganz ähnlich auf dem Omphalos sitzend erscheint der leierspielende Apollon auf einem anderen ebenfalls in der Nähe des ikarischen Pythions ausgegrabenen Relief, das ebendort auf S. 473 beschrieben und auf Taf. 11 unter nr. 1 abgebildet ist. Hinter Apollon stehen zwei weibliche Gottheiten, wohl Artemis und Leto (oder zwei Musen?) An diesem Omphalos sind noch Spuren des in Marmor ausgeführten Netzwerks (ἀγρηνόν) erhalten (a. a. O. 473 A. 45). Beide Reliefs zeigen einen ziemlich hohen, zum Sitzen geeigneten, basislosen, oben flachgewölbten, von unten nach oben sich sehr allmählich verjüngenden Nabelstein; von Adlern gewahrt man keine Spur.\footnote{Die beiden Reliefs aus Ikaria sind auch aufgeführt und kurz besprochen von Wace im Annual of the Brit. School at Athens 9 (1902/3) p. 213.} S. Taf. 7, 5.

10. Mehrere rohe Nachbildungen (`rude Roman copies') des delphischen Omphalos haben sich nach Middleton im Journ. of Hellen. Stud. 9 (1888) S. 301 im Apollotempel von Pompeji gefunden, doch ist es mir leider bis jetzt nicht möglich gewesen, über deren Form Genaueres zu erfahren. Wahrscheinlich handelt es sich um Exemplare, die den rohen in Delphi ausgegrabenen (s. ob. S. 83) ähnlich sind.

11. Der Sammlung Barracco gehört an ein prächtiger Torso aus guter Zeit, beschrieben von Helbig-Amelung, Führer 3 1 (1912) Nr. 1096: `Fragmentierte Statue des Apollon. Der Gott sitzt auf einem Felsen, in dessen vorderer Höhlung der Omphalos steht. Das Motiv erinnert stark an den sogenannten Kekrops im Ostgiebel des Parthenon. Doch weisen die Formen, soweit sie noch kenntlich sind, und die Verwendung des Motivs zu einer Einzelstatue des Gottes auf spätere Zeit. Eigenartig, aber kaum von besonderer Bedeutung ist die Stellung des [bienenkorbförmigen] O. in der Höhlung des Felsens (in Delphi stand er im Innern des Tempels).' Nach der mir durch P. Hermanns Güte zur Verfügung gestellten Photographie zu urteilen, scheint der O. auf einer viereckigen Basis zu stehen. Die `Stellung in der Höhlung des Felsens' ist insofern wohl nicht ohne Bedeutung, als sich der delphische O. höchst wahrscheinlich in dem als eine Art Grotte zu denkenden Adyton des Tempels befand (man denke auch an den O. des auf dem Relief des Archelaos von Priene dargestellten grottenförmigen Adytons).

12. Kleine, nur 0,48m hohe Statue des auf dem O. sitzenden Apollon aus hellenistischer Zeit (3. Jahrh.?) im Museum von Alexandria, beschrieben und besprochen von Wace im Annual of the Brit. School at Athens 9 (1902/3) p. 211 ff. und abgebildet ebenda auf Taf. 4.

Von dem O. sagt Wace a. a. O. S. 212: `The o. itself is a plain truncated cone about 0,22 metre high and calls for no special remark.' Ebendort S. 213 ff. hat Wace eine Anzahl Reliefs und Münzen zusammengestellt, welche ebenfalls den Gott auf dem O. sitzend zeigen. Die beiden Reliefs von Ikaria sind schon oben besprochen worden, noch nicht erwühnt ist aber

13. das Votivrelief des Britischen Museums (Cat. Sculpture 1 nr. 776 = Overbeck, Kunstmythol. Atlas 21, 8). Nach Overbeck, Apollon S. 284 stammt das Bildwerk aus griechisch-römischer Periode und stelli dar den am rechten Ende der Platte auf dem O. sitzenden Apollon, der in der rechten Hand nach Wace einen Lustrationszweig (nach Overbeck a. a. O. ein Szepter) hält. Vor ihm stehen zwei attributlose mit Stephanen geschmückte Frauen (Artemis und Leto?), weiter links ein menschlicher Vater mit zwei Söhnen in römisch-militärischer Tracht. `The o. is a plain conical stone' (Wace). Ganz ähnlich soll nach Overbeck S. 285 und Wace (S. 213)

14. ein aus Modena stammendes Relief in Wien (Antikensammlung 11 nr. 154, abgebildet bei Cavedoni, Marmi Modenesi tav. 1, vgl. p. 192, und bei v. Sacken, Die antiken Skulpturen in Wien Taf. 18; vgl. S. 38) sein. Waces Beschreibung lautet: `On the extreme left on a square plinth stands a circular altar decorated with the usual bucrania and garland pattern. A fire burns on it. On the right is a netted omphalos rather flat in outline. On it sits Apollo to the left... It is undoubtedly Graeco-Roman work of about the second century A. D. and probably a modification of an earlier type.'

15. Auf ein nicht unbedeutendes Original noch guter Zeit scheint zurückzugehen die bekannte Statue des langgewandeten (d. h. mit einer Chlamys von außerordentlicher Größe bekleideten) Kitharoden Apollon in Petworth, dem `hinter dem r. Fuß ein mit Stemmaten geschmückter O. beigegeben ist, der wesentlich als Stütze des hinten auf ihn herabfallenden Gewandes dient und dem, wie er von vorn nur wenig sichtbar ist, schwerlich eine tiefere Bedeutung, diejenige den Gott in eine besonders nahe Beziehung zu Delphi zu setzen, beigelegt werden darf' (Overbeck, a. a. O. S. 185; vgl. die Abbildungen im Atlas Taf. 21, 33, Müller-Wieseler, D. a. K. 2, 133, Clarac 496, 966 etc.).

16. und 17. Hier sind ferner zu erwähnen zwei Statuen in Villa Albani und in Neapel, die beide den Gott auf dem Dreifuß sitzend und die Füße auf dem O. ruhen lassend darstellen. `Der Dreifuß, auf dem der Gott sitzt, ist mit einer aus geknoteten Wollbinden netzförmig geknüpften Hülle überhängt, welche auch den Omphalos bedeckt... und der außerdem in der albanischen Statue von einer breiten, horizontalen Binde, in der neapolitanischen von zweien desgleichen umgeben ist, deren eine den heiligen Stein vertikal, die andere schräg an der Vorderseite umwindet. Dass mit diesem Dreifuß der mantische von Delphi gemeint sei, wird durch den... Omphalos erwiesen' (Overbeck a. a. O. S. 231 ff.; vgl. die Abbildungen in dessen Atlas Taf. 23 nr. 30; Clarac 486, B 937 A; Denkm. a. K. 2 nr. 137 (Albani) und Mus. Borb. T. 13 tav. 41; Clarac 486 A 937 u. 485, 937 (Neapel)). Der O. hat in beiden Fällen die Gestalt einer Halbkugel und ist ohne Basis.

18. Schönes Relief im Louvre, darstellend den langgewandeten Apollon Kitharodos, dem die geflügelte Nike eine Spende eingießt, zwischen ihnen der halbkugelförmige mit Tänien geschmückte Omphalos, der dem auf der dresdner Dreifußbasis dargestellten ganz ähnlich ist: Taf. 7, 1. Vgl. Overbeck, Apollon, S. 261 ff. (wo noch weitere hierher gehörige Reliefs aufgezählt sind), Atlas Taf. 21 nr. 11; Müller-Wieseler D. d. a. K. 1, 47; Clarac 122, 41.

19. Marmorrelief an der dresdner Basis (Overbeck a. a. O. S. 405, Atlas 24, 14, Müller-Wieseler 1, 41 etc.); Herakles l., weicht vor Apollon, indem er den Dreifuß über der Schulter im l. Arme hält. In der R. schwingt er die Keule, in der L. trägt er den Bogen. Apollon greift mit der R. in einen der Ringe des Dreifußes, in der L. führt auch er den Bogen. Zwischen den beiden am Boden der halbkugelförmige, wie in nr. 18 mit Tänien geschmückte, basislose Omphalos. Auf dasselbe Original gehen auch noch andere von Overbeck S. 406 aufgezählte Reliefs in Venedig (aus Kythera) usw. zurück. S. Taf. 7 Fig. 2.

\subsubsection{Die Omphalosdarstellungen in Wandgemälden usw.}
\paragraph{}
20. Hier ist an erster Stelle zu erwähnen das schöne neuerdings im Hause der Vettii entdeckte, Apollon als Sieger über den Drachen Python darstellende Gemälde, das P. Herrmann, Denkmäler der Malerei Farbentaf. 2 und Taf. 20 abgebildet und im Text S. 29 ff. eingehend besprochen hat.\footnote{Danach auch die Abbildung im Artikel `Python' des Lexikons der Mythol. 3, Sp. 3407/8. Ich verdanke die Kenntnis des bedeutenden Bildes der Freundlichkeit P. Herrmanns.} Herrmann sagt darüber: `Apollon, in lebhafter... Bewegung weit ausschreitend, hält im linken Arm die Leier, während die rechte Hand mit dem Plektron über die Saiten gleitet... Links von Apollon erscheint der netzumsponnene [fast halbkugelförmige] Omphalos [auf einer viereckigen Basis stehend], um den der Schlangenleib des [sterbenden oder eben gestorbenen] Drachen sich windet. Dahinter sind an einer Säule... Bogen und Köcher aufgehängt. Rechts steht... auf einen Pfeiler gelehnt Artemis... Links ist ein mächtiger hellfarbiger Stier von einer jugendlichen Tempeldienerin herangeführt worden... In der Linken schultert das Mädchen an langem Stiele eine Doppelaxt. Fast im Mittelpunkt des Bildes, aber etwas in den Hintergrund gerückt, erscheint endlich noch eine jugendliche Männerfigur, bekränzt und dicht eingehüllt in ein langes weißes Gewand... Der Sinn der Darstellung ist klar: Apollon hat den Drachen erlegt und stimmt zur Feier seines Sieges den Paian an, der nach der Sage zum ersten Male nach dem Drachenkampf erklungen sein soll... Die Festfeier, welche für Apollons Drachensieg in Delphi eingesetzt wurde, die Pythien, fielen in den Monat Bukatios, ein Name, dessen Form auf Stieropfer hinweist. Auf sie und ihre Einsetzung... soll die Stiergruppe unseres Gemäldes hinweisen.'\footnote{Ähnlich Helbig a. a. O. S. 63 nr. 231: Apoll, lorbeerbekränzt mit flatternder roter Chlamys, nach r. schreitend, das Plektron in der R., rührt mit der L. die Kithara; r. am Boden der Omph., um den sich eine Schlange windet. Über die Gestalt des O. kann ich leider nichts sagen. S. uns. Taf. 9, 1.}

\begin{figure}[H]
\centering
\includegraphics[width=0.85\textwidth,keepaspectratio]{figs/fig03.jpg}
\caption*{\frakfamily Apollon feiert seinen Sieg über Python. (Nach P. Herrmann-Bruckmann, Denkmäler der Malerei des Altertums 3, 20.)}
\end{figure}
\paragraph{}
21. Ein zweites Wandgemälde von Pompeji (abgeb. Museo Borbon. 10. Taf. 20, Müller-Wieseler 2 nr. 136, vgl. Helbig, Wandgem. nr. 184) zeigt den jugendlichen lorbeerbekränzten Apollon mit einem Lorbeerzweig in der Rechten, seine Leier auf den Omphalos stützend. Dieser ist niedrig, mit einem Wollennetz umgeben, ohne Basis, oben etwas abgeplattet (um besser als Stütze für die Leier zu dienen) und verjüngt sich auch nach unten ein wenig, so dass er, bei weiterer Verlängerung der Kurve nach unten, sich der Eiform nähern würde.

22. Auf einer `etruskischen' Cista (abgeb. Mon. d. Inst. 8 pl. 25-30 = Daremberg-Saglio, Dict. 1 p. 321 Fig. 383) erscheint Apollon vor dem Omphalos auf einem Sessel linkshin sitzend. Er hält in der R. eine Schale, in der L. einen Lorbeerzweig. Vor ihm steht im vollen Waffenschmuck, aber ohne Helm der bärtige `Oedipus' [?], um den Gott zu befragen. Der Omphalos ist ziemlich hoch, ohne Basis, von konischer Gestalt und mit einem Netze umsponnen, das dem in nr. 21 ganz ähnlich ist. Auf der Spitze des Nabelsteins sitzt ein größerer Vogel (Rabe oder Adler) der seinen Kopf nach dem Gotte hinwendet. Zwischen Apollon und Oedipus erscheint im Hintergrunde eine vollkommen nackte, mir unverständliche Jünglingsgestalt, die sich dem Gott zuwendet.

23. Terrakotta in St. Petersburg, beschrieben von Stephani im Compte Rendu de St. P. 1870/71 (Petersb. 1874) S. 164 und abgebildet in dem dazu gehörigen Atlas Taf. 2, Fig. 3. Die Beschreibung Stephanis lautet: `Eine dritte auf Taf. 2 nr. 3 abgebildete Gruppe stellt Apollon auf einem leider stark beschädigten Würfel sitzend dar. Zwischen diesem Würfel und den Füßen des Apollon\footnote{Dieser Omphalos hat also beinahe wie in nr. 16 u. 17 die Bedeutung einer Fußbank, doch ruhen die Sohlen nicht auf ihm, sondern nur die Fußgelenke, indem die Füße selbst noch ein wenig über den O. hinausragen.} sieht man den Delphischen Omphalos [niedrig, ohne Basis und Netz, halbkugelförmig], dessen rote Färbung wohl erhalten ist; und da nicht bezweifelt werden kann, dass in Delphi der Altar und der Omph. räumlich eng verbunden waren,\footnote{Stephani beruft sich hier auf Wieseler Annali d. 1 29 p. 160. 80. Gött. Gel. Anz. 1860 p. 161-196. Nachr. d. K. Ges. d. Wiss. zu Gött. 1872 nr. 7. Arch. Ztg. 1872 p. 69. S. aber oben S. 63 Anm. 115.} so werden wir wohl nicht irren, wenn wir in dem Würfel, auf welchem der Gott hier sitzt, den Altar jenes Heiligtums vermuten.'

\subsubsection{Der delphische Omphalos auf Münzen.}
\paragraph{}
An erster Stelle sind hier natürlich die Münzen von Delphi zu nennen.\footnote{Vgl. dazu die treffliche Abhandlung von Svoronos (Νομισματικὴ) im Bull. de Corr. Hellén. 1896 [20] mit Tafeln.}

24. Die älteste, unzweifelhaft den delphischen Nabelstein darstellende Münze ist der nach allgemeiner Annahme um 346 v. Chr. geschlagene schöne Silberstäter der Amphiktionen, dessen Obvers das Haupt der Demeter mit Schleier und Ährenkranz schmückt, während der Revers darstellt: `Apollo wearing long chiton with closefitting sleeves seated l. on the Delphian omphalos over which hang fillets; his r. arm rests on lyre, in his l. a long branch of laurel. In field l., tripod.' (Catal. of greek coins in the Brit. Mus. Central Greece p. 27 = Pl. 4 13. $\svgAAR$.\footnote{Vgl. ferner: Svoronos a. a. O. Imhoof-Blumer and P. Gardner, Numismat. Comm. on Paus. Taf. Y nr. 7. Overbeck, Kunstmythol. Apollo S. 307 u. Münztaf. 3 nr. 35. Müller-Wieseler D. d. a. K. 2 93 u. 134 b. Studniczka im `Hermes' 1902 (37) S. 261 Fig. 2 (vergrößert). Head, Hist. num. 2 S. 342 f. Fig. 192.}) Der Omphalos hat keine Basis, ist von annähernd halbkugelförmiger Gestalt und fast kniehoch, so dass er sich bequem zum Sitzen eignet. S. Taf. 1 nr. 7.

25. Aus der Zeit Hadrians stammt die ebenda Pl. 4 20 abgebildete Erzmünze, deren Rev. p. 29 so beschrieben wird: $\svgAAS$ Rock, upon which Delphian omphalos, around which serpent twines itself. S. Taf. 1, 8. Hier ist der O. ebenfalls ohne Basis, zeigt aber kein Netz und stellt einen etwas spitzer auslaufenden Kegel dar. --- Nach Head Hist. num. 2 S. 342 (vgl. Rev. Num. 1860 Pl. 12, 8) kommt dieser Typus schon auf den Triobolen der Amphiktionenzeit vor, also in der Zeit zwischen 346 und 339 v. Chr.\footnote{Vgl. auch Imhoof-Blumer and P. Gardner a. a. O. S. 121, die auch verzeichnen: `Omphalos, entwined by snake, and covered with net work. AR. Anton. Berlin. Rev. Num. 1860 pl. 12, 8. AE Hadrian' und `Omphalos on basis. AE. Hadrian. Imh.'}

26. Als zweifelhaft muss es hingestellt werden, ob man den auf dem Revers der ältesten delphischen Münzen (deren Obvers einen Dreifuß darstellt) abgebildeten Kreis mit einem Punkt in der Mitte als `Orbis terrarum'\footnote{Vgl. Herod. 4, 36: γελῶ δὲ ὁρέων γῆς περιόδους γράψαντας πολλοὺς ἤδη... οἳ Ὠκεανόν τε ῥέοντα γράφουσι πέριξ τὴν γῆν, ἐοῦσαν κυκλοτερέα ὡς ἀπὸ τόρνου κ. τ. λ.} mit dem Omphalos [= Delphi] im Zentrum oder als eine φιάλη ὀμφαλωτή (O. Jahn, Versammlung zu München p. 98 u. Taf. 1 Fig. 3) aufzufassen hat. Für die erstere Deutung treten ein Head im Catal. of greek coins. Centr. Greece (1884) S. 33 u. S. 24 (vgl. Pl. 4, 4), Imhoof-Blumer and P. Gardner, Num. Comm. on Paus. p. 121, Head, Hist. num. 1 p. 289, für die letztere Head, Hist. num. 2 p. 340 ff. und Svoronos Bull. Corr. Hell. 20 (1896), der die betreffenden Münzen in die Zeit zwischen 520 und 500 v. Chr. setzt, aber daneben auch an die Möglichkeit denkt, dass das $\odot$ als Buchstabe (= θ) aufgefasst werden könnte.\footnote{Vgl. Agathon im Telephos (= Nauck fr., trag. 1 593) bei Athen. 454 d: γραφῆς ὁ πρῶτος ἦν μεσόμφαλος κύκλος [= θ].} Ich gestehe, dass ich die frühere Auffassung deshalb für die wahrscheinlichere halte, weil neben dem Dreifuß des Obverses der Omphalos ein viel charakteristischeres Symbol für Delphi und sein Orakel darstellt als die ziemlich nichtssagende Trinkschale (s. nr. 29). Übrigens wäre es nicht ganz undenkbar, dass auch die φιάλη μεσόμφαλος oder ὀμφαλωτή ein Sinnbild für den orbis terrarum und den in dessen Mitte befindlichen ὀμφαλὸς γῆς bedeutet hätte. In diesem Falle würde auch die in der Mitte des Tempelgiebels angebrachte Omphalosschale auf delphischen Münzen als Symbol der zentralen Lage Delphis aufgefasst werden können (vgl. Catal. of the gr. coins in the Brit. Mus. a. a. O. Taf. 4 nr. 22 u. daselbst S. 29 unten) S. Taf. 1 nr. 6.

Mit großer Wahrscheinlichkeit können wir ferner die Omphalosdarstellungen auf Münzen derjenigen Städte auf Delphi beziehen, die nachweislich mit dem delphischen Orakel in Beziehung gestanden haben. Hierher gehören vor allen folgende, Städten des griechischen Westens angehörige Münzen:

27. Auf dem Revers von Erzmünzen von Rhegion, die in der Periode von 415-387 v. Chr. geprägt worden sind,\footnote{Vgl. jedoch auch Head, Hist. num. S. 111, nach dem die folgenden Münzen erst dem 3., 2. und 1. Jahrh. v. Chr. angehören.} mit der Beischrift $\svgAAT$ findet sich öfters der Kopf Apollons mit langem Haar und Lorbeerkranz, dahinter der Omphalos (Cat. Brit. Mus. Italy S. 378), oder auch ein Dreifuß, darunter der Omphalos (ebenda S. 379), oder auch `Apollo, naked, seated l. on omphalos, holding arrow and strung bow' (S. 380; s. Taf. 1, 14). Da Rhegion für eine ἀποικία ἐκ Δελφῶν,\footnote{Timaios b. Antigon. Paradox. 1, 1: τῆς ὅλης Ῥηγίνων ἀποικίας ἐκ Δελφῶν καὶ παρὰ τοῦ θεοῦ γεγενημένης. Heracl. Pont. π. πολιτ. 25: Ρήγιον ᾤκισαν Χαλκιδεῖς οἱ ἀπ᾿ Εὐρίπου... παρέλαβον δὲ καὶ ἐκ Πελοποννήσου τοὺς Μεσσηνίους. καὶ χρησμὸν ἔλαβον· Ὅπου ἂν ἡ θήλεια τὸν ἄρρενα κ. τ. λ. Mehr b. O. Müller, Dorier 1 1, 260, 3 ff.} so kann nicht bezweifelt werden, dass der Omphalos auf seinen Münzen der delphische sein soll.

28. Ungefähr dasselbe gilt von den Münzen von Neapolis in Campanien (einer κατὰ χρησμὸν gegründeten Kolonie der Chalkidenser von Kyme),\footnote{Scymn. 250: Ἐκ τῆς δὲ Κύμης τῆς πρὸς Ἀόρνῳ κειμένης | κτίσιν κατὰ χρησμὸν ἔλαβεν ἡ Νεάπολις...} die sehr oft den Apollokopf zeigen (Cat. Brit. Mus. Italy 108 ff.). Vgl. a. a. O. S. 116: `Obv. Head of Apollo, l., laur.; behind $\svgAAU$ --- Rev. $[\svgAAV]\svgAAW\enspace\svgAAX$ (in ex.) Omphalos and lyre...' oder: `omphalos, above which, crested serpent, l.; and lyre...' usw.

29. Erzmünzen der Mamertiner von Messana zeigen nach Catal. Brit. Mus. Sicily 113 auf dem Obvers `Head of Apollo l., laur., auf dem Revers: $\svgAAY$ Omphalos: border of dots.' Da Messana ebenfalls eine Gründung der chalkidischen Kymaier war und hier nach Ausweis der Münzen ebenso wie in Kyme und Chalkis selbst ein Kult des Apollon und der Artemis blühte (a. a. O. 110 ff.),\footnote{Vgl. auch Roscher im Philologus 71 (25) 1912 S. 307 f.} so kann es sich auch hier nur um den delphischen Omphalos handeln.

30. Sehr alt ist der Kult des Apollon und besonders des Apollon Pythios (nach Paus. 1, 42, 5) in Megara. Wir dürfen also wohl unbedenklich (mit Imhoof-Blumer und P. Gardner im Num. Comm. on Paus. S. 6; vgl. Taf. A 9) die Erzmünze des Geta, die Apollon, langgewandet und eine Schale und Leier haltend, vor einem Omphalos (oder Altar,\footnote{Nach Head im Catal. Brit. Mus. Attica S. 124 (vgl. Taf. 22 nr. 7) ist der Omphalos eigentlich ein Altar, während Imhoof-Bl. und Gardner a. a. O. schwanken, ob es sich um einen Altar oder einen Omphalos handelt.} auf dem zwei Vögel [Adler oder Raben?] sitzen) zeigt, auf den delphischen Pythios und seinen Omphalos beziehen.

31. Silbermünzen (des 3. Jahrh. v. Chr.) von Kalchedon, der Kolonie der Megareer, zeigen nach Catal. Brit. Mus. Pontus etc. S. 126 (vgl. Taf. 27 nr. 12 f.) auf der Rückseite `Apollo naked, seated r. on chlamys placed on netted omphalos; in r., arrow; in l., bow,' darunter $\svgAAZ$. Vgl. auch Overbeck, Kunstmyth. Apollon, Münztafel 3, 38 und dazu Text S. 300. Hier hat der basislose Omphalos eine ziemlich spitz zulaufende Form. S. Taf. 1, 10.

Sehr bekannt und alt sind die zahlreichen Beziehungen, welche das delphische Orakel mit Kreta verbinden.\footnote{Vgl. Otfr. Müller, Dorier 1 1, 206 f. 212 f. Svoronos im Bull. de Corr. Hellén. 20 (1896) S. 8 f. Gruppe, Gr. Mythol. n. Rel.-Gesch. 1 101 ff., der (S. 103) auch den delphischen Omphalos vom kretischen Omphalion (nicht weit von Knosos, ziemlich in der Mitte von Kreta gelegen) ableiten möchte. Vgl. auch Aristot. b. Plut. Thes. 16: καί ποτε Κρῆτας εὐχὴν παλαιὰν ἀποδιδόντας ἀνθρώπων ἀπαρχὴν εἰς Δελφοὺς ἀποστέλλειν u. Höfer im Lex. d. Myth. 3 Sp. 3260.} Ich erinnere vor allem an den homerischen Hymnus auf den pythischen Apollon, nach dem der Gott selbst in Gestalt eines Delphins kretische Männer aus Knossos über das Meer nach Krisa und Delphi geleitet und daselbst als Priester einsetzt (v. 210 ff.), an den Kreter Karmanor, der Apollon und Artemis von dem Morde des Python gereinigt haben sollte, an dessen Sohn Chrysothemis, der in den pythischen Spielen zu Delphi durch einen Hymnus auf Apollon den ersten Sieg davontrug, an Eleuther, den Eponymos von Eleutherna auf Kreta, der ebenfalls im musischen Agon siegte (Paus. 10, 7, 3), an die von Pausanias (10, 5, 10) berichteten eigentümlichen Beziehungen, welche Pteras oder Apteros, der Erbauer des ersten delphischen Tempels, zu der kretischen Stadt Aptera haben sollte,\footnote{Vgl. W. Wroth in Catal. Brit. Mus. Crete etc. Introd. p. 30, der auf Asklep. b. Parthenius (π. ἐρωτ. παθ. 35 = p. 32, 11 ff. Hercher) und Eusebios verweist. Mehr b. Höfer im Lex. d. Myth. 3 unter Pteras u. Crusius ebenda 1 2808.} an den Kreter Kastalios, den Führer des knosischen Schiffes, das Apollon als Delphin nach Krisa geleitete (Tzetz. z. Lyk. 208). Bei diesen überaus nahen und alten Beziehungen des delphischen Apollonkultus zu dem kretischen, die sogar die Frage nahelegen, ob nicht der delphische Kult eigentlich ein Ableger des kretischen sei,\footnote{Gleichzeitig mache ich auf die schon von Otfr. Müller, Dorier 1 1 206 ff. 215 ff. festgestellten zahlreichen Beziehungen kretischer und kleinasiatischer Apollokulte aufmerksam, die eine durch Kreta vermittelte indirekte Beeinflussung Delphis von Kleinasien aus sehr wahrscheinlich machen, zumal wenn man an die vielfachen Ähnlichkeiten der Riten und Mythen denkt.} werden wir kein Bedenken tragen, auch den auf Münzen mehrerer kretischer Städte, insbesondere der eben genannten Eleutherna und Aptera (Aptara; vgl. Patara) erscheinenden Omphalos mit dem delphischen zu identifizieren.

32. Revers einer Hemidrachme von Aptara auf Kreta aus der Zeit zwischen 250 u. 67 v. Chr. mit Inschrift $\svgABA$: Apollon nackt, sitzend auf dem etwas undeutlich gebildeten, basislosen, kniehohen (wie es scheint, mit einem Kissen bedeckten) Omphalos, den l. Arm auf die Leier stützend, in der r. Hand eine Schale haltend; auf dem Obv. Kopf der Artemis; vgl. Overbeck, Kunstmyth. Apollon Münztaf. 3, 22 u. Text S. 300 nr. 9 u. S. 307; Catal. Brit. Mus. Crete etc. Pl. 2, 9 u. Text S. 9, wo der Omphalos wohl fälschlich als Felsen (`rock') gedeutet ist.

33. Revers einer Silbermünze des 4. vorchr. Jahrhunderts von Chersonasos auf Kreta, deren Obvers ebenfalls mit dem Kopf der Artemis (Britomartis) geschmückt ist, beschrieben von W. Wroth im Catal. Brit. Mus. Crete etc. S. 16: $\svgABB[\svgABC]$ Apollo, naked, seated r. on netted omphalos (fast kniehoch, basislos und ziemlich halbkugelförmig,\footnote{Wie eine umgestürzte Kesselpauke!} der unterste Teil ist als deutlicher Streifen gebildet); holding in r., plectrum, and with l. supporting lyre, which rests upon his knee; in field thymiaterion. S. die Abbildg. auf Taf. 4 nr. 1. Vgl. Overbeck, Apollo, Münztaf. 3 nr. 36, Text S. 300 u. 307 (hier handelt es sich um eine Silbermünze aus der Wiener Sammlung). S. uns. Taf. 1 nr. 15.

34. Rückseite einer Bronzemünze des 3. Jahrh. v. Chr. von Eleuthernai auf Kreta: $\svgABD$ Apollo wearing bow and quiver, seated l. on netted omphalos (dieser ist basislos und fast würfelförmig gebildet!) before which is lyre; he holds in r., stone... Vgl. die Abbildung Pl. 8 nr. 13 und Text S. 34 a. a. O. Overbeck a. a. O. Münztaf. 3 nr. 23 u. Text S. 300 u. 307 (Kupfermünze aus Imhoofs Sammlung); Müller-Wieseler, Denkm. a. K. 2 nr. 136.

Dies sind die Omphalosdarstellungen auf Münzen, in denen wir entweder mit voller Sicherheit oder großer Wahrscheinlichkeit den delphischen Nabelstein anzuerkennen haben; bei anderen Münzen, z. B. denen von Milet und dessen Kolonien (z. B. Kyzikos; s. ob.), von Patara und benachbarten Städten, von Kypros, wo Paphos sich rühmte der Nabel der Erde zu sein, usw. ist die Beziehung auf Delphi mindestens zweifelhaft: daher wir diese Münzdarstellungen, ebenso wie die auf den Asklepioskult bezüglichen, an andern Orten zu besprechen haben.

\subsubsection{Der delphische Omphalos in Vasenbildern.}
\paragraph{}
Ziemlich zahlreich sind die hier aufzuführenden Vasengemälde, die meist die Orestessage darstellen. Hier erscheint der Nabelstein in den mannigfaltigsten Formen, bald niedrig, basislos und fast halbkugelförmig, bald als `Bienenkorb,' bald als hoher auf Stufen sich erhebender ziemlich spitzer Kegel, oder wie ein gewaltiges am unteren Ende plattgedrücktes Ei gestaltet, fast immer aber mit deutlichem Netzwerk (ἀγρηνόν) oder Tänien geschmückt, bisweilen auch wie aus einem Blumenkelche emporsteigend. Aus dieser Mannigfaltigkeit der Formen erkennt man deutlich, dass viele Vasenmaler bei der Darstellung des Omphalos sehr frei verfuhren und ihre Phantasie ziemlich zügellos walten ließen. Wir wollen, um das recht deutlich zu machen, zuerst diejenigen Vasen aufzählen, die den Nabelstein in einer Gestalt wiedergeben, die einigermaßen den Omphaloi der besten und authentischsten oben aufgeführten Skulpturwerke entspricht, und später zu den mehr phantastischen Formen fortschreiten. Ich zitiere die Vasen, wo es irgend angeht, nach dem trefflichen und weitverbreiteten Répertoire des vases peints von Sal. Reinach, Paris 1899-1900, wo übrigens auch eine Anzahl von Vasenbildern zu finden ist, die Orestes nicht auf den Stufen des Omphalos, sondern auf dem Altar\footnote{Vgl. Reinach a. a. O. 1, 53. 1, 390, 1. 1, 400. 467. 316. Overbeck, Gall. S. 708 f. 711 (Taf. 29, 8). S. auch unt. Anm. 192 f.} knieend oder sitzend darstellen. Wir werden später sehen, dass auch sonst hie und da auf Vasenbildern der Omphalos als Altar (βωμός) aufgefasst wird.

35. Reinach a. a. O. 1 S. 390 Nr. 2, 3 = Arch. Zeitg. 1860 Taf. 138, 1. Basilicate. Naples. Amphore à colonettes; rehauts blancs. A. Oreste devant Apollon, assis sur l'omphalos de Delphes [dieser ist halbkugelförmig ohne Basis und mit Tänien geschmückt, die von der Spitze herabhängen]; à dr., Pylade et la Pythie sur le trépied; à g. Électre... Vgl. Arch. Ztg. 1860 p. 49. Raoul-Rochette, Mon. Ined. pl. 36. 37. Heydemann, D. Vasensammlungen zu Neapel 1984. Overbeck, Gall. her. Bildw. Taf. 29, 11, Text S. 715 f. Baumeister, Denkm. 2 S. 1110, Fig. 1307. S. uns. Taf. 2, 1.

36. Reinach a. a. O. 1 S. 8 = Compte Rendu... de St. Pétersbourg, Atlas 1861 pl. 4: Apollon et Dionysos se donnent la main de part et d'autre d'un palmier; au dessous, l'omphalos [halbkugelförmig, basislos, mit Wollnetz und einer Lorbeerguirlande geschmückt]. Derrière Dionysos, une Ménade et deux Silènes musiciens; derrière Apollon, une Ménade jouant du tympanon, une autre plaçant un coussin sur un siège, un Silène assis, un trépied. Overbeck, Apollon p. 331; Atlas 21 25. Kekulé, Hebe p. 31, pl. 5, 3. Arch. Zeitg. 1866 Taf. 211. Stephani, Vasens. d. Ermitage Nr. 1807. Baumeister, Denkm. S. 103 f. Unsere Taf. 2, 2.

37. Reinach a. a. O. 1 S. 3 = Compte Rendu de St. Pétersb., Atlas 1860 pl. 2: Iouz-oba. Ermitage. Polychromie et dorure. Zeus assis (Admète? suivant Stephani); à sa g., Athéné debout; au-dessus, Niké. Sur la dr., Artémis ou Séléné à cheval (Érinys ou Apaté suivant Klein), précédée d'un éphèbe. Sur la g. Hestia (?) assise sur l'omphalos [halbkugelförmig, ohne Basis, mit Netzwerk]; au-dessus, Hermès; plus loin à g., Aphrodite assise (Alceste? suivant Stephani) et une femme debout (Peitho?)... Suivant Strube (Bilderkreis von Eleusis p. 86), consultation de Zeus avec Thémis et d'autres divinités au sujet de la guerre de Troie. Overbeck, Zeus p. 181. K. Robert, Archaeol. Märchen, Taf. 3. Klein, Jahrb. d. Inst. 1894 p. 250. Stephani, Vasens. d. Ermitage Nr. 1793.

38. Reinach a. a. O. 1 S. 313 = Annali d. Inst. 1865 Taf. H: Cumes. Naples. Péliké. Rehaussements blancs. A. Corbeau sur l'omphalos de Delphes [dieser ist basislos, halbkugelförmig und mit dem Agrenon behängt], entre Artémis et Apollon (ou sacrifice d'Hécate? Archäol.-epigraph. Mitteilungen aus Österreich-Ungarn 5 p. 40. 6 p. 55).

39. Reinach a. a. O. 2 S. 183 = Laborde, Collect. des vases grecs de M. le comte de Lamberg 1 pl. 27: Vienne. Cratère. Aphrodite sur un cygne; au dessous, l'omphalos [halbkugelförmig, ohne Basis, mit Bindennetz]; à g., Zeus; à dr., Apollon. Les autres personnages sont Peitho (?), Hermès, Athéné (?). Arch. Zeite 1858, pl. 120. Annali 1845 p. 364. Jahrb. d. Inst. 1866 p. 258. Benndorf, Griech. u. sicil. Vasenbilder p. 78. Sacken-Kenner, D. Sammlungen d. k. k. Münz- u. Antiken-Cab. S. 217. Inghirami, Vasi fittili 3 Taf. 235.

40. Reinach a. a. O. 1 S. 397 = Arch. Ztg. 1865 Taf. 203 = Elite céramogr. 2, 45: S. Agata. Berlin (nr. 2645 Furtw.): Devant un temple, Apollon assis sur l'omphalos (kniehoch, bienenkorbförmig, wie es scheint, mit Lorbeer bekränzt), nourrissant un daim; de g. à dr., Hermès, Artemis (torches), Nymphe, Silene.

41. Rhyton in Neapel, abgebildet bei Hancarville, Antiq. étrusque, du cab. de M. Hamilton 2 30 f. `Orestes, von zwei Erinyen verfolgt, kniet auf dem heiligen Erdnabel in Bienenkorbform': Overbeck, Gall. S. 707.

42. Reinach a. a. O. 1 S. 19 = Compte Rendu de St. Pétersb. 1863 Atlas Taf. 6, 5: Campana [Tarentiner Vase nach Karo im Dict. d. antiq. s. v. Omphalos]. Fig. polychromes sur fond noir. A. Sous un temple ionique, dont la paroi est ornée d'un bouclier, Oreste, armé d'un glaive, est appuyé sur l'omphalos [dieser steht auf einer hohen dreistufigen Basis, ist eiförmig und mit einem Wollnetz behängt]; cinq Erinyes dorment alentour; sur la dr., la prêtresse porte-clefs fuit effrayée...

43. Reinach a. a. O. 1 S. 132 = Monumenti d. Inst. 4 Taf. 48: Campana. Louvre. Cratère. Oreste assis sur l'autel[?] de Delphes [vielmehr auf der Basis des hinter Orestes aufsteigenden, hohen, kegelförmigen, mit Netzwerk versehenen Omphalos]; derrière lui, Apollon tenant une branche de laurier, agite un goret au dessus de la tête du coupable (pour en répandre le sang sur son corps?). A dr. Artémis; à g., deux Erinyes endormies, devant lesquelles paraît l'ombre de Clytemnestre; plus bas, une troisième Érinys sort de terre. Annali 1847 p. 413. Arch. Ztg. 1860 Taf. 138, 2. Baumeister, Denkm. 2 p. 1117 Fig. 1314. Rayet et Collignon, Céramique p. 297. J. de Witte, Études sur les vases peints p. 108. Roscher, Lex. d. Mythol. 3 Sp. 983. Overbeck, Gall. her. Bildw. Taf. 29, 7. S. unsere Tafel 3, 3.

44. Reinach a. a. O. 1 S. 321 = Annali 1868 Taf. E. H: Ruvo. Coll. Jatta. Amphore. Νεοπτολεμος s'est réfugié sur l'autel d'Apollon; Ορεστας se dissimule derrière l'omphalos [dieser steht vor dem Tempel, ist hoch, eiförmig, mit Netzwerk versehen und erhebt sich aus einem blumenkelchförmigen Gebilde,\footnote{Ganz ähnlich auch auf der Orestes-Vase (etruskischer Fabrik?) im Brit. Museum = Annali 1847 pl. 10 = Overbeck, Gall. Taf. 29 Fig. 12 (Text S. 717) = Reinach 1 S. 276, der aber wohl den Omphalos verkannt hat (vgl. Overbeck S. 717, 2b).} das aus einer hohen 3-4stufigen Basis herauswächst]. A dr. du temple Απολλων assis; à g. du temple et d'un trépied on aperçoit la prêtresse avec sa grande clef. A gauche de Néoptolème, un jeune guerrier brandit un javelot. Palmier et trépied à droite. Donnée analogue à celle de l'Andromaque d'Euripide. --- Annali 1868 p. 235. Vogel, Scenen eurip. Trag. p. 36. Baumeister, Denkmäler Fig. 1215. Roscher, Lex. 3 S. 175. Huddliston, Greek tragedy p. 84. Catal. Jatta 239. S. unsere Tafel 2, 3.

45. Fehlt bei Reinach. Overbeck, Gallerie Taf. 29, 4 = Rochette, M. I. pl. 35 und Gerhard, Apul. Vasenb. Taf. 6: große Amphore von Ruvo in Berlin. Orestes kniet mit gezücktem Schwert auf der Basis (2stufig) des hohen, eiförmigen, mit Wollflocken bedeckten Omphalos, den er mit der L. umfasst. Rechts entweicht die von einer Tempeldienerin begleitete, verschleierte Pythia. Von l. stürmt eine geflügelte, mit Fackel und Schwert bewaffnete Erinys heran, welcher jedoch der auf seinem Dreifuß sitzende Apollon die Rechte gebietend und zurückweisend entgegenstreckt.

46. Unteritalischer Krater (vaso a campana) in Kopenhagen = Müller-Wieseler, D. a. K. 2, 13, 148 = Thorlacius, Vas pictum Italo-Graecum... exhibens 1826; vgl. Overbeck, Gall. S. 710 [fehlt bei Reinach]: Orestes, von zwei Erinyen verfolgt, sitzt ermattet hingesunken auf der zweistufigen Basis des hohen netzbedeckten eiförmigen Omphalos, neben dem der heilige Lorbeerbaum emporsprießt und hinter dem der große Dreifuß steht. Rechts davon steht Apollon mit Lorbeerzweig in der L., die R. schützend nach Orestes hinausstreckend.

47. Amphora der Sammlung Hope = Millin, Mon. inéd. 2, 29 u. Peint. de vases 2, 68 = Overbeck, Gall. Taf. 29, 9 (vgl. Text S. 712 f.) = Baumeister, Denkm. S. 1118 Fig. 1117, fehlt bei Reinach a. a. O. In der Mitte kniet Orestes an dem hohen, basislosen, netzbedeckten, eiförmigen Omphalos, hinter dem der Dreifuß steht. Hinter dem Dreifuß eine Erinys mit Schlange. Links von Orestes Apollon vor einem Lorbeerbaum zum Schutze des Orestes hintretend, links von ihm eine Erinys mit Schlange, rechts von Orestes Athene. Oben in den Ecken zwei Brustbilder (Klytaimnestras Schatten und Pylades darstellend). S. uns. Taf. 3, 1.

48. Reinach a. a. O. 1 S. 419 nr. 2 = Arch. Zeite. 1877 Taf. 4: Vienne. Cratère. Reh. blancs. A. Oreste à Delphes, réfugié sur l'omphalos (2stufige Basis, hoch, kegelförmig mit ziemlich spitzem Ende und Agrenon); une Érinys se précipite sur lui. B. La prêtresse s'enfuit, portant la clef et précédée du chien du temple. Sacken-Kenner, D. Sammlungen des k. K. Münz- u. Ant. Cab. S. 238 (243).

49. O. Jahn, Vasenbilder Taf. 1 = Bötticher, Der Omphalos des Zeus zu Delphi, 19. Winckelmannsprogr. Berlin 1859 Tafel: Orestes umklammert den hohen, kegelförmigen, beinahe an einen Zuckerhut gemahnenden, basislosen, in der mit ionischen Säulen geschmückten Cella des Tempels stehenden Omphalos; r. Artemis, l. Apollon und die Pythia (fliehend), links oben eine Erinys mit Schlange in der Hand, dem O. drohend. S. uns. Taf. 3, 2.

50. Besonderes Interesse beansprucht das Gemälde der schwarzfigurieren Lekythos in Paris, abgebildet in der Élite Céramographique 2 Taf. 1 A. Es stellt den noch auf den Armen der Leto getragenen kindlichen Apollon dar, wie er den in einer Felsengrotte sich verbergenden Pythondrachen mit einem Pfeilschusse zu erlegen sucht.\footnote{Vielleicht dient die hier angedeutete Felsenlandschaft zur Illustration von Hesych. s. v. Τοξίου βουνός· τοῦ Ἀπόλλωνος τοῦ ἐν Σικυῶνι. βέλτιον δὲ ἀκούειν τὴν ἐν Δελφοῖς Νάπην [νάπην?] λεγομένην· ἐκεῖ γὰρ καὶ ὁ δράκων κατετοξεύθη. καὶ ὁ ὀμφαλὸς τῆς γῆς τάφος ἐστὶ τοῦ Πύθωνος.} Anwesend ist die schon halb erwachsene Artemis, die aufmerksam nach dem Drachen und dessen Höhle hinblickt. Vor der als dachartig überhängender Fels dargestellten Grotte (= Adyton?) befindet sich ein ziemlich hoher, kegelförmiger und oben spitzgewölbter Omphalos, in dem man bisher einen Felsblock erblickt hat, hinter dem die gekrümmte Gestalt des Drachen sichtbar wird. Vgl. Overbeck, Apollon S. 377 f. Türk im Lex. d. Mythol. 3 Sp. 3408 f., wo auch eine Abbildung gegeben ist, und Schreiber, Apollon Pythoktonos S. 92. Das Bild ist deshalb von besonderer Wichtigkeit für uns, weil es, wenn meine Deutung des `Felsblocks' als Omphalos zutrifft, die älteste Darstellung des delphischen Nabelsteins bietet, der bisher auf schwarzfigurieren Vasen nicht nachgewiesen werden konnte.\footnote{Vor allem nicht auf den zahlreichen schwarzfig. Vasen, die den Raub des Dreifußes durch Herakles darstellen. Dies ist vielleicht für das Alter des delphischen O. nicht ohne Bedeutung.} S. unsere Tafel 4 Fig. 4.
\clearpage
\section{Weitere, wahrscheinlich nicht von Delphi abhängige Kulte des Apollon, Asklepios usw., in denen Omphaloi vorkamen.}
\paragraph{}
Wir haben oben gezeigt, dass Delphi keineswegs der einzige Ort der antiken Welt war, der sich rühmte, Nabel oder Mittelpunkt der Erde oder eines größeren und wichtigeren Teiles der Erde zu sein, und deshalb beanspruchte, als sichtbares Zeichen dessen einen heiligen Nabelstein zu besitzen, sondern dass neben Delphi und vielleicht noch vor der Entwicklung seines Apollokults Orte wie Branchidai und Paphos genau dieselbe Bedeutung in Anspruch nehmen durften. Wir sahen ferner (s. oben S. 54 f.), dass bereits im 7.-6. Jahrhundert der in theologischen Fragen so maßgebende Kreter Epimenides ernstlich an jener Bedeutung des delphischen Omphalos gezweifelt hat, wohl hauptsächlich deshalb, weil er noch andere Orte kannte, die ebenfalls beanspruchten, für Mittelpunkte der Erde zu gelten. Jetzt kommt es uns darauf an, womöglich mit Hilfe der Monumente noch weitere Kulte ausfindig zu machen, in denen ebenfalls heilige Nabelsteine von derselben Bedeutung wie der Omphalos in Delphi vorkamen.

1. An erster Stelle ist hier wohl das uralte, bereits in der Ilias wiederholt erwähnte Apollonheiligtum von Thymbra, in unmittelbarer Nähe von Ilion, zu nennen. Dass auch hier ein Omphalos bestand, scheint mit ziemlicher Gewissheit hervorzugehen aus dem höchst altertümlichen, schwarzfigurieren Bilde der jetzt in München befindlichen Vase Nr. 124 b. O. Jahn, Beschreibg. der Vasensammlung K. Ludwigs S. 37 f.\footnote{Abgebildet b. Gerhard, Auserl. Vasenb. 223. Overbeck, her. Gall. Taf. 15, 12. Baumeister, Denkm. S. 1902 Fig. 2001. Reinach, Rep. de vases 2 p. 113.} Jahns Beschreibung lautet: `Neben einem omphalosförmigen, mit einem gegitterten Überwurf bedeckten Altar ($\svgABE$)\footnote{Dieser `Altar' hat ungefähr die Gestalt eines Bienenkorbes und ist ohne Basis.} liegt der nackte, weiß gemalte Leichnam des Troilos ($\svgABF$ v. r.) auf dem Rücken ausgestreckt. Über ihm steht Achilleus ($\svgABG$ v. r.) mit Helm, Schwert, ausgeschnittenem Schild und gezückter Lanze, auf welche er den Kopf des Troilos gespießt hat. Hinter ihm steht Athene mit Helm, in der R. eine Lanze und einen Kranz, hinter dieser Hermes ($\svgABH$ v. r); in der R. das Kerykeion. Achilles gegenüber stehen vier gerüstete Krieger, Hektor, Aineas, hierauf Deiphobos ($\svgABI$ v. r.); der vierte ist nicht näher charakterisiert; von seinem Namen sind nicht mehr lesbare Spuren da.' Es unterliegt nicht dem geringsten Zweifel, dass die dargestellte Szene im Apollontempel zu Thymbra spielt, wo nach den Kyprien Troilos von Achilleus ermordet wurde.\footnote{Kyprien (?) b. Apollod. epit. 3, 32: Ἀχιλλεὺς ἐνεδρεύσας Τρωίλον ἐν τῷ τοῦ Θυμβραίου Ἀπόλλωνος ἱερῷ φονεύει. Vgl. Kinkel, Epic. gr. fr. 1 p. 20. Lykoph. 313 (Prophezeiung von Troilos' Tod): καρατομηθείς τύμβον [= βωμόν; vgl. v. 335 u. 613 u. Tzetz. z. d. St.: τύμβον καλεῖ τὸν βωμόν, καὶ Δοῦρίς φησιν ἐν τῷ περὶ ἀγώνων τοὺς βωμοὺς τάφους καλεῖσθαι] αἰμάξεις πατρός (über Troilos als Sohn Apollons: Apollod. 3, 12, 5, 7). Schol. A Il. Ω 257: ἐντεῦθεν Σοφοκλῆς ἐν Τρωίλῳ φησὶν αὐτὸν <λ>λο<γ>χευθῆναι ὑπὸ Ἀχιλλέως ἵππους γυμνάζοντα παρὰ τὸ Θυμβραῖον καὶ ἀποθανεῖν. Mehr über das Thymbraion b. Klausen, Aeneas S. 184 ff. S. auch Overbeck, hero. Gall. 338 ff. u. 359 ff. zu Atlas Taf. 15, 12. Beachtenswert erscheint, dass auch hier wie bei den Darstellungen von Orestes' Sühnung in Delphi als der geheiligte Platz, zu dem der Schutzbedürftige flieht, bald der Altar bald der Omphalos fungiert.} Es fragt sich nur, ob der `omphalosförmige, mit einem gegitterten Überwurf bedeckte Bomos' als Altar oder als Nabelstein aufzufassen ist.\footnote{An eine dritte Möglichkeit denkt Karo im Artikel Omphalos (Dictionn. d. antiq. 6, 1 p. 198a), der diesen `Bomos' wegen seiner großen Ähnlichkeit mit dem Grab des Achilleus auf der altattischen Amphora, mit der Schlachtung der Polyxena, besprochen und abgebildet von Walters, Journ. of Hellen. Stud. 1898 pl. 15, für einen Grabaufsatz (`tombeau-autel') erklären möchte, obwohl doch an ein Grab im Tempel von Thymbra kaum zu denken ist. S. unsere Tafel 4, 1.} Mir erscheint die zweite Auffassung als die bei weitem wahrscheinlichere, und zwar ausfolgenden Gründen. Erstens passt die Form des vermeintlichen βωμός viel besser zu einem Omphalos als zu einem Feueraltar, da sich seine gewölbte Gestalt nur sehr schlecht zu Brandopfern eignet. Zweitens verträgt sich der gegitterte Überwurf gar nicht mit den Zwecken eines Brandopferaltars, dagegen ganz vortrefflich mit der Bedeutung des Nabelsteins. Endlich braucht die Bezeichnung βωμός durchaus nicht mit der Auffassung des betreffenden Gegenstandes in unlösbarem Widerspruch zu stehen, da ja βωμός mit βαίνω, βῆμα, βηλός zusammenhängt und schon bei Homer öfters Tritt, Stufe (η 100), Gestell (Θ 441) bedeutet.\footnote{Auch erscheint es, im Hinblick auf die von O. Jahn, Vasensammlg. in München S. 115 A. 838 u. 854 angeführten Beispiele für verkehrte Inschriften auf Vasen, wohl denkbar, dass auch hier ein Versehen vorliegen könnte.} Fraglich kann nur sein, ob die Vorstellung von einem Omphalos in Thymbra von Delphi oder Branchidai unabhängig ist oder nicht. Vielleicht führt ein neues literarisches oder monumentales Zeugnis die Entscheidung dieser Frage herbei. S. Taf. 4 Fig. 2.

2. Der Omphalos im Apollotempel zu Patara in Lykien wird fraglos bezeugt durch Münzen dieser Stadt aus der Kaiserzeit, deren eine im Catal. of the greek coins in the Brit. Mus. Lycia etc. S. 77 so beschrieben wird: $\svgABJ$: Bust of Gordian r... --- Rev. Apollo standing to l. wearing long chiton and himation; in r., laurel-branch; in l. bow; in field, to r. eagle on omphalos. Vgl auch die Beschreibung der folgenden Münze, wo die `column' (?) entwined by serpent' wohl zweifellos als Omphalos aufzufassen ist,\footnote{Dieselbe Münze beschreibt Imhoof-Blumer, Kleinasiat. Münzen 2 S. 307 Nr. 2 so: `Links zu Füßen des Gottes Adler links hin auf Omphalos (?), den Kopf zurückwendend und die Flügel schlagend, rechts schmaler Dreifuß, an dem sich eine Schlange emporringelt.'} sowie die unmittelbar vorhergehende, wo links neben Apollon `eagle on omphalos entwined by snake' erblickt wird (vgl. Taf. 16, 2 u. 3). S. auch Müller-Wieseler, Denkm. 2 Nr. 135, der den `Adler' als `Raben' deutet und sich dafür auf einige Analogien beruft. Overbeck, Apollon S. 310, Münztaf. 5, 6, uns. Taf. 1, 17. --- Ebenso erscheint auf dem Revers der Münzen der lykischen Stadt Masikytes aus dem 2. Jahrh. v. Chr., deren Obvers das lorbeerbekränzte Haupt Apollons schmückt, ein von einer Schlange umringelter Omphalos (basislos, ziemlich hoch und schlank, spitzgewölbt), daneben eine Leier und die Beischrift $\svgABK$ (Catal. a. a. O. S. 63; Taf. 13, 11). Das Orakel von Patara war uralt und durchaus selbständig, d. h. von Delphi unabhängig, wie schon aus der einheimischen Legende von Εἰκάδιος (= Icadius) bei Serv. z. Verg. Aen. 3, 332 hervorgeht: Icadius, Apollinis et Nymphae Lyciae filius, cum in adultam aetatem venisset, primo regionem, in qua natus erat, a matre Lyciam nominavit, deinde in ea urbem quoque Apollini condidit, sortes et cortinam consecravit, et, ut illum patrem esse testaretur, `Patora' cognominavit (= Patara).\footnote{Einen προφήτης τοῦ πατρῴου Ἀπόλλωνος zu Patara erwähnt die Inschrift Journ. of hell. Stud. 10 (1889) 76.} Inde cum Italiam peteret, naufragio vexatus delphini tergo exceptus dicitur, ac prope Parnassum montem delatus, patri Apollini templum constituisse et a delphino locum Delphos appellasse: aras deinde Apollini tamquam patri consecrasse quas ferunt vulgo patrias dictas. Hinc ergo et delphinum aiunt inter sacra Apollinis receptum, cuius rei vestigium est, quod hodie <quo>que 15 virorum cortinis delphinus in summo honore ponitur et pridie quam sacrificium faciunt, velut symbolum delphinus circumfertur, ob hoc scilicet quia 15 viri libror. Sibyllinorum sunt antistites. Sibylla autem Apollinis vates... est. Invenitur tamen apud Cornificium Longum Iapydem et Icadium profectos a Creta in diversas regiones venisse, Iapydem ad Italiam, Icadium vero duce delphino ad montem Parnassum, et a duce Delphos cognominasse et in memoriam gentis, ex qua profectus erat, subiacentes campos Crisaeos vel Cretaeos appellasse et aras constituisse.

So unklar auch manche Einzelheiten dieser Legende sein mögen, so ist doch über ihre Haupttendenz kein Zweifel möglich, das delphische Orakel mit seinem Kult für eine Filiale von Patara-Kreta, und nicht umgekehrt, zu erklären.\footnote{Vgl. dazu Treuber, Gesch. d. Lykier 48-68, der gegen Bouché-Leclercq, Hist. de la divination 3, 255 das hohe Alter und die Unabhängigkeit des Orakels zu Patara vom delphischen nachweist. S. auch Pomp. Mela 1, 15: Pataram... nobilem facit delubrum Apollinis quondam opibus et oraculi fide Delphico simile. Aus diesen Worten scheint auf eine gewisse Rivalität zwischen Patara und Delphi in älterer Zeit geschlossen werden zu müssen. Vgl. ob. S. 44 das über Branchidai Gesagte!} Geradezu bestätigt wird diese Auffassung durch die im homerischen Hymnus auf den Pythischen Apoll mitgeteilte Legende von der Gründung des apollinischen Orakels in Delphi, das direkt als eine Filiale von Knossos in Kreta, der mit Kleinasien und seinen zahlreichen Apollokulten so eng verbundenen Insel und Sitze ältester Kultur (O. Müller, Dorier 1 1, 206 ff., 215 ff.), hingestellt wird. Die vielfachen, später ausführlich und in größerem Zusammenhange zu behandelnden Analogien des Kultus und Mythus zwischen Delphi und Patara deuten daher vielmehr auf die Priorität Pataras in diesen Beziehungen als umgekehrt. Wir ziehen daraus den naheliegenden Schluss, dass ebenso wie in Branchidai auch in Patara der Gedanke eines Mittelpunkts der bewohnten Erdscheibe, d. h. eines Omphalos, autochthon sein kann und keineswegs auf delphischen Einflüssen zu beruhen braucht.

3. Bei Hesychius lesen wir die Glosse Ἐρεθύμιος· ὁ Απόλλων παρὰ Λυκίοις καὶ ἑορτὴ Ἐρεθύμια. Offenbar derselbe Kult bestand auch zu Kamiros auf Rhodos (Ross, Hellenika 1, 2 (1846), 112 und Reisen nach Kos usw. (1852) 58; C. Inscr. Insul. 1, 733), und aus dem dortigen Heiligtum des Apollon Erethimios stammt das Bruchstück eines netzumsponnenen Omphalos, auf dessen unterem Rande zu lesen ist (s. Taf. 6 Fig. 3):
\begin{quotation}
[Ἀπόλλωνι Ἐρ]ΕΘΙΜ[ίωι]

[ὁ δεῖνα... ο]ΥΙΕΡΑΤ[εύσας?]
\end{quotation}
\paragraph{}
Auch diese Tatsache macht es sehr wahrscheinlich, dass der netzumsponnene Nabelstein in den lykischen Apollonkulten eine ähnliche Rolle spielte wie in Delphi und Branchidai, ohne dass sich in diesem Punkte eine entschiedene Beeinflussung von einem der beiden Kulte behaupten oder nachweisen lässt. Vielmehr scheinen die Omphaloi der ältesten kleinasiatischen Apollonkulte mindestens ebenso alt und ursprünglich zu sein wie der von Delphi.\footnote{Nach Nilsson, Griech. Feste 143 war der Kult des Apollon Erethimios der Hauptkult von Kameiros, wie schon aus der (trieterischen?) Feier der Erethimia hervorgeht, eines Festes, das gar keinen delphischen Charakter trägt.}

4. Das sehr alte (ἀρχαῖον Steph. Byz. s. v. Γρῦνοι u. Hekat. fr. 211)\footnote{Strab. 622: πολίχνιον Μυριναίων Γρύνιον καὶ ἱερὸν Ἀπόλλωνος καὶ μαντεῖον ἀρχαῖον καὶ νεὼς πολυτελὴς λίθου λενκοῦ.} Orakel von Gryneion in Aiolis gehörte zu der in unmittelbarer Nähe gelegenen Stadt Myrina. Deren schöne Silbermünzen (aus dem 2. Jahrh. v. Chr.) zeigen im Obvers: Head of Apollo of Grynium r., wearing laurel-wreath with ends falling behind, im Rev. $\svgABL$ Apollo of Grynium, wearing laurelwreath and himation which leaves upper limbs bare, standing r.; in r., patera; in l. laurel branch to which two fillets are attached; before him, omphalos (bienenkorbförmig, ohne Basis, bisweilen mit Netzwerk bedeckt) and amphora: the whole in laurel-wreath (Head, Hist. nu. 2 555 f. Catal of greek coins in the Brit. Mus. Troas etc. S. 135; Pl 27 nr. 1-6.\footnote{Der Revers der ebendort Pl. 28 nr. 7 abgebildeten Kaisermünze von Myrina stellt dar einen sechssäuligen Tempel, in dem Apollon steht, eine Schale in der R. und einen Lorbeerzweig in der L. haltend. In dem Giebelfeld des Tempels erscheint ein großes erhabenes Rund von einem deutlichen Kreise umgeben $\svgABM$, was entschieden keine φιάλη ὀμφαλωτή sein kann, sondern wohl den Omphalos im Orbis terrarum von oben gesehen darstellen soll (vgl. oben die delphischen Münzen S. 96 f.).} Dass auch hier wohl kaum an eine Entlehnung von Delphi zu denken ist, geht namentlich aus der Tatsache hervor, dass Gryneion sich rühmte, der eigentliche Schauplatz der Drachentötung zu sein, was sicher auf völlige Unabhängigkeit von Delphi schließen lässt,\footnote{Serv. z. Verg. 6, 72: Miraculum (lies: oraculum) Apollinis, qui serpentem ibi interfecit. Mehr b. Otto Jahn in den Ber. d. Sächs. Ges. d. Wiss. 1851 S. 138 ff., der auch die Personifikation der Stadt Myrina an der puteolanischen Basis als weissagende Priesterin des Apollo von Gryneion deutet. Sie blickt ernst vor sich hin, lehnt sich mit dem l. Arm auf einen neben ihr stehenden Dreifuß, hält in der L. einen Lorbeerzweig, in der R. wahrscheinlich eine Schale und ist bei reichster Gewandung barfuß.} da in späterer Zeit, nach der Entwicklung Delphis zum Weltorakel, eine lokale Sage von der Schlangentötung zu Gryneion sich nur dann behaupten konnte, wenn sie wirklich uralt war.

So viel über die wahrscheinlich oder sicher in nicht delphischen Apollonkulten vorkommenden Nabelsteine, deren Zahl sich wahrscheinlich durch weitere Ausgrabungen und Funde noch wesentlich steigern lässt: wir gehen jetzt über zur Betrachtung derjenigen Monumente, die in Ermangelung schriftstellerischer Zeugnisse beweisen, dass auch im Kulte des Asklepios der Omphalos eine ganz ähnliche Rolle gespielt hat wie in dem des Apollon. Die hauptsächlichsten Monumente, um die es sich hier handelt, sind folgende.

1. An die Spitze stellen wir die von der römischen Tiberinsel, der bekannten Stätte des aus Epidauros nach Rom verpflanzten Asklepioskultes, stammende, jetzt in Neapel befindliche Marmorstatue des Gottes,\footnote{Vgl. Wissowa, Rel. u. Kult. d. Römer 2 S. 307 f.} weil wir annehmen dürfen, dass diese den in jenem Hauptund Zentralpunkte des Asklepiosdienstes von jeher traditionellen Typus am treuesten wiedergibt.\footnote{Vgl. die Abbildungen im Lex. d. Mythol. I Sp. 634 = Cranac 550, 1161 = Reinach, Statuaire I p. 289 nr. 1161, am besten b. Baumeister, Denkm. 8. 139 Nr. 148, wo aber S. 137 die Statue mit der Florentiner verwechselt ist.} Diesen Typus beschreibt Thraemer im Lexikon d. Mythologie 1 Sp. 634 folgendermaßen: `Schema 1: Der Körper stützt sich mit der rechten Achsel auf den [von der Schlange umringelten] langen Stab, der linke Arm ist an die Seite gestemmt und meist ganz verhüllt: das unter der r. Achsel vorgezogene Gewandstück geht mit mehrmals eingeschlagener Kante vorn um den Leib und dann, dass von der l. Schulter herabfallende Gewandende überdeckend, um den l. Arm nach dem Rücken herum, wo sein Zipfel von der aufgestemmten l. Hand festgehalten wird. Diese Anordnung des Gewandes gibt der Gestalt etwas sehr geschlossenes und dort, wo die Kante zugleich fest angezogen ist, geradezu Strammes.' Rechts unten am Boden, direkt unter dem herabfallenden Gewande, befindet sich ein annähernd halbkugelförmiger, basisloser, mit dem üblichen Wollnetz bedeckter Omphalos, den wir sonach mit ziemlicher Wahrscheinlichkeit auch im Tempel zu Epidauros, der Metropole fast aller bedeutenderen Asklepieen, voraussetzen dürfen. Vgl. uns. Tafel 9, 2. Diese Annahme wird weiter bestätigt durch

2. die abgesehen vom Schlangenstabe in vieler Hinsicht ähnliche Florentiner Statue des Asklepios, abgebildet bei Müller-Wieseler, D. a. K. 1, 48, 219a, Galleria reale di Firenze Ser. 4, Vol. 1 tav. 27, die ebenfalls rechts unter dem Gewande einen dem vorigen ähnlich gestalteten Omphalos zeigt. Wegen der großen Ähnlichkeit des Typus mit dem einer Münze von Pergamon, abgebildet bei Müller-Wieseler a. a. O. unter 219b, nehmen Müller-Wieseler a. a. O. und Brunn, Künstlergesch. 1, 443 (vgl. auch Baumeister, Denkm. S. 137) wohl mit Recht an, dass diese Statue eine Kopie des berühmten für Pergamos von Phyromachos gefertigten Kultbildes darstelle. Da nun aber der Asklepiosdienst von Pergamon unzweifelhaft von Epidauros ausgegangen ist (Thraemer a. a. O. Sp. 624 f.), so liegt auch hier wieder die Annahme nahe, dass auch in diesem Falle die Verbindung des Gottes mit dem Nabelstein auf jene Metropole des Kultes zurückgeführt werden muss.

3. Hierher gehört auch eine bekannte im Vatikan befindliche Statue des unbärtigen Asklepios, die sonst fast durchweg dem unter 1 beschriebenen Typus ähnelt (Müller-Wieseler 2, 60, 775 = Mus. Chiaramonti T. 2 tav. 9 = Clarac 549, 1159 = Reinach, Statuaire 1 p. 289 nr. 1159). Auch hier findet sich rechts unter dem Gewande ein niedriger, basisloser, mit Netzwerk versehener Nabelstein von konischer Gestalt.

4. Dass der Omphalos auch in dem berühmten, aus Epidauros schon in alter Zeit dorthin verpflanzten Kult von Pergamon eine Rolle spielte, wird nicht bloß indirekt durch die oben unter nr. 2 angeführte Statue, sondern auch ganz ausdrücklich durch Münzen von Pergamon bestätigt. Vgl. Catal. of gr. coins in the Brit. Mus. Mysia S. 129: Obv.: Head of Asklepios r., laur. Rev.: $\svgABN\enspace\svgABO$ Serpent of Asklepios coiled r. round netted omphalos. Pl. 27, 4; vgl. ebenda 27, 5.

5. Kaisermünzen von Pergamon (Head, Hist. nu. 2 S. 536; Catal. Brit. Mus. Mysia S. 129 und Taf. 27, 4-5) zeigen auf dem Obv. das lorbeerbekränzte Haupt des Asklepios, auf dem Rev. die Inschrift $\svgABP\enspace\svgABQ$ und `Serpent of Askl. coiled r. round netted omphalos.' Genau dieselben Typen finden sich auf Münzen von Magnesia ad Sipylum (2. Jahrh. v. Chr.), deren Obversbild fälschlich als Zeus gedeutet ist (Cat. Brit. Mus. Lydia S. 137 u. Taf 15, 1), und aus solchen von Nakrasa (ebenda S. 165 f. Pl. 18, 3), wo der Obverskopf bisher als `Head of bearded Herakles r., bare' gefasst wird, obwohl nach dem Zeugnis der Münze (ebenda S. 169 Pl. 18, 10) ein Asklepioskult in Nakrasa bestanden haben muss, während sich Herakleskult daselbst nicht nachweisen läßt.\footnote{Auch müsste doch wohl, wenn Herakles hier gemeint wäre, dieser eine Andeutung des Löwenfells am Nacken haben, wie auf anderen Münzen derselben Gegend zu sehen ist.} Vgl. auch den Revers der Münze von Rhodos (Brit. Mus. Caria S. 253).

Die schwierige Frage, was der Omphalos im Kult des Asklepios zu bedeuten habe, ist bisher sehr verschieden beantwortet worden. Middleton im Journ. of Hellen. Stud. 9 (1888) S. 300, der den mit Netzwerk versehenen und von einer Schlange umringelten Omphalos des Apollon als ein `emblem of Apollon in the character of the Healer' auffasst, meint dem entsprechend: `thus the omphalos becomes transferred to Asklepios and Telesphoros as the patrons of the healing art.' Auch Karo in seinem stoffreichen Artikel `Omphalos' im Dictionnaire d. antiq. p. 200a erblickt in dem von einer Schlange umringelten Omphalos der eben besprochenen Münzen von Pergamon, Magnesia usw. `plutôt la puissance médicale d'Apollon, frère d'Asklépios, qu'un souvenir du vieux Python, peu probable à cette époque récente.' Löwy (Jahrb. d. K. D. Arch. Inst. 2 (1887) S. 109) scheint geneigt, den Omphalos des Asklepios als Heroeneschara zu erklären, indem er auf Benndorf-Schöne, Later. Mus. nr. 259 und Conze, Reise auf d. Inseln d. thrak. Meeres Taf. 15, 4 S. 84 verweist. Noch andere deuten den O. des Asklepios als Deckel des Dreifußes (cortina) oder als den beim Baden benutzten κλίβανος (clypeus); vgl. Müller-Wieseler a. a. O. 2 770, während Thraemer a. a. O. Sp. 628 gewiss mit Recht bemerkt: `Dass der omphalosförmige Gegenstand neben Asklepios jedenfalls kein ärztliches Instrument ist, sondern ein Symbol des Kultus, beweist seine Umwindung mit Binden.'

Legen wir diese m. E. wohlbegründete Ansicht Thraemers zu Grunde, so kann es kaum zweifelhaft sein, dass der Nabelstein des Asklepios, der dem des Apollon so überaus ähnlich gebildet ist und verwendet wird, wenn er auch nicht geradezu als eine Entlehnung aus dem apollinischen Kultus betrachtet werden darf (vgl. Thraemer a. a. O. Sp. 628), doch ebenso wie dieser erklärt werden muss, nämlich als ein Symbol oder Wahrzeichen der Erdmitte. Sonach weist alles darauf hin, dass das bedeutendste und maßgebendste aller Asklepieen, von dem nach der Annahme der Epidaurier alle übrigen Kulte nach allen Himmelsrichtungen gewissermaßen ausgestrahlt sind,\footnote{Vgl. Paus. 2, 26, 8: Μαρτυρεῖ δέ μοι καὶ τόδε ἐν Ἐπιδαύρῳ τὸν θεὸν γενέσθαι· τὰ γὰρ Ἀσκληπιεῖα εὑρίσκω τὰ ἐπιφανέστατα ἐξ Ἐπιδαύρου. τοῦτο μὲν γὰρ Ἀθηναῖοι τῆς τελετῆς λέγοντες Ἀσκληπιῷ μεταδοῦναι τὴν ἡμέραν ταύτην Ἐπιδαύρια ὀνομάζουσι, καὶ θεὸν ἀπ᾽ ἐκείνου φασὶν Ἀσκληπιόν σφισι νομισθῆναι· τοῦτο δὲ Ἀρχίας ὁ Ἀρισταίχμου τὸ συμβὰν σπάσμα θηρεύοντί οἱ περὶ τὸν Πίνδασον ἰαθεὶς ἐν τῇ Ἐπιδαυρίᾳ τὸν θεὸν ἐπηγάγετο ἐς Πέργαμον. ἀπὸ δὲ τοῦ Περγαμηνῶν Σμυρναίοις γέγονεν ἐφ᾽ ἡμῶν Ἀσκληπιεῖον τὸ ἐπὶ θαλάσσῃ. τὸ δ᾽ ἐν Βαλάγραις ταῖς Κυρηναίων, ἔστιν Ἀσκληπιὸς καλούμενος Ἰατρὸς ἐξ Ἐπιδαύρου καὶ οὗτος. ἐκ δὲ τοῦ παρὰ Κυρηναίοις τὸ ἐν Λεβήνῃ τῇ Κρητῶν ἐστὶν Ἀσκληπιεῖον. Vgl. auch die Legende des Asklepieions von Sikyon b. Paus. 2, 10, 3 u. die des römischen b. Liv. 10, 47 usw. Vgl. auch Paus. 2, 26, 5: ὁ δὲ [Ἀσκληπιός nach seiner Geburt in Epidauros] αὐτίκα ἐπὶ γῆν καὶ θάλασσαν πᾶσαν ἠγγέλλετο τά τε ἄλλα ὁπόσα βούλοιτο εὑρίσκειν ἐπὶ τοῖς κάμνουσι, καὶ ὅτι ἀνίστησι τεθνεῶτας und das pythische Orakel ebenda 7: ὦ μέγα χάρμα βρότοις βλαστὼν Ἀσκληπιὲ πᾶσιν.} nämlich das von Epidauros, durch das Wahrzeichen des Nabelsteins diese seine zentrale Lage und Bedeutung zum Ausdruck hat bringen wollen. Eine erfreuliche Bestatigung dieser Annahme liefern uns gewisse Bildwerke, die statt des Omphalos neben Asklepios eine Weltkugel (Müller-Wieseler 2 776 u. 791, Thraemer a. a. O. 628, 35)\footnote{Eine ganz ähnliche Bedeutung hat wohl auch die Kugel (die beinahe wie ein Omphalos aussieht), auf der die sehr altertümliche Aphrodite der Münzen von Elaiusa Sebaste steht (Imhoof-Blumer, Nomisma 8 (1913) S. 19 = Taf. 2, 24). Ebenso die Hekate der Münzen von Bruzos. Ebenda S. 20 = Taf. 2, 26.} zeigen oder Asklepios und Hygieia vom Zodiakos umgeben darstellen, wodurch nach O. Müller die beiden Heilgottheiten als Mittelpunkt des Weltsystems bezeichnet werden sollen (Müller-Wieseler 2 785).

Eine ganz ähnliche Bedeutung wie bei Asklepios scheint der Nabelstein auch im Kult der Lares Compitales gehabt zu haben. Auf einem bekannten pompejanischen Wandgemälde\footnote{Vgl. Herculanum u. Pompeji... von H. Roux ainé und L. Barré, deutsch v. A. Kaiser. Hamburg 1841. 2. Taf. 57. Helbig, Wandgemälde d. verschätt, Städte Campaniens S. 12 Nr. 37.} sehen wir in der Mitte einen kniehohen, bienenkorbförmigen, basislosen, mit Netzwerk umhüllten und von einer gewaltigen lebendigen Schlange (offenbar dem Genius loci) umringelten Omphalos stehen. Rechts und links von ihm stehen zwei Laren in der üblichen Stellung mit Trinkhorn und situla. Dass man sich die Szene im Freien zu denken hat, beweisen die Bäumchen und Sträucher, die den Hintergrund bilden. Hier bezeichnet offenbar der Omphalos den geheiligten Mittelpunkt, oder `compitum,' d. h. das Zentrum oder die gemeinsame Grenzscheide eines größeren Bezirks, wo mehrere Besitzungen oder Grundstücke zusammenstießen (Gromat. lat. p. 302, 20 ff.; Wissowa im Lex. d. Myth. 2, 1873, 35 ff.) und wo das ländliche Fest der Compitalia oder Laralia nach Bezirken gefeiert wurde. S. uns. Taf. 9 Fig. 6.

Endlich kommt der von einer Schlange umringelte Nabelstein auch als Attribut des Hermes (Mercurius) auf pompejanischen Wandgemälden vor. Vgl. Helbig, Wandgemälde S. 7 Nr. 15: `Mercur, mit beflügeltem Petasos... laufend und in Stiefeln, dahinter auf einer Basis ein Hahn, davor ein Omphalos, um welchen sich eine Schlange windet.' Ebenda Nr. 17: `Mercur... schreitend, in gelbem beflügelten Petasos und Flügelschuhen..., in der L. den Caduceus, in der R. die Börse. Davor ein von einer Schlange umwundener Omphalos.' Es muss als eine offene Frage bezeichnet werden, was in diesem Fall der Omphalos zu bedeuten hat. Am nächsten liegt wohl auch hier die Annahme, dass er einen Mittelpunkt bezeichnen soll, vielleicht den der Erde oder des Weltalls, an dem sich der Bote der Götter aufzuhalten hat, um von da aus seine Botschaften nach allen Richtungen hin auszurichten. Doch vermisse ich für diese Vermutung bis jetzt ein literarisches Zeugnis.
\clearpage
\section{Grabmonumente, Baitylien und Altäre in Omphalosform.}
\paragraph{}
Eine sehr große Verwirrung hat in der Omphalosfrage neuerdings die Erkenntnis angerichtet, dass es in Hellas und anderwärts Grabmonumente gegeben hat, die deutliche Omphalosform zeigen, so dass im Hinblick auf die seit Varro (s. oben S. 66 f.) nachweisbare Anschauung, der delphische Nabelstein stelle eigentlich das Grab des Pythondrachens dar, daraus neuerdings die Ansicht entstehen konnte, dass der Omphalos eigentlich und ursprünglich als Grabdenkmal aufzufassen sei. Vor allen hat die berühmte englische Archäologin Miss Jane Harrison mit Scharfsinn und Gelehrsamkeit diese Auffassung vertreten,\footnote{Vgl. J. Harrison, Journ. of Hellen. Stud. 19 (1899) S. 225 ff., Dieselbe im Bull. de Corresp. Hellén. 24 (1900) S. 254 ff. Allerdings hat noch vor Miss. H. kein Geringerer als E. Rohde (Psyche 2 1 S. 132) ziemlich dieselbe Ansicht ausgesprochen.} und mehrere ausgezeichnete Gelehrte haben sich in der Hauptsache an sie angeschlossen, z. B. G. Karo in seinem lehrreichen Omphalosartikal des Dictionnaire des antiquités (6 1 p. 197 ff.) und der hervorragende Numismatiker und Archäologe Svoronos (Journ. Intern. d'archéol. numism. 13 (1911) S. 313 f.). Die Beweisführung Miss Harrisons ist im wesentlichen die folgende.

Sie geht aus von der Auffassung der Erinyen als Seelengeister (`ghosts') und will beweisen, dass der Omphalos als Grab deren Heiligtum (`sanctuary') und Wohnsitz sei. Zu diesem Zwecke beruft sie sich vor allem auf die von uns oben (S. 66 f.) behandelten späten Zeugnisse des Varro (de l. l. 7, 17)\footnote{Den Ausdruck `thesaurus' in den Worten Varros: `in aede ad latus [? arcuatum ?] est quiddam ut thesauri specie, quod Graeci vocant ὀμφαλόν, quam Pythonos aiunt esse tumulum' fasst sie als `Grabgewölbe,' dagegen Karo a. a. O. p. 198a wohl richtiger als `Sparbüchse' (s. ob. S. 66 Anm. 121). Dadurch werden die von J. H. angeführten Stellen Paus. 9, 38, 2; Aristot. de mu. 6, 20; Hippolyt 5, 20 beweisunkräftig.} und Hesychius (s. v. Τοξίου βουνός), die allerdings in schroffem Gegensatz zu den sonstigen älteren Zeugnissen die Behauptung aussprechen, dass der Pythondrache unter dem Omphalos begraben liege.

Den schlagenden Beweis für die Richtigkeit der Annahme Varros liefert nun nach Miss H. die Tatsache, dass unzweifelhaft Grabhügel auf Vasen genau in denselben Formen erscheinen, wie sie den Darstellungen des delphischen Omphalos eigentümlich sind. So entspricht z. B. der eiförmige Omphalos auf Vasen dem eiförmigen Grabmonument auf der attischen Lekythos bei J. Kell im Journ. of Hell. Stud. 19 Taf. 2\footnote{S. auch a. a. O. die Vignette S. 169 Fig. 1: Eine Frau vor einem Grabe, das als mannshoher eiförmiger `Omphalos' auf einer quadratischen Basis dargestellt ist (von einer Lekythos der Athener Sammlung Nr. 1960). S. uns. Taf. 5 Fig. 2.} (= uns. Taf. 5, 3), und das Grab des Achilleus, über dem Polyxena geopfert wird, erscheint auf der von Walters (Journ. of Hell. Stud. 18 (1898) Taf. 15 = uns. Taf. 4, 1) publizierten sehr altertümlichen Vase fast genau in derselben Gestalt und Ausstattung wie der omphalosförmige `Bomos'\footnote{Man beachte, dass Miss H. hier statt des Begriffes `Grab in Omphalosform,' den man eigentlich nach dem Zusammenhang erwarten sollte, plötzlich den Begriff `Altar in Omphalosform' substituiert.} der ebenfalls sehr altertümlichen Vase in München (Nr. 124), auf der die Ermordung des Troilos im Tempel von Thymbra dargestellt ist (s. ob. 106 f.).\footnote{Der Unterschied zwischen beiden Objekten besteht nur darin, dass das Grab des Achilleus etwas breiter und niedriger dargestellt ist als der `Bomos' der Troilosvase und also mehr den sogen. Heroenescharai entspricht (vgl. Deneken im Lex. d. Myth. 1 Sp. 2499 nebst Abbildg. 1; Löwy, Jahrb. d. arch. Inst. 2 (1887) S. 109 ff. Pauly-Wissowa 1 Sp. 1665 Artikel Altar; Stengel, Gr. Kultusaltertümer 2 S. 126 Taf. 1, 2).} Wie hier so sehen wir auch sonst öfters den Omphalos zum Altar (und umgekehrt) werden. Dafür, dass der Nabelstein auch als Altar betrachtet wurde, beruft sich Miss H. einerseits auf den Schol. z. Aesch. Eum. 40 (ὁρῶ δ᾽ ἐπ᾽ ὀμφαλῷ μὲν ἄνδρα θεομυσῆ ἕδραν ἔχοντα προστρόπαιον): ἰδοῦσα γὰρ Ὀρέστην ἐπὶ τοῦ βωμοῦ..., anderseits auf die namentlich aus der Orestessage bekannte Geltung des Omphalos als eines `mercy seat,' in welcher Hinsicht er einem Altar völlig gleichkomme,\footnote{Hier hätte Miss H. wohl besser auf die zahlreichen Bildwerke, namentlich Vasenbilder, hinweisen können, die Orestes als Schutzflehenden auf dem Altar statt auf dem Omphalos sitzend zeigen: vgl. Overbeck, Gallerie Taf. 29, 5. 8. 10. 12., Text S. 706 ff. Gerhard, Etrusk. Spiegel 1, 21.} wie denn auch oft die Münzen (s. ob.) Apollon selbst auf dem Omphalos wie auf dem Altar (oder dem Dreifuß!) sitzend zeigen. So vereinigen sich drei Begriffe `Altar,' `Grab' und `mercy-seat' in dem eines `holy place,' aber offenbar sei der Begriff des Grabes hier der ursprünglichste. Zu diesen drei Begriffen komme aber noch ein vierter, nämlich der des Orakels (μαντεῖον). Einen βωμοειδὴς τάφος als μαντεῖον sehen wir deutlich vor uns in einem von Heydemann total missverstandenen sf. Vasenbilde (Taf. 5, 4) in Neapel (Nr. 2458). Nach Heydemann handelt es sich hier um `eine Felsenhöhle, in der ein Reh steht,' in Wahrheit aber ist die vermeintliche Felsenhöhle ein `tumulus with a coat of λεύκωμα, decorated on one side with a stag, on the other with a large snake.' Die Szenen auf Vorder- und Rückseite sind gleichbedeutend. In einem Hain befindet sich ein [omphalosförmiger] Grabhügel, zu dessen beiden Seiten je ein Krieger sitzt, der zurückblickend eifrig das Grab beobachtet. Auf dem einen Grabhügel sitzt ein gerade einen Hasen zerfleischender Adler, auf dem andern ein Adler, der sich, mit einer Schlange zu schaffen macht (vgl. Head, Hist. nu. p. 105). Ihre τύμβος-ὀμφαλὸς-Theorie findet Miss H. weiter bestätigt durch das bekannte Theseusrelief (uns. Taf. 8, 1) mit der vor Theseus befindlichen niedrigen Heroeneschara (Lex. d. Mythol. 1 Sp. 2499 f.), sowie durch die neuerdings gefundene sf. Vase in Neapel Nr. 111609, abgebildet a. a. O. S. 228 Fig. 9. In der Mitte befindet sich ein weißer Grab-Omphalos, überragt von einem schwarzen `Bätyl.' Die das Grab umgebende Szene erklärt Miss H. als eine jener an Gräbern Verstorbener vorgenommenen feierlichen Eidesleistungen, wie sie nach Herod. 4, 172 bei den Nasamones in Libyen üblich waren, obwohl sie zugeben muss, dass eine gleiche Sitte bei den Griechen sonst nicht nachweisbar ist [vgl. Stengel, Kultusaltertümer § 78]. S. Taf. 5, 1.

Der delphische Omphalos war also nach Miss H.s Ansicht ursprünglich ein einfaches χῶμα γῆς, überzogen mit einer Decke von weißem Stuck (λεύκωμα),\footnote{Vgl. Cic. de leg. 2, 26, 65: opus tectorium u. dazu Brückner im Jahrb. d. Kais. Deutsch. Arch. Inst. 6 (1891) S. 197 ff.} das schließlich in einen Stein von entsprechender Form verwandelt wurde. Die Deutung des Omphalos als Symbol der Erdmitte halt sie (mit E. Rohde) für unursprünglich und sekundär; sie will daher, weil die gewöhnliche Bedeutung von ὀμφαλός = `Nabel,' `Mittelpunkt' mit ihrer Erklärung in schroffem Widerspruch steht, in diesem Falle ὀμφαλός von ὀμφή ableiten, indem sie an die Bedeutung des prophetischen λίθος αύδήεις erinnert, den nach den orphischen Lithika Phoibos-Apollon dem Helenos gab (Lith. 360 ff. = 354 H.).\footnote{Demgemäß erklärt Miss H. das Wollnetz, mit dem der ὀμφαλός d. i. `la pierre de l'ὀμφή, la sainte voix prophétique' (= λίθος ἔμψυχος) bekleidet war, als identisch mit dem ἀγρηνόν = πλέγμα ἐξ ἐρίων δικτυοειδὲς περὶ πᾶν τὸ σῶμα, ὃ Τειρεσίας ἐπεβάλλετο ἤ τις ἄλλος μάντις (Poll. on. 4, 116) und zugleich mit der αἰγίς = ὅπλον ἐξ αἰγοῦ καὶ τὸ ἐκ τῶν στεμμάτων διαπεπλεγμένον δίκτυον (Hesych.; vgl. Suid. s. v. αἰγίδας u. Bekk. Anecd. s. v. αἰγίδες). S. auch Bull. de Corr. Hellén. 1900 S. 254 u. Daremberg-Saglio, Dict. d. ant. s. v. ἀγρηνόν.} Vgl. J. Harrison im Bull. de Corr. Hellén. 24 (1900) S. 258 f. ---

Ehe ich aber an eine Kritik dieser Hypothese von Miss Harrison gehe und deren Unhaltbarkeit im einzelnen nachweise, möchte ich noch einmal betonen, dass der ausgezeichneten englischen Forscherin das Verdienst zuzuerkennen ist, das schwierige Omphalosproblem neu angeregt und die wichtige Frage nach der Gestalt der ältesten griechischen τύμβοι von neuem aufgeworfen zu haben. Zugleich möchte ich diese Gelegenheit dazu benutzen, die von Miss H. angefangene Liste der omphalosförmigen Grabmonumente um einige weitere Belege zu vermehren.

1. Zu den von Miss H. angeführten schwarzfigurieren Vasenbildern mit den Darstellungen von omphalosförmigen Grabmonumenten kommt jetzt hinzu das von Studniczka im Hermes 1902 (37) S. 265 unter nr. 4 zum ersten Male veröffentlichte, und wie ich glaube, richtig erklärte Vasenbild des Britischen Museums, das man bisher fälschlich auf den delphischen Omphalos und auf Orestes und Pylades als delphische Schutzflehende bezogen hat. Studniczka fasst auch hier den vermeintlichen Omphalos einfach als einen `getünchten Grabhügel,' auf den sich zwei Augurienvögel (Raben?) niedergelassen haben, die von den beiden links und rechts sitzenden geharnischten Kriegern eifrig beobachtet werden. Svoronos dagegen denkt (Journ. Internat. d'archéol. numism. 13 [1911] S. 313) an den Mythus von Kalchas und Mopsos, die vor dem Omphalos des Apolloheiligtums in Klaros bei Kolophon sitzen, verwirft also Studniczkas mir viel wahrscheinlichere Deutung des `Omphalos' als Grabmonument. S. uns. Taf. 4 Fig. 3.

2. Brückner, Jahrb. d. Kais. Deutsch. arch. Inst. 6 (1891) S. 197 ff. bespricht eine daselbst auf Taf. 4 veröffentlichte aus dem äußern Kerameikos stammende Lekythos, auf der ein weißes, bienenkorbförmiges Grabmal dargestellt ist. `Solch ein weißes, mannshohes Mal mit einer großen Schlange im freien Felde daran, umflattert von dem noch gerüsteten Eidolon, bedeutet den Grabhügel des Patroklos im Bilde der Schleifung Hektors = Gerhard, Auserl. Vasenb. 198/9 = Reinach, Repert. 2 p. 99 Nr. 5.' [Vgl. auch Overbeck, Gallerie Taf. 19, 7].

3. Besonders interessant ist das Gemälde einer attischen Lekythos, dass in den Monum. d. I. 8 Taf. 4. 5 abgebildet und von Wolters Ath. Mitteil. 16, 379 und O. Crusius im Lex. d. Myth. 2 Sp. 1149, 12 ff. (daselbst auch Sp. 1147 Fig. 5 eine Abbildung) besprochen worden ist. Rechts und links von einem bienenkorbförmigen Grabmal, das von einer Lutrophoros bekrönt ist, stehen zwei klagende Frauen. An dem Grabmal gewahrt man eine große Schlange und vier flatternde Seelen (= Keren?), bei denen man zweifeln kann, ob sie am oder im Grabe herumflattern. S. uns. Taf. 4 Fig. 5.

4. Auf Kasos sind viele halbkugelförmige, also an zahlreiche Omphalosdarstellungen erinnernde, Grabsteine gefunden worden nach Ross, Archäol. Aufs. 1, 65; vgl. Pfuhl im Jahrb. d. K. D. Arch. Inst. 1905 S. 88 f., der noch weitere Beispiele anführt (z. B. Millingen, Peint. ant. Taf. 17, Brit. Mus. Cat. of vases 3 S. 412 D. 83) und im allgemeinen auf Schroeders Studien z. d. Grabdenkmälern d. röm. Kaiserzeit. Bonn 1902 = Bonner Jahrb. Heft 108 S. 25 ff. verweist.

5. Br. Schröder verdanke ich die Photographie eines bei Hagia Triada in Athen gefundenen Grabmals, das aus einem basislosen halbeiförmigen Marmoromphalos besteht, der sich direkt von dem jetzt mit Rasen bewachsenen Boden erhebt und von einer quadratischen steinernen Einfassung umgeben ist. S. Fig. 7 auf Taf. 6.\footnote{Bruno Schröder hat mir auch eine Reihe ähnlicher von ihm in Athen, Thespiai und Theben gezeichneter omphalosförmiger Grabmäler freundlichst zur Verfügung gestellt, die ich hier gerne abbilden würde, wenn ich könnte und dürfte. Hoffentlich entschließt er sich selbst dazu, das von ihm gesammelte Material zu veröffentlichen.} Es dürfte kaum zweifelhaft sein, dass sich bei eifrigem Nachforschen noch zahlreiche Belege für Miss H.s These der formalen Übereinstimmung zwischen Omphalos und Grabdenkmälern bzw. Altären (ἐσχάραι) finden werden.

Darf aber aus dieser rein äußerlichen Ähnlichkeit oder Gleichheit auch auf innere Verwandtschaft oder gar ursprüngliche Identität im Sinne von Miss H. geschlossen werden? Ich glaube nicht und gehe jetzt daran meine Ansicht im einzelnen zu begründen. Vor allem lassen sich gegen Miss H. folgende Argumente anführen.

a. Wie vorsichtig man im allgemeinen mit der Annahme innerer Identität bei vollkommener äußerer Gleichheit sein muss, lehrt namentlich die Geschichte des griechischen Alphabets. Hier kann z. B. das Zeichen $\svgABR$ bald β bald ε, das Zeichen $\svgABS$ bald γ bald λ, das Zeichen $\odot$ bald θ bald ο,\footnote{Dasselbe Zeichen, als Bild gefasst, stellt bald eine φιάλη ὀμφαλωτή (s. oben S. 51 f. A. 99), bald ein Brot (vgl. die πόπανα ὀμφαλωτά b. Pol. 6, 25, 7; mehr b. Lobeck, Agl. 1078 f. u. O. Jahn in d. Ber. d. Sächs. Ges. d. Wiss. 1861 (14) S. 348), vielleicht auch auf altdelphischen Münzen als Pendant zum Dreifuß den Erdnabel inmitten des Orbis terrarum dar (s. oben S. 51 f.). Vgl auch die mehrdeutigen Formen παρεῖτε (3), ἦσθε (3), εἰστεῖσθε (5), ᾖσαν (3), παρῇ (3), πάρει (2), παρεῖεν (2) usw. S. Krüger, Griech. Sprachl. § 39, 8.} das Zeichen $\svgABT$ bald ι bald σ bedeuten usw. Ebenso wurde das berühmte delphische $\svgABU$ bald als Zahlzeichen für 5, bald als εἰ, bald als εἶ (`du bist') gefasst, während es in Wirklichkeit wohl weiter nichts als der Imperativ von ἰέναι (= `Gehe,' `Komm'!) ist\footnote{Vgl. Roscher im Philologus 1900 S. 21 ff. 1901 S. 81 ff. 1902 S. 513 ff. Hermes 36 (1901) S. 470 ff.}; das Wort τέλος = `Abgabe' ist etymologisch von τέλος = `Ende' ganz verschieden (Curtius, Grundz. d. gr. Etymol. 5 S. 221 f.) usw. Ganz ähnlich müssen wir auch über das Verhältnis des delphischen Omphalos zu den omphalosförmigen Grabdenkmälern urteilen: aus ihrer formalen Übereinstimmung folgt noch lange nicht ihre ursprüngliche Identität; diese muss vielmehr erst mit historischen und logischen Gründen erwiesen werden, und gerade das ist in diesem Falle unmöglich.

b. Die von Miss H. für ihre Deutung nach dem Vorgange E. Rohdes angeführten späten Zeugnisse des Varro und Hesychius (s. v. Τοξίου βουνός) stehen mit der für uns maßgebenden delphischen von Pausanias (10, 16, 3),\footnote{Τὸν δὲ ὑπὸ Δελφῶν καλούμενον ὀμφαλόν, λίθου πεποιημένον λευκοῦ, τοῦτο εἶναι τὸ ἐν μέσῳ γῆς πάσης αὐτοί τε λέγουσιν οἱ Δελφοί καὶ ἐν ᾠδῇ τινὶ Πίνδαρος ὁμολογοῦντά σφισιν ἐποίησε.} Pindar (s. oben) und den Tragikern (s. oben) ausdrücklich bezeugten Tradition in einem unlösbaren Widerspruch. Sie erklären sich aber leicht einerseits aus der von Miss H. gut nachgewiesenen Ähnlichkeit des Omphalos mit gewissen Grabmonumenten, anderseits aus der von mir oben (S. 41 ff.) hervorgehobenen Tatsache, dass in der späteren Zeit durch die immer allgemeiner gewordene Lehre von der Kugelgestalt der Erde die alte dem Omphalosgedanken zugrunde liegende Theorie von der Erdscheibe (`orbis terrarum') hinfällig geworden war und demnach durch eine neue Deutung des Omphalos ersetzt werden musste.

c. Die neue zugunsten ihrer Auffassung von M. H. vorgeschlagene Ableitung des Wortes ὀμφαλός von ὀμφή = λίθος αὐδήεις vertragt sich weder mit der allgemein anerkannten Bedeutung von ὀμφαλός = `Nabel,' `Mittelpunkt' noch auch mit der richtigen Etymologie des Wortes (s. oben) und kann deshalb unmöglich richtig sein.

d. Das Bild der sf. Neapler Vase Nr. 111609 nach Analogie eines nichtgriechischen Brauches der libyschen Nasamonen zu erklären, ist doch wohl zu gewagt, um Beifall zu finden.\footnote{Dies ist eines von den neuerdings, namentlich in England, immer zahlreicher werdenden Beispielen für eine Überschätzung des Folklore.}

e. Der schon früher einmal von Wieseler ausgesprochene Gedanke, dass der delphische Omphalos mit dem delphischen Altar (ἑστία) identisch sei, ist schon längst von Preuner und andern widerlegt worden (s. oben S. 63 Anm. 115) Beide Kultobjekte müssen vielmehr streng voneinander unterschieden werden. Aus den Vasenbildern, die den Orestes als Schutzflehenden bald auf der ἑστία, bald auf der Basis des ὀμφαλός sitzend zeigen, folgt noch lange nicht die Identität beider, sondern nur, dass die Tradition in diesem Punkte auseinander ging.

f. Mehrere Monumente, darunter das schöne pompejanische Wandgemälde, das Apollon als Sieger über den Pythondrachen darstellt (oben Nr. 20), ferner die sehr alte schwarzfiguriere Lekythos mit der Darstellung des hinter dem Omphalos vor den Pfeilschüssen des Gottes sich verbergenden Drachen (oben Nr. 50) und die delphische Münze im Catal. of gr. coins in the Brit. Mus. Central Greece Pl. 4 Nr. 20 (s. oben Nr. 27) zeigen den delphischen Omphalos entweder von der Schlange umringelt oder diese sich hinter ihm versteckend. Wie schlecht sich diese Auffassungen und Darstellungen mit Miss H.s Deutung des Omphalos als Grab des Python vertragen, dürfte einleuchtend sein und bedarf keiner weiteren Auseinandersetzung.\footnote{Nachträglich weise ich noch auf das bedenkliche Schwanken der späteren Traditionen über das Grab des Python hin. Nach Tatian adv. Gr. 8 p. 40 Otto war der O. das Grab nicht des Python, sondern des Dionysos, der nach andern Zeugnissen vielmehr παρὰ τὸν Ἀπόλλωνα τὸν χρυσοῦν (Philoch. fr. 22. b. Syncell. 307, 4 D. etc. Rohde, Psyche 2 1, 132, 2) oder παρὰ τὸ χρηστήριον (Plut. Is. et Os. 35), παρὰ τὸν τρίποδα (Callim. b. Tzetz. Lyc. 208) begraben sein sollte. Nach Porphyr. vit. Pyth. 16 sollte Apoll selbst im Dreifuß bestattet sein, der nach Hygin. fab. 140 u. Serv. z. Aen. 3, 92; 3, 360; 6, 347 vielmehr das Grab des Python darstellte. Wieviel einheitlicher und glaubwürdiger stellt sich diesem Schwanken gegenüber die ältere Tradition vom Mittelpunkt der Erde dar! Man beachte endlich auch die in Kap. 2 von uns nachgewiesene starke Verbreitung des Erdnabelgedankens bei allen möglichen Völkern.}

g. Die verschiedenen Typen des Omphalos, der allmählich von der ursprünglichen niedrigen, sich der Halbkugel nähernden Form in die eines mehr oder weniger hohen Bienenkorbes (oder pilleus) und weiter in die eines hohen und ziemlich spitzen, metaähnlichen Kegels (so namentlich auf Vasenbildern) übergeht und immer mit Tänien oder einem Netzwerk aus Wolle (ἀγρηνόν) versehen zu denken ist, erklären sich mindestens ebenso gut wie aus den Formen der τύμβοι aus den stets runden, oben gewölbten und im Zentrum eines Raumes oder Gegenstandes befindlichen, bald niedrigen, bald hoch erhabenen Objekten, für die der Ausdruck ὀμφαλοί (umbilici) der allgemein üblichste und bezeichnendste ist. Ich denke dabei vor allem an die ὀμφαλοί in der Mitte der φιάλαι ὀμφαλωταί (s. oben S. 51 f. A. 99), an die ὀμφαλοί (umbones) in der Mitte der antiken Schilde, die bald niedrig, bald hoch und ziemlich spitzig gebildet sind,\footnote{Vgl. z. B. den gewaltigen halbkugelförmigen Buckel des Schildes b. Baumeister, Denkm. S. 2071 Fig. 1b und daneben den hohen und spitzen kegelförmigen Omphalos des `Laiseion' b. Daremberg-Saglio, Dict. d. ant. 2 S. 1250a Fig. 1638, ferner ebenda S. 1255 Fig. 1653; S. 1257, Fig. 1660. Rich, Illustr. Wörterb. d. röm. Alterth. s. v. umbo.} an die ὀμφαλοί (umbilici, cornua) der antiken Bücherrollen, endlich auch an die kegelförmigen, runden, in der Mittellinie der Rennbahnen (Hippodrome) angebrachten νύσσαι, τέρματα, στῆλαι, καμπτῆρες, metae,\footnote{Die Gestalt der metae (νύσσαι) war wohl immer konisch, bald schlank wie bei den Zielsäulen im römischen Zirkus, bald breiter wie bei dem `meta' genannten unteren Teil der Mühlen; vgl. O. Jahn in den Sächs. Ber. 1861 (14) S. 341 A. 192. Rich, Illustr. Wörterb. s. v. meta.} die wie Grenzsteine (ὅροι, termini) die Bahn in zwei gleiche Hälften zerlegten und schon deshalb mit dem die Erdscheibe in eine östliche und westliche Hälfte teilenden Omphalos von Delphi eine große innere Ähnlichkeit besitzen.

Solche `termini' oder ὅροι waren heilig, genossen göttliche Ehren ebenso wie der Herd (ἑστία), wurden noch zu Ovids Zeiten (Ov. fast. 2, 641 ff.) alljährlich in Latium bekränzt, mit dem Blute eines Lammes oder Ferkels besprengt, unter feierlichen Zeremonien (Salbung und Schmückung mit Tänien und Kränzen usw.) geweiht\footnote{Vgl. Sicul. Flacc. Gromat. vet. ed. Lachm. 1 p. 141: cum (antiqui) terminos disponerent, ipsos quidem lapides in solidam terram rectos conlocabant... et unguento velaminibusque et coronis eos coronabant ete, Vgl. dazu Wissowa, Rel. u. Kult. d. Römer 2 136 f. Samter, Arch. f. Rel.-Wiss. 16 (1913) S. 137 ff. K. Fr. Hermann, Gottesdienstl. Alt. § 15, 8 und De terminis eorumque religione apud Graecos. Gott. 1847. Ders., Lehrb. d. griech. Privatalt. § 49, 11. § 63, 17. Nilsson, Griech. Feste 167, 2 u. 168.} und standen unter dem Schutze des Zeus ὅριος = Juppiter Terminus.\footnote{Plat. leg. 842 E: Διὸς ὁρίου μὲν πρῶτος νόμος ὅδε εἰρήσθω· μὴ κινείτω γῆς ὅρια μηδεὶς μήτε οἰκείου πολίτου γείτονος μήτε ὁμοτέρμονος ἐπ᾽ ἐσχατιᾶς κεκτημένος ἄλλῳ ξένῳ γειτονῶν, νομίσας τὸ τἀκένητα κινεῖν ἀληθῶς τοῦτο εἶναι κ. τ. λ. Demosth. or. 7 39: Χερρονήσου οἱ ὅροι εἰσὶν οὐκ Ἀγορά, ἀλλὰ βωμὸς τοῦ Διὸς τοῦ ὁρίου κ. τ. λ. woraus ersichtlich ist, dass statt der στῆλαι, στυλίδες usw. mehrfach auch Altäre als Grenzsteine geweiht wurden. Poll. 9, 8: ἀπὸ τῶν ὅρων Ζεὺς ὅριος καὶ στήλη ἐφορία... 9, 9: καὶ ἡ ἐνεστηκυῖα στήλη ὅρος.}

Ungefähr das gleiche gilt auch von dem Omphalos im delphischen Adyton. Auch dieser spielte, wie die beiden rechts und links von ihm angebrachten goldenen Adler des Zeus beweisen, die Rolle eines dem Zeus (ὅριος) geheiligten ὅρος-Steines oder Terminus, der die Mitte\footnote{Strab. 3, 171: ἔθος παλαιὸν ὑπῆρχε τὸ τίθεσθαι τοιούτους ὅρους καθάπερ οἱ Ῥηγῖνοι τὴν στυλίδα ἔθεσαν τὴν ἐπὶ τῷ πορθμῷ κειμένην, πυργίον τι, καὶ ὁ τοῦ Πελώρου λεγόμενος πύργος ἀντίκειται ταύτῃ τῇ στυλίδι, καὶ οἱ Φιλαίνων λεγόμενοι βωμοὶ κατὰ μέσην πον τὴν μεταξὺ τῶν σύρτεων γῆν κ. τ. λ. (vgl. dazu Sall. Jug. 79. Val. Max. 5, 6, 4). Ähnlich auch die Sage von Leuke: Nilsson, Griech. Feste S. 175.} oder Grenze zwischen der östlichen und westlichen Hälfte der Erdscheibe bezeichnen sollte, galt neben dem prophetischen Dreifuß und der ἑστία des Tempels als ein Heiligtum ersten Ranges und war ständig mit Tänien oder einem Wollnetze (ἀγρηνόν) geschmückt, um seine Heiligkeit und Unantastbarkeit so deutlich wie möglich zu bezeichnen. Dass wir nichts von seiner täglichen Salbung hören, wie sie Pausanias (10, 24, 6) von dem vor dem delphischen Apollotempel liegenden Stein berichtet, den Kronos ausgespien haben sollte, ist wohl nur ein Zufall.

Auch mit gewissen uralten Steinfetischen, den sogenannten βαιτύλια (βαίτυλοι), die göttliche Ehren genossen\footnote{Luk. Alex. 30: εἰ μόνον ἀληλιμμένον που λίθον ἢ ἐστεφανωμένον θεάσαιτο, προσπίπτων εὐθὺς καὶ προσκυνῶν καὶ ἐπὶ πολὺ παρεστὼς καὶ εὐχόμενος καὶ τἀγαθὰ παρ᾽ αὐτοῦ αἰτῶν. --- Theophr. char. 16, 5 (π. δεισιδαιμονίας): καὶ τῶν λιπαρῶν λίθων τῶν ἐν ταῖς τριόδοις παριὼν ἐκ τῆς ληκύθου ἔλαιον καταχεῖν καὶ ἐπὶ γόνατα πεσὼν καὶ προσκυνήσας ἀπαλλάττεσθαι und Immisch z. d. St. --- Auch der Kronosstein zu Delphi, von dem Paus. 10, 24, 6 berichtet: τούτου [τοῦ λίθου οὐ μεγάλου] καὶ ἔλαιον ὁσημέραι καταχέουσι καὶ κατὰ ἑορτὴν ἑκάστην ἔρια ἐπιτιθέασι τὰ ἀργά· ἔτι δὲ καὶ δόξα ἐς αὐτὸν, δοθῆναι Κρόνῳ τὸν λίθον ἀντὶ [τοῦ] παιδός, καὶ ὡς αὖθις ἤμεσεν αὐτὸν ὁ Κρόνος, hieß βαίτυλος nach Hesych. s. v. βαίτ. οὕτως ἐκαλεῖτο ὁ δοθεὶς λίθος τῷ Κρόνῳ ἀντὶ Διός... (παρὰ τὸ τύλον ὄντα κεκρύφθαι Bekk. Anecd. 224, 10). Vgl. dazu Daremberg-Saglio, Dict. d. ant. 1 S. 645 Fig. 742. Übrigens wurde nach Lact. i. div. 1, 20 auch der Terminus im Capitolinischen Juppitertempel mit dem Kronosstein identifiziert.} und wohl durchweg als vom Himmel gefallene Meteorsteine angesehen wurden, z. B. mit dem von Emisa in Syrien (= Elagabal), Perge (= Artemis Pergaia), Kypros (= Aphrodite), Aphrodisias in Kilikien (? = Aphrodite?), besitzt der delphische Omphalos eine große äußere und innere Ähnlichkeit. So hat der durch Elagabal berühmt gewordene Steinfetisch von Emisa nach Ausweis der dort geschlagenen Münzen (Baumeister, Denkm. S. 603 Fig. 649: s. Taf. 1 Fig. 18) fast genau dieselbe Gestalt wie der delphische Nabelstein in den besten und ältesten seiner Darstellungen; ziemlich ähnlich, nur etwas schlanker und oben an der Spitze mit einer Art Knauf versehen, erscheint auch das alte Idol der `Artemis' von Perge auf den Münzen dieser Stadt (a. a. O. Fig. 645 ff.; uns. Taf. 1, 19 f.), ferner das Standbild der Göttin von Iasos in Karien (ebenda Fig. 650), das Idol der Aphrodite von Kypros (s. oben Seite 30 = Lex. d. Mythol. 1 Sp. 747),\footnote{Vgl. auch Catal. gr. coins Brit. Mus. Cyprus p. 80 u. 132.} endlich das ganz ähnliche Bild der früher Mallos jetzt Aphrodisias in Kilikien zugeteilten Münzen (s. Imhoof-Blumer, Kleinas. Münzen S. 435 f. Catal. gr. coins Brit. Mus. Lycaonia etc. S. 95 ff. Pl. 15, 10 ff. 16, 1 ff. S. 117 f.). Nur insofern besteht zwischen dem delphischen Nabelstein und den Baitylen, soweit sie wirklich Meteorsteine sind, ein wesentlicher Unterschied, als jener ebenso wie die beiden Prellsteine an der νύσσα der Rennbahn des Achilleus in der Ilias (Ψ 329) von weißer Farbe war,\footnote{Paus. 10, 16, 2: Τὸν δὲ ὑπὸ Δελφῶν καλούμενον ὀμφαλόν, λίθου πεποιημένον λευκοῦ τοῦτο εἶναι τὸ ἐν μέσῳ γῆς πάσης αὐτοί τε λέγουσιν οἱ Δελφοί κ. τ. λ. Zwar beziehen sich diese Worte nur auf die vor dem Tempel stehende Kopie des O., aber aus dieser Kopie darf unbedingt auch auf jenen geschlossen werden (s. oben S. 81 f.).} während die Meteorsteine bekanntlich alle eine schwärzliche Farbung besitzen, was von den Steinen des Elagabal und der Magna Mater von Pessinus ganz ausdrücklich berichtet wird.\footnote{Herodian 5, 3, 5: ἄγαλμα, ὥσπερ καρ᾽ Ἕλλησιν ἢ Ῥωμαίοις, οὐδὲν ἕστηκε χειροποίητον, θεοῦ φέρον εἰκόνα· λίθος δέ τις ἔστε μέγιστος, κάτωθεν περιφερής, λήγων ἐς ὀξύτητα· κωνοειδὲς αὐτῷ σχῆμα, μέλαινά τε ἡ χροιά, διιπετῆ τε αὐτὸν εἶναι σεμνολογοῦσιν... Dasselbe gilt von dem Stein der Magna Mater Idaea, der von schwärzlicher Farbe (Preller-Jordan, Röm. Mythol. 3 2, 55, 2) und ein ἄγαλμα διοπετές war. Vgl. auch Lenormant im Dict. d. ant. 1 p. 644 u. Arnob. adv. gent. 7, 49: lapis... coloris furvi atque atri. --- Auch Pley, De lanae in antiquor. ritibus usu. Gießen 1911 (= Wünsch-Deubner, Religionsgesch. Versuche u. Vorarb. 11, 2) S. 30 f., der richtig gegenüber Hock (Griech. Weihegebräuche. Münch. 1905 S. 36 ff.) und Miss Harrison die Bedeutung des Omph. als Grab des Python bestreitet, irrt insofern, als er ihn als Baityl auffasst.}

Auf Grund aller dieser Darlegungen und Erwägungen glaube ich mit ziemlicher Zuversicht behaupten zu dürfen, dass die zuerst von Varro bezeugte Auffassung des delphischen Nabelsteines als Grab des Python gegenüber der viel älteren und besser bezeugten delphischen Lokalüberlieferung, die in ihm den ὀμφαλὸς γῆς erblickte, unhaltbar ist.
\clearpage
\section{Nachträge.}
\paragraph{}
Der Güte Paul Herrmanns in Dresden verdanke ich den Hinweis auf zwei neue sehr wertvolle Publikationen von `Omphaloi,' die unbedingt in dieser meiner Abhandlung, solange sie noch nicht völlig abgeschlossen ist, Erwähnung und Berücksichtigung verdienen.

\subsection{(S. uns. Taf. 8, 3 u. 1, 9.)}
\paragraph{}
Der schon wiederholt genannte ausgezeichnete Archäologe und Numismatiker J. N. Svoronos in Athen hat kürzlich in der Ἀρχαιολογικὴ Ἐφημερίς (Περίοδος τρίτη 1912, τεῦχος 3\textsuperscript{ον} καὶ 4\textsuperscript{ον}, πίν. 22) ein sehr interessantes und im ganzen wohlerhaltenes, jetzt im Museum von Aigina befindliches Omphalosrelief veröffentlicht und ebenda S. 254 ff. unter der Überschrift Αἰγίνης ἀνάγλυφον ἀναθηματικόν eingehend besprochen. Das Bildwerk wurde entdeckt auf der Insel Aigina beim Reinigen eines Brunnens, der nur ungefähr 300 Schritte von einem Orte entfernt ist, wo s. Z. ein "`ὅρος τεμένους Ἀπόλλωνος, Ποσειδῶνος"' (I. G. 4 33. 36) aufgefunden wurde. Das sorgfältig und gut gearbeitete Relief besteht aus weißem Marmor, gehört kunstgeschichtlich und technisch etwa der Mitte des 4. vorchristl. Jahrhunderts an und ist 0,50 hoch, 0,45 1/2 breit und 0,11 dick. Rechts steht der langgewandete Kitharode Apollon (en face) (= Ἀπ. Πύθιος), erhält im linken Arm die Kithar und gießt mit der rechten Hand aus einer Schale eine Spende aus über dem delphischen Omphalos. Dieser ist basislos, bienenkorbförmig und mit dem Agrenon bekleidet. Das Eigentümliche dieser Omphalosdarstellung besteht darin, dass hier die beiden Adler des Mythus nicht unten rechts und links vom Nabelsteine sitzen, wie auf den oben S. 84 ff. besprochenen Reliefs des 5. Jahrh. aus Sparta und Athen, sondern vielmehr oben auf der Spitze des O. angebracht sind, aber sonst in ihrer Haltung und Form ganz genau den Adlern der eben erwähnten Bildwerke entsprechen.\footnote{Vielleicht hängt das mit dem Umstande zusammen, dass dies Relief in einer Zeit geschaffen wurde, wo der delph. O. der goldenen Adler beraubt war, also der betr. Künstler nicht mehr den ursprüngl. O. vor Augen hatte.} Links vom Omphalos steht ein Adorant, der den Gott anblickt und ihm seine Rechte entgegenstreckt. Von besonderem Interesse ist, dass diese Reliefdarstellung ganz genau mit dem Revers einer unter Septimius Severus geschlagenen Münze von Megara übereinstimmt, wo nach Pausanias 1, 42, 5\footnote{Paus. a. a. O. Τοῦ δὲ Ἀπόλλωνος πλίνθου μὲν ἦν ὁ ἀρχαῖος ναός... ὁ μὲν δὴ Πύθιος καλούμενος καὶ ὁ Δεκατηφόρος τοῖς Αἰγυπτίοις μάλιστα ἐοίκασι ξοάνοις.} ein alter Kult des Apollon Pythios blühte. S. uns. Taf. 8, 3. Svoronos verweist in dieser Beziehung nicht nur auf Imhoof-Gardner, Num. Comm. on Pausanias p. 6 f. und Taf. A Fig. 9 u. 10, sondern hat auch, um die Übereinstimmung von Relief und Münze recht augenfällig nachzuweisen, die letztere nach einem besonders gut erhaltenen Exemplar auf Taf. 22 in der oberen linken Ecke noch einmal abbilden lassen. Der auf der Münze dargestellte Adorant ist nach Svoronos' einleuchtender Erklärung höchst wahrscheinlich kein anderer als der Kaiser Septimius Severus selbst. S. uns. Taf. 1 nr. 9.

Wie erklärt sich nun aber diese auffallende und merkwürdige Übereinstimmung zwischen dem Relief von Aigina und der Münze von Megara? Nach Svoronos sind in diesem Falle zwei Erklarungen möglich:

a. Das Relief war ursprünglich dem Apollon Pythios von Megara geweiht, diente einem dortigen Münzstempelschneider zur Zeit des Septimius Severus zum Vorwurf und wurde später (unter Kapodistrias) mit anderen Altertümern von Megara nach dem nahe gelegenen Aigina transportiert.

b. Erheblich wahrscheinlicher ist aber nach Svoronos eine zweite Erklärung, die ich hier mit seinen eigenen Worten wiedergeben will. Ἔν τε τοῖς Μεγάροις καὶ τῇ Αἰγίνῃ θὰ ἀνέκειντο πανόμοια ἀγάλματα καὶ ἀνάγλυφα τοῦ αὐτοῦ τύπου, καθ᾽ ἃ καὶ ἐν ἄλλαις πόλεσι τῆς Ἑλλάδος... Ἡ λατρεία τοῦ Ἀπόλλωνος ἐν Αἰγίνῃ εἶναι μεμαρτυρημένη καὶ δὴ ἀρχαιοτάτη· αὐτὸ τὸ ἀνάγλυφον ἡμῶν εὑρέθη, ὡς εἴπομεν, πλησίον τῆς θέσεως, ἐν ᾗ ἀνεκαλύφθη εἷς τῶν "`ὅρων τεμένους Ἀπόλλωνος,"' ἤτοι τοῦ "`ἐν τῷ ἐπιφανεστάτῳ τόπῳ τῆς πόλεως Ἀπολλωνίου"' (I. G. 4, 2. Paus. 2, 30, 1; Imhoof-Gardner a. a. O. pl. 50 nr. 2). Κατ᾽ ἀκολουθίαν ἡ ἐν τῇ νήσῳ ὕπαρξις λατρείας Ἀπόλλωνος Πυθίου εἶναι πιθανωτάτη, ἀφ᾽ οὗ μάλιστα καὶ αὐτὴ ἡ Πυθία παρέπεμψέ ποτε εἰς Αἴγιναν τοὺς ζητοῦντας χρησμὸν ἀπαλλαγῆς ἀπὸ τοῦ πιέζοντος αὐτοὺς αὐχμοῦ Ἕλληνας (Paus. 2, 29, 7).

Zuletzt weist Svoronos noch hin auf einen von ihm unter der Überschrift `Φῶς ἐπὶ Παρθενῶνος' im Journ. Internat. d'archéol. numismat. 14 (1912) p. 226 ff. veröffentlichten Aufsatz (mir unzugänglich), worin er gezeigt haben will, ὅτι τοιοῦτος ὀμφαλὸς ὑπῆρχε καὶ ἐν τῷ κέντρῳ τοῦ ἐμβαδοῦ τῆς Ἀκροπόλεως τῶν Ἀθηνῶν.

\subsection{}
\paragraph{}
Zu den oben S. 89 f. beschriebenen außerhalb Delphis gefundenen plastischen Kopien des delphischen Nabelsteines kommt jetzt noch ein in Thermos (Aitolien) ausgegrabener Omphalos hinzu, von dem es in dem Berichte der Ἀρχαιολ. Ἐφημερίς 1912 S. 267 heißt: Ἐν Θέρμῳ ἐξ ἀδιαγνώστων συντριμμάτων ἀπετελέσθη τὸ μεγαλύτερον μέρος ὀμφαλοῦ περιβαλλομένου διὰ ταινιῶν καὶ μεγάλου ὄφεως.\footnote{Über den sehr alten hölzernen Apollotempel von Thermos mit seinen hocharchaischen Terrakottametopen s. Sotiriadis Ἐφημ. Ἀρχαιολ. 1900 p. 161-212. 1903 p. 96 u. Taf. 1 u. 2 ff. Perrot et Chipiez, Hist. de l'art 8 516 f. Roscher, Hermes 36 482.}

\subsection{(Siehe Taf. 6 Fig. 8 u. Taf. 1 Fig. 23-30.)}
\paragraph{}
Ein sehr eigentümlicher `Omphalos' wurde kürzlich in dem Auktionskataloge der Firma Helbing in München vom Jahre 1910 ("`Griech. Ausgrabungen, Keramik usw. Auktion in München in der Gallerie Helbing. München 1910"' S. 30) abgebildet und kurz beschrieben. Es heißt dort (S. 30) unter Nr. 322: "`Kegelförmige Säule von Schlange umwunden. Oben reliefierter Zweig [Lorbeer?]. Delos. Höhe 38 cm."' Um womöglich Genaueres über dies interessante Objekt zu erfahren, habe ich mich brieflich an Paul Arndt in München gewandt. Dieser hatte die Güte mir mitzuteilen, dass der `Omphalos' von Delos ursprünglich dem griechischen Antiquar C. A. Lambessis in Paris (22 Rue Royale) gehört habe und von diesem nach München zur Auktion gesandt worden sei. Da er aber hier nicht den gewünschten Preis habe erzielen können, hätte ihn L. zurückgenommen und auf einer Pariser Auktion versteigern lassen. Wo er sich jetzt befinde und wer der Käufer sei, habe L. nicht erfahren können. Ferner schreibt mir Arndt, dass Kollegen in München angesichts des Originals die Frage aufgeworfen hätten, ob der vermeintliche Omphalos nicht vielmehr ein `Phallos' sei. P. Herrmann und mir scheint freilich jene Frage kaum in Betracht gezogen werden zu dürfen, weil ein von einer Schlange umringelter und noch dazu an seiner Spitze mit einem reliefierten Zweige geschmückter Phallos eine Singularität ohnegleichen sein würde. Aber auch ein Grabmonument scheint kaum angenommen werden zu dürfen, wenn der Stein wirklich aus Delos stammt, wo bekanntlich Gräber und Grabstelen streng verpönt waren (Herod. 1, 64. Thuk. 3, 104. Strab. 486. Diod. 12, 58. Paus. 2, 27, 1), doch ist, wie mir Arndt schreibt, auf die Herkunftsangabe "`Delos"' nicht allzu viel zu geben, da nach seinen Erfahrungen der griechische Handler nur in dem Falle einen bestimmten Fundort angibt, wenn er den wirklichen verschleiern will.\footnote{Sollte der Stein tatsächlich aus Delos stammen, so würde es nahe liegen, ihn als `Omphalos,' d. h. als ein Symbol der zentralen Lage der Insel (s. oben S. 9 A. 14. S. 39 A. 74), der ἱστίη νήσων nach Kallimachos, aufzufassen. Vgl. nr. 4.} Wie dem aber auch sein möge: in jedem Falle darf an eine gewisse Ähnlichkeit des `Omphalos' von Delos einerseits mit manchen unzweifelhaften Nabelsteinen, anderseits mit den namentlich aus Münzen bekannten Stelen des Apollon Agyieus erinnert werden (s. uns. Taf. 1 Fig. 23 ff.), deren eigentliche Bedeutung zwar auch noch nicht festgestellt ist, die aber hinsichtlich ihrer Gestalt und ihrer Zugehörigkeit zum Kult Apollons entschieden an die apollinischen Nabelsteine gemahnen. Es wäre gewiss der Mühe wert, den jetzigen Aufbewahrungsort des interessanten Steines zu erfahren und eine genauere Beschreibung und Abbildung davon zu veranlassen, insbesondere kommt es darauf an, die Frage zu beantworten, ob der eigentümliche Knauf an der Spitze des `Omphalos' der verstümmelte Kopf der Schlange ist und ob der reliefierte Zweig unterhalb desselben ein Lorbeerzweig sein soll, oder nicht. Weiteres s. jetzt S. 132 a. E.

\subsection{}
\paragraph{}
Kaum waren obige drei Nachträge zu Papier gebracht, als mir die beiden neuen Bände des wertvollen und reichhaltigen Répertoire de reliefs von Sal. Reinach in die Hände fielen, worin noch folgende von mir bisher nicht berücksichtigte Reliefdarstellungen von Omphaloi abgebildet und kurz beschrieben sind.

a. Tom. 2 p. 320 nr. 4: `Délos. Omphalos et serpent. Bull. Corr. Hellén. 1906 p. 561.' Dieser delische Omphalos ist ziemlich ähnlich dem soeben unter 3 erwähnten, nur etwas weniger schlank. Rechts und links von ihm ist en relief je ein Baum dargestellt. Handelt es sich vielleicht in diesem Falle um die Darstellung des Omphalos, der Delos, die ἱστίη νήσων, als Zentrum des ägäischen Meeres oder der Erde bezeichnen sollte?

b. Tom. 3 p. 160 ist nach dem Catal. Barracco nr. 129 ein schönes Votivrelief der Sammlung Barracco abgebildet, dessen Beschreibung lautet: `Relief votif grec; dédicace à Apollon par quatre jeunes gens qui avaient consulté un oracle (Pythaistes). De gauche à droite, Omphalos [fast würfelförmig], Latone [sitzt auf dem O.], Artemis, Apollon, adorants, prêtre.' Oben und unten eine Inschrift: $\svgABV\enspace\svgABW\enspace\svgABX\enspace\svgABY$ || $\svgABZ$ etc. ---

c. Tom. 2 p. 137: `Vienne. De Bologne. Apollon sur l'omphalos devant un autel. Sacken, Skulpturen pl. 18.'

d. Tom. 2 p. 272: `Campana Paris Louvre: Oreste sur l'omphalos à Delphes. Campana, pl. 73. Rohden, p. 117.' Der Omphalos dieses schönen Reliefs ist halbkugelförmig, ohne Basis und mit einem Agrenon bedeckt.

\subsection{}
\paragraph{}
Wer sich für die Vorstellungen der Babylonier und Israeliten vom Nabel der Welt interessiert, den verweise ich jetzt auf A. Jeremias' mir in Korrekturbogen gütigst übersandtes, bald erscheinendes Handbuch der altorientalischen Geisteskultur S. 38 f. Dort ist auch (S. 34 Anm. 4) der eigentümliche Nabelstein in Jerusalem erwähnt mit den Worten: `Vgl. die jüdische Anschauung, nach der Jerusalem höher liegt als alle Länder: Sifre debe Rab Deuter. § 152 (bin Gorion, Jüdische Sagen 1 57). In Jerusalem steht in der griechischen Kathedrale eine Art Becher\footnote{Vgl. dazu Hohel. Salom. 7, 2: `Dein Nabel [Schoß?] ist wie ein runder Becher, dem nimmer Getränk mangelt.'} mit einer umflochtenen gedrückten Kugel, der nach sehr alter Fabel die Mitte der Welt bezeichnete (Baedeker, Palästina 7 39).' Vgl. oben S. 26 und unsere Tafel 9 Fig. 3.
\clearpage
\section{Berichtigungen und Zusätze.}
\paragraph{}
Zu S. 16 f. Dieselbe Anschauung von der Bedeutung der `Glückshaube' findet sich noch im heutigen Griechenland. Ich berufe mich dafür auf das aus Aigina, Argos usw. stammende sehr wertvolle Zeugnis des Herrn Panag. D. Sepherles, der die Güte hatte, auf meine Umfrage in der Λαογραφία τ. γʹ τεῦχ. δʹ (1912) S. 697 f. daselbst τ. δʹ τεῦχ. αʹ καὶ βʹ (1913) S. 322 Folgendes zu antworten:

Μεγάλην σηµασίαν δίδουν εἷς τὴν λεγομένην προσωπίδα (τεµάχιον λεπτῆς µεµβράνης καλυπτούσης τὸ πρόσωπον τοῦ γεννηθέντος βρέφους), μεθ᾽ ἧς γεννῶνται μερικὰ παιδιά, προνοµιοῦχα ἐκ φύσεως θεωρούµενα, καὶ πολὺ τυχηρά. Ἀφοῦ τὴν ἀφαιρέσουν προσεκτικὰ ἐκ τοῦ προσώπου, τὴν τοποθετοῦν ἀπλωτὰ ὅπως ἀποξηρανθῇ καὶ κατόπιν ἀφοῦ τὴν στείλουν εἰς τὴν ἐκκλησίαν καὶ λειτουργηθῇ, τὴν φυλάττουν εἰς τὸ εἰκονοστάσιον τῆς οἰκίας καὶ ἀποκόπτουν μικρὸν τεµάχιον ἐξ αὐτῆς, τὸ ὁποῖον κάµνουν φυλαχτὸ καὶ τὸ κρεμοῦν εἰς τὸν λαιμὸν τοῦ παιδίου.\footnote{Man denke dabei an die `bullae' der römischen Kinder!} Συνήθως αἱ γυναῖκες ὑπεξαιροῦν καὶ ἀποκρύπτουν τὴν προσωπίδα πρὸς ἴδιον ὄφελος· διότι θεωρείται, ὅτι αὕτη ὄχι µόνον φέρει εὐτυχίαν εἰς τὸ γεννηθὲν καὶ φέρον αὐτὴν παιδίον, ἀλλὰ καὶ ὅτι µεταβιβάζεται ἡ εὐτυχία καὶ εἰς τὸν κατόπιν κάτοχον αὐτῆς, διὰ τοῦτο αἱ ἐν τῷ δωματίῳ παραµένουσαι κατὰ τὴν στιγμὴν τοῦ τοκετοῦ συγγενεῖς τῆς τεκούσης, μετὰ προσοχῆς παρατηροῦν τὸ νεογέννητον, µήπως ἔχει προσωπίδα, καὶ προσπαθοῦν νὰ τὴν διασώσουν ἐκ τοῦ κινδύνου τῆς κλοπῆς. Ἡ αὐτὴ δοξασία περὶ τῶν ἰδιοτήτων τῆς προσωπίδος ἐπικρατεῖ καὶ ἐν Ἄργει, Κεφαλληνίᾳ καὶ εἰς ἄλλα τῆς Ἑλλάδος µέρη.

Zu S. 18 f. Wertvolle Zeugnisse für die Geltung der abgeschnittenen Nabelschnur als Amulett bei den heutigen Griechen bringt derselbe Sepherles a. a. O. aus Aigina, Argos und Syros bei:

Ἐν Αἰγίνῃ εὐθὺς μετὰ τὸν τοκετὸν ἡ μαῖα ἁποκόπτει τὸν ὀμφάλιον λῶρον, τὸν ὁποῖον ὁμοῦ μὲ τὸ ὕστερο ἢ ῥίπτουν εἰς τὴν θάλασσαν ἢ θάπτουν πὲρι τῆς οἰκίας, ἵνα μὴ ὑπὸ κυνῶν διασυρθῇ· τὸ δὲ τεµάχιον ἐκεῖνο τοῦ λώρου τὸ ἀπομένον ἐπὶ τοῦ ὀμφαλοῦ, τὸ ὁποῖον µετά τινας ἡμέρας, ὀκτὼ συνήθως [hier ist wohl die 7 tägige Woche zu verstehen!] μετὰ τὸν τοκετὸν τοῦ βρέφους ἐκπίπτει, ἐντελῶς ἀπεξηραμμένον, τὸ φυλάττουν καὶ τὸ τοποθετοῦν εἰς τὸ φυλαχτὸ τοῦ παιδίου διὰ καλό. λέγει δὲ συνήθως ἡ μητέρα εἰς τὸ τέκνον της, ὅταν τὸ βλέπῃ νὰ συχνάζῃ εἰς τόπον ἢ οἰκίαν τινὰ "`ἐκεῖ σοῦ κόψανε τὸν ἀφαλό;"' ἐννοοῦσα ὅτι ὀφείλει νὰ διαμένῃ τὸ περισσότερον διάστηµα ὑπὸ τὴν ἰδίαν της σκέπην, ὑπὸ τὴν ὁποίαν ἐγεννήθη, καὶ τοῦ ἐγένετο ἡ ἀποκοπὴ τοῦ ὀμφαλίου λώρου. Καὶ ὅταν τὸ παιδίον ἐρωτᾶται περί τινος πράγματος καὶ ἀπαντᾷ "`δὲν ξέρω"' τοῦ λέγουν "`νὰ ξεραθῇ ποῦ σ᾿ ἀφαλόκοβε,"' καὶ ἐννοοῦν τὴν μαῖαν.

Ἐν Ἄργει φυλάττουν τὸ τεµάχιον τοῦ ὀμφαλίου λώρου ὡς γίνεται καὶ ἐν Αἰγίνῃ, τὸ δὲ ὕστερο τὸ λέγουν "`ἄκλουθο"' (ἐκ τῆς λέξεως ἀκόλουθο), πιστεύουν δὲ, ὅτι ὅταν µία γυνὴ δὲν θέλει νὰ τεκνοποιήσῃ πλέον, ἐπιτυγχάνει τοῦτο, ἐὰν, καθ᾽ ἣν στιγμὴν γεννᾷ, δώσῃ εἰς µίαν συγγενῆ της ἕνα σουγιᾶ καινούργιο, ἐντελῶς ἁμεταχείριστο, "`ἀφόρηγο,"' ὡς τὸν λέγουν, ἵνα δι᾿ αὐτῆς ἀποκόψῃ τὸν ὀμφάλιον λῶρον τοῦ γεννηθέντος τέκνου. Εὐθὺς δὲ ἀμέσως ἡ ἀποκόψασα τὸν λῶρον πρέπει νὰ σταυρώσῃ τρεῖς φορὰς τὸν σουγιᾶ, καὶ νὰ εἴπῃ ἑνῷ θὰ τὸν κλείῃ συγχρόνως "`κλείω τὴ µήτρα τῆς (δεῖνα)."' Δίδεται δὲ ἀμέσως κατόπιν εἰς τὴν λεχώ, ἥτις ἐνόσῳ θὰ τὸν φυλάτεῃ κλεισµένον, δὲν φοβεῖται νὰ συλλάβη πλέον, διότι πιστεύει ὅτι οὕτω κρατεῖ κλεισμένην τὴν µήτραν της. Ὅταν δὲ τύχῃ κανὲν παιδὶ νὰ εἶναι πολὺ ἄτακτον ἢ ὅταν γίνῃ κακὸς ἄνθρωπος, λέγουν περὶ αὐτοῦ τὴν ἑξῆς παροιµίαν: Ἐκεῖ ποῦ σοῦ κοβαν τ᾽ ἀφάλι, δὲν σοῦ κοβαν τὸ κεφάλι;

Εἰς δὲ τὴν Σύρον, ὡς ἔμαθον ἐν Πειραιεῖ, παρὰ γυναικὸς ἐκ Σύρου καταγοµένης, τὸ ἀπομεῖναν ἐπὶ τοῦ ὀμφαλοῦ τεµάχιον τοῦ ὀμφαλίου λώρου, ἅμα ἀποξηρανθῇ καὶ πἐσῃ, τὸ κάµουν φυλαχτὸ τοῦ παιδίου, καὶ πιστεύουν ὅτι ὅταν τὸ ἔχῃ ἐπάνω του, γίνεται πολὺ ἔξυπνο εἰς τὸ σχολεῖον καὶ εἰς τὰ γράµµατα.

Aus Bourboura in Kinouria stammen nach Herrn K. I. Mantzouranes (ebendort S. 323 f.) folgende Bräuche und Anschauungen:

Ὁ ὀμφαλὸς εἶναι τὸ µέσον τοῦ σώματος. Ὅταν λύεται, ὡς πιστούουσι, τὸν δένουσιν ὡς ἑξῆς. Θέτουσι τὸν δείκτην τῆς ἀριστερᾶς χειρὸς ἐπὶ τοῦ ὀμφαλοῦ καὶ ἐν τῇ στάσει ταύτῃ κάµνουσι τρεῖς ὁλοκλήρους στροφὰς πρὸς δεξιά. Διὰ τοῦ µέσου τούτου ὁ ὀμφαλὸς δένεται καὶ διὰ νὰ μὴ λυθῇ καὶ πάλιν δένουσι διὰ χωρίδος ἐπ᾽ αὐτοῦ τεμάχια σκορόδου.

Ὅταν ὁ λοῦρος ὁ συγκρατῶν τὸ νεογνὸν ἔχει µελάνα στίγματα (ἐλιές), τοῦτο εἶναι ἀσφαλὲς σημεῖον ὅτι ἡ λεχὼ θὰ ἀποκτῄσῃ εἰς τὸ μέλλον ἄρρενα τέκνα.

Zu S. 35. Die Vermutung, dass es sich bei dem kleinen Orte Naszály in Ungarn wohl nur um einen Scherz handelt, verdanke ich meinem Freunde K. Seeliger, der als Analogie dazu die in der Lausitz verbreitete scherzhafte Redensart anführt: "`In Bernstadt [einem kleinen Städtchen b. Löbau] wird die Erdachse geschmiert."'

Zu S. 39 Anm. 74. Dass Delos, wo sich bekanntlich auch ein in älterer Zeit hochangesehenes apollinisches Orakel befand (vgl. Hom. hy. in Ap. Del. 81 u. Gemoll z. d. St. Hermann, Gottesd. Alt. § 40, 23. Verg. Aen. 3, 92 etc.), geradezu als Erdnabel (ὀμφαλὸς γῆς καὶ θαλάσσης) angesehen wurde, bezeugt der Scholiast zu Eurip. Or. 331 mit den Worten: ἡ Δῆλος... μεσαιτάτη ἐστὶ τοῦ παντὸς κόσμου, ἢ τῶν Κυκλάδων νήσων. Wir haben oben (S. 128 f.) gezeigt, dass an diese Bedeutung von Delos auch noch tatsächlich vorhandene Monumente (d. h. plastische ὀμφαλοί) erinnern.

Zu S. 78 Anm. 146. Für die Lokalisierung des Omphalos in unmittelbarer Nähe des χάσμα γῆς und des pythischen Dreifußes, also im Adyton, sprechen auch diejenigen Bildwerke, welche Apollon auf dem Dreifuß sitzend und seine Füße auf den Omphalos wie auf eine Fußbank setzend darstellen (s. S. 87, uns. Taf. 8, 2 u. S. 92).

Zu S. 118-120. Zu den omphalosartigen Grabmälern gehört auch das auf der athenischen Lekythos bei Benndorf, Griech. u. sizil. Vasenbilder Taf. 24, 3 = Schreiber, Kulturhistor. Bilderatlas 1 Taf. 94, 6 abgebildete, vor dem sich eine hohe, schlanke, ziemlich spitz zulaufende Stele erhebt. Bei dieser Gelegenheit, sei auch die Frage aufgeworfen, ob diese eigentümliche Eiform gewisser Grabmäler sich nicht einfach aus der Form der uralten Kuppelgräber zu Mykenae usw. erklären lässt. Vgl. Reber, Baukunst im Alterth. Fig. 136 = Schreiber a. a. O. Taf. 94 Fig. 9.

Zu S. 128 f. Paul Arndt in München verdanke ich die Zusendung des Bulletin of the Metropolitan Museum of Art Vol. 8 nr. 8 (New York, August 1913), wo S. 174 Folgendes zu lesen ist: `A pointed pillar, 14 5/8 in. (37,2 cm) high, with a snake coiled round it and an ivy wreath at the top, probably served as a symbol of Apollon Agyieus, who, we are told, was worshiped under the form of a pointed column... The base of our example is left rough and was evidently intended to be sunk in the ground.' Das interessante Objekt befindet sich also jetzt im Metrop. Mus. of Art in New York.
\clearpage
\section{Bilder im Texte.}
\paragraph{}
1. S. 30: Münze von Kypros mit dem in einem Tempel stehenden, mit einer meta oder einem Omphalos (umbilicus) verglichenen Idol der Aphrodite von Paphos, nach Roschers Lexikon d. Mythol. 1, Sp. 747.

2. S. 32: Zwei Himmelsgöttinnen und Erdgott der Ägypter. Nach Brugsch, Religion u. Mythologie d. alt. Ägypter S. 211. Vgl. auch Roscher, Die neuentdeckte Schrift eines altmilesischen Naturphilosophen etc., Bildertafel Fig. 1 u. 2 u. daselbst S. 6. Derselbe, Über Alter, Ursprung u. Bedeutung d. hippokrat. Schr. v. d. Siebenzahl S. 12 Anm. 15. Boll, Die Lebensalter S. 50 f. Roscher, Die hippokrat. Schrift von der Siebenzahl, in ihrer vierfachen Überlieferung zum erstenmal herausgegeben und erläutert von W. H. R. Paderborn 1913. S. 12 u. 156.

3. S. 93: Wandbild aus Pompeji, nach P. Herrmann, Denkmäler der Malerei des Altertums 3 20: Apollon feiert seinen Sieg über Python, der sterbend oder tot den auf einer viereckigen niedrigen Basis stehenden, halbkugelförmigen, netzbedeckten Omphalos umringelt. S. ob. S. 93 f.
\clearpage
\section{Verzeichnis der Abbildungen nebst Erläuterungen.}
\subsection{Erklärungen zu Tafel 1 (Münztafel).}
\paragraph{}
1. Elektronmünze (vergrößert) des 5. Jahrh. von Kyzikos, der Pflanzstadt Milets, nach Studniczka im Hermes 37 (1902) S. 266: netzbedeckter, bienenkorbförmiger, basisloser Omphalos (von Milet oder Branchidai?), darauf zwei große die Köpfe einander zukehrende Adler, darunter der Thunfisch, das Wahrzeichen von Kyzikos (s. ob. S. 50).

2. Bronzemünze von Milet aus der Zeit des Commodus, nach Overbeck, Kunstmythol. Apollon Münztafel 4 nr. 47 (vgl. daselbst S. 304 u. 308): ziemlich hoher, auf niedriger Basis stehender, ganz oben sich plötzlich stark verjüngender, schlangenumwundener Omphalos, auf den der attributlos und nackt in bequemer Lage auf einem Felsen sitzende und rechtshin (in die Ferne?) blickende Apollon von Didyma sich mit dem l. Arme stützt (s. ob. S. 47).

3. Elektronmünze des 5. Jahrh. von Kyzikos, nach Catal. of the gr. coins in the Brit. Mus. Mysia Taf. 7, 1 (vgl. daselbst S. 28): Orestes mit gezogenem Schwert in der R., knieend am Omphalos (von Branchidai oder Kyzikos oder Delphi?), der, wie es scheint (s. a. a. O. Fig. 2!), mit einem Netze bedeckt und basislos ist. Darunter Thunfisch. S. ob. S. 52 f.

4. Bronzemünze aus der Kaiserzeit von Tarsos, nach Overbeck a. a. O. Münztaf. 1 nr. 30 (vgl. das. S. 25 u. 29): Apollon (Lykeios?) in jeder Hand einen Wolf an den Vorderbeinen haltend, steht auf dem bienenkorbförmigen basislosen Omphalos (s. ob. S. 89 Anm. 163).

5. Bronzemünze aus der Zeit Hadrians von Kreta, nach Overbeck a. a. O. Münztaf. 1 nr. 27 (s. das. S. 25): Apollon nackt, in der Haltung und mit den Attributen des Didymäischen A. auf dem niedrigen Omphalos (von Branchidai?) stehend. S. ob. S. 89 Anm. 163.

6. Silbermünze des 5. vorchristl. Jahrh. von Delphi, nach Catal. of the gr. coins in the Brit. Mus. Central Greece Taf. 4, 4 (s. das. S. 24): Obverse: Tripod. --- Rev.: Incuse square, within which a cirele with a point in the centre (= Omphalos). S. ob. S. 96 f.

7. Silbermünze der Amphiktionen von Delphi aus dem 4. Jahrh. nach Overbeck a. a. O. Münztaf. 3 nr. 35 (s. das. S. 307 f.): Der pythische Apollon (Kitharodos) sitzt auf dem annähernd halbkugelförmigen, netzbedeckten, basislosen Omphalos von Delphi: s. ob. S. 95 f.

8. Bronzemünze aus hadrianischer Zeit von Delphi nach Catal. a. a. O. Taf. 4, 20 (s. das, S. 29): Rev. $\svgACA$ Rock, upon which Delphian omphalos [ohne Basis, ziemlich hoch, fast bienenkorbförmig, aber oben etwas spitz auslaufend], around which serpent twines itself. S. ob. S. 96.

9. Münze des Septimius Severus von Megara, nach Svoronos in der Ἀρχαιολ. Ἐφημ. 3 (1912) 3/4 πίν. 22: Septimius Sev. als Adorant vor dem langgewandeten Apollon Kitharodos (= Pythios?) stehend; in der Mitte zwischen ihnen der basislose, netzbedeckte, fast halbkugelförmige Omphalos (von Delphi), auf dessen Spitze 2 Adler sitzen, die ihre Köpfe voneinander abwenden. Die Münze hat, wie Svoronos erkannt hat, die größte Ähnlichkeit mit dem kürzlich aufgefundenen Relief von Aigina; s. ebenda sowie unsere Tafel 8 Fig. 3 und oben S. 126 f.

10. Tetradrachme von Kalchedon, nach Overbeck a. a. O. Münztaf. 3 nr. 38 (s. das. S. 300 u. 308): nackter Apollon rechtshin auf dem basislosen, bienenkorbförmigen Omphalos sitzend, in der vorgestreckten r. Hand einen Pfeil haltend: s. ob. S. 98.

11. Tetradrachme von Sinope, der Kolonie Milets, nach Overbeck a. a. O. Münztaf. 3 nr. 37 (s. das. S. 300): Apollon Kitharodos sitzt auf dem basislosen, bienenkorbförmigen, netzbedeckten Omphalos (von Didyma?); s. ob. S. 53.

12. Tetradrachme von Kyzikos, der Kolonie Milets, nach Overbeck a. a. O. Münztaf. 3, 39 (s. das. S. 300): Apollon Kitharodos auf dem netzbedeckten, basislosen, fast halbkugelförmigen Omphalos sitzend. Vgl. ob. S. 51.

13. Tetradrachme des Antiochos 1. von Syrien, nach Overbeck a. a. O. Münztaf. 3, 31 (vgl. das. S. 307 f. und Catal. Brit. Mus. Syria Taf. 3, 4): s. oben S. 30 Anm. 58: Apollon nackt auf dem basislosen, netzbedeckten, fast bienenkorbförmigen Omphalos sitzend und in der R. einen Pfeil haltend (vgl. nr. 46).

14. Erzmünze von Rhegion, nach Overbeck a. a. O. Münztaf. 3, 43 (s. das. S. 308): Apollon sitzt nackt auf dem basislosen, netzbedeckten, bienenkorbförmigen Omphalos und hält in der R. einen Pfeil: s. ob. S. 97.

15. Didrachme von Chersonasos auf Kreta, nach Overbeck a. a. O. Münztaf. 3, 36 (s. das. S. 307): Apollon Kitharodos rechtshin auf dem basislosen, fast bienenkorbförmigen, wie es scheint, netzbedeckten Omphalos sitzend, s. ob. S. 79.

16. Tetradrachme des Antiochos 3. von Syrien, nach Overbeck a. a. O. Münztaf. 3 nr. 42 (s. das. S. 308): Apollon sitzt, nur den Unterkörper mit dem Himation umhüllend, auf dem netzbedeckten, basislosen, bienenkorbförmigen Omphalos (vgl. nr. 13 u. ob. S. 30 Anm. 58).

17. Erzmünze des Gordianus 3 von Patara in Lykien, nach Overbeck a. a. O. Münztaf. 5 nr. 6 (s. das. S. 304 u. S. 310 oben): Apollon mit langem Chiton bekleidet, stehend und in der R. einen Lorbeerzweig haltend; vor ihm ein kleiner, bienenkorbförmiger, basisloser Omphalos, auf dem ein ziemlich großer, eben seine Flügel entfaltender Adler (oder Rabe? Overbeck) sitzt. S. ob. S. 107 f.

18. Münze des Elagabal von Emisa, nach Baumeister, Denkmäler S. 603 nr. 649: Der bienenkorbförmige, omphalosartige Steinfetisch von Emisa in einem sechssäuligen Tempel stehend, im Giebelfeld des Tempels ein Halbmond. S. ob. S. 124.

19. Münze von Perge, nach Baumeister, Denkmäler S. 603 nr. 645: Das bienenkorbförmige, oben mit einer Art Knauf versehene Idol der Artemis Pergaia in einem zweisäuligen Tempel stehend; oben im Giebelfeld ein Adler(?), rechts und links oben vom Idol je ein Stern, unten je ein `Trabant.' S. ob. S. 124.

20. Desgl. nach Baumeister a. a. O. S. 603 nr. 646: ähnliche Darstellung des Artemisidols von Perge, nur ist es hier wesentlich breiter oder dicker gebildet.

21. Tetradrachme des Königs Nikokles von Paphos, nach Overbeck a. a. O. Münztaf. 3 nr. 40 (s. das. S. 300 u. 308): Apollon (nackt) sitzt links hin auf dem basislosen, netzbedeckten, bienenkorbförmigen Omphalos und hält in der R. den Bogen. S. ob. S. 30.

22. Münze von Seleukeia (Pieria) mit dem Idol des Zeus Kasios (in dessen Tempel) nach Roscher, Lexikon der Mythologie 1 Sp. 747. Vgl. Head, Hist. nu. 1 S. 661 unt.

23. Silberstater von Ambrakia (Akarn.) aus dem 4. Jahrh., nach Overbeck a. a. O. Münztaf. 1, 1: Behelmter Pallaskopf, links daneben die Spitzsäule des Apollon Agyieus mit Wollbinden (Stemmata) geschmückt. S. ob. S. 129 u. 132 a. E. u. vgl. Overbeck a. a. O. S. 4 Anm. a.

24. Silberdrachme von Ambrakia aus dem 3. Jahrh. nach Overbeck a. a. O. Münztaf. 1, 2: Spitzsäule des Agyieus mit Stemmata auf zweistufiger Basis (daneben Palmzweig), von einem Lorbeerkranz umgeben; vgl. Head, Hist. nu. 1 270. S. ob S. 129 u. 132 u. vgl. Overbeck a. a. O. S. 4 Anm. a, sowie unt. Taf. 6 Fig. 8.

25. Desgl. nach Overbeck a. a. O. Fig. 3 (ohne Palmzweig). S. ob. S. 129 u. 132 a. E.

26. Drachme von Apollonia Illyr. nach Overbeck a. a. O. Fig. 4; Kopf des Apollon links hin, rechts daneben Spitzsäule des Agyieus. S. ob. S. 129 u. 132 u. vgl. Head, Hist. nu. 1 S. 265, sowie unt. Taf. 6 Fig. 8.

27. Münze von Apollonia Illyr., nach Overbeck a. a. O. Fig. 5. Desgl.

28. Vierteldrachme derselben Stadt, nach Overbeck a. a. O. Fig. 6 a b.

29. Desgl. nach Overbeck a. a. O. Fig. 7.

30. Kupfermünze derselben Stadt, nach Overbeck a. a. O. Fig. 8.
\clearpage
\vspace*{\fill}
\begin{figure}[H]
\centering
\includegraphics[width=0.85\textwidth,keepaspectratio]{figs/table01.jpg}
\caption{\frakfamily Tafel 1.}
\end{figure}
\vspace*{\fill}
\clearpage
\subsection{Erklärungen zu Tafel 2.}
\begin{center}
(Vasengemälde mit dem delphischen Omphalos).
\end{center}
\paragraph{}
1. Rotfiguriere Vase in Neapel, nach Baumeister, Denkm. 2 S. 1110 Fig. 1307: Apollon mit Leier und Lorbeerzweig sitzt auf dem halbkugelförmigen, basislosen, mit Tänien und Lorbeerzweigen geschmückten Omphalos von Delphi; vor ihm Orestes, hinter diesem Elektra. Hinter Apollon Pylades und die Pythia, auf dem Dreifuß thronend. S. ob. S. 101.

2. Desgl. in Petersburg, nach Baumeister a. a. O. 1 S. 104 Fig. 110: Bündnis zwischen Dionysos und dem in Delphi einziehenden Apollon. Vorn unten in der Mitte der basislose, netzbedeckte, halbkugelförmige Omphalos von Delphi; darüber eine Palme, rechts von dieser Dionysos, links Apollon, einander die rechte Hand zum Bündnis reichend. Rings umher Satyrn und Bacchantinnen, links oben der delphische Dreifuß. S. ob. S. 101.

3. Desgl. in Ruvo, Sammlung Jatta, nach Roschers Lexikon der Mythol. 3 Sp. 175: Orestas verbirgt sich hinter dem hohen eiförmigen, netzbedeckten Omphalos, der aus einem blumenkelchförmigen auf einer 3-4stufigen Basis stehenden Gebilde gewissermaßen hervorwächst; links vom Omphalos Neoptolemos mit dem r. Knie auf einen Altar sich stützend und sein Schwert schwingend, links von ihm ein lanzenschwingender Krieger. In der Mitte oben der Tempel von Delphi, rechts von ihm Apollon mit dem Bogen sitzend, neben ihm eine Palme. Links vom Tempel der Dreifuß und Artemis. S. ob. S. 103.
\clearpage
\begin{landscape}
\vspace*{\fill}
\begin{figure}[H]
\centering
\includegraphics[height=0.75\textheight,keepaspectratio]{figs/table02-01.jpeg}
\caption{\frakfamily Tafel 2. --- 1.}
\end{figure}
\vspace*{\fill}
\clearpage
\vspace*{\fill}
\begin{figure}[H]
\centering
\includegraphics[height=0.75\textheight,keepaspectratio]{figs/table02-02.jpeg}
\caption{\frakfamily Tafel 2. --- 2.}
\end{figure}
\vspace*{\fill}
\clearpage
\vspace*{\fill}
\begin{figure}[H]
\centering
\includegraphics[height=0.75\textheight,keepaspectratio]{figs/table02-03.jpeg}
\caption{\frakfamily Tafel 2. --- 3.}
\end{figure}
\vspace*{\fill}
\clearpage
\end{landscape}
\subsection{Erklärungen zu Tafel 3.}
\begin{center}
(Vasengemälde mit dem delphischen Omphalos).
\end{center}
\paragraph{}
1. Rotfiguriere Vase der Sammlung Hope nach Baumeister a. a. O. 2 S. 1118 Fig. 1315: Orestes' Sühnung in Delphi. Unten in der Mitte kniet Orestes vor dem hohen, eiförmigen, basislosen, netzbedeckten Omphalos, hinter dem der Dreifuß steht. Rechts davon Athene, links Apollon. Hinter diesem eine Erinys, ebenso eine desgl. hinter dem Dreifuß. Links oben das Brustbild des Pylades, rechts oben das einer verschleierten Frau (Klytaimnestra?). S. ob. S. 104.

2. Vasenbild nach O. Jahn, Vasenbilder Taf. 1 = Bötticher, Der Omphalos des Zeus zu Delphi = 19. Winckelmannsprogramm Berl. 1859 Tafel: Orestes umklammert den hohen, kegelförmigen, netzbedeckten, basislosen, wie es scheint, im Tempel befindlichen Omphalos. Anwesend rechts Artemis, links Apollon und die entsetzt fliehende Pythia; links oben eine Erinys. S. ob. S. 104.

3. Vase im Louvre zu Paris, nach Roschers Lexikon der Mythol. 3 Sp. 983/4: Orestes' Sühnung. In der Mitte Orestes, auf der Basis des hohen, kegelförmigen, netzbedeckten Omphalos sitzend. Anwesend Apollon, Artemis, 3 Erinyen und der Schatten Klytaimnestras. S. ob. S. 102 f.
\clearpage
\vspace*{\fill}
\begin{figure}[H]
\centering
\includegraphics[width=0.85\textwidth,keepaspectratio]{figs/table03-01.jpeg}
\caption{\frakfamily Tafel 3. --- 1.}
\end{figure}
\vspace*{\fill}
\clearpage
\begin{landscape}
\vspace*{\fill}
\begin{figure}[H]
\centering
\includegraphics[height=0.75\textheight,keepaspectratio]{figs/table03-02.jpeg}
\caption{\frakfamily Tafel 3. --- 2.}
\end{figure}
\vspace*{\fill}
\clearpage
\vspace*{\fill}
\begin{figure}[H]
\centering
\includegraphics[height=0.75\textheight,keepaspectratio]{figs/table03-03.jpeg}
\caption{\frakfamily Tafel 3. --- 3.}
\end{figure}
\vspace*{\fill}
\clearpage
\end{landscape}
\subsection{Erklärungen zu Tafel 4.}
\begin{center}
(Vasengemälde mit Omphalos oder omphalosartigen Grabmälern).
\end{center}
\paragraph{}
1. Omphalosförmiges, an den `Bomos' (= Omphalos?) des Tempels von Thymbra (s. das folgende Bild!) erinnerndes Grab des Achilleus, über dem Polyxene geschlachtet wird, anwesend Nestor, Diomedes, Neoptolemos, Amphilochos, Antiphates, Aias Oiliades, Phoinix, hocharchaisches Vasenbild nach Journ. of Hellen. Stud. 18 (1898) Taf. 15; s. ob. S. 116.

2. Der `Bomos' (= Omphalos?) im Apollontempel von Thymbra, neben dem der gemordete Troilos liegt. Anwesend Hermes, Athena, Achilleus, Hektor, Aineas, Deithynos (= Deiphobos) Hochaltertümliches Vasenbild in München nach Baumeister, Denkm, S. 1902 Fig. 2001. S. ob. S. 106 f. u. 116 Anm. 209 f.

3. Omphalosförmiges Grab, auf dem zwei Raben (Augurienvögel) sitzen, rechts und links je ein sitzender gerüsteter Krieger, das Augurium beobachtend, schwarzfigurieres Vasenbild im Brit. Museum nach Studniczka im Hermes 37 (1902) S. 265 nr. 4. S. ob. S. 117 f. u. vgl. unt. Taf. 5 Fig. 4.

4. Die Tötung des Python, der sich hinter dem Omphalos (in der delphischen Felsengrotte) zu verbergen sucht, durch den noch auf dem Arme der Leto getragenen kindlichen Apollon; anwesend Artemis. Schwarzfiguriere (archaische??) Lekythos in Paris nach Roschers Lexikon d. Mythol. 3 Sp. 3408 Fig. 4. S. ob. S. 104 f.

5. Omphalosförmiger, gegitterter (s. ob. Fig. 1 u. 2) Gegenstand (Grabstele??) als Dekoration eines Gefäßes in Straßburg nach Jahrb. für d. klass. Alt. 15 (1912) Taf. 2 Fig. 8.

6. Omphalosförmiges Grabmal, an dem oder in dem kleine Flügelwesen (Keren, Seelen?) umhersehwirren und eine Schlange sichtbar ist, daneben zwei Trauernde, attische Lekythos nach O. Crusius in Roschers Lexikon d. Mythol. 2 Sp. 1147 Fig. 5. S. ob. S. 119.
\clearpage
\begin{landscape}
\vspace*{\fill}
\begin{figure}[H]
\centering
\includegraphics[height=0.65\textheight,keepaspectratio]{figs/table04-01.jpeg}
\caption{\frakfamily Tafel 4. --- 1.}
\end{figure}
\vspace*{\fill}
\clearpage
\vspace*{\fill}
\begin{figure}[H]
\centering
\includegraphics[height=0.55\textheight,keepaspectratio]{figs/table04-02.jpeg}
\caption{\frakfamily Tafel 4. --- 2.}
\end{figure}
\vspace*{\fill}
\clearpage
\vspace*{\fill}
\begin{figure}[H]
\centering
\includegraphics[height=0.75\textheight,keepaspectratio]{figs/table04-03.jpeg}
\caption{\frakfamily Tafel 4. --- 3.}
\end{figure}
\vspace*{\fill}
\clearpage
\vspace*{\fill}
\begin{figure}[H]
\centering
\includegraphics[height=0.75\textheight,keepaspectratio]{figs/table04-04.jpeg}
\caption{\frakfamily Tafel 4. --- 4.}
\end{figure}
\vspace*{\fill}
\clearpage
\vspace*{\fill}
\begin{figure}[H]
\centering
\includegraphics[height=0.75\textheight,keepaspectratio]{figs/table04-05.jpeg}
\caption{\frakfamily Tafel 4. --- 5.}
\end{figure}
\vspace*{\fill}
\clearpage
\end{landscape}
\vspace*{\fill}
\begin{figure}[H]
\centering
\includegraphics[height=0.75\textheight,keepaspectratio]{figs/table04-06.jpeg}
\caption{\frakfamily Tafel 4. --- 6.}
\end{figure}
\vspace*{\fill}
\clearpage
\subsection{Erklärungen zu Tafel 5.}
\begin{center}
(Vasengemälde mit omphalosförmigen Grabmälern).
\end{center}
\paragraph{}
1. Weißes, omphalosartiges Grabmal mit Schlange daran (oder darin?), überragt von einem `Baityl' (?), rechts und links davon je ein gerüsteter Krieger, schwarzfigurieres Vasenbild nach Journ. of Hellen. Stud. 19 (1899) S. 228. S. ob S. 117.

2. Eiförmiges Grabmal auf Basis, links davon eine (opfernde?) Frau, Vasenbild nach Journ. of Hellen. Stud. 19 S. 169 Fig. 1. S. ob. S. 116 Anm. 208.

3. Eiförmiges, auf Stufenbasis sich erhebendes Grabmal mit Stele, rechts davon eine Opfergaben darbringende Frau, links ein Jüngling mit Lanze, Vasenbild nach Kell, Journ. of Hellen. Stud. 19 Taf. 2. S. ob. S. 116.

4. Zwei omphalosförmige Gräber (s. ob. Taf. 4, 3), auf denen je ein Augurienvogel sitzt, rechts und links von jedem je ein gerüsteter Krieger, sitzend das Wahrzeichen beobachtend, schwarzfiguriere Vase in Neapel nach Journ. of Hellen. Stud. 19 (1899) S. 227. S. ob. S. 117.
\clearpage
\begin{landscape}
\vspace*{\fill}
\begin{figure}[H]
\centering
\includegraphics[height=0.7\textheight,keepaspectratio]{figs/table05-01.jpeg}
\caption{\frakfamily Tafel 5. --- 1.}
\end{figure}
\vspace*{\fill}
\end{landscape}
\clearpage
\vspace*{\fill}
\begin{figure}[H]
\centering
\includegraphics[height=0.55\textheight,keepaspectratio]{figs/table05-02.jpeg}
\caption{\frakfamily Tafel 5. --- 2.}
\end{figure}
\vspace*{\fill}
\clearpage
\begin{landscape}
\vspace*{\fill}
\begin{figure}[H]
\centering
\includegraphics[height=0.75\textheight,keepaspectratio]{figs/table05-03.jpeg}
\caption{\frakfamily Tafel 5. --- 3.}
\end{figure}
\vspace*{\fill}
\clearpage
\vspace*{\fill}
\begin{figure}[H]
\centering
\includegraphics[height=0.65\textheight,keepaspectratio]{figs/table05-04.jpeg}
\caption{\frakfamily Tafel 5. --- 4.}
\end{figure}
\vspace*{\fill}
\end{landscape}
\clearpage
\subsection{Erklärungen zu Tafel 6.}
\begin{center}
(plastische Omphaloi und omphalosähnliche Grabmäler).
\end{center}
\paragraph{}
1. Netzbedeckter, bienenkorbförmiger Omphalos von weißem Marmor, gefunden auf dem Platze vor dem Apollontempel in Delphi, Federzeichnung von Alice Roscher nach einer Photographie Pomtows. S. ob. S. 81 f. Die beste und deutlichste Abbildung dieses Omphalos s. jetzt in `Über Land und Meer' 1913 nr. 3 S. 1083.

2. Zuckerhutförmiger, innen hohler Omphalos[?] von glattem Kalkstein, gefunden in Delphi beim Thesauros von Syrakus, Federzeichnung von Alice Roscher nach einer Photographie Pomtows. S. ob. S. 83.

3. Marmoromphalos mit Netzwerk aus dem Heiligtum des Apollon Erethimios zu Kamiros auf Rhodos nach Corp. Inscr. Insul. 1, 733; s. ob. S. 109 f.

4. Großer, konischer Marmoromphalos mit Netzwerk, oben abgeplattet, um eine Apollonstatue zu tragen, gefunden im Dionysostheater Athens, nach Studniczka im Hermes 37 (1902) S. 261 Fig. 3. S. ob. S. 88 f.

5. Großer, konischer, von einer Schlange umwundener Marmoromphalos aus der Nekropole Milets, nach einer Photographie Br. Schröders. S. ob. S. 38 u. 48 oben. Es ist natürlich zweifelhaft, ob es sich in diesem Falle um einen richtigen Omphalos oder um ein Grabmal handelt.

6. Großer, konischer, mit Netzwerk versehener Omphalos von Marmor, gefunden in Vathia bei Eretria unweit des Heiligtums der Artemis Amarysia, nach Ἐφημερὶς Ἀρχαιολ. 1900 S. 19. S. ob. S. 89.

7. Grabomphalos, gefunden zu Athen, Federzeichnung von Alice Roscher nach einer Photographie Br. Schroeders. S. ob. S. 119 f.

8. Omphalosähnliche, konische, ziemlich schlanke, an die Säule des Apollon Agyieus erinnernde Säule, die von einer Schlange umwunden ist, gefunden in Delos(?), nach `Griech. Ausgrabungen... Auktion in München in der Gallerie Helbing.' München 1910 S. 30. S. ob. S. 128 u. 132.
\clearpage
\vspace*{\fill}
\begin{figure}[H]
\centering
\includegraphics[width=0.75\textwidth,keepaspectratio]{figs/table06-01.jpeg}
\caption{\frakfamily Tafel 6. --- 1.}
\end{figure}
\vspace*{\fill}
\clearpage
\vspace*{\fill}
\begin{figure}[H]
\centering
\includegraphics[height=0.55\textheight,keepaspectratio]{figs/table06-02.jpeg}
\caption{\frakfamily Tafel 6. --- 2.}
\end{figure}
\vspace*{\fill}
\clearpage
\vspace*{\fill}
\begin{figure}[H]
\centering
\includegraphics[width=0.6\textwidth,keepaspectratio]{figs/table06-03.jpeg}
\caption{\frakfamily Tafel 6. --- 3.}
\end{figure}
\vspace*{\fill}
\clearpage
\vspace*{\fill}
\begin{figure}[H]
\centering
\includegraphics[width=0.75\textwidth,keepaspectratio]{figs/table06-04.jpeg}
\caption{\frakfamily Tafel 6. --- 4.}
\end{figure}
\vspace*{\fill}
\clearpage
\vspace*{\fill}
\begin{figure}[H]
\centering
\includegraphics[width=0.8\textwidth,keepaspectratio]{figs/table06-05.jpeg}
\caption{\frakfamily Tafel 6. --- 5.}
\end{figure}
\vspace*{\fill}
\clearpage
\vspace*{\fill}
\begin{figure}[H]
\centering
\includegraphics[height=0.55\textheight,keepaspectratio]{figs/table06-06.jpeg}
\caption{\frakfamily Tafel 6. --- 6.}
\end{figure}
\vspace*{\fill}
\clearpage
\vspace*{\fill}
\begin{figure}[H]
\centering
\includegraphics[width=0.9\textwidth,keepaspectratio]{figs/table06-07.jpeg}
\caption{\frakfamily Tafel 6. --- 7.}
\end{figure}
\vspace*{\fill}
\clearpage
\vspace*{\fill}
\begin{figure}[H]
\centering
\includegraphics[height=0.75\textheight,keepaspectratio]{figs/table06-08.jpeg}
\caption{\frakfamily Tafel 6. --- 8.}
\end{figure}
\vspace*{\fill}
\clearpage
\subsection{Erläuterungen zu Tafel 7.}
\begin{center}
(Reliefs mit dem Omphalos von Delphi und Didyma).
\end{center}
\paragraph{}
1. Votivrelief im Louvre: Apollon Kitharodos, dem Nike eine Schale darreicht, zwischen ihnen der basislose, fast halbkugelförmige, mit Tänien geschmückte Omphalos, nach Overbeck, Kunstmythol. Atlas Taf. 21 nr. 11. S. ob. S. 92.

2. Relief der Dresdener Dreifußbasis: Apollon und Herakles, den Dreifuß raubend; zwischen ihnen der Omphalos, ganz ähnlich gebildet wie in Fig. 1, nach Overbeck a. a. O. Taf. 24 nr. 14. S. ob. S. 92 f.

3. Relief des Archelaos von Priene: Apotheose Homers. Im zweiten Streifen von unten die Grotte von Branchidai: darin der fast halbkugelförmige, basislose Omphalos von Branchidai, links von ihm Apollon Kitharodos, rechts die Priesterin, eine Trinkschale zum Munde(?) führend. S. ob. S. 48 f.

4. Votivrelief des 5. Jahrh. aus Sparta, darstellend Apollon Kitharodos, dem Artemis einen Trunk kredenzt, zwischen ihnen der delphische Omphalos auf viereckiger Basis, annähernd halbkugelförmig, flankiert von zwei Adlern, die die Köpfe nach rückwärts wenden. Nach Wolters in den Athen. Mitteilungen 1887 (12) Taf. 12. S. ob. S. 84 u. vgl. Taf. 8 Fig. 2.

5. Votivrelief aus dem Pythion von Ikaria (Attika), nach Americ. Journal of Archaeol. 5 (1889) Taf. 11 nr. 3. S. ob. S. 89 f.
\clearpage
\vspace*{\fill}
\begin{figure}[H]
\centering
\includegraphics[width=0.8\textwidth,keepaspectratio]{figs/table07-01.jpeg}
\caption{\frakfamily Tafel 7. --- 1.}
\end{figure}
\vspace*{\fill}
\clearpage
\vspace*{\fill}
\begin{figure}[H]
\centering
\includegraphics[height=0.55\textheight,keepaspectratio]{figs/table07-02.jpeg}
\caption{\frakfamily Tafel 7. --- 2.}
\end{figure}
\vspace*{\fill}
\clearpage
\vspace*{\fill}
\begin{figure}[H]
\centering
\includegraphics[width=0.9\textwidth,keepaspectratio]{figs/table07-03.jpeg}
\caption{\frakfamily Tafel 7. --- 3.}
\end{figure}
\vspace*{\fill}
\clearpage
\vspace*{\fill}
\begin{figure}[H]
\centering
\includegraphics[width=0.8\textwidth,keepaspectratio]{figs/table07-04.jpeg}
\caption{\frakfamily Tafel 7. --- 4.}
\end{figure}
\vspace*{\fill}
\clearpage
\vspace*{\fill}
\begin{figure}[H]
\centering
\includegraphics[width=0.8\textwidth,keepaspectratio]{figs/table07-05.jpeg}
\caption{\frakfamily Tafel 7. --- 5.}
\end{figure}
\vspace*{\fill}
\clearpage
\subsection{Erläuterungen zu Tafel 8.}
\begin{center}
(Reliefs mit Eschara und delphischem Omphalos).
\end{center}
\paragraph{}
1. Theseus mit Eschara vor ihm, adoriert von Sosippos, Sohn des Nauarchides, Votivrelief aus Athen, nach Roschers Lex. d. Mythol. 1 Sp. 2499. S. ob. S. 117.

2. Linke Hälfte eines schönen figurenreichen Votivreliefs aus Phaleron, nach Ἐφημ. Ἀρχαιολ. 1909 Taf. 8: s. ob. S. 86 f. Hier ist der Omphalos ganz ähnlich gebildet wie in Taf. 7 Fig. 4.

3. Votivrelief aus Aigina: oben auf dem netzbedeckten, fast halbkugelförmigen Omphalos sitzen die zwei Adler, die Köpfe voneinander abwendend, nach Svoronos in der Ἐφημ. Ἀρχαιολ. 1912 πίν. 22. S. ob. S. 126 u. vgl. ob. Taf. 1 Fig. 9 (Münze von Megara).
\clearpage
\clearpage
\vspace*{\fill}
\begin{figure}[H]
\centering
\includegraphics[width=0.8\textwidth,keepaspectratio]{figs/table08-01.jpeg}
\caption{\frakfamily Tafel 8. --- 1.}
\end{figure}
\vspace*{\fill}
\clearpage
\vspace*{\fill}
\begin{figure}[H]
\centering
\includegraphics[height=0.55\textheight,keepaspectratio]{figs/table08-02.jpeg}
\caption{\frakfamily Tafel 8. --- 2.}
\end{figure}
\vspace*{\fill}
\clearpage
\vspace*{\fill}
\begin{figure}[H]
\centering
\includegraphics[width=0.8\textwidth,keepaspectratio]{figs/table08-03.jpeg}
\caption{\frakfamily Tafel 8. --- 3.}
\end{figure}
\vspace*{\fill}
\clearpage

\subsection{Erläuterungen zu Tafel 9.}
\begin{center}
(Verschiedenes).
\end{center}
\paragraph{}
1. Apollon als Pythonbesieger, den Siegespaian anstimmend. Vor ihm der halbkugelförmige, basislose Omphalos, vom toten Python umwunden. Wandgemälde aus Pompeji, nach Photographie P. Herrmanns. S. ob. S. 93 u. 94 Anm. 169.

2. Asklepiosstatue in Neapel, von der römischen Tiberinsel stammend. Neben Asklepios ein fast halbkugelförmiger netzbedeckter, basisloser Omphalos. Nach Roschers Lex. d. Mythol. 1 Sp. 634. S. ob. S. 111.

3. Der Omphalos von Jerusalem. Er befindet sich in der griechischen Kathedrale und zwar in "`einer Art Becher, mit einer umflochtenen gedrückten Kugel, der nach sehr alter Fabel die Mitte der Welt bezeichnete"' (Baedeker, Palästina 7, 39). Vgl. A. Jeremias, Handbuch d. altoriental. Geisteskultur S. 34 Anm. 4. Nach einer mir von Jeremias gütigst überlassenen Photographie gezeichnet von Alice Roscher. S. ob. S. 26 u. 130.

4. Mittelalterlicher Orbis terrarum mit dem Zentrum Jerusalem. Nach A. Jeremias, D. A. Testament im Lichte d. alt. Or. 2 S. 584. S. ob. S. 26 Anm. 48.

5. Votivrelief aus Athen nach Svoronos im Journ. Internat. d'archéol. num. 13 (1911) S. 302: Apollon Pythios, Leto und Artemis; zwischen ihnen der fast halbkugelförmige, auf einer Basis stehende, von zwei Adlern (die zurückblicken) flankierte Omphalos von Delphi. S. ob. S. 84 f.

6. Bienenkorbförmiger, netzbedeckter, von einer lebendigen Schlange umringelter Omphalos (= Terminus) Rechts und Links von ihm je ein Lar Compitalis. Wandgemälde in Pompeji nach Herculanum u. Pompeji... von H. Roux ainé u. L. Barré, deutsch von Kaiser, Hamburg 1841 2 Taf. 57. S. ob. S. 114.
\clearpage
\vspace*{\fill}
\begin{figure}[H]
\centering
\includegraphics[width=0.8\textwidth,keepaspectratio]{figs/table09-01.jpeg}
\caption{\frakfamily Tafel 9. --- 1.}
\end{figure}
\vspace*{\fill}
\clearpage
\vspace*{\fill}
\begin{figure}[H]
\centering
\includegraphics[height=0.75\textheight,keepaspectratio]{figs/table09-02.jpeg}
\caption{\frakfamily Tafel 9. --- 2.}
\end{figure}
\vspace*{\fill}
\clearpage
\vspace*{\fill}
\begin{figure}[H]
\centering
\includegraphics[width=0.75\textwidth,keepaspectratio]{figs/table09-03.jpeg}
\caption{\frakfamily Tafel 9. --- 3.}
\end{figure}
\vspace*{\fill}
\clearpage
\vspace*{\fill}
\begin{figure}[H]
\centering
\includegraphics[width=0.75\textwidth,keepaspectratio]{figs/table09-04.jpeg}
\caption{\frakfamily Tafel 9. --- 4.}
\end{figure}
\vspace*{\fill}
\clearpage
\vspace*{\fill}
\begin{figure}[H]
\centering
\includegraphics[width=0.85\textwidth,keepaspectratio]{figs/table09-05.jpeg}
\caption{\frakfamily Tafel 9. --- 5.}
\end{figure}
\vspace*{\fill}
\clearpage
\vspace*{\fill}
\begin{figure}[H]
\centering
\includegraphics[width=0.85\textwidth,keepaspectratio]{figs/table09-06.jpeg}
\caption{\frakfamily Tafel 9. --- 6.}
\end{figure}
\vspace*{\fill}
\clearpage
\end{document}
